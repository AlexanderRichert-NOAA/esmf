% $ Id: $

Each specific requirement (in the case above, support for each vendor) 
possesses the following attributes:  priority, source, verification, status, 
and notes, the last of which is optional.  These are typical for requirements
analysis ~\cite{wiegers}.  We'll now look at them in more detail.

\begin{description}
\item [Priority] The purpose of the priority attribute is to associate
each requirement with the milestones and longer term project goals that 
it satisfies.  Each requirement is assigned a number from 1-3, with
values defined as follows:
\begin{itemize}

\item 1:  This capability is directly required for a milestone {\it OR}
\newline Within the set of JMC applications that could use the 
utility or class in which this capability is embedded:
all that adopted the utility or class would require this 
capability in order to maintain their existing functionality;

\item 2: Within the set of JMC applications that could use the 
utility or class in which this capability is embedded:
some but not all of those that adopted the utility or class would 
require this capability in order to maintain their existing 
functionality.

\item 3:  This capability is desired in order to extend
the existing functionality of one or more JMC codes.

\end{itemize}

If some capability merits additional explanation to describe 
its priority, those preparing requirements are encouraged 
to elaborate.
 
\item [Source] The source attribute traces each capability
to the applications to which it applies.  In addition to applications
particular people or organizations may be noted.   This attribute 
helps to identify those that can provide further 
information and who may also be potential testers and users.  It
prevents the inclusion of features that have little likelihood of
being used.

\item [Verification] It's sensible to try to evaluate in some
objective and quantitative fashion whether a requirement is 
satisfied.  The verification attribute specifies how this assessment
will be performed.  Typical values include {\it code inspection}, 
{\it unit test} and {\it system test}.
Some capabilities may require the preparation of special data sets.

\item [Status] Throughout the course of this project it will be 
useful for us to track what has been accomplished and to archive 
ideas for extensions and improvements.  The status attribute identifies
each capability as:
\begin{itemize}
\item proposed;
\item approved-1 (approved for 1st code release at Milestone F);
\item approved-2 (approved for 2nd code release at Milestone G);
\item implemented; 
\item verified; or
\item rejected.  
\end{itemize}
Whether the cabability exists in other packages or models
is also helpful to note.

\item [Notes] This is a catch-all for additional information such
as background, references, any related design and implementation issues, 
risk factors, and so on.

\end{description}
