% $Id: comp_refdoc.tex,v 1.1 2001/12/13 20:03:39 cdeluca Exp $

\documentclass[]{article}

\usepackage{epsf}
\usepackage{html}
\usepackage[T1]{fontenc}

\textwidth 6.5in
\textheight 8.5in
\addtolength{\oddsidemargin}{-.75in}

\begin{document}

\bodytext{BGCOLOR=white LINK=#083194 VLINK=#21004A}

\begin{titlepage}

\begin{center}
{\Large Earth System Modeling Framework } \\
\vspace{.25in}
{\Large {\bf <Module, Library, Component or Model Name> <Language> Reference}} \\
\vspace{.25in}
{\large {\it Authors}}
\vspace{.5in}
\end{center}

\begin{latexonly}
\vspace{5.5in}
\begin{tabular}{p{5in}p{.9in}}
\hrulefill \\
\noindent {\bf NASA High Performance Computing and Communications Program} \\
\noindent Earth and Space Sciences Project \\
\noindent CAN 00-OES-01 \\
\noindent http://www.esmf.ucar.edu \\
\end{tabular}
\end{latexonly}

\end{titlepage}

\tableofcontents

\newpage
%\section{Synopsis}
% $Id: comp_syn.tex,v 1.6 2008/04/05 03:37:54 cdeluca Exp $
%
% Earth System Modeling Framework
% Copyright 2002-2008, University Corporation for Atmospheric Research, 
% Massachusetts Institute of Technology, Geophysical Fluid Dynamics 
% Laboratory, University of Michigan, National Centers for Environmental 
% Prediction, Los Alamos National Laboratory, Argonne National Laboratory, 
% NASA Goddard Space Flight Center.
% Licensed under the University of Illinois-NCSA License.

%\section{Synopsis}

<Brief synopsis of component or library.>


%\section{Architecture}
% $Id: comp_arch.tex,v 1.2 2002/07/25 17:15:39 eschwab Exp $

%\section{Architecture}

<Describe layering strategy and interaction of major components,
provide examples of high-level interfaces.>


\section{<Name1> Class Description}
% $Id: class_desc.tex,v 1.3 2002/10/14 21:54:05 cdeluca Exp $
%
% Earth System Modeling Framework
% Copyright 2002-2003, University Corporation for Atmospheric Research, 
% Massachusetts Institute of Technology, Geophysical Fluid Dynamics 
% Laboratory, University of Michigan, National Centers for Environmental 
% Prediction, Los Alamos National Laboratory, Argonne National Laboratory, 
% NASA Goddard Space Flight Center.
% Licensed under the GPL.

%\subsection{Description}

<Describe class function and relation to other classes.>


\subsection{Restrictions}
input{class_rest}

\section{<Name1> Interface}

%\subsection{Use and Examples}
% $Id: class_fex.tex,v 1.7 2008/04/05 03:37:52 cdeluca Exp $
%
% Earth System Modeling Framework
% Copyright 2002-2008, University Corporation for Atmospheric Research, 
% Massachusetts Institute of Technology, Geophysical Fluid Dynamics 
% Laboratory, University of Michigan, National Centers for Environmental 
% Prediction, Los Alamos National Laboratory, Argonne National Laboratory, 
% NASA Goddard Space Flight Center.
% Licensed under the University of Illinois-NCSA License.

%\subsection{F90 Use and Examples}

<Detailed examples of F90 usage of the class.>
 
% or
% $ Id: $

<Detailed examples of usage of the class.>











%\subsection{Parameters and Definitions}
% $Id: class_fparam.tex,v 1.5 2007/03/31 05:50:45 cdeluca Exp $
%
% Earth System Modeling Framework
% Copyright 2002-2007, University Corporation for Atmospheric Research, 
% Massachusetts Institute of Technology, Geophysical Fluid Dynamics 
% Laboratory, University of Michigan, National Centers for Environmental 
% Prediction, Los Alamos National Laboratory, Argonne National Laboratory, 
% NASA Goddard Space Flight Center.
% Licensed under the University of Illinois-NCSA License.

%\subsection{Parameters}

\begin{description}

\item [<ITEM1>] <Description of ITEM1.>

\item [<ITEM2>] <Description of ITEM2.>

\end{description}









% or
% $ Id: $

<Detailed examples of usage of the class.>











%\subsection{Class API}
% $Id: class_fapi.tex,v 1.1 2002/11/13 22:08:47 ekluz Exp $
%
% Earth System Modeling Framework
% Copyright 2002-2003, University Corporation for Atmospheric Research, 
% Massachusetts Institute of Technology, Geophysical Fluid Dynamics 
% Laboratory, University of Michigan, National Centers for Environmental 
% Prediction, Los Alamos National Laboratory, Argonne National Laboratory, 
% NASA Goddard Space Flight Center.
% Licensed under the GPL.

%\subsection{Parameters}

\begin{description}

\item [<ITEM1>] <Description of ITEM1.>

\item [<ITEM2>] <Description of ITEM2.>

\end{description}









% or
\input{class_capi}

\section{<Name2> Class}

% Subsections same as <Name1> Class ...

%\section{Glossary}
% $Id: comp_glos.tex,v 1.4 2006/11/16 05:20:53 cdeluca Exp $
%
% Earth System Modeling Framework
% Copyright 2002-2008, University Corporation for Atmospheric Research, 
% Massachusetts Institute of Technology, Geophysical Fluid Dynamics 
% Laboratory, University of Michigan, National Centers for Environmental 
% Prediction, Los Alamos National Laboratory, Argonne National Laboratory, 
% NASA Goddard Space Flight Center.
% Licensed under the University of Illinois-NCSA License.

% USAGE NOTE:
%
% The first use of a term in the text of a document can be 
% linked to the corresponding glossary item using the item label.
% 
% For example,
%
% Original document text: code must include item1
%
% Linked to glossary:     code must include \htmlref{item1}{glos:item1}
%
% The link will appear in the html version of the document.
% The print version of the document will appear unchanged.

\begin{description}

\item [item1] \label{glos:item1} <Definition of item1.>

\item [item2] \label{glos:item2} <Definition of item2.>

\end{description}










%\section{Bibliography}
\bibliography{comp} 
\bibliographystyle{plain}
\addcontentsline{toc}{section}{Bibliography}

\end{document}











