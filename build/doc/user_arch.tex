% $Id: user_arch.tex,v 1.19 2007/02/27 17:31:56 svasquez Exp $

% List of architectures supported.  This file is 
% meant to be included in a user doc.

The following two tables list various combinations of environment 
variable settings used by the ESMF build system. A {\tt default}
value in the compiler column indicates the vendor compiler. A {\tt mpi}
value in the comm column indicates the vendor MPI implementation.

The first table lists the exact combinations which are tested regularly and are
fully supported. The second table lists all possible combinations which are 
included in the build system.

\vspace{1ex}
{\bf Fully tested combinations}:
\vspace{1ex}

\begin{tabular}{lcccc}
  &{\bfseries ESMF\_OS} &{\bfseries ESMF\_COMPILER} & {\bfseries ESMF\_COMM} & {\bfseries ESMF\_ABI} \\

IBM SP      &  AIX     &  default      &  mpi         &  32,64  \\
Mac         &  Darwin  &  absoft       &  lam,mpiuni  &  32  \\
Mac         &  Darwin  &  nag          &  lam,mpiuni  &  32  \\
Mac         &  Darwin  &  xlf          &  lam,mpiuni  &  32  \\
Mac         &  Darwin  &  xlfgcc       &  lam,mpiuni  &  32  \\
SGI Mips    &  IRIX64  &  default      &  mpi         &  32,64  \\
PC          &  Linux   &  absoftintel  &  mpich       &  32  \\
PC          &  Linux   &  g95          &  mpich       &  32  \\
HP IA64     &  Linux   &  intel        &  lam         &  64  \\
SGI Altix   &  Linux   &  intel        &  mpi         &  64  \\
PC          &  Linux   &  intel        &  mpich       &  32  \\
PC 	    &  Linux   &  lahey        &  mpiuni      &  32  \\
PC 	    &  Linux   &  nagintel     &  mpich       &  32  \\
IBM Opteron &  Linux   &  pathscale    &  mpich       &  64  \\
IBM Opteron &  Linux   &  pgi          &  mpich       &  64  \\
PC 	    &  Linux   &  pgi          &  mpich,mpiuni&  32  \\
HP/Compaq   &  OSF1    &  default      &  mpi         &  64  \\
Cray X1     &  Unicos  &  default      &  mpi         &  64
\end{tabular}

\vspace{1ex}

{\bf All possible options}. Where multiple options exist 
and the default is independent of {\tt ESMF\_MACHINE} the default value is in bold:

\vspace{1ex}


\begin{tabular}{lcccc}
  {\bfseries ESMF\_OS} &{\bfseries ESMF\_COMPILER} & {\bfseries ESMF\_COMM} & {\bfseries ESMF\_ABI} \\

AIX     &  default      &  {\bf mpi},mpiuni,user         &  32,{\bf 64}  \\
Darwin  &  absoft       &  {\bf mpiuni},mpich,mpich2,lam,openmpi,user  &  32  \\
Darwin  &  g95          &  {\bf mpiuni},mpich,mpich2,lam,openmpi,user  &  32  \\
Darwin  &  gfortran     &  {\bf mpiuni},mpich,mpich2,lam,openmpi,user  &  32  \\
Darwin  &  intel        &  {\bf mpiuni},mpich,mpich2,lam,openmpi,user  &  32  \\
Darwin  &  nag          &  {\bf mpiuni},mpich,mpich2,lam,openmpi,user  &  32  \\
Darwin  &  xlf          &  {\bf mpiuni},mpich,mpich2,lam,openmpi,user  &  32  \\
Darwin  &  xlfgcc       &  {\bf mpiuni},mpich,mpich2,lam,openmpi,user  &  32  \\
IRIX64  &  default      &  {\bf mpi},mpiuni,user         &  32,{\bf 64}  \\
Linux   &  absoft       &  {\bf mpiuni},mpich,mpich2,lam,openmpi,user  &  32, 64  \\
Linux   &  absoftintel  &  {\bf mpiuni},mpich,mpich2,lam,openmpi,user  &  32, 64  \\
Linux   &  intel        &  {\bf mpiuni},mpi,mpich,mpich2,intelmpi,lam,openmpi,user  &  32, 64 \\
Linux   &  intelgcc     &  {\bf mpiuni},mpi,mpich,mpich2,lam,openmpi,user  &  32, 64 \\
Linux   &  lahey        &  {\bf mpiuni},mpich,mpich2,lam,openmpi,user  &  32  \\
Linux   &  nag          &  {\bf mpiuni},mpich,mpich2,lam,openmpi,user  &  32  \\
Linux   &  nagintel     &  {\bf mpiuni},mpich,mpich2,lam,openmpi,user  &  32  \\
Linux   &  pathscale    &  {\bf mpiuni},mpich,mpich2,lam,openmpi,user  &  64  \\
Linux   &  pgi          &  {\bf mpiuni},mpich,mpich2,lam,openmpi,user  &  32, 64  \\
Linux   &  pgigcc       &  {\bf mpiuni},mpich,mpich2,lam,openmpi,user  &  32, 64  \\
Linux   &  xlf          &  {\bf mpiuni},mpich,mpich2,lam,openmpi,user  &  32  \\
OSF1    &  default      &  {\bf mpi},mpiuni,user         &  64  \\
Unicos  &  default      &  {\bf mpi},mpiuni,user         &  64  \\
Unicos  &  pgi          &  {\bf mpi},mpiuni,user         &  64

\end{tabular}

\vspace{1ex}

