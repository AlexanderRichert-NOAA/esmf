%  $Id: ESMF_builddetail.tex,v 1.13 2005/02/11 20:42:58 nscollins Exp $


\subsection{Make System}
\label{sec:make}
For most users the description of the build system in previous
sections should be sufficient.  Some users, however, may wish to have
a more detailed knowledge of the make system either for configuring
different build options or for porting to unsupported platforms.
\subsubsection{General Structure}

The ESMF build system is divided into two parts.  The first is the
series of makefiles located with the source code.  The second is a set
of makefile fragment files designed to be used by the source code
makefiles.  Makefile fragment files are files that contain makefile
syntax defining build rules and actions, but do not constitute a
complete build system.

The main components of the make system are:
\label{sec:BuildOptions}
\begin{itemize}
\item{{\bf Build directories}}

There are two directories containing makefile fragment files used by
the make system.  

The {\tt build} directory contains the generic makefile fragment file
common.mk that is included by the top level makefile in the source
tree.  common.mk contains generic build system settings and build
rules used across all platforms.  A user should have no reason to edit
common.mk.

The {\tt build\_config} directory contains makefile fragments for each
supported platform defining compilers, compiler flags, and the various
other definitions that are necessary to build on each platform.  Files
can also be added to this directory for specific machines where the
build settings are different from the standards of the architecture.
One of the files in this directory will be included by the
build/common.mk file depending the values of the environment variables
ESMF\_ARCH, ESMF\_COMPILER and ESMF\_SITE.  See below for more details
on environment variables.

\item{{\bf Environment Variables}}

The three sets of source codes that the build system supports all need
environment variables set to point to their top level source code
directories.

\begin{description}

\item{ESMF Library} 

To build the ESMF library, ESMF\_DIR needs to be set to the top level ESMF
library source code directory.

\item{Implementation Report} 

The build system needs ESMF\_IMPL\_DIR set to the top level source
code directory of the Implementation Report source tree to build the
report and to build and run the examples.

\item{EVA Applications} 

An EVA source code tree does not contain a copy of the ESMF build
system.  Instead it uses a copy found in an ESMF library source code
tree.  Building the EVA applications requires that ESMF\_EVA\_DIR and
ESMF\_DIR be set.  ESMF\_EVA\_DIR has to be set to the top directory
of the EVA source code.  ESMF\_DIR has to be set to the top directory
of an ESMF source code tree.

\end{description}

There are several other variables that the build system uses.  If they
are not set, then the build system will assign default values to
them.  In most cases the default values will be fine.

\begin{description}

\item{ESMF\_ARCH} 

Defines the current architecture. Default value is the value returned
by the command uname -s.  There should be almost no reason for
ESMF\_ARCH to be set by a user.

\item{ESMF\_BOPT} 

Variable specifying that the build system use either debugging or
optimization options.  Value of g specifies the debugging options.
Value of O (capital oh) specifies optimization options.  Default value
is O.

\item{ESMF\_COMM}

Defines which MPI communications library to use.  Many times a machine
will come with its own MPI library and in those cases the default
setting will be mpi.  Otherwise the default setting will be mpiuni so
that the mpiuni library will be used.  Other possible settings are
mpich and lam.

\item{ESMF\_COMPILER}

Variable specifying which compiler to use.  Value can be default,
absoft, intel, lahey, xlf, or nag.  If the value is default or the
ESMF\_COMPILER is left unset, then the default compiler will be
used. On systems which usually come with a single vendor-supplied
compiler, the default is to use this compiler.  On systems like
Linux clusters where there is no single vendor compiler installed
on all systems, you will generally want to set this.
The default for Linux systems is lahey and on Darwin (Mac OS X)
systems it is absoft.

\item{ESMF\_C\_COMPILER}

Variable specifying which C/C++ compiler to use.   In most cases
this should not be set; the default is to use the vendor-supplied
compilers on those systems which normally come with a single 
development environment; on other systems like Linux clusters,
the default is to use the ESMF\_COMPILER setting.  However, some
Fortran compilers support linking with C and C++ code compiled
with the GNU compilers; in these cases, you can set this variable
to the value gnu in order to compile with the vendor Fortran
compiler and the GNU C/C++.   This option is only supported for
ESMF\_ARCH=Darwin, ESMF\_COMPILER=xlf, and ESMF\_ARCH=Linux,
ESMF\_COMPILER=intel.

\item{ESMF\_C\_LIBRARY}

Variable specifying which C/C++ libraries to link with.  In most cases
this should not be set; the default is to use the vendor-supplied
libraries on those systems which normally come with a single 
development environment; on other systems like Linux clusters,
the default is to use the ESMF\_COMPILER setting.  However, some
vendor compilers support compiling with their compilers but linking
with the GNU C/C++ libraries.
In these cases, you can set this variable
to the value gnu in order to link with the GNU C/C++ libraries.
This option is only supported for ESMF\_ARCH=Linux, ESMF\_COMPILER=intel.

\item{ESMF\_EXHAUSTIVE}

The unit tests by default compile only a subset of the unit tests.
If ESMF\_EXHAUSTIVE is set to the value {\tt ON}, then when
compiling the unit tests, all tests will be included.  Note that this
is a compile time, not run time, option.

\item{ESMF\_NO\_IOCODE}

The current release of the system is prepared to link with the
{\tt netCDF} I/O libraries, but since the installation of the
libraries and include files varies widely from system to system
support for them is disabled by default.  To enable I/O support,
edit the {\tt build/common.mk} file and comment out the setting
of both the CPPFLAG and environment variable.  Additional customization
will be needed in the build\_config makefile fragments to point the
framework to the location of the include and library files.

\item{ESMF\_PREC}

Variable specifying the size of an address on systems which can build
either 32 or 64 bit executables.
When possible the default value will be 64, otherwise it will be 32.

\item{ESMF\_SITE}

If ESMF\_SITE is not set or has the value of default, default build
settings for the current machine architecture and compiler will be
used.  Values are created from the user's site.

\end{description}


\item{{\bf Makefiles}}

Every source tree contains a makefile in its top level directory.  This
makefile includes the common.mk file from the {\tt build} directory.
The top level makefile contains makefile settings specific for the
source code that it is found in.

Each directory in the source tree contains a makefile which includes
the top level makefile.  These local makefiles include definitions that
allow the local files and documents to be built.
\end{itemize}

\subsubsection{Build Configuration}

A single makefile or makefile fragment from the build system never
constitutes a complete set of build rules and settings.  Starting from
the local makefile, successive include commands are used to string
together makefiles and makefile fragments to create a complete system
of build rules and settings.  Configuration of the build system is
done by including a configuration makefile fragment.  The build
system can be configured for a machine's architecture or, if needed,
for a particular machine and its compiler. A configuration for a
specific machine or compiler is referred to as a site configuration.

The string of files included is fairly short.  Makefiles below the top
level makefile include the top level makefile. The top level makefile
includes build/common.mk and then build/common.mk includes a
configuration file from the build\_config directory.  The configuration
files in the build\_config directory contain the architecture and site
specific build settings.  The architecture, compiler and site that a file
configures is determined by its name.  The configuration makefile
fragments follow this naming convention:

\begin{verbatim}
    ESMF_ARCH.ESMF_COMPILER.ESMF_SITE/build_rules.mk
\end{verbatim}

Where ESMF\_ARCH, ESMF\_COMPILER, and ESMF\_SITE are environment
variables either set by the user or given default values by the build
system. ESMF\_ARCH is the current architecture and will have the value
returned by the command {\tt uname -s}.  ESMF\_COMPILER is the compiler
name.  ESMF\_SITE is the current machine name. If there are no site
specific files for a particular architecture, then ESMF\_COMPILER and
ESMF\_SITE will be set to default values.  Examples:

\begin{verbatim}
    AIX.default.default/build_rules.mk      ! Default configuation for RS6000.
    Linux.lahey.default/build_rules.mk      ! Linux configuation using lahey compilers.
\end{verbatim}

\subsubsection{Source Code Configuration}

C++ and C source code written to build on a range of platforms many
times require preprocessor directives to configure the source code for
specific platforms.  The directives are included in the source code
and are processed by the C preprocessor (cpp) before the source code
is compiled.  The directives are used to determine among other things,
the memory requirements of variable types and the system resources
that are available.

The ESMF build system provides preprocessor directives in 
{\tt ESMC\_Conf.h} and {\tt ESMF\_Conf.inc} files
that are included in the source code.  The path to these files is

\begin{verbatim}
    build_config/ESMF_ARCH.ESMF_COMPILER.ESMF_PREC.ESMF_SITE
\end{verbatim}

Where ESMF\_ARCH, ESMF\_COMPILER, ESMF\_PREC and ESMF\_SITE are
environment variables set by the user or given default values be the
build system.  Based on the settings of these environment variables,
the build system provides a path to the correct files during
source code compiles.

\subsubsection{Building on New Platforms}

The build system can be ported to other Unix platforms by adding new
makefile fragments and configuration files.  The new makefile fragment 
file has to follow the naming
convention used by the existing makefile fragment files and be created in the
directory build\_config.  The new file will also have to define
the same makefile variables as the existing makefile fragment files.

Porting to a new machine will require new configuration files as well. 
New configuration files have to define the same machine attributes as
existing configuration files. Example:

\begin{verbatim}
      build_config/Linux.pgi.mysite/build_rules.mk
      build_config/Linux.pgi.mysite/ESMF_Conf.inc
      build_config/Linux.pgi.mysite/ESMC_Conf.h
\end{verbatim}

