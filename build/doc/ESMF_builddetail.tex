%  $Id: ESMF_builddetail.tex,v 1.1 2003/08/26 17:40:16 flanigan Exp $


\subsection{Make System}
For most users the description of the build system in the Quick Start
section should be sufficient.  Some users, however, may wish to have a
more detailed knowledge of the make system that is used by the library
either for configuring different build options or other reasons.
\subsubsection{General Structure}

The ESMF build system is broken into two parts.  The first is the
series of makefiles located with the source code.  The second is a set
of makefile fragements files designed to be used by the source code
makefiles.  Makefile fragment files are files that contain makefile
syntax defining build rules and actions, but do not constitute a
complete build system.

The main components of the make system are:
\begin{itemize}
\item{Build directories}

There are two directories containing makefile fragment files used by
the make system.  Makefile fragment files are files that contain
makefile syntax defining build rules and variables, but do not
constitute a complete build system.

The {\tt esmf\_build} directory contains a generic makefile fragment
file common.mk that is included by the top level Makefile in the
source tree.  common.mk contains generic build system settings and
build rules used across all platforms.

The {\tt esmf\_build\_config} directory contains makefile fragments
for each supported platform defining compilers, compiler flags, and
the various other definitions that are necessary to build on each
platform.  Files can also be added to this directory for specific
machines where the build settings are different from the standards of
the architecture.  One of the files in this directory will be included
by the esmf\_build/common.mk file depending the values of the environment
variable EMSF\_ARCH and ESMF\_SITE.  See below for more details on this
topic.

\item{Environment Variables}

The three sets of source code that the build system support all need
environment variables set to point to their top level source code
directories. 

\begin{itemize}

\item{ESMF Library} The ESMF library needs ESMF\_DIR set .

\item{Implementation Report} The Implementation Report needs ESMF\_IMPL\_DIR set.  


\item{EVA Applications} An EVA source code tree does not contain a copy of the ESMF build
system.  Instead it uses a copy found in an ESMF source code tree.
ESMF\_EVA\_DIR has to be set to the top most directory of the EVA source
code.  ESMF\_DIR has to be set to the top most directory of an ESMF
source code tree.


\end{itemize}

\begin{description}

\item{ESMF\_ARCH} 
Defines current architecture. Default value is value returned by the
command uname -s.  There should be almost no reason for ESMF\_ARCH to
be set by a user.


\item{ESMF\_SITE}

If ESMF\_SITE is not set or has the value of default, default build
settings for the current machine architecture will be used.  Values
are created from a machine's name and if needed, the compiler used.
Example ESMF\_SITE values could be blackforest or jazz\_lahey.

\item{ESMF\_PREC}

Variable specifying the precession build arguments.  Value can be 32
or 64.  When possible the default value will be 64, otherwise it will
be 32.

\item{ESMF\_BOPT} 

Variable specifying that the build system use either debugging or
optimization options.  Value of g specifies the debugging options.
Value of O (capital oh) specifies optimization options.  Default value
is O.

\end{description}


\item{Top level Makefile}

The top level makefile includes the common.mk file from the {\tt
esmf\_build} directory.  Many of the commands in this Makefile spawn a recursive make
through the directory structure.

\item{Source tree Makefiles}

Each directory contains a Makefile which includes the {\tt build} common
makefiles.  These local Makefiles include defintions that allow the local files
and documents to be built.
\end{itemize}

\subsubsection{Configuration}



\subsection{Build Options}
\label{BuildOptions}

There is an install target which will copy the library and *.mod files to
specified directories.  To invoke this target use:
\begin{verbatim}
  gmake BOPT=[O,g] ESMF_LIB_INSTALL=<path for library>
                   ESMF_MOD_INSTALL=<path for *.mod files> install 
\end{verbatim}

Some users may wish for the library to be built in a directory different from 
where the source code resides.  To do this, build using:
\begin{verbatim}
   gmake ESMF_BUILD=<path> BOPT=[O,g]
\end{verbatim}

The {\tt ESMF\_BUILD} variable gives an alternate path in which to
place the libraries, *.mod files and object files.  This variable
defaults to {\tt ESMF\_DIR}.  If it is assigned another value, the
{\tt ESMF\_BUILD} variable will need to be passed as an additional
argument to the the above make commands.  (Alternatively the variable
{\tt ESMF\_BUILD} can be set in the environment (using setenv or
export) and then it need not be passed to any make calls).
