%  $Id$


\subsection{Make System}
\label{sec:make}
For most users the description of the build system in previous
sections should be sufficient.  Some users, however, may wish to have
a more detailed knowledge of the make system either for configuring
different build options or for porting to unsupported platforms.
\subsubsection{General Structure}

The ESMF build system is divided into two parts.  The first is the
series of makefiles located with the source code.  The second is a set
of makefile fragment files designed to be used by the source code
makefiles.  Makefile fragment files are files that contain makefile
syntax defining build rules and actions, but do not constitute a
complete build system.

The main components of the make system are:
\label{sec:BuildOptions}
\begin{itemize}
\item{{\bf Build directories}}

There are two directories containing makefile fragment files used by
the make system.  

The {\tt build} directory contains the generic makefile fragment file
common.mk that is included by the top level makefile in the source
tree.  common.mk contains generic build system settings and build
rules used across all platforms.  A user should have no reason to edit
common.mk.

The {\tt build\_config} directory contains makefile fragments for each
supported platform defining compilers, compiler flags, and the various
other definitions that are necessary to build on each platform.  Files
can also be added to this directory for specific machines where the
build settings are different from the standards of the platform.
One of the files in this directory will be included by the
build/common.mk file depending the values of the environment variables
ESMF\_OS, ESMF\_COMPILER and ESMF\_SITE.  See below for more details
on environment variables.

\item{{\bf Environment Variables}}

The three sets of source codes that the build system supports all need
environment variables set to point to their top level source code
directories.

\begin{description}

\item{ESMF Library} 

To build the ESMF library, ESMF\_DIR needs to be set to the top level ESMF
library source code directory.

\item{Implementation Report} 

The build system needs ESMF\_IMPL\_DIR set to the top level source
code directory of the Implementation Report source tree to build the
report and to build and run the examples.

\item{EVA Applications} 

An EVA source code tree does not contain a copy of the ESMF build
system.  Instead it uses a copy found in an ESMF library source code
tree.  Building the EVA applications requires that ESMF\_EVA\_DIR and
ESMF\_DIR be set.  ESMF\_EVA\_DIR has to be set to the top directory
of the EVA source code.  ESMF\_DIR has to be set to the top directory
of an ESMF source code tree.

\end{description}

There are several other variables that the build system uses.  If they
are not set, then the build system will assign default values to
them. In most cases the default values will be fine.

\begin{description}

\item[ESMF\_DIR]
The environment variable {\tt ESMF\_DIR} must be set to the full pathname 
of the top level ESMF directory before building the framework.  This is the 
only environment variable which is required to be set on all platforms under 
all conditions.

\item[ESMF\_BOPT]
This environment variable controls the build option. To make a debuggable
version of the library set {\tt ESMF\_BOPT} to {\tt g} before building. The
default is {\tt O} (capital oh) which builds an optimized version of the 
library. If {\tt ESMF\_BOPT} is {\tt O}, {\tt ESMF\_OPTLEVEL} can also be set
to a numeric value between 0 and 4 to select a specific optimization level.

\item[ESMF\_COMM]
On systems with a vendor-supplied MPI communications library the vendor library 
is chosen by default for communications and {\tt ESMF\_COMM} need not be set.
For other systems (e.g. Linux or Darwin) a multitude of MPI implementations is
available and {\tt ESMF\_COMM} must be set to indicate which implementation is
used to build the ESMF library. Set {\tt ESMF\_COMM} according to your situation
to: {\tt mpich, mpich2, lam, openmpi} or {\tt intelmpi}. {\tt ESMF\_COMM} may
also be set to {\tt user} indicating that the user will set all the required
flags using advanced ESMF environment variables.

Alternatively, ESMF comes with a single-processor MPI-bypass library which is
the default for Linux and Darwin systems. To force the use of this bypass
library set {\tt ESMF\_COMM} equal to "mpiuni".

\item[ESMF\_COMPILER]
The ESMF library build requires a working Fortran90 and C++ compiler. On 
platforms that don't come with a single vendor supplied compiler suite
(e.g. Linux or Darwin) {\tt ESMF\_COMPILER} must be set to select which Fortran
and C++ compilers are being used to build the ESMF library. Notice that setting
the {\tt ESMF\_COMPILER} variable does {\em not} affect how the compiler
executables are located on the system. {\tt ESMF\_COMPILER} (together with
{\tt ESMF\_COMM}) affect the name that is expected for the compiler executables.
Furthermore, the {\tt ESMF\_COMPILER} setting is used to select compiler and
linker flags consistent with the compilers indicated.

By default Fortran and C++ compiler executables are expected to be located in
a location contained in the user's {\tt PATH} environment variable. This means
that if you cannot locate the correct compiler executable via the {\tt which}
command on the shell prompt the ESMF build system won't find it either!

There are advanced ESMF environment variables that can be used to select 
specific compiler executables by specifying the full path. This can be used to
pick specific compiler executables without having to modify the {\tt PATH}
environment variable.

Use 'gmake info' to see which compiler executables the ESMF build system will
be using according to your environment variable settings.

To see possible values for {\tt ESMF\_COMPILER}, cd to 
{\tt \$ESMF\_DIR/build\_config} and list the directories there. The first part 
of each directory name corresponds to the output of 'uname -s' for this 
platform. The second part contains possible values for {\tt ESMF\_COMPILER}. In
some cases multiple combinations of Fortran and C++ compilers are possible, e.g.
there is {\tt intel} and {\tt intelgcc} available for Linux. Setting 
{\tt ESMF\_COMPILER} to {\tt intel} indicates that both Intel Fortran and 
C++ compilers are used, whereas {\tt intelgcc} indicates that the Intel Fortran
compiler is used in combination with GCC's C++ compiler.

If you do not find a configuration that matches your situation you will need to
port ESMF.

\item[ESMF\_ABI]
If a system supports 32-bit and 64-bit (pointer wordsize) application binary
interfaces (ABIs), this variable can be set to select which ABI to use. Valid 
values are {\tt 32} or {\tt 64}. By default the most common ABI is chosen.

\item[ESMF\_SITE]
The Sourceforge {\tt esmfcontrib} repository contains makefiles which have 
already been customized for certain machines.  If one exists for your site 
and you wish to use it, download the corresponding files into the 
{\tt build\_contrib} directory and set {\tt ESMF\_SITE} to your location
(which corresponds to the last part of the directory name).  See the 
Sourceforge site {\tt http://sourceforge.net/projects/esmfcontrib} for more 
information.


\end{description}


\item{{\bf Makefiles}}

Every source tree contains a makefile in its top level directory.  This
makefile includes the common.mk file from the {\tt build} directory.
The top level makefile contains makefile settings specific for the
source code that it is found in.

Each directory in the source tree contains a makefile which includes
the top level makefile.  These local makefiles include definitions that
allow the local files and documents to be built.
\end{itemize}

\subsubsection{Build Configuration}

A single makefile or makefile fragment from the build system never
constitutes a complete set of build rules and settings.  Starting from
the local makefile, successive include commands are used to string
together makefiles and makefile fragments to create a complete system
of build rules and settings.  Configuration of the build system is
done by including a configuration makefile fragment. A configuration for a
specific machine or compiler is referred to as a site configuration.

The string of files included is fairly short.  Makefiles below the top
level makefile include the top level makefile. The top level makefile
includes build/common.mk and then build/common.mk includes a
configuration file from the build\_config directory.  The configuration
files in the build\_config directory contain the platform and site
specific build settings.  The os, compiler and site that a file
configures is determined by its name.  The configuration makefile
fragments follow this naming convention:

\begin{verbatim}
    ESMF_OS.ESMF_COMPILER.ESMF_SITE/build_rules.mk
\end{verbatim}

Where ESMF\_OS, ESMF\_COMPILER, and ESMF\_SITE are environment
variables either set by the user or given default values by the build
system. ESMF\_OS is the target operating system. If the build is performed
on the target system ESMF\_OS will typically have the value returned by the
command {\tt uname -s}. ESMF\_COMPILER is the compiler name. ESMF\_SITE, if set,
is generally the current machine name, the location, or the organization (e.g.
mit, cola).  If there are no site specific files for a particular platform, then
ESMF\_COMPILER and ESMF\_SITE will be set to {\tt default}.  Examples:

\begin{verbatim}
    AIX.default.default/build_rules.mk      ! Default configuation for IBM SP
    Linux.lahey.default/build_rules.mk      ! Linux configuation using lahey compilers.
\end{verbatim}

\subsubsection{Source Code Configuration}

C++ and C source code written to build on a range of platforms many
times require preprocessor directives to configure the source code for
specific platforms.  The directives are included in the source code
and are processed by the C preprocessor (cpp) before the source code
is compiled.  The directives are used to determine among other things,
the memory requirements of variable types and the system resources
that are available.

The ESMF build system provides preprocessor directives in 
{\tt ESMC\_Conf.h} and {\tt ESMF\_Conf.inc} files
that are included in the source code.  The path to these files is

\begin{verbatim}
    build_config/ESMF_OS.ESMF_COMPILER.ESMF_ABI.ESMF_SITE
\end{verbatim}

Where ESMF\_OS, ESMF\_COMPILER, ESMF\_ABI and ESMF\_SITE are
environment variables set by the user or given default values be the
build system.  Based on the settings of these environment variables,
the build system provides a path to the correct files during
source code compiles.

\subsubsection{Building on New Platforms}

The build system can be ported to other Unix platforms by adding new
makefile fragments and configuration files.  The new makefile fragment 
file has to follow the naming
convention used by the existing makefile fragment files and be created in the
directory build\_config.  The new file will also have to define
the same makefile variables as the existing makefile fragment files.

Porting to a new machine will require new configuration files as well. 
New configuration files have to define the same machine attributes as
existing configuration files. Example:

\begin{verbatim}
      build_config/Linux.pgi.mysite/build_rules.mk
      build_config/Linux.pgi.mysite/ESMF_Conf.inc
      build_config/Linux.pgi.mysite/ESMC_Conf.h
\end{verbatim}

