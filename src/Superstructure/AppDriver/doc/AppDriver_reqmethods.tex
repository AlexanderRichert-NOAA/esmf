% $Id: AppDriver_reqmethods.tex,v 1.19 2011/01/05 20:05:46 svasquez Exp $
%
% Earth System Modeling Framework
% Copyright 2002-2011, University Corporation for Atmospheric Research, 
% Massachusetts Institute of Technology, Geophysical Fluid Dynamics 
% Laboratory, University of Michigan, National Centers for Environmental 
% Prediction, Los Alamos National Laboratory, Argonne National Laboratory, 
% NASA Goddard Space Flight Center.
% Licensed under the University of Illinois-NCSA License.

\subsubsection{User-code {\tt SetServices} method}

Many programs call some library routines.  The library
documentation must explain what the routine name is, what arguments 
are required and what are optional, and what the code does.  

In contrast, all ESMF components must be written to {\it be called}
by another part of the program; in effect, an ESMF component takes the 
place of a library.  The interface is prescribed by the framework,
and the component writer must provide specific subroutines which 
have standard argument lists and perform specific operations.
For technical reasons {\em none} of the arguments in user-provided subroutines
must be declared as {\em optional}.

The only {\em required} public interface of a Component is its
set services method.  This subroutine must have an
externally accessible name (be a public symbol), take a component
as the first argument, and an integer return code as the second. 
Both arguments are required and must {\em not} be declared as 
{\tt optional}. If an intent is specified in the interface it must be 
{\tt intent(inout)} for the first and {\tt intent(out)} for the 
second argument. The subroutine name is not predefined, it is set by the
component writer, but must be provided as part of the component 
documentation.

The required function that the set services subroutine must provide is to
specify the user-code entry points for the standard ESMF Component methods. To
this end the user-written set services routine calls the

{\tt ESMF\_<Grid/Cpl>CompSetEntryPoint()} method to set each 
Component entry point.

See sections \ref{sec:GridSetServ} and \ref{sec:CplSetServ} for examples of
how to write a user-code set services routine.

Note that a component does not call its own set services routine;
the AppDriver or parent component code, which is creating a component, 
will first call {\tt ESMF\_<Grid/Cpl>CompCreate()} to create an "empty" 
component, and then must call into {\tt ESMF\_<Grid/Cpl>CompSetServices()}, 
supplying the user-code set services routine as an argument. The framework
calls into the user-code set services, after the Component's VM has been
started up.


\subsubsection{User-code {\tt Initialize}, {\tt Run}, and {\tt Finalize} methods}

The required standard ESMF Component methods, for which user-code entry
points must be set, are Initialize, Run, and Finalize. Currently optional,
a Component may also set entry points for the WriteRestart and
ReadRestart methods.

Sections \ref{sec:GridSetServ} and \ref{sec:CplSetServ} provide examples
of how the entry points for Initialize, Run, and Finalize are set during
the user-code set services routine, using the 
{\tt ESMF\_<Grid/Cpl>CompSetEntryPoint()} library call.

All standard user-code methods must abide {\em exactly} to the prescribed
interfaces. {\em None} of the arguments must be declared as {\em optional}.

The names of the Initialize, Run, and Finalize user-code subroutines do
not need to be public; in fact it is far better for them to be private to
lower the chances of public symbol clashes between different components.

See sections \ref{sec:GridInitialize}, \ref{sec:GridRun},
\ref{sec:GridFinalize}, and \ref{sec:CplInitialize}, \ref{sec:CplRun},
\ref{sec:CplFinalize} for examples of how to write entry points for the 
standard ESMF Component methods.


\subsubsection{User-code {\tt SetVM} method}
\label{sec:AppDriverSetVM}

When the AppDriver or parent component code calls
{\tt ESMF\_<Grid/Cpl>CompCreate()} it has the option to specify a 
{\tt petList} argument. All of the parent PETs contained in this list become
resources of the child component. By default all of the parent PETs are 
provided to the child component.

Unless the optional {\tt contextflag} argument is used during 
{\tt ESMF\_<Grid/Cpl>CompCreate()}, to indicate that the child component will
execute in the same VM as the parent, the child component has the option
to set certain aspects of how the provided resources are to be used when
executing child component methods. The resources provided via the parent PETs
are the associated processing elements (PEs) and virtual address spaces (VASs).

The optional user-written set vm routine is called from the parent
through the {\tt ESMF\_<Grid/Cpl>CompSetVM()} library code, and is the only
place where the child component can set aspects of its own VM before it is 
started up. The child component's VM must be running before its set services
routine can be called, and thus the optional {\tt ESMF\_<Grid/Cpl>CompSetVM()}
call must be placed {\em before} {\tt ESMF\_<Grid/Cpl>CompSetServices()}.

If called by the parent, the user-code set vm routine has the option to
specify how the PETs of the child component share the provided parent PEs.
Further, PETs on the same single system image can be set to run multi-threaded,
within a reduced number of VAS, allowing a component to leverage shared memory
concepts.

Sections \ref{sec:GridSetVM} and \ref{sec:CplSetVM} provide examples for
simple user-written set vm routines.
