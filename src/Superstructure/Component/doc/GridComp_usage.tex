% $Id: GridComp_usage.tex,v 1.4 2004/04/21 22:05:58 cdeluca Exp $
%
% Earth System Modeling Framework
% Copyright 2002-2003, University Corporation for Atmospheric Research, 
% Massachusetts Institute of Technology, Geophysical Fluid Dynamics 
% Laboratory, University of Michigan, National Centers for Environmental 
% Prediction, Los Alamos National Laboratory, Argonne National Laboratory, 
% NASA Goddard Space Flight Center.
% Licensed under the GPL.

%\subsection{Use and Examples}

A Gridded Component is a computational entity which
consumes and produces data.  It uses a State object to
exchange data between itself and other Components.  It 
uses a Clock object to manage time.

\subsubsection{The SetServices Method}

Most programs call some sort of library routines.  The library
documentation must explain what the routine name is, what arguments 
are required and what are optional, and what the code does.  

In contrast, all ESMF Components must be written to {\it be called}
by another part of the program; in effect, an ESMF Component takes the 
place of a library.  In this case the interface is already prescribed,
so the Component writer must provide specific subroutines which 
have fixed argument lists and perform specific operations.

One of the required interfaces a Component must provide is 
the {\tt SetServices} method.  This subroutine must have an
externally accessible name (be a public symbol), take an {\tt ESMF\_GridComp}
as the first argument, and an integer return code as the second.
The subroutine name is not predefined, it is set by the component
writer, but must be provided as part of the component documentation.

The required function of the {\tt SetServices} subroutine is to
register the rest of the required functions in the Gridded Component,
such as the Initialize, Run, and Finalize subroutines.  The subroutine
{\tt ESMF\_GridCompSetEntryPoint} should be called for
each of the required subroutines.  

The names of the Initialize, Run, and Finalize subroutines do not
need to be public, and in fact are encourage to be private to lower
the chances of public symbol clashes between different components.

The {\tt SetServices} routine can also register a private data block
by calling the {\tt ESMF\_GridCompSetInternalState} subroutine.

Note that a Component does not call its own {\tt SetServices} routine;
the higher level code which is creating this Component will first call
{\tt ESMF\_GridCompCreate} to create an "empty" component, and then call
the Component specific {\tt SetServices} routine to attach the code
to the Component.  After {\tt SetServices} has been called, the Framework
now will be able to call the proper Initialize, Run, and Finalize routines
when required.  


