% $Id: xform_desc.tex,v 1.3 2003/02/03 17:08:27 nscollins Exp $
%
% Earth System Modeling Framework
% Copyright 2002-2003, University Corporation for Atmospheric Research, 
% Massachusetts Institute of Technology, Geophysical Fluid Dynamics 
% Laboratory, University of Michigan, National Centers for Environmental 
% Prediction, Los Alamos National Laboratory, Argonne National Laboratory, 
% NASA Goddard Space Flight Center.
% Licensed under the GPL.

%\subsection{Description}

The Transform class is designed to encapsulate a   
(single or series of)
data transformations needed to couple
data between Components when the Components are 
running concurrently.

In a sequential model Components return an Export State
which contains the data needed to couple with other
Components.  Data transformations such as regridding, interpolation,
unit conversion, accumulation, and averaging are called directly
from user code, usually in a user-written Coupler Component which
has been customized for the particular combination of Components
in the current Application.

However, in a concurrent model the Components all run concurrently 
and Components do not return to the calling code each time data 
exchange is required.  
Therefore the Component requires a call-back function in
order to perform the data transformations for coupling Components
in an independent manner.

The Transform class provides methods which allow a series 
of transformations to be stored and then 
applied to Export States in order to create new Import States, and
to store the address of a call-back function to execute the transforms.
After being initialized, a Transform object is passed into the Component
to provide this call-back function which performs the transformations.
This method allows the Component code to be independent of
other Components it is being coupled with.


