% $Id$
%
% Earth System Modeling Framework
% Copyright 2002-2014, University Corporation for Atmospheric Research, 
% Massachusetts Institute of Technology, Geophysical Fluid Dynamics 
% Laboratory, University of Michigan, National Centers for Environmental 
% Prediction, Los Alamos National Laboratory, Argonne National Laboratory, 
% NASA Goddard Space Flight Center.
% Licensed under the University of Illinois-NCSA License.

%TODO: This file started as an exact copy of the Fortran version of this file.
%TODO: It was changed to the C version by ESMF->ESMC subsitution.
%TODO: We should check into using latex command defintion to generate this
%TODO: sort of language specific document from a single generic document.

%\subsection{Description}
\label{sec:CplComp}

In a large, multi-component application such as a weather 
forecasting or climate prediction system running within ESMF, 
physical domains and major system functions are represented 
as Gridded Components 
(see Section \ref{sec:GridComp}).  A Coupler Component, or 
{\tt ESMC\_CplComp}, arranges and executes the data 
transformations between the Gridded Components.  Ideally, 
Coupler Components should contain all the information 
about inter-component communication for an application.
This enables the Gridded Components in the application to be 
used in multiple contexts; that is, used in different coupled 
configurations without changes to their source code. 
For example, the same atmosphere might in one case be coupled 
to an ocean in a hurricane prediction model, and to a 
data assimilation system for numerical weather prediction in
another.  A single Coupler Component can couple 
two or more Gridded Components.

Like Gridded Components, Coupler Components have two parts, one
that is provided by the user and another that is part of the 
framework.  The user-written portion of the software is the coupling
code necessary for a particular exchange between Gridded Components.  
This portion of the Coupler Component code must be divided into 
separately callable initialize, run, and finalize methods.  The 
interfaces for these methods are prescribed by ESMF.

The term ``user-written'' is somewhat misleading here, since within 
a Coupler Component the user can leverage ESMF infrastructure 
software for regridding, redistribution, lower-level communications, 
calendar management, and other functions.  However, ESMF is unlikely 
to offer all the software necessary to customize a data transfer
between Gridded Components.  For instance, ESMF does not currently 
offer tools for unit tranformations or time averaging operations, 
so users must manage those operations themselves.

The second part of a Coupler Component is the {\tt ESMC\_CplComp}
derived type within ESMF.  The user must create one of these types
to represent a specific coupling function, such as the regular
transfer of data between a data assimilation system and an 
atmospheric model.  \footnote{It is not necessary to create 
a Coupler Component for each individual data {\it transfer.}}

The user-written part of a Coupler Component is associated with an
{\tt ESMC\_CplComp} derived type through a routine called 
{\tt ESMC\_SetServices()}.
This is a routine that the user must write and declare public.
Inside the {\tt ESMC\_SetServices()} routine the user must call 
{\tt ESMC\_SetEntryPoint()} methods that associate a standard ESMF 
operation with the name of the corresponding Fortran subroutine in 
their user code.  For example, a user routine called ``couplerInit''
might be associated with the standard initialize routine in a 
Coupler Component.
