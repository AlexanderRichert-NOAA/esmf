% $Id: GridComp_desc.tex,v 1.4 2004/04/20 19:20:05 cdeluca Exp $
%
% Earth System Modeling Framework
% Copyright 2002-2003, University Corporation for Atmospheric Research, 
% Massachusetts Institute of Technology, Geophysical Fluid Dynamics 
% Laboratory, University of Michigan, National Centers for Environmental 
% Prediction, Los Alamos National Laboratory, Argonne National Laboratory, 
% NASA Goddard Space Flight Center.
% Licensed under the GPL.

%\subsection{Description}

In Earth system modeling, the most natural way to think about an ESMF 
Gridded Component, or {\tt ESMF_GridComp}, is as a piece of code 
representing a particular physical domain; for example, an atmospheric 
model or an ocean model.  
In many large modeling systems, each components is developed by its
own group of domain experts.  The ESMF Gridded Component construct 
provides domain experts with a structured, consistent set of component 
interfaces so that it is straightforward, at least technically, to 
combine software from a number of groups, representing different physical 
domains, to form a complex application.  

Earth system software components tend to share a number of basic 
features.  Most contain a variety of physical fields; refer to 
a (possibly noncontiguous) spatial region and a grid that is 
partitioned across a set of computational resources; and require 
a clock, usually for stepping a governing set of PDEs forward in time.  
Most can also be divided into distinct initialize, run, and finalize 
computational phases.  These common characteristics are used 
within ESMF to define a Gridded Component data structure that 
is both tailored for Earth system modeling and yet is still flexible
enough to represent a variety of domains.

More broadly, an ESMF Gridded Component can be based on any 
software with a computational function that is associated with 
a grid.  This might be a convection or radiation scheme, a 
dynamical core, or a data assimilation system.  ESMF allows you
to nest Gridded Components, so that the physics and dynamics within 
an atmospheric model can be considered Gridded Components, along
with the atmospheric model itself.

A well-designed Gridded Component does not store information 
internally about how it couples to other Gridded Components.  That
allows it to be used in different contexts without changes to source
code.\footnote{The idea here is to avoid situations in which slightly
different versions of the same model source are maintained for use in 
different contexts - standalone vs. coupled versions, for example.}
Data is passed between Gridded Components using an intermediary 
Coupler Component, described in Section \ref{sec:CplComp}.

An ESMF Gridded Component has two parts, one which is user-written
and another which is part of the framework.  The user-written
part is software representing a physical domain or performing some
other computational function.  It forms the body of the Gridded 
Component.  It may be a piece of legacy code, or it may be developed 
expressly for use with the ESMF.  It must contain routines with
standard ESMF interfaces that can be called to initialize, run, and
finalize the Gridded Component.  These routines can have separate 
callable phases, such as distinct first and second initialization steps.

The part provided by ESMF is the Gridded Component derived type 
itself, {\tt ESMF\_GridComp}.  An {\tt ESMF\_GridComp} must be created 
for every portion of the application that will be represented 
as a separate component; for example, in a climate model, there may 
be Gridded Components representing the land, ocean, sea ice, and 
atmosphere.  If the application contains an ensemble of identical 
Gridded Components, every one has its own associated {\tt ESMF\_GridComp}.
Each Gridded Component has its own name and is allocated
a set of computational resources, in the form of an ESMF Virtual
Machine, or VM.

The user-written part of a Gridded Component is associated with the
{\tt ESMF\_GridComp} derived type through a routine called SetServices.
This is a routine that the user must write, and declare public.
Inside the SetServices routine the user makes calls to 
{\tt ESMF\_SetEntryPoint} methods that associate a standard ESMF 
operation with the name of the corresponding Fortran subroutine in their user code.  

For example, a user-written initialization routine called {\tt popOceanInit} 
might be associated with the standard initialize routine of an ESMF 
Gridded Component named ``POP'' that represents an ocean model.

\subsubsection{Gridded Components and States}







