% $Id$
%
% Earth System Modeling Framework
% Copyright 2002-2013, University Corporation for Atmospheric Research, 
% Massachusetts Institute of Technology, Geophysical Fluid Dynamics 
% Laboratory, University of Michigan, National Centers for Environmental 
% Prediction, Los Alamos National Laboratory, Argonne National Laboratory, 
% NASA Goddard Space Flight Center.
% Licensed under the University of Illinois-NCSA License.

%TODO: This file started as an exact copy of the Fortran version of this file.
%TODO: It was changed to the C version by ESMF->ESMC subsitution.
%TODO: We should check into using latex command defintion to generate this
%TODO: sort of language specific document from a single generic document.

%\subsection{Description}
\label{sec:GridComp}

In Earth system modeling, the most natural way to think about an ESMF 
Gridded Component, or {\tt ESMC\_GridComp}, is as a piece of code 
representing a particular physical domain, such as an atmospheric 
model or an ocean model.  Gridded Components may also represent individual
processes, such as radiation or chemistry.  It's up to the application
writer to decide how deeply to ``componentize.''

Earth system software components tend to share a number of basic 
features.  Most ingest and produce a variety of physical fields, refer to 
a (possibly noncontiguous) spatial region and a grid that is 
partitioned across a set of computational resources, and require 
a clock for things like stepping a governing set of PDEs forward in time.  
Most can also be divided into distinct initialize, run, and finalize 
computational phases.  These common characteristics are used 
within ESMF to define a Gridded Component data structure that 
is tailored for Earth system modeling and yet is still flexible
enough to represent a variety of domains.

A well designed Gridded Component does not store information 
internally about how it couples to other Gridded Components.  That
allows it to be used in different contexts without changes to source
code.  The idea here is to avoid situations in which slightly
different versions of the same model source are maintained for use in 
different contexts - standalone vs. coupled versions, for example.
Data is passed in and out of Gridded Components using an ESMF State,
this is described in Section~\ref{sec:State}.

An ESMF Gridded Component has two parts, one which is user-written
and another which is part of the framework.  The user-written
part is software that represents a physical domain or performs some
other computational function.  It forms the body of the Gridded 
Component.  It may be a piece of legacy code, or it may be developed 
expressly for use with ESMF.  It must contain routines with
standard ESMF interfaces that can be called to initialize, run, and
finalize the Gridded Component.  These routines can have separate 
callable phases, such as distinct first and second initialization steps.

ESMF provides the Gridded Component derived type, 
{\tt ESMC\_GridComp}.  An {\tt ESMC\_GridComp} must be created 
for every portion of the application that will be represented 
as a separate component.  For example, in a climate model, there may 
be Gridded Components representing the land, ocean, sea ice, and 
atmosphere.  If the application contains an ensemble of identical 
Gridded Components, every one has its own associated {\tt ESMC\_GridComp}.
Each Gridded Component has its own name and is allocated
a set of computational resources, in the form of an ESMF Virtual
Machine, or {\tt VM}.

The user-written part of a Gridded Component is associated with an
{\tt ESMC\_GridComp} derived type through a routine called 
{\tt ESMC\_SetServices()}.
This is a routine that the user must write, and declare public.
Inside the SetServices routine the user must call  
{\tt ESMC\_SetEntryPoint()} methods that associate a standard ESMF 
operation with the name of the corresponding Fortran subroutine 
in their user code.
