% $Id$


%% I will make use of the following ESMF Design Elements, which I assume come
%% from the General Design Document:
%% ----------
%% Text for these is in the Terms Section--
%% Gridded Components (Dsgn\_GENn.m)
%% Couplers (Dsgn\_GENn.m)
%% ---------
%% Text for these needs to be written--
%% Integrated Applications (Dsgn\_GENn.m)
%% Object-Oriented Approach (Dsgn\_GENn.m)
%% Superstructure (Dsgn\_GENn.m)
%% Infrastructure (Dsgn\_GENn.m)



An important function of ESMF is to support the integration of
complex computational modules developed by earth scientists into
larger models and data assimilation systems, and to facilitate
the interoperability of these modules. 
In the ESMF design, these integration functions 
are achieved by combining Couplers
and Gridded Components into Integrated Applications.

Gridded Components will contain the 
computational modules and will thus be the main arena for interaction
between computational scientists and the Framework.  Couplers will
provide all necessary functionality required to integrate Gridded Components
into a larger application (e.g., ocean and atmosphere components into
a climate model).

Following the ESMF general design, Couplers and Gridded Components
are to be implemented using an object-oriented approach  
in which the data and actions of these components are encapsulated in 
ESMF specified ways.

\subsection{Location}

Couplers and Gridded Components, being key elements in the
Framework's integration functions fall in the Superstructure.
They will be invoked in ESMF specified ways by Integrated Applications 
and will, in turn, rely on other Superstructure and Infrastructure
elements in their internal design.  Coupling between computational
modules, for example, often requires interpolation of data and
its reorganization in memory. To perform these tasks, couplers will
rely on Superstructure Regridding methods, which in turn rely on the 
Infrastructure's uniform  implementation of various machine memory
organizations and user-chosen data decompositions.




\subsection{Scope}



Requirements on Gridded Components represent a compromise between
the desire to facilitate their interoperability and the need for
computational modules to have essentially no restriction on their
physical content or computational approach.
A fully interoperable framework would
have to prescribe the actual physical quantities to be exchanges
by components, as well as placing strong inclusive and exclusive 
restrictions on the physics that they represent. The General Requirements
 preclude ESMF from acheiving this  ``strong'' interoperability.
ESMF is intended for  diverse
applications and must support  the rapidly  evolving 
models and data handling techniques.
The scope of requirements on components will thus be limited to the
way in which data is exchange between them and some restriction
on how their internal data must be structured to access Framework
services and to allow its instanziation by Integrated Applications.
They will {\em not} include requirements on what physical quantities
components must provide or accept, or what calculations they
must perform.

















