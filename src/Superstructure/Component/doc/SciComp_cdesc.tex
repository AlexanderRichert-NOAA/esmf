% $Id: SciComp_cdesc.tex,v 1.1 2012/10/02 02:25:34 ksaint Exp $
%
% Earth System Modeling Framework
% Copyright 2002-2012, University Corporation for Atmospheric Research, 
% Massachusetts Institute of Technology, Geophysical Fluid Dynamics 
% Laboratory, University of Michigan, National Centers for Environmental 
% Prediction, Los Alamos National Laboratory, Argonne National Laboratory, 
% NASA Goddard Space Flight Center.
% Licensed under the University of Illinois-NCSA License.

%TODO: This file started as an exact copy of the Fortran version of this file.
%TODO: It was changed to the C version by ESMF->ESMC subsitution.
%TODO: We should check into using latex command defintion to generate this
%TODO: sort of language specific document from a single generic document.

%\subsection{Description}
\label{sec:SciComp}

In Earth system modeling, a particular piece of code representing a physical 
domain, such as an atmospheric model or an ocean model, is typically 
implemented as an ESMF Gridded Component, or {\tt ESMC\_GridComp}.  
However, there are times when physical domains, or realms, need to be 
represented, but aren't actual pieces of code, or software.  These domains 
cam be implmentd as ESMF Science Components, or {\\tt ESMC_SciComp}.

Unlike Gridded and Coupler Components, Science Components are not associated 
with software, so they don't require the overhead of handling execution 
routines, such as initialize, run and finalize.  Instead, a Science Component 
is a simple container object, that contains basic information, such as name,
as well as lists of other objects, such as Attributes.  In addition, Science
Components can contain other Components, so they can be nested within a
Component hierarchy.

