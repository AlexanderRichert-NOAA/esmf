% $Id$
%
% Earth System Modeling Framework
% Copyright 2002-2018, University Corporation for Atmospheric Research, 
% Massachusetts Institute of Technology, Geophysical Fluid Dynamics 
% Laboratory, University of Michigan, National Centers for Environmental 
% Prediction, Los Alamos National Laboratory, Argonne National Laboratory, 
% NASA Goddard Space Flight Center.
% Licensed under the University of Illinois-NCSA License.

%\subsection{Use and Examples}

ESMF tracing is disabled by default. To enable tracing, set the
{\tt ESMF\_RUNTIME\_TRACE} environment variable to {\tt ON}. You
do not need to recompile your code to enable tracing.

\begin{verbatim}
# csh shell
$ setenv ESMF_RUNTIME_TRACE ON

# bash shell
$ export ESMF_RUNTIME_TRACE=ON
\end{verbatim}

When enabled, the default behavior is to trace all PETs of the
ESMF application. Although the ESMF tracer is designed to write 
events in a compact form, tracing can produce an extremely
large number of events depending on the total number of PETs and
the length of the run. To reduce output, it is possible to restrict
the PETs that produce trace output by setting the {\tt ESMF\_RUNTIME\_TRACE\_PETLIST}
environment variable. For example, this setting:

\begin{verbatim}
$ setenv ESMF_RUNTIME_TRACE_PETLIST "0 101 192-196"
\end{verbatim}

will instruct the tracer to only trace PETs 0, 101, and 192 through 196
(inclusive). The syntax of this environment variable is to list
PET numbers separated by spaces. PET ranges are also supported using
the ``X-Y'' syntax where X < Y. For PET counts greater than 100, it is
recommended to set this environment variable. The one exception is that
PET 0 is always traced, regardless of the {\tt ESMF\_RUNTIME\_TRACE\_PETLIST}
setting.

When tracing is enabled, {\tt phase\_enter} and {\tt phase\_exit} events will
automatically be recorded for all initialize, run, and finalize phases of all
Components in the application. To trace {\em only} user-instrumented regions (via
the {\tt ESMF\_TraceRegionEnter()} and {\tt ESMF\_TraceRegionExit()} calls),
Component-level tracing can be turned off by setting:

\begin{verbatim}
$ setenv ESMF_RUNTIME_TRACE_COMPONENT OFF
\end{verbatim}

After running an ESMF application with tracing enabled, a directory
called {\em traceout} will be created in the run directory and it will
contain a {\em metadata} file and an event stream file {\em esmf\_stream\_XXXX}
for each PET with tracing enabled. Together these files form a valid
CTF trace which may be analyzed with any of the tools listed above.

Trace events are flushed to file at a regular interval. If the application
crashes, some of the most recent events may not be flushed to file. To
maximize the number of events appearing in the trace, an option is available
to flush events to file more frequently. Because this option may have
negative performance implications due to increased file I/O, it is not
recommended unless needed. To turn on eager flushing use:

\begin{verbatim}
$ setenv ESMF_RUNTIME_TRACE_FLUSH EAGER
\end{verbatim}

\subsubsection{Clocks}
\label{sec:TracingClocks}

There are three options for the kind of clock to use to timestamp
events. These options are controlled by setting the environment variable
{\tt ESMF\_RUNTIME\_TRACE\_CLOCK}.
\begin{itemize}
\item [{\tt REALTIME}] The {\tt REALTIME} clock timestamps events with the current time on
      the system.  This is the default clock if the above environment
      variable is not set.  This setting can be useful when tracing PETs that
      span multiple physical computing nodes assuming that the system clocks
      on each node are adequately synchronized.  On most HPC systems, system
      clocks are periodically updated to stay in sync.  A disadvantage of this
      clock is that periodic adjustments mean the clock is not monotonically
      increasing so some timings may be inaccurate if the system clock jumps
      forward or backward significantly. Testing has shown that this is not
      typically an issue on most systems.
\item [{\tt MONOTONIC}] The {\tt MONOTONIC} clock is guaranteed to be monotonically increasing
      and does not suffer from periodic adjustments.  The timestamps represent
      an amount of time since some arbitrary point in the past.  There is no
      guarantee that these timestamps will be synchronized across physical
      computing nodes, so this option should only be used for tracing a set of PETs
      running on a single physical machine.
\item [{\tt MONOTONIC\_SYNC}] The {\tt MONOTONIC\_SYNC} clock is similar to the {\tt MONOTONIC} clock
      in that it is guaranteed to be monotonically increasing. In addition, at
      application startup, all PET clocks are synchronized to a common time
      by determining a PET-local offset to be applied to timestamps. Therefore this option
      can be used to compare trace streams across physical nodes.     
\end{itemize}
