% $Id: State_rest.tex,v 1.9 2008/04/05 03:39:16 cdeluca Exp $
%
% Earth System Modeling Framework
% Copyright 2002-2008, University Corporation for Atmospheric Research, 
% Massachusetts Institute of Technology, Geophysical Fluid Dynamics 
% Laboratory, University of Michigan, National Centers for Environmental 
% Prediction, Los Alamos National Laboratory, Argonne National Laboratory, 
% NASA Goddard Space Flight Center.
% Licensed under the University of Illinois-NCSA License.

%\subsubsection{Restrictions and Future Work}

\begin{enumerate}
\item{\bf Flags not fully implemented.}
The flags for indicating various qualities associated with 
data items in a State - validity, whether or not the item is
required for restart, read/write status - are not fully implemented.
Although their defaults can be set, the associated methods for 
setting and getting these flags have not been implemented.
(The {\tt needed} flag is fully supported.)

\item{\bf No synchronization at object create time.}
Object IDs are using during the reconcile process to identify objects
which are unknown to some subset of the PETs in the currently running VM.
Object IDs are assigned in sequential order at object create time.
User input at design time requested there be no communication overhead
during the create of an object, so there is no opportunity to
synchronize IDs if one or more PETs create objects which
are not in unison (not all PETs in the VM make the same calls).

Even if the user follows the unison rules, if components are running on 
a subset of the PETs, when they return to the parent (calling) component
the next available ID will potentially not be the same across all
PETs in the VM.  Part of the reconcile process or part of the return
to the parent will need to have a broadcast which sends the current
ID number, and all PETs can reset the next available number to the highest
number broadcast.  This could be an async call to avoid as much as
possible serialization and barrier issues.

Default object names are based on the object id (e.g. "Field1", "Field2")
to create unique object names, so basing the detection of unique objects 
on the name instead of on the object id is no better solution.

\end{enumerate}



