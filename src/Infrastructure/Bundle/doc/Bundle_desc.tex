% $Id: Bundle_desc.tex,v 1.3 2004/06/09 23:26:44 cdeluca Exp $

The Bundle class represents ``bundles'' of Fields that are 
discretized on the same Grid and distributed in the same manner.  
Fields within a Bundle may be located at different locations relative 
to the vertices of their common Grid.  Fields within a Bundle may
be of different dimensions, as long as the Grid dimensions that 
are distributed are the same.  For example, a surface Field on 
a distributed lat/lon Grid and a 3D Field with an added vertical 
dimension on the same distributed lat/lon Grid can be placed together
in a Bundle.
 
Bundles currently function mainly as convenient containers for storing 
Fields.  Bundles can be created and destroyed, can have attributes 
added or retrieved, and can have Fields added, removed, or retrieved.  
The Fortran data pointer of a Field within a Bundle can be obtained 
by passing the Bundle a Field name.  Memory layout information is 
stored in a BundleDataMap object which is attached to the Bundle.  It 
can be accessed by querying the Bundle.

Bundles are one of the data objects that can be added to States,
which are used for sending to or receiving data from another 
component.

In the future Bundles will serve as a mechanism for performance
optimization.  ESMF will take advantage of the similarities of the
Fields within a Bundle in order to implement collective communication,
IO, and regridding.  See Section \ref{sec:bundlerest} for a 
description of features that are being planned.





