% $Id: Perf_design.tex,v 1.1 2002/11/20 15:24:05 ekluz Exp $
%
% Earth System Modeling Framework
% Copyright 2002-2003, University Corporation for Atmospheric Research, 
% Massachusetts Institute of Technology, Geophysical Fluid Dynamics 
% Laboratory, University of Michigan, National Centers for Environmental 
% Prediction, Los Alamos National Laboratory, Argonne National Laboratory, 
% NASA Goddard Space Flight Center.
% Licensed under the GPL.

%\subsection{Design}

The Perf class has four high level interfaces accessable by the user:
PerfInit, PerfStart, PerfEnd, and PerfPrint. PerfInit does the needed
initialization, and sets the monitoring, and doHW flags. If the monitoring
flag is set to false by the user performance monitoring will be turned
off. If the doHW flag is set, it's value will be used as the default value
to determine if hardware performance monitoring is done. Inidividual code
segments that are timed can also turn hardware monitoring on or off for
a particular code segment. PerfStart begins timing a code segment, PerfEnd
ends timing a code segment. PerfPrint prints a report on the timing results
for all code segments timed.

More information on PerfInit. PerfInit sets MaxThreads the Maximum number of 
threads that will be used, and allocates an array of Segments that is two
dimensional in code segment and thread.

More information on PerfStart. When called PerfStart figures out if the given
code segment has been activated before or not. If it hasn't it increments
the codeSegment counter and begins timing and Hardware monitoring of this
code segment. If the code segment was active before, if it's still active
it triggers an error, since this means a PerfStart has been called more than
once, without a PerfEnd imbetween. If it's not active it activates timing
for this code segment.

More information on PerfEnd. PerfEnd deactivates a code segment, and calls
the SegmentEnd method to accumulate the relevent counters.

More information on PerfPrint. PerfPrint figures out the maximum, minimum, and
average times across both threads and distributed PE's. The results are then printed
to the Log file.
