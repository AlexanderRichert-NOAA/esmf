% $Id: HWMonitor_design.tex,v 1.2 2002/11/18 20:06:52 ekluz Exp $
%
% Earth System Modeling Framework
% Copyright 2002-2003, University Corporation for Atmospheric Research, 
% Massachusetts Institute of Technology, Geophysical Fluid Dynamics 
% Laboratory, University of Michigan, National Centers for Environmental 
% Prediction, Los Alamos National Laboratory, Argonne National Laboratory, 
% NASA Goddard Space Flight Center.
% Licensed under the GPL.

%\subsection{Design}

The HWMonitor class uses Hardware monitoring libraries (PCL or PAPI)
to do performance profiling of code segments. Floating point operations
per second, data cache utilization and miss rates, floating point unit
utilization, and cycle counts are the desired statistics to be reported.
The HWMonitor class is meant to be created by the CodeSeg
class. There are three important methods for HWMonitor: Init, BeginMonitoring,
EndMonitoring, and Report. The Init method does the initialization needed
to figure out which hardware monitoring counters need to be used and verify
that they are available. It also prepares arrays for the counters that
will be activated. BeginMonitoring turns performance monitoring on, for
a given code segment. EndMonitoring turns performance monitoring off, for
a given code segment. The Report method then gives a report on the statistics
for the code segment timed. The Init and Report methods should not be
called in threaded sections and should return a error if they are. The BeginMonitoring
and EndMonitoring methods can be called from within threaded sections. However,
they can not be called in code sections that are partially threaded and
partially unthreaded. If these methods determine that the user has put calls
partially threaded and partially unthreaded sections -- then a warning message
will be sent to the LogErr file.

More information on the Init method. The init method figures out the PCL or
PAPI integer event types and fills the counter array with the event type that
corresponds to the hw\_option\_names numerated type. Since, many of the events
that will be reported are actually rates, there may be two event-types that
need to be recorded for a single item. After the counter array is filled PCL\_init
is called when using PCL, or PAPI\_start\_counters is called when using PAPI.
If one of the desired events is unavailable on a given machine, an error
code will be returned and the function will abort. Technically, we could
add events individually, and keep track of which events are available. Since
we are tracking very common events we expect that if hardware counters are
available on a machine -- the counters we are tracking will all work or all fail.
When using PCL, the PCLinit call needs to be done for all possible threads.
So in this case PCLinit will be called for each thread. Note, that if later
the number of threads increases beyond the value used in initialization -- this
will be a problem and the code will need to abort. Also when using PCL memory
for the desc pointer will need to be allocated. For all performance libraries the 
Init method also initializes the accum array to zero.

More information on the BeginMonitoring method. BeginMonitoring stores the initial
counter values in the initial array. Later when EndMonitoring is called the initial
array will be used to figure out the difference in the counters over the code 
section being timed. When using PCL, PCLread will be used and when using PAPI, 
PAPI\_read\_counters is used to store the initial counter values into the initial array 
for each event type in the counter array.

More information on the EndMonitoring method. When EndMonitoring is called
the counters are again checked to find the results at the end of the code
section. The difference between the counters and the initial array is then
computed and this difference is added to the accum array. When using PCL,
PCL\_read is used to get the counter results, and when using PAPI, PAPI\_read\_counters
is used to get the counter results.

More information on the Report method. The report method sends the results
of the hardware timing to the Log file using the Log methods. The report method
will take the values in the accum array to figure out the underlying rates that
will be reported.

