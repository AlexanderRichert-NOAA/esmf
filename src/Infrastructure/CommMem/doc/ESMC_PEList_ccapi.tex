%                **** IMPORTANT NOTICE *****
% This LaTeX file has been automatically produced by ProTeX v. 1.1
% Any changes made to this file will likely be lost next time
% this file is regenerated from its source. Send questions 
% to Arlindo da Silva, dasilva@gsfc.nasa.gov
 
\parskip        0pt
\parindent      0pt
\baselineskip  11pt
 
%--------------------- SHORT-HAND MACROS ----------------------
\def\bv{\begin{verbatim}}
\def\ev{\end{verbatim}}
\def\be{\begin{equation}}
\def\ee{\end{equation}}
\def\bea{\begin{eqnarray}}
\def\eea{\end{eqnarray}}
\def\bi{\begin{itemize}}
\def\ei{\end{itemize}}
\def\bn{\begin{enumerate}}
\def\en{\end{enumerate}}
\def\bd{\begin{description}}
\def\ed{\end{description}}
\def\({\left (}
\def\){\right )}
\def\[{\left [}
\def\]{\right ]}
\def\<{\left  \langle}
\def\>{\right \rangle}
\def\cI{{\cal I}}
\def\diag{\mathop{\rm diag}}
\def\tr{\mathop{\rm tr}}
%-------------------------------------------------------------

\markboth{Left}{Source File: ESMC\_PEList.C,  Date: Mon Dec  9 19:56:32 MST 2002
}

 
%/////////////////////////////////////////////////////////////
\subsubsection{ESMC\_PEListCreate - Create a new PEList}


  
\bigskip{\sf INTERFACE:}
\begin{verbatim}       ESMC_PEList *ESMC_PEListCreate(\end{verbatim}{\em RETURN VALUE:}
\begin{verbatim}       pointer to newly allocated ESMC_PEList\end{verbatim}{\em ARGUMENTS:}
\begin{verbatim}       int numpes,          // in - number of PEs in list
       int *rc) {           // out - return code\end{verbatim}
{\sf DESCRIPTION:\\ }


        Create a new PEList from ... Allocates memory for a new PEList
        object and uses the internal routine ESMC\_PEListContruct to
        initialize it.  Define for deep classes only, for shallow classes only
        define and use ESMC\_PEListInit.
        There can be multiple overloaded methods with the same name, but
        different argument lists.
   
%/////////////////////////////////////////////////////////////
 
\mbox{}\hrulefill\ 
 
\subsubsection{ESMC\_PEListCreate - Create a new PEList}


  
\bigskip{\sf INTERFACE:}
\begin{verbatim}       ESMC_PEList *ESMC_PEListCreate(\end{verbatim}{\em RETURN VALUE:}
\begin{verbatim}       pointer to newly allocated ESMC_PEList\end{verbatim}{\em ARGUMENTS:}
\begin{verbatim}       int firstpe,          // in - first PE in list
       int lastpe,           // in - last PE in list
       int *rc) {            // out - return code\end{verbatim}
{\sf DESCRIPTION:\\ }


        Create a new PEList from ... Allocates memory for a new PEList
        object and uses the internal routine ESMC\_PEListContruct to
        initialize it.  Define for deep classes only, for shallow classes only
        define and use ESMC\_PEListInit.
        There can be multiple overloaded methods with the same name, but
        different argument lists.
   
%/////////////////////////////////////////////////////////////
 
\mbox{}\hrulefill\ 
 
\subsubsection{ESMC\_PEListDestroy - free a PEList created with Create}


  
\bigskip{\sf INTERFACE:}
\begin{verbatim}       int ESMC_PEListDestroy(\end{verbatim}{\em RETURN VALUE:}
\begin{verbatim}      int error return code\end{verbatim}{\em ARGUMENTS:}
\begin{verbatim}       ESMC_PEList *pelist) {    // PE list to destroy\end{verbatim}
{\sf DESCRIPTION:\\ }


        ESMF routine which destroys a PEList object previously allocated
        via an ESMC\_PEListCreate routine.  Define for deep classes only.
   
%/////////////////////////////////////////////////////////////
 
\mbox{}\hrulefill\ 
 
\subsubsection{ESMC\_PEListConstruct - fill in an already allocated PEList}


  
\bigskip{\sf INTERFACE:}
\begin{verbatim}       int ESMC_PEList::ESMC_PEListConstruct(\end{verbatim}{\em RETURN VALUE:}
\begin{verbatim}      int error return code\end{verbatim}{\em ARGUMENTS:}
\begin{verbatim}       int numpes) {          // in - number of PEs in list\end{verbatim}
{\sf DESCRIPTION:\\ }


        ESMF routine which fills in the contents of an already
        allocated PEList object.  May need to do additional allocations
        as needed.  Must call the corresponding ESMC\_PEListDestruct
        routine to free the additional memory.  Intended for internal
        ESMF use only; end-users use ESMC\_PEListCreate, which calls
        ESMC\_PEListConstruct.  Define for deep classes only.
   
%/////////////////////////////////////////////////////////////
 
\mbox{}\hrulefill\ 
 
\subsubsection{ESMC\_PEListDestruct - release resources associated w/a PEList}


  
\bigskip{\sf INTERFACE:}
\begin{verbatim}       int ESMC_PEList::ESMC_PEListDestruct(void) {\end{verbatim}{\em RETURN VALUE:}
\begin{verbatim}      int error return code\end{verbatim}{\em ARGUMENTS:}
\begin{verbatim}      none\end{verbatim}
{\sf DESCRIPTION:\\ }


        ESMF routine which deallocates any space allocated by
        ESMC\_PEListConstruct, does any additional cleanup before the
        original PEList object is freed.  Intended for internal ESMF
        use only; end-users use ESMC\_PEListDestroy, which calls
        ESMC\_PEListDestruct.  Define for deep classes only.
   
%/////////////////////////////////////////////////////////////
 
\mbox{}\hrulefill\ 
 
\subsubsection{ESMC\_PEListInit - initializes a PEList element}


  
\bigskip{\sf INTERFACE:}
\begin{verbatim}       int ESMC_PEList::ESMC_PEListInit(\end{verbatim}{\em RETURN VALUE:}
\begin{verbatim}      int error return code\end{verbatim}{\em ARGUMENTS:}
\begin{verbatim}       int i,                // in - ith element
       int esmfid,           // in - ESMF ID
       int cpuid,            // in - machine cpu ID
       int nodeid) {         // in - machine node ID\end{verbatim}
{\sf DESCRIPTION:\\ }


        ESMF routine which only initializes PE values; it does not
        allocate any resources.
   
%/////////////////////////////////////////////////////////////
 
\mbox{}\hrulefill\ 
 
\subsubsection{ESMC\_PEListGetConfig - get configuration info from a PEList}


  
\bigskip{\sf INTERFACE:}
\begin{verbatim}       int ESMC_PEList::ESMC_PEListGetConfig(\end{verbatim}{\em RETURN VALUE:}
\begin{verbatim}      int error return code\end{verbatim}{\em ARGUMENTS:}
\begin{verbatim}       ESMC_PEListConfig *config) const {  // out - resources\end{verbatim}
{\sf DESCRIPTION:\\ }


      Returns the set of resources the PEList object was configured with.
   
%/////////////////////////////////////////////////////////////
 
\mbox{}\hrulefill\ 
 
\subsubsection{ESMC\_PEListSetConfig - set configuration info for a PEList}


  
\bigskip{\sf INTERFACE:}
\begin{verbatim}       int ESMC_PEList::ESMC_PEListSetConfig(\end{verbatim}{\em RETURN VALUE:}
\begin{verbatim}      int error return code\end{verbatim}{\em ARGUMENTS:}
\begin{verbatim}       const ESMC_PEListConfig *config) {     // in - resources\end{verbatim}
{\sf DESCRIPTION:\\ }


      Configures the PEList object with set of resources given.
   
%/////////////////////////////////////////////////////////////
 
\mbox{}\hrulefill\ 
 
\subsubsection{ESMC\_PEListGet<Value> - get <Value> for a PEList}


  
\bigskip{\sf INTERFACE:}
\begin{verbatim}       int ESMC_PEList::ESMC_PEListGet<Value>(\end{verbatim}{\em RETURN VALUE:}
\begin{verbatim}      int error return code\end{verbatim}{\em ARGUMENTS:}
\begin{verbatim}       <value type> *value) const {     // out - value\end{verbatim}
{\sf DESCRIPTION:\\ }


       Returns the value of PEList member <Value>.
       Can be multiple routines, one per value
   
%/////////////////////////////////////////////////////////////
 
\mbox{}\hrulefill\ 
 
\subsubsection{ESMC\_PEListSet<Value> - set <Value> for a PEList}


  
\bigskip{\sf INTERFACE:}
\begin{verbatim}       int ESMC_PEList::ESMC_PEListSet<Value>(\end{verbatim}{\em RETURN VALUE:}
\begin{verbatim}      int error return code\end{verbatim}{\em ARGUMENTS:}
\begin{verbatim}       <value type> value) {     // in - value\end{verbatim}
{\sf DESCRIPTION:\\ }


       Sets the PEList member <Value> with the given value.
       Can be multiple routines, one per value
   
%/////////////////////////////////////////////////////////////
 
\mbox{}\hrulefill\ 
 
\subsubsection{ESMC\_PEListGetPE - get PE pointer from a PEList}


  
\bigskip{\sf INTERFACE:}
\begin{verbatim}       int ESMC_PEList::ESMC_PEListGetPE(\end{verbatim}{\em RETURN VALUE:}
\begin{verbatim}      int error return code\end{verbatim}{\em ARGUMENTS:}
\begin{verbatim}       int i,                    // in  - ith PE
       ESMC_PE **pe) const {     // out - pointer to ith PE\end{verbatim}
{\sf DESCRIPTION:\\ }


       Returns a pointer to the ith PE in the list.
   
%/////////////////////////////////////////////////////////////
 
\mbox{}\hrulefill\ 
 
\subsubsection{ESMC\_PEListValidate - internal consistency check for a PEList}


  
\bigskip{\sf INTERFACE:}
\begin{verbatim}       int ESMC_PEList::ESMC_PEListValidate(\end{verbatim}{\em RETURN VALUE:}
\begin{verbatim}      int error return code\end{verbatim}{\em ARGUMENTS:}
\begin{verbatim}       void) const {    // in - validate options\end{verbatim}
{\sf DESCRIPTION:\\ }


        Validates that a PEList is internally consistent.
        Returns error code if problems are found.  ESMC\_Base class method.
   
%/////////////////////////////////////////////////////////////
 
\mbox{}\hrulefill\ 
 
\subsubsection{ESMC\_PEListPrint - print contents of a PEList}


  
\bigskip{\sf INTERFACE:}
\begin{verbatim}       int ESMC_PEList::ESMC_PEListPrint(\end{verbatim}{\em RETURN VALUE:}
\begin{verbatim}      int error return code\end{verbatim}{\em ARGUMENTS:}
\begin{verbatim}       void) const {     //  in - print options\end{verbatim}
{\sf DESCRIPTION:\\ }


        Print information about a PEList.  The options control the
        type of information and level of detail.  ESMC\_Base class method.
   
%/////////////////////////////////////////////////////////////
 
\mbox{}\hrulefill\ 
 
\subsubsection{ESMC\_PEList - native C++ constructor}


  
\bigskip{\sf INTERFACE:}
\begin{verbatim}       ESMC_PEList::ESMC_PEList(\end{verbatim}{\em RETURN VALUE:}
\begin{verbatim}      none\end{verbatim}{\em ARGUMENTS:}
\begin{verbatim}       void) {  // in\end{verbatim}
{\sf DESCRIPTION:\\ }


        Calls standard ESMF deep or shallow methods for initialization
        with default or passed-in values
   
%/////////////////////////////////////////////////////////////
 
\mbox{}\hrulefill\ 
 
\subsubsection{~ESMC\_PEList - native C++ destructor}


  
\bigskip{\sf INTERFACE:}
\begin{verbatim}       ESMC_PEList::~ESMC_PEList(void) {\end{verbatim}{\em RETURN VALUE:}
\begin{verbatim}      none\end{verbatim}{\em ARGUMENTS:}
\begin{verbatim}      none\end{verbatim}
{\sf DESCRIPTION:\\ }


        Calls standard ESMF deep or shallow methods for destruction
   
%/////////////////////////////////////////////////////////////
 
\mbox{}\hrulefill\ 
 
\subsubsection{ESMC\_PEListSort - sort contents of a PEList by node affinity}


  
\bigskip{\sf INTERFACE:}
\begin{verbatim}       int ESMC_PEList::ESMC_PEListSort(\end{verbatim}{\em RETURN VALUE:}
\begin{verbatim}      int error return code\end{verbatim}{\em ARGUMENTS:}
\begin{verbatim}       void) {     //  in - none\end{verbatim}
{\sf DESCRIPTION:\\ }


        Sorts a PEList by node affinity; all PE's within same node
        will appear contiguously in the PEList  
   
%/////////////////////////////////////////////////////////////
 
\mbox{}\hrulefill\ 
 
\subsubsection{ESMC\_PEListPECompare - compares 2 PE's by node affinity}


  
\bigskip{\sf INTERFACE:}
\begin{verbatim}       int ESMC_PEListPECompare(\end{verbatim}{\em RETURN VALUE:}
\begin{verbatim}      int comparison result as required by qsort()\end{verbatim}{\em ARGUMENTS:}
\begin{verbatim}       const void *pe1,       //  in - PE 1
       const void *pe2) {     //  in - PE 2\end{verbatim}
{\sf DESCRIPTION:\\ }


        Compares 2 PEs by node affinity; used by qsort in ESMC\_PEListSort()
  
%...............................................................
