%                **** IMPORTANT NOTICE *****
% This LaTeX file has been automatically produced by ProTeX v. 1.1
% Any changes made to this file will likely be lost next time
% this file is regenerated from its source. Send questions 
% to Arlindo da Silva, dasilva@gsfc.nasa.gov
 
\parskip        0pt
\parindent      0pt
\baselineskip  11pt
 
%--------------------- SHORT-HAND MACROS ----------------------
\def\bv{\begin{verbatim}}
\def\ev{\end{verbatim}}
\def\be{\begin{equation}}
\def\ee{\end{equation}}
\def\bea{\begin{eqnarray}}
\def\eea{\end{eqnarray}}
\def\bi{\begin{itemize}}
\def\ei{\end{itemize}}
\def\bn{\begin{enumerate}}
\def\en{\end{enumerate}}
\def\bd{\begin{description}}
\def\ed{\end{description}}
\def\({\left (}
\def\){\right )}
\def\[{\left [}
\def\]{\right ]}
\def\<{\left  \langle}
\def\>{\right \rangle}
\def\cI{{\cal I}}
\def\diag{\mathop{\rm diag}}
\def\tr{\mathop{\rm tr}}
%-------------------------------------------------------------

\markboth{Left}{Source File: ESMC\_PE.h,  Date: Mon Dec  9 19:56:32 MST 2002
}

 
%/////////////////////////////////////////////////////////////
\subsection{C++:  Class Interface ESMC\_PE - one line general statement about this class (Source File: ESMC\_PE.h)}


  
  
   The code in this file defines the C++ PE members and declares method 
   signatures (prototypes).  The companion file ESMC\_PE.C contains
   the definitions (full code bodies) for the PE methods.
  
   
  
  -----------------------------------------------------------------------------
   
\bigskip{\em USES:}
\begin{verbatim}  #include <ESMC_Base.h>  // all classes inherit from the ESMC Base class.
  #include <ESMC_Machine.h>  
  //#include <ESMC_XXX.h>   // other dependent classes (subclasses, aggregates,
                         // composites, associates, friends)
 \end{verbatim}{\sf PUBLIC TYPES:}
\begin{verbatim}   class ESMC_PEConfig;
  class ESMC_PE;
 \end{verbatim}{\sf PRIVATE TYPES:}
\begin{verbatim} 
  // class configuration type
   class ESMC_PEConfig {
     private:
  //   < insert resource items here >
   };
 
  // class declaration type
  class ESMC_PE : public ESMC_Base {    // inherits from ESMC_Base class
 
    private:
  //  < insert class members here >  corresponds to type ESMF_PE members
  //                                 in F90 modules
     int esmfID;    // ESMF assigned ID
     int cpuID;     // Hardware assigned processor ID
     int nodeID;    // hardware assigned associated node ID
     static int peCount;  // number of PEs instantiated
 
     ESMC_Machine *machine;  // interface to specific platform
     // neighbor connections ?  see ESMC_DELayout.h
 \end{verbatim}{\sf PUBLIC MEMBER FUNCTIONS:}
\begin{verbatim}   public:
  // the following method applies to a shallow class
     int ESMC_PEInit(void);
     int ESMC_PEInit(int esmfid, int cpuid, int nodeid);
 
  // optional configuration methods
      int ESMC_PEGetConfig(ESMC_PEConfig *config) const;
      int ESMC_PESetConfig(const ESMC_PEConfig *config);
 
  // accessor methods for class members
     int ESMC_PEGetEsmfID(int *id) const;
     int ESMC_PESetEsmfID(int  id);
 
     int ESMC_PEGetCpuID(int *id) const;
     int ESMC_PESetCpuID(int  id);
 
     int ESMC_PEGetNodeID(int *id) const;
     int ESMC_PESetNodeID(int  id);
     
  // required methods inherited and overridden from the ESMC_Base class
     int ESMC_PEValidate(void) const;
     int ESMC_PEPrint(void) const;
 
  // native C++ constructors/destructors
 	ESMC_PE(void);
 	~ESMC_PE(void);
   
  // < declare the rest of the public interface methods here >
 
     // friend int ESMC_PEListPECompare(const void *pe1, const void *pe2);
     // friend class ESMC_PEList;  // TODO: ?? implement in ESMC_PEList
   \end{verbatim}{\sf PRIVATE MEMBER FUNCTIONS:}
\begin{verbatim}   private: 
  // < declare private interface methods here >\end{verbatim}

%...............................................................
