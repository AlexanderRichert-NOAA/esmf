% $Id: ESMC_DELayout_design.tex,v 1.2 2003/03/10 04:16:22 cdeluca Exp $

The DELayout class organizes a set of DE's and their associated PE's into
1, 2, or 3-dimensional topology according to given x,y,z sizes and a hint
as to the most time-critical communication direction (x, y, or z).

Once a layout is created, a distributed grid, running on a particular DE,
will determine its distributed index space via information from the DELayout.
It will ask the DELayout for the x,y,z extent or size of the layout as a whole,
and, providing its DE ID as input, it will ask for its own i,j,k coordinates
or position within the DELayout.  Combined with its knowledge of the Physical
Grid, the Distributed Grid can then use this information to calculate the
i,j,k sub-space for each DE.
