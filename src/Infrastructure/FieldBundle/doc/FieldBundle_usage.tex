% $Id: FieldBundle_usage.tex,v 1.5 2010/10/08 20:15:49 rokuingh Exp $

%\subsection{Use and Examples}

Examples of creating, destroying and accessing FieldBundles and their
constituent Fields are provided in this section, along with some
notes on FieldBundle methods.

\subsubsection{Create a FieldBundle}

After creating multiple Fields, a FieldBundle
can be created by passing a list of the Fields into the 
method {\tt ESMF\_FieldBundleCreate()}.  The FieldBundle will contain
references to the Fields.  An empty FieldBundle can also be created
and Fields added one at a time or in groups.

\subsubsection{Access FieldBundle data}

To access data in a FieldBundle the user can provide a Field
name and retrieve the Field's Fortran data pointer.  Alternatively,
the user can retrieve the data in the form of an ESMF 
Field and use the Field-level interfaces.

\subsubsection{Destroy a FieldBundle}

The user must call {\tt ESMF\_FieldBundleDestroy()} before 
deleting any of the Fields it contains.  Because Fields
can be shared by multiple FieldBundles and States, they are
not deleted by this call.


