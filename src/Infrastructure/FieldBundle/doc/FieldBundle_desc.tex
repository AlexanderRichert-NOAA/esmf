% $Id: FieldBundle_desc.tex,v 1.2 2008/04/02 20:42:51 cdeluca Exp $

The FieldBundle class represents ``bundles'' of Fields that are 
discretized on the same Grid and distributed in the same manner.  
Fields within a FieldBundle may be located at different locations relative 
to the vertices of their common Grid.  The Fields in a FieldBundle may
be of different dimensions, as long as the Grid dimensions that 
are distributed are the same.  For example, a surface Field on 
a distributed lat/lon Grid and a 3D Field with an added vertical 
dimension on the same distributed lat/lon Grid can be included
in the same FieldBundle.
 
FieldBundles currently function mainly as convenient containers for storing 
Fields.  FieldBundles can be created and destroyed, can have attributes 
added or retrieved, and can have Fields added or retrieved.
Methods include a variety of queries that return information about 
the attributes and the Fields that a FieldBundle contains.  The Fortran 
data pointer of a Field within a FieldBundle can be obtained 
by passing the FieldBundle a Field name.  

Memory layout information is stored in a FieldBundleDataMap object 
which is attached to the FieldBundle.  It can be accessed by querying the 
FieldBundle.  Although we have made the FieldBundleDataMap public, many of 
the memory layout options have not been implemented.

FieldBundles are one of the data objects that can be added to States,
which are used for sending to or receiving data from other components.

In the future FieldBundles will serve as a mechanism for performance
optimization.  ESMF will take advantage of the similarities of the
Fields within a FieldBundle in order to implement collective communication,
IO, and regridding.  See Section \ref{sec:bundlerest} for a 
description of features that are being planned.





