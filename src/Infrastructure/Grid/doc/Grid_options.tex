% $Id: Grid_options.tex,v 1.18 2005/07/01 21:23:25 jwolfe Exp $


 \subsubsection{ESMF\_CoordOrder}

 {\sf DESCRIPTION:\\}
 By default, ESMF assumes coordinates are ordered XYZ, which is also known as
 IJK ordering.  This means that the first coordinate axis (for example, X in
 a cartesian system or latitude in a lat-lon system) is indexed first, the
 second coordinate axis is indexed second, and the third coordinate axis is
 indexed third in all internal arrays.  The Grid class can be set to different
 ordering of coordinates, for example ZXY.  This sets the default coordinate
 ordering for all Fields created from the Grid as well (although there is also
 a way to override this default with a user-specified mapping between rank
 numbers in the Field and Grid during Field creation -- please see the Field
 section for details).  This feature is designed to provide the user with an
 simple mechanism to change the overall ordering of indices.  The
 {ESMF\_CoordOrder} parameter describes ordering options supported by ESMF.

 Valid values are:
 \begin{description}
    \item [ESMF\_COORD\_ORDER\_UNKNOWN]
          Unknown or undefined coordinate ordering.

    \item [ESMF\_COORD\_ORDER\_XYZ]
          The coordinates are ordered XYZ.  For a 2D Grid, this defaults to
          XY mapping.

    \item [ESMF\_COORD\_ORDER\_XZY]
          The coordinates are ordered XZY.  For a 2D Grid, this defaults to
          XY mapping.

    \item [ESMF\_COORD\_ORDER\_YXZ]
          The coordinates are ordered YXZ.  For a 2D Grid, this defaults to
          YX mapping.

    \item [ESMF\_COORD\_ORDER\_YZX]
          The coordinates are ordered YZX.  For a 2D Grid, this defaults to
          YX mapping.

    \item [ESMF\_COORD\_ORDER\_ZXY]
          The coordinates are ordered ZXY.  For a 2D Grid, this defaults to
          XY mapping.

    \item [ESMF\_COORD\_ORDER\_ZYX]
          The coordinates are ordered ZYX.  For a 2D Grid, this defaults to
          YX mapping.
 
 \end{description}


 \subsubsection{ESMF\_CoordSystem}

 {\sf DESCRIPTION:\\}
 The Grid class supports a variety of coordinate systems, which are typically
 set by specific {\tt ESMF\_GridCreate} methods.

 Valid values are:
 \begin{description}
    \item [ESMF\_COORD\_SYSTEM\_CARTESIAN]
          Cartesian coordinates (x,y).

    \item [ESMF\_COORD\_SYSTEM\_CYLINDRICAL] 
          Cylindrical coordinates.

%   \item [ESMF\_COORD\_SYSTEM\_DEPTH]
%         Vertical z coordinate depth (0 at top surface).
%   \item [ESMF\_COORD\_SYSTEM\_ETA]
%         Vertical eta coordinate.
    \item [ESMF\_COORD\_SYSTEM\_HEIGHT]
          Vertical z coordinate height (0 at bottom).

%   \item [ESMF\_COORD\_SYSTEM\_HYBRID]
%         Hybrid vertical coordinates.
%   \item [ESMF\_COORD\_SYSTEM\_ISOPYCNAL]
%         Vertical density coordinate.
%   \item [ESMF\_COORD\_SYSTEM\_LAGRANGIAN] 
%         Lagrangian coordinates.
%   \item [ESMF\_COORD\_SYSTEM\_LATFOURIER]
%         Mixed latitude/Fourier spectral space.
%   \item [ESMF\_COORD\_SYSTEM\_PRESSURE]
%         Vertical pressure coordinate.
%   \item [ESMF\_COORD\_SYSTEM\_SIGMA] 
%         Vertical sigma coordinate.
%   \item [ESMF\_COORD\_SYSTEM\_SPECTRAL]
%         Wavenumber space.
    \item [ESMF\_COORD\_SYSTEM\_SPHERICAL]
          Spherical coordinates (longitude, latitude).

%   \item [ESMF\_COORD\_SYSTEM\_THETA]
%         Vertical theta coordinate.
    \item [ESMF\_COORD\_SYSTEM\_UNKNOWN]
          Unknown or undefined coordinate system.
%   \item [ESMF\_COORD\_SYSTEM\_USER]
%         User-defined coordinate system.

 \end{description}


 \subsubsection{ESMF\_GridHorzStagger}

 {\sf DESCRIPTION:\\}
 The Grid class supports a variety of horizontal Grid staggerings.  The
 {ESMF\_GridHorzStagger} parameter describes the options, following typical
 Arakawa nomenclature.

 Valid values are:
 \begin{description}
    \item [ESMF\_GRID\_HORZ\_STAGGER\_A]
          Arakawa A staggering, where all Fields, including velocities, are
          located at cell centers.  A Grid created with this staggering can
          only be used to create {\tt ESMF\_Fields} with the following
          horizontal {\tt ESMF\_RelLocs} (relative locations):
          \begin{description}
            \item ESMF\_CELL\_CENTER
          \end{description}

    \item [ESMF\_GRID\_HORZ\_STAGGER\_B\_NE]
          Arakawa B staggering, where both the U and V velocities are located at
          each cell's NorthEast corner.  A Grid created with this staggering can
          only be used to create {\tt ESMF\_Fields} with the following
          horizontal {\tt ESMF\_RelLocs} (relative locations):
          \begin{description}
            \item ESMF\_CELL\_CENTER
            \item ESMF\_CELL\_NECORNER
          \end{description}

    \item [ESMF\_GRID\_HORZ\_STAGGER\_B\_NW]
          Arakawa B staggering, where both the U and V velocities are located at
          each cell's NorthWest corner.  A Grid created with this staggering can
          only be used to create {\tt ESMF\_Fields} with the following
          horizontal {\tt ESMF\_RelLocs} (relative locations):
          \begin{description}
            \item ESMF\_CELL\_CENTER
            \item ESMF\_CELL\_NWCORNER
          \end{description}
 
    \item [ESMF\_GRID\_HORZ\_STAGGER\_B\_SE]
          Arakawa B staggering, where both the U and V velocities are located at
          each cell's SouthEast corner.  A Grid created with this staggering can
          only be used to create {\tt ESMF\_Fields} with the following
          horizontal {\tt ESMF\_RelLocs} (relative locations):
          \begin{description}
            \item ESMF\_CELL\_CENTER
            \item ESMF\_CELL\_SECORNER
          \end{description}

    \item [ESMF\_GRID\_HORZ\_STAGGER\_B\_SW]
          Arakawa B staggering, where both the U and V velocities are located at
          each cell's SouthWest corner.  A Grid created with this staggering can
          only be used to create {\tt ESMF\_Fields} with the following
          horizontal {\tt ESMF\_RelLocs} (relative locations):
          \begin{description}
            \item ESMF\_CELL\_CENTER
            \item ESMF\_CELL\_SWCORNER
          \end{description}

    \item [ESMF\_GRID\_HORZ\_STAGGER\_C\_NE]
          Arakawa C staggering, where the U velocity is located at the East face
          and the V velocity is located at the North face.  A Grid created with
          this staggering can only be used to create {\tt ESMF\_Fields} with the
          following horizontal {\tt ESMF\_RelLocs} (relative locations):
          \begin{description}
            \item ESMF\_CELL\_CENTER
            \item ESMF\_CELL\_NFACE
            \item ESMF\_CELL\_EFACE
          \end{description}

    \item [ESMF\_GRID\_HORZ\_STAGGER\_C\_NW]
          Arakawa C staggering, where the U velocity is located at the West face
          and the V velocity is located at the North face.  A Grid created with
          this staggering can only be used to create {\tt ESMF\_Fields} with the
          following horizontal {\tt ESMF\_RelLocs} (relative locations):
          \begin{description}
            \item ESMF\_CELL\_CENTER
            \item ESMF\_CELL\_NFACE
            \item ESMF\_CELL\_WFACE
          \end{description}
 
    \item [ESMF\_GRID\_HORZ\_STAGGER\_C\_SE]
          Arakawa C staggering, where the U velocity is located at the East face
          and the V velocity is located at the South face.  A Grid created with
          this staggering can only be used to create {\tt ESMF\_Fields} with the
          following horizontal {\tt ESMF\_RelLocs} (relative locations):
          \begin{description}
            \item ESMF\_CELL\_CENTER
            \item ESMF\_CELL\_SFACE
            \item ESMF\_CELL\_EFACE
          \end{description}

    \item [ESMF\_GRID\_HORZ\_STAGGER\_C\_SW]
          Arakawa C staggering, where the U velocity is located at the West face
          and the V velocity is located at the South face.  A Grid created with
          this staggering can only be used to create {\tt ESMF\_Fields} with the
          following horizontal {\tt ESMF\_RelLocs} (relative locations):
          \begin{description}
            \item ESMF\_CELL\_CENTER
            \item ESMF\_CELL\_SFACE
            \item ESMF\_CELL\_WFACE
          \end{description}

    \item [ESMF\_GRID\_HORZ\_STAGGER\_D\_NE]
          Arakawa D staggering, where the U velocity is located at the North face
          and the V velocity is located at the East face.  A Grid created with
          this staggering can only be used to create {\tt ESMF\_Fields} with the
          following horizontal {\tt ESMF\_RelLocs} (relative locations):
          \begin{description}
            \item ESMF\_CELL\_CENTER
            \item ESMF\_CELL\_NFACE
            \item ESMF\_CELL\_EFACE
          \end{description}

    \item [ESMF\_GRID\_HORZ\_STAGGER\_D\_NW]
          Arakawa D staggering, where the U velocity is located at the North face
          and the V velocity is located at the West face.  A Grid created with
          this staggering can only be used to create {\tt ESMF\_Fields} with the
          following horizontal {\tt ESMF\_RelLocs} (relative locations):
          \begin{description}
            \item ESMF\_CELL\_CENTER
            \item ESMF\_CELL\_NFACE
            \item ESMF\_CELL\_WFACE
          \end{description}

    \item [ESMF\_GRID\_HORZ\_STAGGER\_D\_SE]
          Arakawa D staggering, where the U velocity is located at the South face
          and the V velocity is located at the East face.  A Grid created with
          this staggering can only be used to create {\tt ESMF\_Fields} with the
          following horizontal {\tt ESMF\_RelLocs} (relative locations):
          \begin{description}
            \item ESMF\_CELL\_CENTER
            \item ESMF\_CELL\_SFACE
            \item ESMF\_CELL\_EFACE
          \end{description}
 
    \item [ESMF\_GRID\_HORZ\_STAGGER\_D\_SW]
          Arakawa D staggering, where the U velocity is located at the South face
          and the V velocity is located at the West face.  A Grid created with
          this staggering can only be used to create {\tt ESMF\_Fields} with the
          following horizontal {\tt ESMF\_RelLocs} (relative locations):
          \begin{description}
            \item ESMF\_CELL\_CENTER
            \item ESMF\_CELL\_SFACE
            \item ESMF\_CELL\_WFACE
          \end{description}

%   \item [ESMF\_GRID\_HORZ\_STAGGER\_E]
%         Arakawa E.

    \item [ESMF\_GRID\_HORZ\_STAGGER\_UNKNOWN]
          Unknown or undefined staggering.
%   \item [ESMF\_GRID\_HORZ\_STAGGER\_Z]
%         C grid equivalent for geodesic grid.

 \end{description}


% \subsubsection{ESMF\_GridMaskType}
% 
% {\sf DESCRIPTION:\\}
% Supported types of grid masks.
% 
% Valid values are:
% \begin{description}
%    \item [ESMF\_GRID\_MASKTYPE\_LOGICAL]
%          Logical mask.
%    \item [ESMF\_GRID\_MASKTYPE\_MULT]
%          Multiplicative mask.
%    \item [ESMF\_GRID\_MASKTYPE\_REGION\_ID]
%          Integer assigning unique ID to each point.
%    \item [ESMF\_GRID\_MASKTYPE\_UNKNOWN]
%          Unknown or undefined mask type.
% \end{description}


 \subsubsection{ESMF\_GridStorage}
 
 {\sf DESCRIPTION:\\}
 Distributed Grid storage options supported by ESMF.
 
 Valid values are:
 \begin{description}
    \item [ESMF\_GRID\_STORAGE\_BLOCK]
          Grid is distributed as single rectangular blocks on any, but not
          necessarily all, DEs.  All high level communication methods are
          supported for this storage option.

%    \item [ESMF\_GRID\_STORAGE\_MULTIBLOCK]
%          Logically rectangular grid.  This infers a single block grid.
    \item [ESMF\_GRID\_STORAGE\_UNKNOWN]
          Unknown or undefined Grid storage.

    \item [ESMF\_GRID\_STORAGE\_ARBITRARY]
          Grid is distributed as 1D arbitrary vectors on any, but not
          necessarily all, DEs.  This storage option is intended for
          column based computations, so the only high level communication
          method that is supported is redistribution back and forth with
          its underlying Grid.
 
 \end{description}


% \subsubsection{ESMF\_GridStructure}
% 
% {\sf DESCRIPTION:\\}
% General Grid structures supported by ESMF.
% 
% Valid values are:
% \begin{description}
%    \item [ESMF\_GRID\_STRUCTURE\_LOGRECT]
%          Logically rectangular grid.  This infers a single block grid.
% 
%    \item [ESMF\_GRID\_STRUCTURE\_LOGRECTBLK]
%          Logically rectangular blocked grid.
%    \item [ESMF\_GRID\_STRUCTURE\_UNKNOWN]
%          Unknown or undefined grid.
%    \item [ESMF\_GRID\_STRUCTURE\_UNSTRUCT]
%          Unstructured grid.
%    \item [ESMF\_GRID\_STRUCTURE\_USER]
%          User-defined grid.
% 
% \end{description}


 \subsubsection{ESMF\_GridType}
 
 {\sf DESCRIPTION:\\}
 There are several Grid types supported by ESMF.  In general, we expect each
 {\tt ESMF\_GridType} to have its own explicit 
 {\tt ESMF\_GridCreateHorz<GridType>()} function.
 
 Valid values are:
 \begin{description}
%    \item [ESMF\_GRID\_TYPE\_CART\_SPECT]
%          Spectral space for cartesian coordinates.
%    \item [ESMF\_GRID\_TYPE\_CUBEDSPHERE]
%          Cubed sphere grid.
%    \item [ESMF\_GRID\_TYPE\_DATASTREAM]
%          Data stream - set of locations.
%    \item [ESMF\_GRID\_TYPE\_DIPOLE]
%          Displaced-pole dipole grid.
%    \item [ESMF\_GRID\_TYPE\_EXCHANGE]
%          Intersection of two grids, which is itself a grid.
%    \item [ESMF\_GRID\_TYPE\_GEODESIC]
%          Spherical geodesic grid.
    \item [ESMF\_GRID\_TYPE\_LATLON]
          Latitude/longitude Grid with variable or unequal spacing.
 
%    \item [ESMF\_GRID\_TYPE\_LATLON\_GAUSS]
%          Latitude/Longitude grid with gaussian-spaced latitudes.
%    \item [ESMF\_GRID\_TYPE\_LATLON\_MERC]
%          Latitude/Longitude grid with Mercator-spaced latitudes.
    \item [ESMF\_GRID\_TYPE\_LATLON\_UNI]
          Latitude/longitude Grid with uniform spacing.
 
%    \item [ESMF\_GRID\_TYPE\_PHYSFOURIER]
%          Mixed Fourier Space/Physical Space grid.
%    \item [ESMF\_GRID\_TYPE\_REDUCED]
%          Latitude/Longitude grid where the number of longitudinal points is a
%          function of the latitude.
%    \item [ESMF\_GRID\_TYPE\_SPHER\_SPECT]
%          Spectral space for spherical harmonics.
%    \item [ESMF\_GRID\_TYPE\_TRIPOLE]
%          Tripolar grids.
    \item [ESMF\_GRID\_TYPE\_UNKNOWN]
          Unknown or undefined Grid.

    \item [ESMF\_GRID\_TYPE\_XY]
          XY Cartesian Grid with variable or unequal spacing.
 
    \item [ESMF\_GRID\_TYPE\_XY\_UNI]
          XY Cartesian Grid with uniform spacing.
 
 \end{description}


 \subsubsection{ESMF\_GridVertStagger}

 {\sf DESCRIPTION:\\}
 The Grid class supports a variety of vertical subGrid staggerings.  The
 {ESMF\_GridVertStagger} parameter describes the options.

 Valid values are:
 \begin{description}

    \item [ESMF\_GRID\_VERT\_STAGGER\_BOTTOM]
          Vertical velocity or pressure gradient is located at the bottom
          vertical face of the cell.  A subGrid created with this staggering
          will only accept {\tt ESMF\_Fields} with the following vertical
          {\tt ESMF\_RelLocs} (relative locations):
          \begin{description}
            \item ESMF\_CELL\_CELL
            \item ESMF\_CELL\_BOTFACE
          \end{description}

    \item [ESMF\_GRID\_VERT\_STAGGER\_CENTER] 
          Vertical velocity or pressure gradient is located at vertical
          midpoints.  A subGrid created with this staggering will only accept
          {\tt ESMF\_Fields} with the following vertical {\tt ESMF\_RelLocs}
          (relative locations):
          \begin{description}
            \item ESMF\_CELL\_CELL
          \end{description}

    \item [ESMF\_GRID\_VERT\_STAGGER\_TOP]
          Vertical velocity or pressure gradient is located at the top vertical 
          face of the cell.  A subGrid created with this staggering will only
          accept {\tt ESMF\_Fields} with the following vertical
          {\tt ESMF\_RelLocs} (relative locations):
          \begin{description}
            \item ESMF\_CELL\_CELL
            \item ESMF\_CELL\_TOPFACE
          \end{description}

    \item [ESMF\_GRID\_VERT\_STAGGER\_UNKNOWN]
          Unknown or undefined staggering.

 \end{description}


 \subsubsection{ESMF\_GridVertType}
 
 {\sf DESCRIPTION:\\}
 Several vertical subGrid types are supported by ESMF.  In general, we expect
 each {\tt ESMF\_GridVertType} to have its own explicit
 {\tt ESMF\_GridAddVert<GridVertType>()} subroutine.
 
 Valid values are:
 \begin{description}
%   \item [ESMF\_GRID\_TYPE\_VERT\_DEPTH]
%         Vertical subGrid with zero coordinate at top surface.
%   \item [ESMF\_GRID\_TYPE\_VERT\_ETA]
%         Vertical subGrid with eta coordinates.
    \item [ESMF\_GRID\_TYPE\_VERT\_HEIGHT]
          Vertical subGrid with zero coordinate at bottom.
%   \item [ESMF\_GRID\_TYPE\_VERT\_HYBRID]
%         Vertical subGrid with hybrid coordinates.
%   \item [ESMF\_GRID\_TYPE\_VERT\_ISOPYCNAL]
%         Vertical subGrid with density coordinates.
%   \item [ESMF\_GRID\_TYPE\_VERT\_LAGRANGIAN]
%         Vertical subGrid with lagrangian coordinates.
%   \item [ESMF\_GRID\_TYPE\_VERT\_PRESSURE]
%         Vertical subGrid with pressure coordinates.
%   \item [ESMF\_GRID\_TYPE\_VERT\_SIGMA]
%         Vertical subGrid with sigma coordinates.
%   \item [ESMF\_GRID\_TYPE\_VERT\_THETA]
%         Vertical subGrid with theta coordinates.
    \item [ESMF\_GRID\_TYPE\_VERT\_UNKNOWN]
          Unknown or undefined vertical subGrid.
%   \item [ESMF\_GRID\_TYPE\_VERT\_USER]
%         User-defined vertical subGrid.

 \end{description}

%\subsubsection{ESMF\_RegionType}
%
%{\sf DESCRIPTION:\\}
%Supported types of grid regions.
%
%Valid values are:
%\begin{description}
%   \item [ESMF\_REGION\_TYPE\_ELLIPSE]
%         Elliptical regions centered on grid point, defined by two 
%         additional parameters.
%   \item [ESMF\_REGION\_TYPE\_POLYGON]
%         Polygonal regions defined by vertex coordinates.
%   \item [ESMF\_REGION\_TYPE\_UNKNOWN]
%         Unknown or undefined region type.
%\end{description}


