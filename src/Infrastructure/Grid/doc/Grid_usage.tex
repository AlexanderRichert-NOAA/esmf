% $Id: Grid_usage.tex,v 1.6 2004/05/24 22:58:05 jwolfe Exp $

%\subsection{Design and Implementation Notes}


In typical applications, Grids are created either internally or read in from
a file.  The {\tt ESMF\_Grid} class will provide methods for both, though at 
the moment is only has routines for simple internal Grid generation.  It also
has a variety of methods to set and get Grid parameters like the number of cells
on any particular processor.  Some methods which have a Grid interface are
actually implemented at the underlying DistGrid or PhysGrid level; they will be
inherited by the Grid class.  This allows the API to present functions at the
level which is most consistent to the application without restricting where
inside the ESMF the actual implementation is done.

The creation of a distributed {\tt ESMF\_Grid} requires multiple steps, as
illustrated in the example code below.  The GridCreate call, which has an
explicit interface for each GridType, allocates space for the Grid class and sets
parameters defining the horizontal grid.  A vertical subGrid can then be attached
to the Grid via a GridAddVert call.  Currently a Grid can have only a single
vertical subGrid.  The last call, {\tt ESMF\_GridDistribute()}, allocates some of
the Grid subclasses and distributes the Grid in either a default or
user-specified decomposition.  Currently, these calls must be made in this order
(i.e. it is not possible to add a vertical subGrid to an already distributed
Grid).

