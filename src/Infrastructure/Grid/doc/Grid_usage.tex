% $Id: Grid_usage.tex,v 1.11 2005/01/10 21:42:01 jwolfe Exp $

%\subsection{Design and Implementation Notes}


In typical applications, Grids are created either internally or read in from
a file.  The {\tt ESMF\_Grid} class will provide methods for both, though 
currently it only has routines for simple internal Grid generation.  It also
has a variety of methods to set and get Grid parameters such as the number of
cells associated with a particular DE.  

The creation of a distributed Grid requires multiple steps, as
illustrated in the example code below.  The {\tt ESMF\_GridCreateHorz<GridType>()}
call, which has an explicit interface for each GridType, allocates space for the
Grid class and sets parameters defining the horizontal grid.  A vertical grid
can then be attached to the Grid via an {\tt ESMF\_GridAddVert<VertGridType>()}
call.  Currently a Grid can have only a single vertical grid.  The last call,
{\tt ESMF\_GridDistribute()}, allocates some of the Grid subclasses and
distributes the Grid in either a default or user-specified decomposition.

The default decomposition distributes the Grid cells as evenly as possible across
the DEs in the attached delayout.  The user can define a specified distribution
in the {\tt ESMF\_GridDistribute()} call through optional arguments.  ESMF
currently supports two different basic distribution patterns.  In the first,
logically-rectangular blocks of data are distributed, one per DE.  Here, any two
dimensions of a 3D grid may be distributed.  By default, the first grid
dimension is distributed by the first delayout dimension and the second grid
dimension is distributed by the second delayout dimension.  Users can specify
other distributions via another optional argument to the {\tt ESMF\_GridDistribute()}
call.  The second distribution pattern assigns arbitrary points to different DEs
and is intended for vertical column calculations where horizontal communication
is not necessary.  Please see the {\tt ESMF\_GridDistribute()} interface
description for further details.

Please note that {\tt ESMF\_GridDistribute()} must be called, even if mpiuni is
used for communication.  Currently, these calls must be made in this order
(i.e. it is not possible to add a vertical grid to an already distributed
Grid), and while an {\tt ESMF\_GridAddVert<VertGridType>()} call is optional,
both an {\tt ESMF\_GridCreate<GridType>()} and {\tt ESMF\_GridDistribute()}
call are required.

