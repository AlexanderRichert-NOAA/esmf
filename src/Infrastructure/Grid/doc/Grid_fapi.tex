%                **** IMPORTANT NOTICE *****
% This LaTeX file has been automatically produced by ProTeX v. 1.1
% Any changes made to this file will likely be lost next time
% this file is regenerated from its source. Send questions 
% to Arlindo da Silva, dasilva@gsfc.nasa.gov
 
\parskip        0pt
\parindent      0pt
\baselineskip  11pt
 
%--------------------- SHORT-HAND MACROS ----------------------
\def\bv{\begin{verbatim}}
\def\ev{\end{verbatim}}
\def\be{\begin{equation}}
\def\ee{\end{equation}}
\def\bea{\begin{eqnarray}}
\def\eea{\end{eqnarray}}
\def\bi{\begin{itemize}}
\def\ei{\end{itemize}}
\def\bn{\begin{enumerate}}
\def\en{\end{enumerate}}
\def\bd{\begin{description}}
\def\ed{\end{description}}
\def\({\left (}
\def\){\right )}
\def\[{\left [}
\def\]{\right ]}
\def\<{\left  \langle}
\def\>{\right \rangle}
\def\cI{{\cal I}}
\def\diag{\mathop{\rm diag}}
\def\tr{\mathop{\rm tr}}
%-------------------------------------------------------------

\markboth{Left}{Source File: ESMF\_Grid.F90,  Date: Wed Jan 15 15:07:53 MST 2003
}

 
%/////////////////////////////////////////////////////////////
\subsection{Fortran:  Module Interface ESMF\_GridMod - Grid class (Source File: ESMF\_Grid.F90)}


  
  
   The code in this file implements the {\tt Grid} class.  This class
   provides a unified interface for both {\tt PhysGrid} and {\tt DistGrid}
   information for model grids.  Functions for defining and computing {\tt Grid}
   information are available through this class.
  
  ------------------------------------------------------------------------------
\bigskip{\em USES:}
\begin{verbatim}       use ESMF_ArrayMod       ! ESMF array class
       use ESMF_BaseMod        ! ESMF base class
       use ESMF_DistGridMod    ! ESMF distributed grid class
       use ESMF_IOMod          ! ESMF I/O class
       use ESMF_LayoutMod      ! ESMF layout class
       use ESMF_PhysGridMod    ! ESMF physical grid class
       implicit none
 
  ------------------------------------------------------------------------------\end{verbatim}{\sf PRIVATE TYPES:}
\begin{verbatim}       private
  ------------------------------------------------------------------------------
       ! ESMF_GridConfig
       ! Description of ESMF_GridConfig
 
       type ESMF_GridConfig
       sequence
       private
         integer :: dummy
         < insert other class members here >
       end type
 
  ------------------------------------------------------------------------------
       !  ESMF_GridType
       ! Definition for the Grid class.  A Grid
       ! is passed back to the user at Grid creation.
 
       type ESMF_GridType
       sequence
       private
 
         type (ESMF_Base) :: base            ! base class object
         type (ESMF_Status) :: gridstatus    ! uninitialized, init ok, etc
         integer :: horz_gridtype            ! enum for type of horizontal grid
         integer :: vert_gridtype            ! enum for type of vertical grid
         integer :: horz_stagger             ! enum for horizontal grid staggering
         integer :: vert_stagger             ! enum for vertical grid staggering
         integer :: horz_coord_system        ! enum for horizontal physical
                                             ! coordinate system
         integer :: vert_coord_system        ! enum for vertical physical
                                             ! coordinate system
         integer :: coord_order              ! enum for mapping of xyz 
                                             ! to ijk
         integer :: num_physgrids            ! number of grid descriptors
                                             ! necessary to support
                                             ! staggering, vertical
                                             ! grids, background grids
         type (ESMF_PhysGridType), dimension(:), pointer :: &
            physgrids         ! grid info for all grid descriptions necessary
                              ! to define horizontal, staggered and vertical grids
         type (ESMF_DistGrid) :: distgrid    ! decomposition and other
                                             ! logical space info for grid
 
       end type
 
  ------------------------------------------------------------------------------
       !  ESMF_Grid
       ! The Grid data structure that is passed between languages.
 
       type ESMF_Grid
       sequence
       private
         type (ESMF_GridType), pointer :: ptr     ! pointer to a grid type
       end type
 
  ------------------------------------------------------------------------------\end{verbatim}{\sf PUBLIC TYPES:}
\begin{verbatim} 
       public ESMF_GridConfig
       public ESMF_Grid
       public ESMF_GridType
 
  ------------------------------------------------------------------------------\end{verbatim}{\sf PUBLIC MEMBER FUNCTIONS:}
\begin{verbatim} 
     public ESMF_GridCreate
     public ESMF_GridDestroy
     public ESMF_GridAddPhysGrid
     public ESMF_GridGetConfig
     public ESMF_GridSetConfig
     !public ESMF_GridGetCoord
     public ESMF_GridSetCoord
     public ESMF_GridGetDE    ! temporary to access DistGrid from above
     !public ESMF_GridGetInfo
     public ESMF_GridSetInfo
     !public ESMF_GridGetLMask
     public ESMF_GridSetLMask
     !public ESMF_GridGetMMask
     public ESMF_GridSetMMask
     !public ESMF_GridGetMetric
     public ESMF_GridSetMetric
     !public ESMF_GridGetRegionID
     public ESMF_GridSetRegionID
     public ESMF_GridValidate
     public ESMF_GridPrint
 
  ------------------------------------------------------------------------------\end{verbatim}{\sf PUBLIC DATA MEMBERS:}
\begin{verbatim} 
    integer, parameter, public ::            &! recognized grid types
       ESMF_GridType_Unknown           =  0, &! unknown or undefined grid
       ESMF_GridType_LatLon            =  1, &! equally-spaced lat/lon grid
       ESMF_GridType_Mercator          =  2, &! Mercator lat/lon grid
       ESMF_GridType_Dipole            =  3, &! Displaced-pole dipole grid
       ESMF_GridType_Tripole           =  4, &! Tripolar grids
       ESMF_GridType_XY                =  5, &! Cartesian equally-space x-y grid
       ESMF_GridType_DataStream        =  6, &! Data stream
       ESMF_GridType_PhysFourier       =  7, &! Mixed Fourier Space/Phys Space grid
       ESMF_GridType_LatLonGauss       =  8, &! lat/lon grid with Gaussian latitudes
       ESMF_GridType_SphericalSpectral =  9, &! spectral space for spherical harmonics
       ESMF_GridType_Geodesic          = 10, &! spherical geodesic grid
       ESMF_GridType_CubedSphere       = 11   ! cubed sphere grid
 
    integer, parameter, public ::            &! recognized staggering types
       ESMF_GridStagger_Unknown        =  0, &! unknown or undefined staggering
       ESMF_GridStagger_A              =  1, &! Arakawa A (centered velocity)
       ESMF_GridStagger_B              =  2, &! Arakawa B (velocities at grid corner)
       ESMF_GridStagger_C              =  3, &! Arakawa C (velocities at cell faces)
       ESMF_GridStagger_Z              =  4, &! C grid equiv for geodesic grid
       ESMF_GridStagger_VertCenter     =  5, &! vert velocity at vertical midpoints
       ESMF_GridStagger_VertFace       =  6   ! vert velocity/Pgrad at top(bottom)face
 
    integer, parameter, public ::            &! recognized coordinate systems
       ESMF_CoordSystem_Unknown        =  0, &! unknown or undefined coord system
       ESMF_CoordSystem_Spherical      =  1, &! spherical coordinates (lat/lon)
       ESMF_CoordSystem_Cartesian      =  2, &! Cartesian coordinates (x,y)
       ESMF_CoordSystem_Cylindrical    =  3, &! cylindrical coordinates
       ESMF_CoordSystem_LatFourier     =  4, &! mixed latitude/spectral space
       ESMF_CoordSystem_Spectral       =  5, &! wavenumber space
       ESMF_CoordSystem_Depth          =  6, &! vertical z coord. depth (0 at surface)
       ESMF_CoordSystem_Height         =  7, &! vertical z coord. height (0 at bottom)
       ESMF_CoordSystem_Pressure       =  8, &! vertical pressure coordinate
       ESMF_CoordSystem_Sigma          =  9, &! vertical sigma coordinate
       ESMF_CoordSystem_Theta          = 10, &! vertical theta coordinate
       ESMF_CoordSystem_Eta            = 11, &! vertical eta coordinate
       ESMF_CoordSystem_Isopycnal      = 12, &! vertical density coordinate
       ESMF_CoordSystem_Hybrid         = 13, &! hybrid vertical coordinates
       ESMF_CoordSystem_Lagrangian     = 14   ! Lagrangian coordinates
       ! I'm sure there are more - I'm not sure
       ! what the atmospheric ESMF models are using for vertical coords
 
    integer, parameter, public ::            &! recognized coordinate orderings
       ESMF_CoordOrder_Unknown         =  0, &! unknown or undefined coord ordering
       ESMF_CoordOrder_XYZ             =  1, &! IJK maps to XYZ
       ESMF_CoordOrder_XZY             =  2, &! IJK maps to XZY
       ESMF_CoordOrder_YXZ             =  3, &! IJK maps to YXZ
       ESMF_CoordOrder_YZX             =  4, &! IJK maps to YZX
       ESMF_CoordOrder_ZXY             =  5, &! IJK maps to ZXY
       ESMF_CoordOrder_ZYX             =  6   ! IJK maps to ZYX\end{verbatim}
 
%/////////////////////////////////////////////////////////////
 
\mbox{}\hrulefill\ 
 

\bigskip{\sf INTERFACE:}
\begin{verbatim}       interface ESMF_GridCreate
 \end{verbatim}{\sf PRIVATE MEMBER FUNCTIONS:}
\begin{verbatim}          module procedure ESMF_GridCreateEmpty
          module procedure ESMF_GridCreateInternal
          module procedure ESMF_GridCreateRead
          module procedure ESMF_GridCreateCopy
          module procedure ESMF_GridCreateCutout
          module procedure ESMF_GridCreateChangeResolution
          module procedure ESMF_GridCreateExchange
 \end{verbatim}
{\sf DESCRIPTION:\\ }


       This interface provides a single entry point for {\tt Grid} create
       methods.
   
%/////////////////////////////////////////////////////////////
 
\mbox{}\hrulefill\ 
 

\bigskip{\sf INTERFACE:}
\begin{verbatim}       interface ESMF_GridConstruct
 \end{verbatim}{\sf PRIVATE MEMBER FUNCTIONS:}
\begin{verbatim}          module procedure ESMF_GridConstructNew
          module procedure ESMF_GridConstructInternal
 \end{verbatim}
{\sf DESCRIPTION:\\ }


       This interface provides a single entry point for methods that construct a
       complete {\tt Grid}.
   
%/////////////////////////////////////////////////////////////
 
\mbox{}\hrulefill\ 
 

\bigskip{\sf INTERFACE:}
\begin{verbatim}       interface ESMF_GridSetCoord
 \end{verbatim}{\sf PRIVATE MEMBER FUNCTIONS:}
\begin{verbatim}          module procedure ESMF_GridSetCoordFromArray
          module procedure ESMF_GridSetCoordFromBuffer
          module procedure ESMF_GridSetCoordCompute
          module procedure ESMF_GridSetCoordCopy
 \end{verbatim}
{\sf DESCRIPTION:\\ }


       This interface provides a single entry point for methods that set
       coordinates as part of a {\tt Grid}.
   
%/////////////////////////////////////////////////////////////
 
\mbox{}\hrulefill\ 
 

\bigskip{\sf INTERFACE:}
\begin{verbatim}       interface ESMF_GridSetLMask
 \end{verbatim}{\sf PRIVATE MEMBER FUNCTIONS:}
\begin{verbatim}          module procedure ESMF_GridSetLMaskFromArray
          module procedure ESMF_GridSetLMaskFromBuffer
          module procedure ESMF_GridSetLMaskFromMMask
          module procedure ESMF_GridSetLMaskCopy
 \end{verbatim}
{\sf DESCRIPTION:\\ }


       This interface provides a single entry point for methods that set
       logical masks as part of a {\tt Grid}.
   
%/////////////////////////////////////////////////////////////
 
\mbox{}\hrulefill\ 
 

\bigskip{\sf INTERFACE:}
\begin{verbatim}       interface ESMF_GridSetMMask
 \end{verbatim}{\sf PRIVATE MEMBER FUNCTIONS:}
\begin{verbatim}          module procedure ESMF_GridSetMMaskFromArray
          module procedure ESMF_GridSetMMaskFromBuffer
          module procedure ESMF_GridSetMMaskFromLMask
          module procedure ESMF_GridSetMMaskCopy
 \end{verbatim}
{\sf DESCRIPTION:\\ }


       This interface provides a single entry point for methods that set
       multiplicative masks as part of a {\tt Grid}.
   
%/////////////////////////////////////////////////////////////
 
\mbox{}\hrulefill\ 
 

\bigskip{\sf INTERFACE:}
\begin{verbatim}       interface ESMF_GridSetMetric
 \end{verbatim}{\sf PRIVATE MEMBER FUNCTIONS:}
\begin{verbatim}          module procedure ESMF_GridSetMetricFromArray
          module procedure ESMF_GridSetMetricFromBuffer
          module procedure ESMF_GridSetMetricCompute
          module procedure ESMF_GridSetMetricCopy
 \end{verbatim}
{\sf DESCRIPTION:\\ }


       This interface provides a single entry point for methods that set
       metrics as part of a {\tt Grid}.
   
%/////////////////////////////////////////////////////////////
 
\mbox{}\hrulefill\ 
 

\bigskip{\sf INTERFACE:}
\begin{verbatim}       interface ESMF_GridSetRegionID
 \end{verbatim}{\sf PRIVATE MEMBER FUNCTIONS:}
\begin{verbatim}          module procedure ESMF_GridSetRegionIDFromArray
          module procedure ESMF_GridSetRegionIDFromBuffer
          module procedure ESMF_GridSetRegionIDCopy
 \end{verbatim}
{\sf DESCRIPTION:\\ }


       This interface provides a single entry point for methods that set
       region id's as part of a {\tt Grid}.
   
%/////////////////////////////////////////////////////////////
 
\mbox{}\hrulefill\ 
 
\subsubsection{ESMF\_GridCreateEmpty - Create a new Grid with no data}


 
\bigskip{\sf INTERFACE:}
\begin{verbatim}       function ESMF_GridCreateEmpty(name, rc)\end{verbatim}{\em RETURN VALUE:}
\begin{verbatim}       type(ESMF_Grid) :: ESMF_GridCreateEmpty\end{verbatim}{\em ARGUMENTS:}
\begin{verbatim}       character (len=*), intent(in), optional :: name
       integer, intent(out), optional :: rc\end{verbatim}
{\sf DESCRIPTION:\\ }


       Allocates memory for a new {\tt Grid} object and constructs its
       internals.  Return a pointer to the new {\tt Grid}.
  
       The arguments are:
       \begin{description}
       \item[[name]]
            {\tt Grid} name.
       \item[[rc]]
            Return code; equals {\tt ESMF\_SUCCESS} if there are no errors.
     \end{description}
  
\bigskip{\sf REQUIREMENTS:}
\begin{verbatim} \end{verbatim}
 
%/////////////////////////////////////////////////////////////
 
\mbox{}\hrulefill\ 
 
\subsubsection{ESMF\_GridCreateInternal - Create a new Grid internally}


\bigskip{\sf INTERFACE:}
\begin{verbatim}       function ESMF_GridCreateInternal(i_max, j_max, &
                                        nDE_i, nDE_j, layout, &
                                        horz_gridtype, vert_gridtype, &
                                        horz_stagger, vert_stagger, &
                                        horz_coord_system, vert_coord_system, &
                                        x_min, x_max, y_min, y_max, name, rc)\end{verbatim}{\em RETURN VALUE:}
\begin{verbatim}       type(ESMF_Grid) :: ESMF_GridCreateInternal\end{verbatim}{\em ARGUMENTS:}
\begin{verbatim}       integer, intent(in) :: i_max
       integer, intent(in) :: j_max
       integer, intent(in), optional :: nDE_i
       integer, intent(in), optional :: nDE_j
       type (ESMF_Layout), intent(in), optional :: layout
       integer, intent(in), optional :: horz_gridtype
       integer, intent(in), optional :: vert_gridtype
       integer, intent(in), optional :: horz_stagger
       integer, intent(in), optional :: vert_stagger
       integer, intent(in), optional :: horz_coord_system
       integer, intent(in), optional :: vert_coord_system
       real, intent(in), optional :: x_min
       real, intent(in), optional :: x_max
       real, intent(in), optional :: y_min
       real, intent(in), optional :: y_max
       character (len=*), intent(in), optional :: name
       integer, intent(out), optional :: rc\end{verbatim}
{\sf DESCRIPTION:\\ }


       Allocates memory for a new {\tt Grid} object, constructs its
       internals, and internally generates the Grid.  Return a pointer to
       the new {\tt Grid}.
  
       The arguments are:
       \begin{description}
       \item[[i\_max]]
            Number of grid increments in the i-direction.
       \item[[j\_max]]
            Number of grid increments in the j-direction.
       \item[[nDE\_i]]
            Number of DE's in 1st dir.
       \item[[nDE\_j]]
            Number of DE's in 2nd dir.
       \item[[layout]]
            Layout of DE's.
       \item[[horz\_gridtype]]
            Integer specifier to denote horizontal gridtype.
       \item[[vert\_gridtype]]
            Integer specifier to denote vertical gridtype.
       \item[[horz\_stagger]]
            Integer specifier to denote horizontal grid stagger.
       \item[[vert\_stagger]]
            Integer specifier to denote vertical grid stagger.
       \item[[horz\_coord\_system]]
            Integer specifier to denote horizontal coordinate system.
       \item[[vert\_coord\_system]]
            Integer specifier to denote vertical coordinate system.
       \item[[x\_min]]
            Minimum physical coordinate in the x-direction.
       \item[[x\_max]]
            Maximum physical coordinate in the x-direction.
       \item[[y\_min]]
            Minimum physical coordinate in the y-direction.
       \item[[y\_max]]
            Maximum physical coordinate in the y-direction.
       \item[[name]]
            {\tt Grid} name.
       \item[[rc]]
            Return code; equals {\tt ESMF\_SUCCESS} if there are no errors.
     \end{description}
  
\bigskip{\sf REQUIREMENTS:}
\begin{verbatim} \end{verbatim}
 
%/////////////////////////////////////////////////////////////
 
\mbox{}\hrulefill\ 
 
\subsubsection{ESMF\_GridCreateRead - Create a new Grid read in from a file}


\bigskip{\sf INTERFACE:}
\begin{verbatim}       function ESMF_GridCreateRead(iospec, name, rc)\end{verbatim}{\em RETURN VALUE:}
\begin{verbatim}       type(ESMF_Grid) :: ESMF_GridCreateRead\end{verbatim}{\em ARGUMENTS:}
\begin{verbatim}       type(ESMF_IOSpec), intent(in) :: iospec   ! file specs
       character (len=*), intent(in), optional :: name
       integer, intent(out), optional :: rc\end{verbatim}
{\sf DESCRIPTION:\\ }


       Allocates memory for a new {\tt Grid} object, constructs its
       internals, and reads a {\tt Grid} in from a file.  Return a pointer to
       the new {\tt Grid}.
  
       The arguments are:
       \begin{description}
       \item[[iospec]]
            File I/O specification.
       \item[[name]]
            {\tt Grid} name.
       \item[[rc]]
            Return code; equals {\tt ESMF\_SUCCESS} if there are no errors.
     \end{description}
  
\bigskip{\sf REQUIREMENTS:}
\begin{verbatim} \end{verbatim}
 
%/////////////////////////////////////////////////////////////
 
\mbox{}\hrulefill\ 
 
\subsubsection{ESMF\_GridCreateCopy - Create a new Grid by copying another Grid}


\bigskip{\sf INTERFACE:}
\begin{verbatim}       function ESMF_GridCreateCopy(grid_in, name, rc)\end{verbatim}{\em RETURN VALUE:}
\begin{verbatim}       type(ESMF_Grid) :: ESMF_GridCreateCopy\end{verbatim}{\em ARGUMENTS:}
\begin{verbatim}       type(ESMF_Grid), intent(in) :: grid_in
       character (len=*), intent(in), optional :: name
       integer, intent(out), optional :: rc\end{verbatim}
{\sf DESCRIPTION:\\ }


       Allocates memory for a new {\tt Grid} object, constructs its
       internals, and copies attributes from another {\tt Grid}.  Return a
       pointer to the new {\tt Grid}.
  
       The arguments are:
       \begin{description}
       \item[[grid\_in]]
            {\tt Grid} to be copied.
       \item[[name]]
            {\tt Grid} name.
       \item[[rc]]
            Return code; equals {\tt ESMF\_SUCCESS} if there are no errors.
     \end{description}
  
\bigskip{\sf REQUIREMENTS:}
\begin{verbatim} \end{verbatim}
 
%/////////////////////////////////////////////////////////////
 
\mbox{}\hrulefill\ 
 
\subsubsection{ESMF\_GridCreateCutout - Create a new Grid as a subset of an existing Grid}


\bigskip{\sf INTERFACE:}
\begin{verbatim}       function ESMF_GridCreateCutout(grid_in, i_min, i_max, j_min, j_max, &
                                      name, rc)\end{verbatim}{\em RETURN VALUE:}
\begin{verbatim}       type(ESMF_Grid) :: ESMF_GridCreateCutout\end{verbatim}{\em ARGUMENTS:}
\begin{verbatim}       type(ESMF_Grid), intent(in) :: grid_in
       integer, intent(in) :: i_min
       integer, intent(in) :: i_max
       integer, intent(in) :: j_min
       integer, intent(in) :: j_max
       character (len=*), intent(in), optional :: name
       integer, intent(out), optional :: rc\end{verbatim}
{\sf DESCRIPTION:\\ }


       Allocates memory for a new {\tt Grid} object, constructs its
       internals, and copies a region from an existing {\tt Grid}.
       Return a pointer to the new {\tt Grid}.
  
       The arguments are:
       \begin{description}
       \item[[grid\_in]]
            {\tt Grid} to be partially copied.
       \item[[i\_min]]
            Minimum global i-index for the region of the grid to be cutout.
       \item[[i\_max]]
            Maximum global i-index for the region of the grid to be cutout.
       \item[[j\_min]]
            Minimum global j-index for the region of the grid to be cutout.
       \item[[j\_max]]
            Maximum global j-index for the region of the grid to be cutout.
       \item[[name]]
            {\tt Grid} name.
       \item[[rc]]
            Return code; equals {\tt ESMF\_SUCCESS} if there are no errors.
     \end{description}
  
\bigskip{\sf REQUIREMENTS:}
\begin{verbatim} \end{verbatim}
 
%/////////////////////////////////////////////////////////////
 
\mbox{}\hrulefill\ 
 
\subsubsection{ESMF\_GridCreateChangeResolution - Create a new Grid by coarsening or refining an existing Grid}


\bigskip{\sf INTERFACE:}
\begin{verbatim}       function ESMF_GridCreateChangeResolution(grid_in, i_resolution, &
                                                j_resolution, name, rc)\end{verbatim}{\em RETURN VALUE:}
\begin{verbatim}       type(ESMF_Grid) :: ESMF_GridCreateChangeResolution\end{verbatim}{\em ARGUMENTS:}
\begin{verbatim}       type(ESMF_Grid), intent(in) :: grid_in
       integer, intent(in) :: i_resolution
       integer, intent(in) :: j_resolution
       character (len=*), intent(in), optional :: name
       integer, intent(out), optional :: rc\end{verbatim}
{\sf DESCRIPTION:\\ }


       Allocates memory for a new {\tt Grid} object, constructs its
       internals, and creates a {\tt Grid} by either coarsening or refining an
       existing {\tt Grid}.  Return a pointer to the new {\tt Grid}.
  
       The arguments are:
       \begin{description}
       \item[[grid\_in]]
            Source {\tt Grid} to be coarsened or refined.
       \item[[i\_resolution]]
            Integer resolution factor in the i-direction.
       \item[[j\_resolution]]
            Integer resolution factor in the j-direction.
            Note:  The above arguments assume refinement by factor if positive
            and coarsening by absolute value of the factor if negative.  For
            example, i\_resolution=4 indicates the new {\tt Grid} will be four
            times as resolved in the i-direction as the source {\tt Grid},
            whereas j\_resolution=-3 means the new {\tt Grid} will sample every
            third point in the j-direction.
       \item[[name]]
            {\tt Grid} name.
       \item[[rc]]
            Return code; equals {\tt ESMF\_SUCCESS} if there are no errors.
     \end{description}
  
\bigskip{\sf REQUIREMENTS:}
\begin{verbatim} \end{verbatim}
 
%/////////////////////////////////////////////////////////////
 
\mbox{}\hrulefill\ 
 
\subsubsection{ESMF\_GridCreateExchange - Create a new Grid from the intersection of two existing grids}


\bigskip{\sf INTERFACE:}
\begin{verbatim}       function ESMF_GridCreateExchange(grid_in1, grid_in2, name, rc)\end{verbatim}{\em RETURN VALUE:}
\begin{verbatim}       type(ESMF_Grid) :: ESMF_GridCreateExchange\end{verbatim}{\em ARGUMENTS:}
\begin{verbatim}       type(ESMF_Grid), intent(in) :: grid_in1
       type(ESMF_Grid), intent(in) :: grid_in2
       character (len=*), intent(in), optional :: name
       integer, intent(out), optional :: rc\end{verbatim}
{\sf DESCRIPTION:\\ }


       Allocates memory for a new {\tt Grid} object, constructs its
       internals, and creates a new {\tt Grid} from the intersection of two
       existing {\tt Grids}.  Return a pointer to the new {\tt Grid}.
  
       The arguments are:
       \begin{description}
       \item[[grid\_in1]]
            First source {\tt Grid}.
       \item[[grid\_in2]]
            Second source {\tt Grid}.
       \item[[name]]
            New {\tt Grid} name.
       \item[[rc]]
            Return code; equals {\tt ESMF\_SUCCESS} if there are no errors.
     \end{description}
  
\bigskip{\sf REQUIREMENTS:}
\begin{verbatim} \end{verbatim}
 
%/////////////////////////////////////////////////////////////
 
\mbox{}\hrulefill\ 
 
\subsubsection{ESMF\_GridDestroy - Free all resources associated with a Grid }


\bigskip{\sf INTERFACE:}
\begin{verbatim}       subroutine ESMF_GridDestroy(grid, rc)\end{verbatim}{\em ARGUMENTS:}
\begin{verbatim}       type(ESMF_Grid), intent(in) :: grid
       integer, intent(out), optional :: rc\end{verbatim}
{\sf DESCRIPTION:\\ }


       Destroys a {\tt Grid} object previously allocated
       via an {\tt ESMF\_GridCreate routine}.
  
       The arguments are:
       \begin{description}
       \item[grid]
            The class to be destroyed.
       \item[[rc]]
            Return code; equals {\tt ESMF\_SUCCESS} if there are no errors.
       \end{description}
   
%/////////////////////////////////////////////////////////////
 
\mbox{}\hrulefill\ 
 
\subsubsection{ESMF\_GridConstructInternal - Construct the internals of an allocated Grid}


 
\bigskip{\sf INTERFACE:}
\begin{verbatim}       subroutine ESMF_GridConstructInternal(grid, i_max, j_max, &
                                             nDE_i, nDE_j, layout, &
                                             horz_gridtype, vert_gridtype, &
                                             horz_stagger, vert_stagger, &
                                             horz_coord_system, vert_coord_system, &
                                             x_min, x_max, y_min, y_max, name, rc)\end{verbatim}{\em ARGUMENTS:}
\begin{verbatim}       type(ESMF_GridType) :: grid
       integer, intent(in) :: i_max
       integer, intent(in) :: j_max
       integer, intent(in), optional :: nDE_i
       integer, intent(in), optional :: nDE_j
       type (ESMF_Layout), intent(in), optional :: layout
       integer, intent(in), optional :: horz_gridtype
       integer, intent(in), optional :: vert_gridtype
       integer, intent(in), optional :: horz_stagger
       integer, intent(in), optional :: vert_stagger
       integer, intent(in), optional :: horz_coord_system
       integer, intent(in), optional :: vert_coord_system
       real, intent(in), optional :: x_min
       real, intent(in), optional :: x_max
       real, intent(in), optional :: y_min
       real, intent(in), optional :: y_max
       character (len = *), intent(in), optional :: name
       integer, intent(out), optional :: rc\end{verbatim}
{\sf DESCRIPTION:\\ }


       ESMF routine which fills in the contents of an already
       allocated {\tt Grid} object.  May perform additional allocations
       as needed.  Must call the corresponding ESMF\_GridDestruct
       routine to free the additional memory.  Intended for internal
       ESMF use only; end-users use {\tt ESMF\_GridCreate}, which calls
       {\tt ESMF\_GridConstruct}.
  
       The arguments are:
       \begin{description}
       \item[grid]
            Pointer to a {\tt Grid}
       \item[[i\_max]]
            Number of grid increments in the i-direction.
       \item[[j\_max]]
            Number of grid increments in the j-direction.
       \item[[nDE\_i]]
            Number of DE's in 1st dir.
       \item[[nDE\_j]]
            Number of DE's in 2nd dir.
       \item[[layout]]
           Layout of DE's.
       \item[[horz\_gridtype]]
            Integer specifier to denote horizontal gridtype.
       \item[[vert\_gridtype]]
            Integer specifier to denote vertical gridtype.
       \item[[horz\_stagger]]
            Integer specifier to denote horizontal grid stagger.
       \item[[vert\_stagger]]
            Integer specifier to denote vertical grid stagger.
       \item[[horz\_coord\_system]]
            Integer specifier to denote horizontal coordinate system.
       \item[[vert\_coord\_system]]
            Integer specifier to denote vertical coordinate system.
       \item[[x\_min]]
            Minimum physical coordinate in the x-direction.
       \item[[x\_max]]
            Maximum physical coordinate in the x-direction.
       \item[[y\_min]]
            Minimum physical coordinate in the y-direction.
       \item[[y\_max]]
            Maximum physical coordinate in the y-direction.
       \item[[rc]]
            Return code; equals {\tt ESMF\_SUCCESS} if there are no errors.
       \end{description}
  
\bigskip{\sf REQUIREMENTS:}
\begin{verbatim} \end{verbatim}
 
%/////////////////////////////////////////////////////////////
 
\mbox{}\hrulefill\ 
 
\subsubsection{ESMF\_GridConstructNew - Construct the internals of an allocated Grid}


\bigskip{\sf INTERFACE:}
\begin{verbatim}       subroutine ESMF_GridConstructNew(grid, name, rc)\end{verbatim}{\em ARGUMENTS:}
\begin{verbatim}       type(ESMF_GridType) :: grid
       character (len = *), intent(in), optional :: name
       integer, intent(out), optional :: rc\end{verbatim}
{\sf DESCRIPTION:\\ }


       ESMF routine which fills in the contents of an already
       allocated {\tt Grid} object.  May perform additional allocations
       as needed.  Must call the corresponding ESMF\_GridDestruct
       routine to free the additional memory.  Intended for internal
       ESMF use only; end-users use {\tt ESMF\_GridCreate}, which calls
       {\tt ESMF\_GridConstruct}.
  
       The arguments are:
       \begin{description}
       \item[grid]
            Pointer to a {\tt Grid}
       \item[[name]]
            {\tt Grid} name.
       \item[[rc]]
            Return code; equals {\tt ESMF\_SUCCESS} if there are no errors.
       \end{description}
  
\bigskip{\sf REQUIREMENTS:}
\begin{verbatim} \end{verbatim}
 
%/////////////////////////////////////////////////////////////
 
\mbox{}\hrulefill\ 
 
\subsubsection{ESMF\_GridDestruct - Free any Grid memory allocated internally}


\bigskip{\sf INTERFACE:}
\begin{verbatim}       subroutine ESMF_GridDestruct(grid, rc)\end{verbatim}{\em ARGUMENTS:}
\begin{verbatim}       type(ESMF_GridType), intent(in) :: grid
       integer, intent(out), optional :: rc\end{verbatim}
{\sf DESCRIPTION:\\ }


       ESMF routine which deallocates any space allocated by
      {\tt  ESMF\_GridConstruct}, does any additional cleanup before the
       original Grid object is freed.  Intended for internal ESMF
       use only; end-users use {\tt ESMF\_GridDestroy}, which calls
       {\tt ESMF\_GridDestruct}.  
  
       The arguments are:
       \begin{description}
       \item[grid]
            The class to be destructed.
       \item[[rc]]
            Return code; equals {\tt ESMF\_SUCCESS} if there are no errors.
       \end{description}
   
%/////////////////////////////////////////////////////////////
 
\mbox{}\hrulefill\ 
 
\subsubsection{ESMF\_GridAddPhysGrid - Add a PhysGrid to a Grid}


 
\bigskip{\sf INTERFACE:}
\begin{verbatim}       subroutine ESMF_GridAddPhysGrid(grid, i_max, j_max, physgrid_id, &
                                       x_min, x_max, y_min, y_max, &
                                       physgrid_name, rc)\end{verbatim}{\em ARGUMENTS:}
\begin{verbatim}       type(ESMF_GridType) :: grid
       integer, intent(in) :: i_max
       integer, intent(in) :: j_max
       integer, intent(out) :: physgrid_id
       real, intent(in), optional :: x_min
       real, intent(in), optional :: x_max
       real, intent(in), optional :: y_min
       real, intent(in), optional :: y_max
       character (len=*), intent(in), optional :: physgrid_name
       integer, intent(out), optional :: rc\end{verbatim}
{\sf DESCRIPTION:\\ }


       Adds a physgrid to a grid.
  
       The arguments are:
       \begin{description}
       \item[grid]
            Class to be queried.
       \item[[i\_max]]
            Number of grid increments in the i-direction.
       \item[[j\_max]]
            Number of grid increments in the j-direction.
       \item [[physgrid\_id]]
            Integer identifier for {\tt PhysGrid}.
       \item[[x\_min]]
            Minimum physical coordinate in the x-direction.
       \item[[x\_max]]
            Maximum physical coordinate in the x-direction.
       \item[[y\_min]]
            Minimum physical coordinate in the y-direction.
       \item[[y\_max]]
            Maximum physical coordinate in the y-direction.
       \item [[physgrid\_name]]
            {\tt PhysGrid} name.
       \item[[rc]]
            Return code; equals {\tt ESMF\_SUCCESS} if there are no errors.
       \end{description}
   
%/////////////////////////////////////////////////////////////
 
\mbox{}\hrulefill\ 
 
\subsubsection{ESMF\_GridGetConfig - Get configuration information from a Grid}


 
\bigskip{\sf INTERFACE:}
\begin{verbatim}       subroutine ESMF_GridGetConfig(grid, config, rc)\end{verbatim}{\em ARGUMENTS:}
\begin{verbatim}       type(ESMF_Grid), intent(in) :: grid
       integer, intent(out) :: config
       integer, intent(out), optional :: rc\end{verbatim}
{\sf DESCRIPTION:\\ }


       Returns the set of resources the Grid object was configured with.
  
       The arguments are:
       \begin{description}
       \item[grid]
            Class to be queried.
       \item[config]
            Configuration information.
       \item[[rc]]
            Return code; equals {\tt ESMF\_SUCCESS} if there are no errors.
       \end{description}
   
%/////////////////////////////////////////////////////////////
 
\mbox{}\hrulefill\ 
 
\subsubsection{ESMF\_GridSetConfig - Set configuration information for a Grid}


 
\bigskip{\sf INTERFACE:}
\begin{verbatim}       subroutine ESMF_GridSetConfig(grid, config, rc)\end{verbatim}{\em ARGUMENTS:}
\begin{verbatim}       type(ESMF_Grid), intent(in) :: grid
       integer, intent(in) :: config
       integer, intent(out), optional :: rc
 \end{verbatim}
{\sf DESCRIPTION:\\ }


       Configures the Grid object with set of resources given.
  
       The arguments are:
       \begin{description}
       \item[grid]
            Class to be configured.
       \item[config]
            Configuration information.
       \item[[rc]]
            Return code; equals {\tt ESMF\_SUCCESS} if there are no errors.
       \end{description}
   
%/////////////////////////////////////////////////////////////
 
\mbox{}\hrulefill\ 
 
\subsubsection{ESMF\_GridGetValue - Get <Value> for a Grid}


 
\bigskip{\sf INTERFACE:}
\begin{verbatim}       subroutine ESMF_GridGetValue(grid, value, rc)\end{verbatim}{\em ARGUMENTS:}
\begin{verbatim}       type(ESMF_Grid), intent(in) :: grid
       integer, intent(out) :: value
       integer, intent(out), optional :: rc
 \end{verbatim}
{\sf DESCRIPTION:\\ }


       Returns the value of Grid attribute <Value>.
       May be multiple routines, one per attribute.
  
       The arguments are:
       \begin{description}
       \item[grid]
            Class to be queried.
       \item[value]
            Value to be retrieved.
       \item[[rc]]
            Return code; equals {\tt ESMF\_SUCCESS} if there are no errors.
       \end{description}
   
%/////////////////////////////////////////////////////////////
 
\mbox{}\hrulefill\ 
 
\subsubsection{ESMF\_GridSetCoordFromArray - Set the coordinates of a Grid from an existing ESMF array}


 
\bigskip{\sf INTERFACE:}
\begin{verbatim}       subroutine ESMF_GridSetCoordFromArray(Grid, array, id, rc)\end{verbatim}{\em ARGUMENTS:}
\begin{verbatim}       type(ESMF_Grid), intent(in) :: grid
       type(ESMF_Array), intent(in) :: array
       integer, intent(in) :: id
       integer, intent(out), optional :: rc\end{verbatim}
{\sf DESCRIPTION:\\ }


       This version of set assumes the coordinates exist already and are being
       passed in through an {\tt Array}.
  
       The arguments are:
       \begin{description}
       \item[grid]
            Pointer to a {\tt Grid} to be modified.
       \item[array]
            ESMF Array of data.
       \item[[id]]
            Identifier for which set of coordinates are being set:
               1  center\_x
               2  center\_y
               3  corner\_x
               4  corner\_y
               5  face\_x
               6  face\_y 
       \item[[rc]]
            Return code; equals {\tt ESMF\_SUCCESS} if there are no errors.
       \end{description}
   
%/////////////////////////////////////////////////////////////
 
\mbox{}\hrulefill\ 
 
\subsubsection{}


       ESMF_GridGetDE - Get DE information for a DistGrid
 
\bigskip{\sf INTERFACE:}
\begin{verbatim}       subroutine ESMF_GridGetDE(grid, MyDE, MyDEx, MyDEy, &
                                 DE_E, DE_W, DE_N, DE_S, &
                                 DE_NE, DE_NW, DE_SE, DE_SW, &
                                 lsize, gstart, &
                                 n_dir1_start, n_dir1_end, n_dir1_size, &
                                 n_dir2_start, n_dir2_end, n_dir2_size, &
                                 rc)\end{verbatim}{\em ARGUMENTS:}
\begin{verbatim}       type(ESMF_Grid) :: grid
       integer, intent(inout), optional :: MyDE
       integer, intent(inout), optional :: MyDEx
       integer, intent(inout), optional :: MyDEy
       integer, intent(inout), optional :: DE_E
       integer, intent(inout), optional :: DE_W
       integer, intent(inout), optional :: DE_N
       integer, intent(inout), optional :: DE_S
       integer, intent(inout), optional :: DE_NE
       integer, intent(inout), optional :: DE_NW
       integer, intent(inout), optional :: DE_SE
       integer, intent(inout), optional :: DE_SW
       integer, intent(inout), optional :: lsize
       integer, intent(inout), optional :: gstart
       integer, intent(inout), optional :: n_dir1_start
       integer, intent(inout), optional :: n_dir1_end
       integer, intent(inout), optional :: n_dir1_size
       integer, intent(inout), optional :: n_dir2_start
       integer, intent(inout), optional :: n_dir2_end
       integer, intent(inout), optional :: n_dir2_size
       integer, intent(out), optional :: rc
 \end{verbatim}
{\sf DESCRIPTION:\\ }


       Get a DistGrid attribute with the given value.
  
       The arguments are:
       \begin{description}
       \item[grid]
            Class to be queried.
       \item[[MyDE]]
            Identifier for this DE.
       \item[[MyDEx]]
            Identifier for this DE's position in the 1st dir decomposition.
       \item[[MyDEy]]
            Identifier for this DE's position in the 2nd dir decomposition.
       \item[[DE\_E]]
            Identifier for the DE to the east of this DE.
       \item[[DE\_W]]
            Identifier for the DE to the west of this DE.
       \item[[DE\_N]]
            Identifier for the DE to the north of this DE.
       \item[[DE\_S]]
            Identifier for the DE to the south of this DE.
       \item[[DE\_NE]]
            Identifier for the DE to the northeast of this DE.
       \item[[DE\_NW]]
            Identifier for the DE to the northwest of this DE.
       \item[[DE\_SE]]
            Identifier for the DE to the southeast of this DE.
       \item[[DE\_SW]]
            Identifier for the DE to the southwest of this DE.
       \item[[lsize]]
            Local (on this DE) number of cells.
       \item[[gstart]]
            Global index of starting count.
       \item[[n\_dir1\_start]]
            Starting index of this DE in 1st dir decomposition.
       \item[[n\_dir1\_end]]
            Ending index of this DE in 1st dir decomposition.
       \item[[n\_dir1\_size]]
            Size of the 1st dir decomposition on this DE.
       \item[[n\_dir2\_start]]
            Starting index of this DE in 2nd dir decomposition.
       \item[[n\_dir2\_end]]
            Ending index of this DE in 2nd dir decomposition.
       \item[[n\_dir2\_size]]
            Size of the 2nd dir decomposition on this DE.
       \item[[rc]]
            Return code; equals {\tt ESMF\_SUCCESS} if there are no errors.
       \end{description}
   
%/////////////////////////////////////////////////////////////
 
\mbox{}\hrulefill\ 
 
\subsubsection{ESMF\_GridSetCoordFromBuffer - Set the coordinates of a Grid from an existing data buffer}


 
\bigskip{\sf INTERFACE:}
\begin{verbatim}       subroutine ESMF_GridSetCoordFromBuffer(Grid, buffer, id, rc)\end{verbatim}{\em ARGUMENTS:}
\begin{verbatim}       type(ESMF_Grid), intent(in) :: grid
       real, dimension (:), pointer :: buffer
       integer, intent(in) :: id
       integer, intent(out), optional :: rc\end{verbatim}
{\sf DESCRIPTION:\\ }


       This version of set assumes the coordinates exist already and are being
       passed in as a raw data buffer.
  
       The arguments are:
       \begin{description}
       \item[grid]
            Pointer to a {\tt Grid} to be modified.
       \item[buffer]
            Raw data buffer.
       \item[[id]]
            Identifier for which set of coordinates are being set:
               1  center\_x
               2  center\_y
               3  corner\_x
               4  corner\_y
               5  face\_x
               6  face\_y 
       \item[[rc]]
            Return code; equals {\tt ESMF\_SUCCESS} if there are no errors.
       \end{description}
   
%/////////////////////////////////////////////////////////////
 
\mbox{}\hrulefill\ 
 
\subsubsection{ESMF\_GridSetCoordCompute - Compute coordinates for a Grid}


 
\bigskip{\sf INTERFACE:}
\begin{verbatim}       subroutine ESMF_GridSetCoordCompute(grid, physgrid_id, rc)\end{verbatim}{\em ARGUMENTS:}
\begin{verbatim}       type(ESMF_GridType) :: grid
       integer, intent(in) :: physgrid_id
       integer, intent(out), optional :: rc\end{verbatim}
{\sf DESCRIPTION:\\ }


       This version of set internally computes coordinates for a Grid via a
       prescribed method.
  
       The arguments are:
       \begin{description}
       \item[grid]
            Pointer to a {\tt Grid} to be modified.
       \item[[physgrid\_id]]
            Identifier of the {\tt PhysGrid} to be modified.
       \item[[rc]]
            Return code; equals {\tt ESMF\_SUCCESS} if there are no errors.
       \end{description}
  
  TODO: figure out the argument list necessary to completely describe the
        internal calculation of the coordinates of a simple grid. 
%/////////////////////////////////////////////////////////////
 
\mbox{}\hrulefill\ 
 
\subsubsection{ESMF\_GridSetCoordCopy - Copies coordinates from one grid to another}


 
\bigskip{\sf INTERFACE:}
\begin{verbatim}       subroutine ESMF_GridSetCoordCopy(Grid, Grid_in, id, rc)\end{verbatim}{\em ARGUMENTS:}
\begin{verbatim}       type(ESMF_Grid), intent(in) :: grid
       type(ESMF_Grid), intent(in) :: grid_in
       integer, intent(in) :: id
       integer, intent(out), optional :: rc\end{verbatim}
{\sf DESCRIPTION:\\ }


       This version of set copies the coordinates of a Grid from another Grid.
  
       The arguments are:
       \begin{description}
       \item[grid]
            Pointer to a {\tt Grid} to be modified.
       \item[grid\_in]
            Pointer to a {\tt Grid} whose coordinates are to be copied.
       \item[[id]]
            Identifier for which set of coordinates are being set:
               1  center\_x
               2  center\_y
               3  corner\_x
               4  corner\_y
               5  face\_x
               6  face\_y 
       \item[[rc]]
            Return code; equals {\tt ESMF\_SUCCESS} if there are no errors.
       \end{description}
   
%/////////////////////////////////////////////////////////////
 
\mbox{}\hrulefill\ 
 
\subsubsection{ESMF\_GridGetInfo - Gets a variety of information about the grid}


 
\bigskip{\sf INTERFACE:}
\begin{verbatim}       subroutine ESMF_GridGetInfo(grid, horz_gridtype, vert_gridtype, &
                                   horz_stagger, vert_stagger, &
                                   horz_coord_system, vert_coord_system, &
                                   coord_order, rc)\end{verbatim}{\em ARGUMENTS:}
\begin{verbatim}       type(ESMF_GridType) :: grid
       integer, intent(inout), optional :: horz_gridtype
       integer, intent(inout), optional :: vert_gridtype
       integer, intent(inout), optional :: horz_stagger
       integer, intent(inout), optional :: vert_stagger
       integer, intent(inout), optional :: horz_coord_system
       integer, intent(inout), optional :: vert_coord_system
       integer, intent(inout), optional :: coord_order
       integer, intent(out), optional :: rc\end{verbatim}
{\sf DESCRIPTION:\\ }


       This version sets a variety of information about a grid, depending
       on a list of optional arguments.
  
       The arguments are:
       \begin{description}
       \item[grid]
            Pointer to a {\tt Grid} to be modified.
       \item[[horz\_gridtype]]
            Integer specifier to denote horizontal gridtype
       \item[[vert\_gridtype]]
            Integer specifier to denote vertical gridtype
       \item[[horz\_stagger]]
            Integer specifier to denote horizontal grid stagger
       \item[[vert\_stagger]]
            Integer specifier to denote vertical grid stagger
       \item[[horz\_coord\_system]]
            Integer specifier to denote horizontal coordinate system
       \item[[vert\_coord\_system]]
            Integer specifier to denote vertical coordinate system
       \item[[coord\_order]]
            Integer specifier to denote coordinate ordering
       \item[[rc]]
            Return code; equals {\tt ESMF\_SUCCESS} if there are no errors.
       \end{description}
   
%/////////////////////////////////////////////////////////////
 
\mbox{}\hrulefill\ 
 
\subsubsection{ESMF\_GridSetInfo - Sets a variety of information about the grid}


 
\bigskip{\sf INTERFACE:}
\begin{verbatim}       subroutine ESMF_GridSetInfo(grid, horz_gridtype, vert_gridtype, &
                                   horz_stagger, vert_stagger, &
                                   horz_coord_system, vert_coord_system, &
                                   coord_order, rc)\end{verbatim}{\em ARGUMENTS:}
\begin{verbatim}       type(ESMF_GridType) :: grid
       integer, intent(in), optional :: horz_gridtype
       integer, intent(in), optional :: vert_gridtype
       integer, intent(in), optional :: horz_stagger
       integer, intent(in), optional :: vert_stagger
       integer, intent(in), optional :: horz_coord_system
       integer, intent(in), optional :: vert_coord_system
       integer, intent(in), optional :: coord_order
       integer, intent(out), optional :: rc\end{verbatim}
{\sf DESCRIPTION:\\ }


       This version sets a variety of information about a grid, depending
       on a list of optional arguments.
  
       The arguments are:
       \begin{description}
       \item[grid]
            Pointer to a {\tt Grid} to be modified.
       \item[[horz\_gridtype]]
            Integer specifier to denote horizontal gridtype
       \item[[vert\_gridtype]]
            Integer specifier to denote vertical gridtype
       \item[[horz\_stagger]]
            Integer specifier to denote horizontal grid stagger
       \item[[vert\_stagger]]
            Integer specifier to denote vertical grid stagger
       \item[[horz\_coord\_system]]
            Integer specifier to denote horizontal coordinate system
       \item[[vert\_coord\_system]]
            Integer specifier to denote vertical coordinate system
       \item[[coord\_order]]
            Integer specifier to denote coordinate ordering
       \item[[rc]]
            Return code; equals {\tt ESMF\_SUCCESS} if there are no errors.
       \end{description}
   
%/////////////////////////////////////////////////////////////
 
\mbox{}\hrulefill\ 
 
\subsubsection{ESMF\_GridSetLMaskFromArray - Set a logical mask in a Grid from an existing ESMF array}


 
\bigskip{\sf INTERFACE:}
\begin{verbatim}       subroutine ESMF_GridSetLMaskFromArray(Grid, array, name, rc)\end{verbatim}{\em ARGUMENTS:}
\begin{verbatim}       type(ESMF_Grid), intent(in) :: grid
       type(ESMF_Array), intent(in) :: array
       character (len=*), intent(in), optional :: name
       integer, intent(out), optional :: rc\end{verbatim}
{\sf DESCRIPTION:\\ }


       This version of set assumes the logical mask data exists already and is
       being passed in through an {\tt Array}.
  
       The arguments are:
       \begin{description}
       \item[grid]
            Pointer to a {\tt Grid} to be modified.
       \item[array]
            ESMF Array of data.
       \item [[name]]
             {\tt LMask} name.
       \item[[rc]]
            Return code; equals {\tt ESMF\_SUCCESS} if there are no errors.
       \end{description}
   
%/////////////////////////////////////////////////////////////
 
\mbox{}\hrulefill\ 
 
\subsubsection{ESMF\_GridSetLMaskFromBuffer - Set a logical mask in a Grid from an existing data buffer}


 
\bigskip{\sf INTERFACE:}
\begin{verbatim}       subroutine ESMF_GridSetLMaskFromBuffer(Grid, buffer, name, rc)\end{verbatim}{\em ARGUMENTS:}
\begin{verbatim}       type(ESMF_Grid), intent(in) :: grid
       real, dimension (:), pointer :: buffer
       character (len=*), intent(in), optional :: name
       integer, intent(out), optional :: rc\end{verbatim}
{\sf DESCRIPTION:\\ }


       This version of set assumes the logical mask data exists already and is
       being passed in as a raw data buffer.
  
       The arguments are:
       \begin{description}
       \item[grid]
            Pointer to a {\tt Grid} to be modified.
       \item[buffer]
            Raw data buffer.
       \item [[name]]
             {\tt LMask} name.
       \item[[rc]]
            Return code; equals {\tt ESMF\_SUCCESS} if there are no errors.
       \end{description}
   
%/////////////////////////////////////////////////////////////
 
\mbox{}\hrulefill\ 
 
\subsubsection{ESMF\_GridSetLMaskFromMMask - Set a logical mask in a Grid from an existing multiplicative mask}


 
\bigskip{\sf INTERFACE:}
\begin{verbatim}       subroutine ESMF_GridSetLMaskFromMMask(Grid, mmask, name, rc)\end{verbatim}{\em ARGUMENTS:}
\begin{verbatim}       type(ESMF_Grid), intent(in) :: grid
       integer, intent(in) :: mmask        ! TODO: name?
       character (len=*), intent(in), optional :: name
       integer, intent(out), optional :: rc\end{verbatim}
{\sf DESCRIPTION:\\ }


       This version of set assumes the logical mask data will be
       created from an existing multiplicative mask.
  
       The arguments are:
       \begin{description}
       \item[grid]
            Pointer to a {\tt Grid} to be modified.
       \item[[mmask]]
            Multiplicative mask identifier.
       \item [[name]]
             {\tt LMask} name.
       \item[[rc]]
            Return code; equals {\tt ESMF\_SUCCESS} if there are no errors.
       \end{description}
   
%/////////////////////////////////////////////////////////////
 
\mbox{}\hrulefill\ 
 
\subsubsection{ESMF\_GridSetLMaskCopy - Copies a logical mask from one grid to another.}


 
\bigskip{\sf INTERFACE:}
\begin{verbatim}       subroutine ESMF_GridSetLMaskCopy(Grid, Grid_in, name, name_in, rc)\end{verbatim}{\em ARGUMENTS:}
\begin{verbatim}       type(ESMF_Grid), intent(in) :: grid
       type(ESMF_Grid), intent(in) :: grid_in
       character (len=*), intent(in), optional :: name
       character (len=*), intent(in), optional :: name_in
       integer, intent(out), optional :: rc\end{verbatim}
{\sf DESCRIPTION:\\ }


       This version of set copies a logical mask for a Grid from another Grid.
  
       The arguments are:
       \begin{description}
       \item[grid]
            Pointer to a {\tt Grid} to be modified.
       \item[grid\_in]
            Pointer to a {\tt Grid} whose coordinates are to be copied.
       \item [[name]]
             {\tt LMask} name to be set.
       \item [[name\_in]]
             {\tt LMask} name to be copied.
       \item[[rc]]
            Return code; equals {\tt ESMF\_SUCCESS} if there are no errors.
       \end{description}
   
%/////////////////////////////////////////////////////////////
 
\mbox{}\hrulefill\ 
 
\subsubsection{ESMF\_GridSetMMaskFromArray - Set a multiplicative mask in a Grid from an existing ESMF array}


 
\bigskip{\sf INTERFACE:}
\begin{verbatim}       subroutine ESMF_GridSetMMaskFromArray(Grid, array, name, rc)\end{verbatim}{\em ARGUMENTS:}
\begin{verbatim}       type(ESMF_Grid), intent(in) :: grid
       type(ESMF_Array), intent(in) :: array
       character (len=*), intent(in), optional :: name
       integer, intent(out), optional :: rc\end{verbatim}
{\sf DESCRIPTION:\\ }


       This version of set assumes the multiplicative mask data exists already
       and is being passed in through an {\tt Array}.
  
       The arguments are:
       \begin{description}
       \item[grid]
            Pointer to a {\tt Grid} to be modified.
       \item[array]
            ESMF Array of data.
       \item [[name]]
             {\tt MMask} name.
       \item[[rc]]
            Return code; equals {\tt ESMF\_SUCCESS} if there are no errors.
       \end{description}
   
%/////////////////////////////////////////////////////////////
 
\mbox{}\hrulefill\ 
 
\subsubsection{ESMF\_GridSetMMaskFromBuffer - Set a multiplicative mask in a Grid from an existing data buffer}


 
\bigskip{\sf INTERFACE:}
\begin{verbatim}       subroutine ESMF_GridSetMMaskFromBuffer(Grid, buffer, name, rc)\end{verbatim}{\em ARGUMENTS:}
\begin{verbatim}       type(ESMF_Grid), intent(in) :: grid
       real, dimension (:), pointer :: buffer
       character (len=*), intent(in), optional :: name
       integer, intent(out), optional :: rc\end{verbatim}
{\sf DESCRIPTION:\\ }


       This version of set assumes the multiplicative mask data exists already
       and is being passed in as a raw data buffer.
  
       The arguments are:
       \begin{description}
       \item[grid]
            Pointer to a {\tt Grid} to be modified.
       \item[buffer]
            Raw data buffer.
       \item [[name]]
             {\tt MMask} name.
       \item[[rc]]
            Return code; equals {\tt ESMF\_SUCCESS} if there are no errors.
       \end{description}
   
%/////////////////////////////////////////////////////////////
 
\mbox{}\hrulefill\ 
 
\subsubsection{ESMF\_GridSetMMaskFromLMask - Set a multiplicative mask in a Grid from an existing logical mask}


 
\bigskip{\sf INTERFACE:}
\begin{verbatim}       subroutine ESMF_GridSetMMaskFromLMask(Grid, lmask, name, rc)\end{verbatim}{\em ARGUMENTS:}
\begin{verbatim}       type(ESMF_Grid), intent(in) :: grid
       integer, intent(in) :: lmask
       character (len=*), intent(in), optional :: name
       integer, intent(out), optional :: rc\end{verbatim}
{\sf DESCRIPTION:\\ }


       This version of set assumes the multiplicative mask data will be
       created from an existing logical mask.
  
       The arguments are:
       \begin{description}
       \item[grid]
            Pointer to a {\tt Grid} to be modified.
       \item[lmask]
            Logical mask identifier.
       \item [[name]]
             {\tt MMask} name.
       \item[[rc]]
            Return code; equals {\tt ESMF\_SUCCESS} if there are no errors.
       \end{description}
   
%/////////////////////////////////////////////////////////////
 
\mbox{}\hrulefill\ 
 
\subsubsection{ESMF\_GridSetMMaskCopy - Copies a multiplicative mask from one grid to another.}


 
\bigskip{\sf INTERFACE:}
\begin{verbatim}       subroutine ESMF_GridSetMMaskCopy(Grid, Grid_in, name, name_in, rc)\end{verbatim}{\em ARGUMENTS:}
\begin{verbatim}       type(ESMF_Grid), intent(in) :: grid
       type(ESMF_Grid), intent(in) :: grid_in
       character (len=*), intent(in), optional :: name
       character (len=*), intent(in), optional :: name_in
       integer, intent(out), optional :: rc\end{verbatim}
{\sf DESCRIPTION:\\ }


       This version of set copies a multiplicative mask for a Grid from another
       Grid.
  
       The arguments are:
       \begin{description}
       \item[grid]
            Pointer to a {\tt Grid} to be modified.
       \item[grid\_in]
            Pointer to a {\tt Grid} whose coordinates are to be copied.
       \item [[name]]
             {\tt MMask} name to be set.
       \item [[name\_in]]
             {\tt MMask} name to be copied.
       \item[[rc]]
            Return code; equals {\tt ESMF\_SUCCESS} if there are no errors.
       \end{description}
   
%/////////////////////////////////////////////////////////////
 
\mbox{}\hrulefill\ 
 
\subsubsection{ESMF\_GridSetMetricFromArray - Set a metric for a Grid from an existing ESMF array}


 
\bigskip{\sf INTERFACE:}
\begin{verbatim}       subroutine ESMF_GridSetMetricFromArray(Grid, array, name, rc)\end{verbatim}{\em ARGUMENTS:}
\begin{verbatim}       type(ESMF_Grid), intent(in) :: grid
       type(ESMF_Array), intent(in) :: array
       character (len=*), intent(in), optional :: name
       integer, intent(out), optional :: rc\end{verbatim}
{\sf DESCRIPTION:\\ }


       This version of set assumes the metric data exists already and is being
       passed in through an {\tt Array}.
  
       The arguments are:
       \begin{description}
       \item[grid]
            Pointer to a {\tt Grid} to be modified.
       \item[array]
            ESMF Array of data.
       \item [[name]]
             {\tt Metric} name.
       \item[[rc]]
            Return code; equals {\tt ESMF\_SUCCESS} if there are no errors.
       \end{description}
   
%/////////////////////////////////////////////////////////////
 
\mbox{}\hrulefill\ 
 
\subsubsection{ESMF\_GridSetMetricFromBuffer - Set a metric for a Grid from an existing data buffer}


 
\bigskip{\sf INTERFACE:}
\begin{verbatim}       subroutine ESMF_GridSetMetricFromBuffer(Grid, buffer, name, rc)\end{verbatim}{\em ARGUMENTS:}
\begin{verbatim}       type(ESMF_Grid), intent(in) :: grid
       real, dimension (:), pointer :: buffer
       character (len=*), intent(in), optional :: name
       integer, intent(out), optional :: rc\end{verbatim}
{\sf DESCRIPTION:\\ }


       This version of set assumes the metric data exists already and is being
       passed in as a raw data buffer.
  
       The arguments are:
       \begin{description}
       \item[grid]
            Pointer to a {\tt Grid} to be modified.
       \item[buffer]
            Raw data buffer.
       \item [[name]]
             {\tt Metric} name.
       \item[[rc]]
            Return code; equals {\tt ESMF\_SUCCESS} if there are no errors.
       \end{description}
   
%/////////////////////////////////////////////////////////////
 
\mbox{}\hrulefill\ 
 
\subsubsection{ESMF\_GridSetMetricCompute - Compute a metric for a Grid}


 
\bigskip{\sf INTERFACE:}
\begin{verbatim}       subroutine ESMF_GridSetMetricCompute(Grid, name, id, rc)\end{verbatim}{\em ARGUMENTS:}
\begin{verbatim}       type(ESMF_Grid), intent(in) :: grid
       integer, intent(in) :: id
       character (len=*), intent(in), optional :: name
       integer, intent(out), optional :: rc\end{verbatim}
{\sf DESCRIPTION:\\ }


       This version of set internally computes a metric for a Grid via a
       prescribed method.
  
       The arguments are:
       \begin{description}
       \item[grid]
            Pointer to a {\tt Grid} to be modified.
       \item[[id]]
            Identifier for predescribed metrics.  TODO: make list
       \item [[name]]
             {\tt Metric} name.
       \item[[rc]]
            Return code; equals {\tt ESMF\_SUCCESS} if there are no errors.
       \end{description}
   
%/////////////////////////////////////////////////////////////
 
\mbox{}\hrulefill\ 
 
\subsubsection{ESMF\_GridSetMetricCopy - Copies a metric from one grid to another}


 
\bigskip{\sf INTERFACE:}
\begin{verbatim}       subroutine ESMF_GridSetMetricCopy(Grid, name, Grid_in, name_in, rc)\end{verbatim}{\em ARGUMENTS:}
\begin{verbatim}       type(ESMF_Grid), intent(in) :: grid
       character (len=*), intent(in) :: name  ! TODO: optional?
       type(ESMF_Grid), intent(in) :: grid_in
       character (len=*), intent(in) :: name_in  ! TODO: optional?
       integer, intent(out), optional :: rc\end{verbatim}
{\sf DESCRIPTION:\\ }


       This version of set copies a metric for a Grid from another Grid.
  
       The arguments are:
       \begin{description}
       \item[grid]
            Pointer to a {\tt Grid} to be modified.
       \item [[name]]
             {\tt Metric} name to be set.
       \item[grid\_in]
            Pointer to a {\tt Grid} whose coordinates are to be copied.
       \item [[name\_in]]
             {\tt Metric} name to be copied.
       \item[[rc]]
            Return code; equals {\tt ESMF\_SUCCESS} if there are no errors.
       \end{description}
   
%/////////////////////////////////////////////////////////////
 
\mbox{}\hrulefill\ 
 
\subsubsection{ESMF\_GridSetRegionIDFromArray - Set a region identifier in a Grid from an existing ESMF array}


 
\bigskip{\sf INTERFACE:}
\begin{verbatim}       subroutine ESMF_GridSetRegionIDFromArray(Grid, array, name, rc)\end{verbatim}{\em ARGUMENTS:}
\begin{verbatim}       type(ESMF_Grid), intent(in) :: grid
       type(ESMF_Array), intent(in) :: array
       character (len=*), intent(in) :: name  ! TODO: optional?
       integer, intent(out), optional :: rc\end{verbatim}
{\sf DESCRIPTION:\\ }


       This version of set assumes the region identifier data exists already
       and is being passed in through an {\tt Array}.
  
       The arguments are:
       \begin{description}
       \item[grid]
            Pointer to a {\tt Grid} to be modified.
       \item[array]
            ESMF Array of data.
       \item [[name]]
             {\tt RegionID} name.
       \item[[rc]]
            Return code; equals {\tt ESMF\_SUCCESS} if there are no errors.
       \end{description}
   
%/////////////////////////////////////////////////////////////
 
\mbox{}\hrulefill\ 
 
\subsubsection{ESMF\_GridSetRegionIDFromBuffer - Set a region identifier in a Grid from an existing data buffer}


 
\bigskip{\sf INTERFACE:}
\begin{verbatim}       subroutine ESMF_GridSetRegionIDFromBuffer(Grid, buffer, name, rc)\end{verbatim}{\em ARGUMENTS:}
\begin{verbatim}       type(ESMF_Grid), intent(in) :: grid
       real, dimension (:), pointer :: buffer
       character (len=*), intent(in) :: name  ! TODO: optional?
       integer, intent(out), optional :: rc\end{verbatim}
{\sf DESCRIPTION:\\ }


       This version of set assumes the multiplicative mask data exists already
       and is being passed in as a raw data buffer.
  
       The arguments are:
       \begin{description}
       \item[grid]
            Pointer to a {\tt Grid} to be modified.
       \item[buffer]
            Raw data buffer.
       \item [[name]]
             {\tt RegionID} name.
       \item[[rc]]
            Return code; equals {\tt ESMF\_SUCCESS} if there are no errors.
       \end{description}
   
%/////////////////////////////////////////////////////////////
 
\mbox{}\hrulefill\ 
 
\subsubsection{ESMF\_GridSetRegionIDCopy - Copies a region identifier from one grid to another}


 
\bigskip{\sf INTERFACE:}
\begin{verbatim}       subroutine ESMF_GridSetRegionIDCopy(Grid, name, Grid_in, name_in, rc)\end{verbatim}{\em ARGUMENTS:}
\begin{verbatim}       type(ESMF_Grid), intent(in) :: grid
       character (len=*), intent(in) :: name  ! TODO: optional?
       type(ESMF_Grid), intent(in) :: grid_in
       character (len=*), intent(in) :: name_in  ! TODO: optional?
       integer, intent(out), optional :: rc\end{verbatim}
{\sf DESCRIPTION:\\ }


       This version of set copies a region identifier for a Grid from another
       Grid.
  
       The arguments are:
       \begin{description}
       \item[grid]
            Pointer to a {\tt Grid} to be modified.
       \item [[name]]
             {\tt RegionID} name to be set.
       \item[grid\_in]
            Pointer to a {\tt Grid} whose coordinates are to be copied.
       \item [[name\_in]]
             {\tt RegionID} name to be copied.
       \item[[rc]]
            Return code; equals {\tt ESMF\_SUCCESS} if there are no errors.
       \end{description}
   
%/////////////////////////////////////////////////////////////
 
\mbox{}\hrulefill\ 
 
\subsubsection{ESMF\_GridValidate - Check internal consistency of a Grid}


 
\bigskip{\sf INTERFACE:}
\begin{verbatim}       subroutine ESMF_GridValidate(grid, opt, rc)\end{verbatim}{\em ARGUMENTS:}
\begin{verbatim}       type(ESMF_Grid), intent(in) :: grid
       character (len=*), intent(in), optional :: opt
       integer, intent(out), optional :: rc\end{verbatim}
{\sf DESCRIPTION:\\ }


       Validates that a Grid is internally consistent.
  
       The arguments are:
       \begin{description}
       \item[grid]
            Class to be queried.
       \item[[opt]]
            Validation options.
       \item[[rc]]
            Return code; equals {\tt ESMF\_SUCCESS} if there are no errors.
       \end{description}
   
%/////////////////////////////////////////////////////////////
 
\mbox{}\hrulefill\ 
 
\subsubsection{ESMF\_GridPrint - Print the contents of a Grid}


 
\bigskip{\sf INTERFACE:}
\begin{verbatim}       subroutine ESMF_GridPrint(grid, opt, rc)\end{verbatim}{\em ARGUMENTS:}
\begin{verbatim}       type(ESMF_Grid), intent(in) :: grid
       character (len=*), intent(in) :: opt
       integer, intent(out), optional :: rc\end{verbatim}
{\sf DESCRIPTION:\\ }


        Print information about a Grid.
  
       The arguments are:
       \begin{description}
       \item[grid]
            Class to be queried.
       \item[[opt]]
            Print options that control the type of information and level of
            detail.
       \item[[rc]]
            Return code; equals {\tt ESMF\_SUCCESS} if there are no errors.
       \end{description}
  
%...............................................................
