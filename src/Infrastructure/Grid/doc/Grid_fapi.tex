%                **** IMPORTANT NOTICE *****
% This LaTeX file has been automatically produced by ProTeX v. 1.1
% Any changes made to this file will likely be lost next time
% this file is regenerated from its source. Send questions 
% to Arlindo da Silva, dasilva@gsfc.nasa.gov
 
\parskip        0pt
\parindent      0pt
\baselineskip  11pt
 
%--------------------- SHORT-HAND MACROS ----------------------
\def\bv{\begin{verbatim}}
\def\ev{\end{verbatim}}
\def\be{\begin{equation}}
\def\ee{\end{equation}}
\def\bea{\begin{eqnarray}}
\def\eea{\end{eqnarray}}
\def\bi{\begin{itemize}}
\def\ei{\end{itemize}}
\def\bn{\begin{enumerate}}
\def\en{\end{enumerate}}
\def\bd{\begin{description}}
\def\ed{\end{description}}
\def\({\left (}
\def\){\right )}
\def\[{\left [}
\def\]{\right ]}
\def\<{\left  \langle}
\def\>{\right \rangle}
\def\cI{{\cal I}}
\def\diag{\mathop{\rm diag}}
\def\tr{\mathop{\rm tr}}
%-------------------------------------------------------------

\markboth{Left}{Source File: ESMF\_Grid.F90,  Date: Tue Dec  3 16:36:07 MST 2002
}

 
%/////////////////////////////////////////////////////////////
\subsection{Fortran:  Module Interface ESMF\_GridMod - One line general statement about this class (Source File: ESMF\_Grid.F90)}


  
  
   The code in this file implements the {\tt Class> class ...
  
   < Insert a paragraph or two explaining the function of this class. >
  
  ------------------------------------------------------------------------------
\bigskip{\em USES:}
\begin{verbatim}       use ESMF_ArrayMod       ! ESMF array class
       use ESMF_BaseMod        ! ESMF base class
       use ESMF_DistGridMod    ! ESMF distributed grid class
       use ESMF_IOMod          ! ESMF I/O class
       use ESMF_PhysGridMod    ! ESMF physical grid class
       implicit none
 
  ------------------------------------------------------------------------------\end{verbatim}{\sf PRIVATE TYPES:}
\begin{verbatim}       private
  ------------------------------------------------------------------------------
       ! ESMF_GridConfig
       ! Description of ESMF_GridConfig
 
       type ESMF_GridConfig
       sequence
       private
         integer :: dummy
         < insert other class members here >
       end type
 
  ------------------------------------------------------------------------------
       !  ESMF_GridType
       ! Definition for the Grid class.  A Grid
       ! is passed back to the user at Grid creation.
 
       type ESMF_GridType
       sequence
       private
 
         type (ESMF_Base) :: base                     ! base class object
         type (ESMF_Status) :: gridstatus             ! uninitialized, init ok,
                                                      ! etc
         integer :: gridtype                          ! enum for type of grid
         integer :: stagger                           ! enum for grid staggering
         integer :: coord_system                      ! enum for physical
                                                      ! coordinate system
         integer :: coord_order                       ! enum for mapping of xyz 
                                                      ! to ijk
         integer :: num_subgrids                      ! number of grid descriptors
                                                      ! necessary to support
                                                      ! staggering
         real :: global_min_x                         ! global extents
         real :: global_max_x                         ! global extents
         real :: global_min_y                         ! global extents
         real :: global_max_y                         ! global extents
         type (ESMF_PhysGridSpec), dimension(:), pointer :: subgrid  !
         type (ESMF_PhysGrid), pointer :: physgrid  !
  jw     type (ESMF_VertGrid), pointer :: vertgrid    !
         type (ESMF_DistGrid), pointer :: distgrid    !
 
       end type
 
  ------------------------------------------------------------------------------
       !  ESMF_Grid
       ! The Grid data structure that is passed between languages.
 
       type ESMF_Grid
       sequence
       private
         type (ESMF_GridType), pointer :: ptr     ! pointer to a grid type
       end type
 
  ------------------------------------------------------------------------------\end{verbatim}{\sf PUBLIC TYPES:}
\begin{verbatim}       public ESMF_GridConfig
       public ESMF_Grid
  ------------------------------------------------------------------------------\end{verbatim}{\sf PUBLIC MEMBER FUNCTIONS:}
\begin{verbatim}    Pick one or the other of the init/create sections depending on
    whether this is a deep class (the class/derived type has pointers to
    other memory which must be allocated/deallocated) or a shallow class
    (the class/derived type is self-contained) and needs no destroy methods
    other than deleting the memory for the object/derived type itself.
 
   the following routines apply to deep classes only
     public ESMF_GridCreate                 ! interface only, deep class
     public ESMF_GridDestroy                ! interface only, deep class
 
     public ESMF_GridGetConfig
     public ESMF_GridSetConfig
     public ESMF_GridGetValue               ! Get<Value>
     public ESMF_GridSetCoordinate 
     public ESMF_GridSetLMask     
     public ESMF_GridSetMMask    
     public ESMF_GridSetMetric  
     public ESMF_GridSetRegionID
  
     public ESMF_GridValidate
     public ESMF_GridPrint
  
   < list the rest of the public interfaces here >\end{verbatim}
 
%/////////////////////////////////////////////////////////////
 
\mbox{}\hrulefill\ 
 

\bigskip{\sf INTERFACE:}
\begin{verbatim}       interface ESMF_GridCreate 
 \end{verbatim}{\sf PRIVATE MEMBER FUNCTIONS:}
\begin{verbatim}          module procedure ESMF_GridCreateEmpty
          module procedure ESMF_GridCreateInternal
          module procedure ESMF_GridCreateRead
          module procedure ESMF_GridCreateCopy
          module procedure ESMF_GridCreateCutout
          module procedure ESMF_GridCreateChangeResolution
          module procedure ESMF_GridCreateExchange
 \end{verbatim}
{\sf DESCRIPTION:\\ }


       This interface provides a single entry point for Grid create
       methods.
   
%/////////////////////////////////////////////////////////////
 
\mbox{}\hrulefill\ 
 

\bigskip{\sf INTERFACE:}
\begin{verbatim}       interface ESMF_GridConstruct
 \end{verbatim}{\sf PRIVATE MEMBER FUNCTIONS:}
\begin{verbatim}          module procedure ESMF_GridConstructNew
 \end{verbatim}
{\sf DESCRIPTION:\\ }


       This interface provides a single entry point for methods that construct a
       complete {\tt Grid}.
   
%/////////////////////////////////////////////////////////////
 
\mbox{}\hrulefill\ 
 

\bigskip{\sf INTERFACE:}
\begin{verbatim}       interface ESMF_GridSetCoordinate
 \end{verbatim}{\sf PRIVATE MEMBER FUNCTIONS:}
\begin{verbatim}          module procedure ESMF_GridSetCoordinateFromArray
          module procedure ESMF_GridSetCoordinateFromBuffer
          module procedure ESMF_GridSetCoordinateCompute
          module procedure ESMF_GridSetCoordinateCopy
 \end{verbatim}
{\sf DESCRIPTION:\\ }


       This interface provides a single entry point for methods that set
       coordinates as part of a {\tt Grid}.
   
%/////////////////////////////////////////////////////////////
 
\mbox{}\hrulefill\ 
 

\bigskip{\sf INTERFACE:}
\begin{verbatim}       interface ESMF_GridSetLMask
 \end{verbatim}{\sf PRIVATE MEMBER FUNCTIONS:}
\begin{verbatim}          module procedure ESMF_GridSetLMaskFromArray
          module procedure ESMF_GridSetLMaskFromBuffer
          module procedure ESMF_GridSetLMaskCompute
          module procedure ESMF_GridSetLMaskCopy
 \end{verbatim}
{\sf DESCRIPTION:\\ }


       This interface provides a single entry point for methods that set
       logical masks as part of a {\tt Grid}.
   
%/////////////////////////////////////////////////////////////
 
\mbox{}\hrulefill\ 
 

\bigskip{\sf INTERFACE:}
\begin{verbatim}       interface ESMF_GridSetMMask
 \end{verbatim}{\sf PRIVATE MEMBER FUNCTIONS:}
\begin{verbatim}          module procedure ESMF_GridSetMMaskFromArray
          module procedure ESMF_GridSetMMaskFromBuffer
          module procedure ESMF_GridSetMMaskCompute
          module procedure ESMF_GridSetMMaskCopy
 \end{verbatim}
{\sf DESCRIPTION:\\ }


       This interface provides a single entry point for methods that set
       multiplicative masks as part of a {\tt Grid}.
   
%/////////////////////////////////////////////////////////////
 
\mbox{}\hrulefill\ 
 

\bigskip{\sf INTERFACE:}
\begin{verbatim}       interface ESMF_GridSetMetric
 \end{verbatim}{\sf PRIVATE MEMBER FUNCTIONS:}
\begin{verbatim}          module procedure ESMF_GridSetMetricFromArray
          module procedure ESMF_GridSetMetricFromBuffer
          module procedure ESMF_GridSetMetricCompute
          module procedure ESMF_GridSetMetricCopy
 \end{verbatim}
{\sf DESCRIPTION:\\ }


       This interface provides a single entry point for methods that set
       metrics as part of a {\tt Grid}.
   
%/////////////////////////////////////////////////////////////
 
\mbox{}\hrulefill\ 
 

\bigskip{\sf INTERFACE:}
\begin{verbatim}       interface ESMF_GridSetRegionID
 \end{verbatim}{\sf PRIVATE MEMBER FUNCTIONS:}
\begin{verbatim}          module procedure ESMF_GridSetRegionIDFromArray
          module procedure ESMF_GridSetRegionIDFromBuffer
          module procedure ESMF_GridSetRegionIDCompute
          module procedure ESMF_GridSetRegionIDCopy
 \end{verbatim}
{\sf DESCRIPTION:\\ }


       This interface provides a single entry point for methods that set
       region id's as part of a {\tt Grid}.
   
%/////////////////////////////////////////////////////////////
 
\mbox{}\hrulefill\ 
 
\subsubsection{ESMF\_GridCreateEmpty - Create a new Grid with no data}


 
\bigskip{\sf INTERFACE:}
\begin{verbatim}       function ESMF_GridCreateEmpty(name, rc)\end{verbatim}{\em RETURN VALUE:}
\begin{verbatim}       type(ESMF_Grid) :: ESMF_GridCreateEmpty\end{verbatim}{\em ARGUMENTS:}
\begin{verbatim}       character (len=*), intent(in), optional :: name
       integer, intent(out), optional :: rc               \end{verbatim}
{\sf DESCRIPTION:\\ }


       Allocates memory for a new {\tt Grid} object and constructs its
       internals.  Return a pointer to the new {\tt Grid}.
  
       The arguments are:
       \begin{description}
       \item[[name]] 
            {\tt Grid} name.
       \item[[rc]] 
            Return code; equals {\tt ESMF\_SUCCESS} if there are no errors.
     \end{description}
  
\bigskip{\sf REQUIREMENTS:}
\begin{verbatim} \end{verbatim}
 
%/////////////////////////////////////////////////////////////
 
\mbox{}\hrulefill\ 
 
\subsubsection{ESMF\_GridCreateInternal - Create a new Grid internally}


\bigskip{\sf INTERFACE:}
\begin{verbatim}       function ESMF_GridCreateInternal(name, gridtype, coord_system, &
                                        x_min, x_max, y_min, y_max, i_max, &
                                        j_max, rc)\end{verbatim}{\em RETURN VALUE:}
\begin{verbatim}       type(ESMF_Grid) :: ESMF_GridCreateInternal\end{verbatim}{\em ARGUMENTS:}
\begin{verbatim}       character (len=*), intent(in) :: name
       integer, intent(in), optional :: gridtype
       integer, intent(in), optional :: coord_system
       real, intent(in), optional :: x_min
       real, intent(in), optional :: x_max
       real, intent(in), optional :: y_min
       real, intent(in), optional :: y_max
       integer, intent(in) :: i_max
       integer, intent(in) :: j_max
       integer, intent(out), optional :: rc\end{verbatim}
{\sf DESCRIPTION:\\ }


       Allocates memory for a new {\tt Grid} object, constructs its
       internals, and internally generates the Grid.  Return a pointer to
       the new {\tt Grid}.
  
       The arguments are:
       \begin{description}
       \item[[name]] 
            {\tt Grid} name.
       \item[[gridtype]] 
            Integer specifier to denote gridtype:
               gridtype=1   lat-lon
               TODO:  fill out
       \item[[coord\_system]]
            Integer specifier to denote coordinate system:
               coord\_system=1   spherical
               coord\_system=2   Cartesian
               coord\_system=3   cylindrical
       \item[[x\_min]]
            Minimum physical coordinate in the x-direction.
       \item[[x\_max]]
            Maximum physical coordinate in the x-direction.
       \item[[y\_min]]
            Minimum physical coordinate in the y-direction.
       \item[[y\_max]]
            Maximum physical coordinate in the y-direction.
       \item[[i\_max]]
            Number of even-spaced grid increments in the i-direction.
       \item[[j\_max]]
            Number of even-spaced grid increments in the j-direction.
       \item[[rc]] 
            Return code; equals {\tt ESMF\_SUCCESS} if there are no errors.
     \end{description}
  
\bigskip{\sf REQUIREMENTS:}
\begin{verbatim} \end{verbatim}
 
%/////////////////////////////////////////////////////////////
 
\mbox{}\hrulefill\ 
 
\subsubsection{ESMF\_GridCreateRead - Create a new Grid read in from a file}


\bigskip{\sf INTERFACE:}
\begin{verbatim}       function ESMF_GridCreateRead(name, iospec, rc)\end{verbatim}{\em RETURN VALUE:}
\begin{verbatim}       type(ESMF_Grid) :: ESMF_GridCreateRead\end{verbatim}{\em ARGUMENTS:}
\begin{verbatim}       character (len=*), intent(in) :: name
       type(ESMF_IOSpec), intent(in) :: iospec   ! file specs
       integer, intent(out), optional :: rc               \end{verbatim}
{\sf DESCRIPTION:\\ }


       Allocates memory for a new {\tt Grid} object, constructs its
       internals, and reads a {\tt Grid} in from a file.  Return a pointer to
       the new {\tt Grid}.
  
       The arguments are:
       \begin{description}
       \item[[name]] 
            {\tt Grid} name.
       \item[[iospec]] 
            File I/O specification.
       \item[[rc]] 
            Return code; equals {\tt ESMF\_SUCCESS} if there are no errors.
     \end{description}
  
\bigskip{\sf REQUIREMENTS:}
\begin{verbatim} \end{verbatim}
 
%/////////////////////////////////////////////////////////////
 
\mbox{}\hrulefill\ 
 
\subsubsection{ESMF\_GridCreateCopy - Create a new Grid by copying another Grid}


\bigskip{\sf INTERFACE:}
\begin{verbatim}       function ESMF_GridCreateCopy(name, grid_in, rc)\end{verbatim}{\em RETURN VALUE:}
\begin{verbatim}       type(ESMF_Grid) :: ESMF_GridCreateCopy\end{verbatim}{\em ARGUMENTS:}
\begin{verbatim}       character (len=*), intent(in) :: name
       type(ESMF_Grid), intent(in) :: grid_in
       integer, intent(out), optional :: rc               \end{verbatim}
{\sf DESCRIPTION:\\ }


       Allocates memory for a new {\tt Grid} object, constructs its
       internals, and copies attributes from another {\tt Grid}.  Return a
       pointer to the new {\tt Grid}.
  
       The arguments are:
       \begin{description}
       \item[[name]] 
            {\tt Grid} name.
       \item[[grid\_in]] 
            {\tt Grid} to be copied.
       \item[[rc]] 
            Return code; equals {\tt ESMF\_SUCCESS} if there are no errors.
     \end{description}
  
\bigskip{\sf REQUIREMENTS:}
\begin{verbatim} \end{verbatim}
 
%/////////////////////////////////////////////////////////////
 
\mbox{}\hrulefill\ 
 
\subsubsection{ESMF\_GridCreateCutout - Create a new Grid as a subset of an existing Grid}


\bigskip{\sf INTERFACE:}
\begin{verbatim}       function ESMF_GridCreateCutout(name, grid_in, i_min, i_max, j_min, &
                                      j_max, rc)\end{verbatim}{\em RETURN VALUE:}
\begin{verbatim}       type(ESMF_Grid) :: ESMF_GridCreateCutout\end{verbatim}{\em ARGUMENTS:}
\begin{verbatim}       character (len=*), intent(in) :: name
       type(ESMF_Grid), intent(in) :: grid_in
       integer, intent(in) :: i_min                       
       integer, intent(in) :: i_max                       
       integer, intent(in) :: j_min                       
       integer, intent(in) :: j_max                       
       integer, intent(out), optional :: rc               \end{verbatim}
{\sf DESCRIPTION:\\ }


       Allocates memory for a new {\tt Grid} object, constructs its
       internals, and copies a region from an existing {\tt Grid}.
       Return a pointer to the new {\tt Grid}.
  
       The arguments are:
       \begin{description}
       \item[[name]] 
            {\tt Grid} name.
       \item[[grid\_in]] 
            {\tt Grid} to be partially copied.
       \item[[i\_min]] 
            Minimum global i-index for the region of the grid to be cutout.
       \item[[i\_max]] 
            Maximum global i-index for the region of the grid to be cutout.
       \item[[j\_min]] 
            Minimum global j-index for the region of the grid to be cutout.
       \item[[j\_max]] 
            Maximum global j-index for the region of the grid to be cutout.
       \item[[rc]] 
            Return code; equals {\tt ESMF\_SUCCESS} if there are no errors.
     \end{description}
  
\bigskip{\sf REQUIREMENTS:}
\begin{verbatim} \end{verbatim}
 
%/////////////////////////////////////////////////////////////
 
\mbox{}\hrulefill\ 
 
\subsubsection{ESMF\_GridCreateChangeResolution - Create a new Grid by coarsening or refining an existing Grid}


\bigskip{\sf INTERFACE:}
\begin{verbatim}       function ESMF_GridCreateChangeResolution(name, grid_in, i_resolution, &
                                                j_resolution, rc)\end{verbatim}{\em RETURN VALUE:}
\begin{verbatim}       type(ESMF_Grid) :: ESMF_GridCreateChangeResolution\end{verbatim}{\em ARGUMENTS:}
\begin{verbatim}       character (len=*), intent(in) :: name
       type(ESMF_Grid), intent(in) :: grid_in
       integer, intent(in) :: i_resolution
       integer, intent(in) :: j_resolution
       integer, intent(out), optional :: rc\end{verbatim}
{\sf DESCRIPTION:\\ }


       Allocates memory for a new {\tt Grid} object, constructs its
       internals, and creates a {\tt Grid} by either coarsening or refining an
       existing {\tt Grid}.  Return a pointer to the new {\tt Grid}.
  
       The arguments are:
       \begin{description}
       \item[[name]] 
            {\tt Grid} name.
       \item[[grid\_in]] 
            Source {\tt Grid} to be coarsened or refined.
       \item[[i\_resolution]] 
            Integer resolution factor in the i-direction.
       \item[[j\_resolution]] 
            Integer resolution factor in the j-direction.
            Note:  The above arguments assume refinement by factor if positive
            and coarsening by absolute value of the factor if negative.  For 
            example, i\_resolution=4 indicates the new {\tt Grid} will be four
            times as resolved in the i-direction as the source {\tt Grid},
            whereas j\_resolution=-3 means the new {\tt Grid} will sample every
            third point in the j-direction.
       \item[[rc]] 
            Return code; equals {\tt ESMF\_SUCCESS} if there are no errors.
     \end{description}
  
\bigskip{\sf REQUIREMENTS:}
\begin{verbatim} \end{verbatim}
 
%/////////////////////////////////////////////////////////////
 
\mbox{}\hrulefill\ 
 
\subsubsection{ESMF\_GridCreateExchange - Create a new Grid from the intersection of two existing grids}


\bigskip{\sf INTERFACE:}
\begin{verbatim}       function ESMF_GridCreateExchange(name, grid_in1, grid_in2, rc)\end{verbatim}{\em RETURN VALUE:}
\begin{verbatim}       type(ESMF_Grid) :: ESMF_GridCreateExchange\end{verbatim}{\em ARGUMENTS:}
\begin{verbatim}       character (len=*), intent(in) :: name
       type(ESMF_Grid), intent(in) :: grid_in1
       type(ESMF_Grid), intent(in) :: grid_in2
       integer, intent(out), optional :: rc               \end{verbatim}
{\sf DESCRIPTION:\\ }


       Allocates memory for a new {\tt Grid} object, constructs its
       internals, and creates a new {\tt Grid} from the intersection of two
       existing {\tt Grids}.  Return a pointer to the new {\tt Grid}.
  
       The arguments are:
       \begin{description}
       \item[[name]] 
            New {\tt Grid} name.
       \item[[grid\_in1]] 
            First source {\tt Grid}.
       \item[[grid\_in2]] 
            Second source {\tt Grid}.
       \item[[rc]] 
            Return code; equals {\tt ESMF\_SUCCESS} if there are no errors.
     \end{description}
  
\bigskip{\sf REQUIREMENTS:}
\begin{verbatim} \end{verbatim}
 
%/////////////////////////////////////////////////////////////
 
\mbox{}\hrulefill\ 
 
\subsubsection{ESMF\_GridDestroy - Free all resources associated with a Grid }


\bigskip{\sf INTERFACE:}
\begin{verbatim}       subroutine ESMF_GridDestroy(grid, rc)\end{verbatim}{\em ARGUMENTS:}
\begin{verbatim}       type(ESMF_Grid), intent(in) :: grid   
       integer, intent(out), optional :: rc        \end{verbatim}
{\sf DESCRIPTION:\\ }


       Destroys a {\tt Grid} object previously allocated
       via an {\tt ESMF\_GridCreate routine}.
  
       The arguments are:
       \begin{description}
       \item[grid] 
            The class to be destroyed.
       \item[[rc]] 
            Return code; equals {\tt ESMF\_SUCCESS} if there are no errors.
       \end{description}
   
%/////////////////////////////////////////////////////////////
 
\mbox{}\hrulefill\ 
 
\subsubsection{ESMF\_GridConstructNew - Construct the internals of an allocated Grid}


 
\bigskip{\sf INTERFACE:}
\begin{verbatim}       subroutine ESMF_GridConstructNew(grid, name, rc)\end{verbatim}{\em ARGUMENTS:}
\begin{verbatim}       type(ESMF_GridType) :: grid 
       character (len = *), intent(in), optional :: name  
       integer, intent(out), optional :: rc\end{verbatim}
{\sf DESCRIPTION:\\ }


       ESMF routine which fills in the contents of an already
       allocated {\tt Grid} object.  May perform additional allocations
       as needed.  Must call the corresponding ESMF\_GridDestruct
       routine to free the additional memory.  Intended for internal
       ESMF use only; end-users use {\tt ESMF\_GridCreate}, which calls
       {\tt ESMF\_GridConstruct}. 
  
       The arguments are:
       \begin{description}
       \item[grid] 
            Pointer to a {\tt Grid}
       \item[arg1]
            Argument 1.
       \item[arg2]
            Argument 2.         
       \item[[name]] 
            {\tt Grid} name.
       \item[[rc]] 
            Return code; equals {\tt ESMF\_SUCCESS} if there are no errors.
       \end{description}
  
\bigskip{\sf REQUIREMENTS:}
\begin{verbatim} \end{verbatim}
 
%/////////////////////////////////////////////////////////////
 
\mbox{}\hrulefill\ 
 
\subsubsection{ESMF\_GridDestruct - Free any Grid memory allocated internally}


\bigskip{\sf INTERFACE:}
\begin{verbatim}       subroutine ESMF_GridDestruct(grid, rc)\end{verbatim}{\em ARGUMENTS:}
\begin{verbatim}       type(ESMF_Grid), intent(in) :: grid    
       integer, intent(out), optional :: rc         \end{verbatim}
{\sf DESCRIPTION:\\ }


       ESMF routine which deallocates any space allocated by
      {\tt  ESMF\_GridConstruct}, does any additional cleanup before the
       original Grid object is freed.  Intended for internal ESMF
       use only; end-users use {\tt ESMF\_GridDestroy}, which calls
       {\tt ESMF\_GridDestruct}.  
  
       The arguments are:
       \begin{description}
       \item[grid] 
            The class to be destructed.
       \item[[rc]] 
            Return code; equals {\tt ESMF\_SUCCESS} if there are no errors.
       \end{description}
   
%/////////////////////////////////////////////////////////////
 
\mbox{}\hrulefill\ 
 
\subsubsection{ESMF\_GridGetConfig - Get configuration information from a Grid}


 
\bigskip{\sf INTERFACE:}
\begin{verbatim}       subroutine ESMF_GridGetConfig(grid, config, rc)\end{verbatim}{\em ARGUMENTS:}
\begin{verbatim}       type(ESMF_Grid), intent(in) :: grid
       integer, intent(out) :: config   
       integer, intent(out), optional :: rc              \end{verbatim}
{\sf DESCRIPTION:\\ }


       Returns the set of resources the Grid object was configured with.
  
       The arguments are:
       \begin{description}
       \item[grid] 
            Class to be queried.
       \item[config]
            Configuration information.         
       \item[[rc]] 
            Return code; equals {\tt ESMF\_SUCCESS} if there are no errors.
       \end{description}
   
%/////////////////////////////////////////////////////////////
 
\mbox{}\hrulefill\ 
 
\subsubsection{ESMF\_GridSetConfig - Set configuration information for a Grid}


 
\bigskip{\sf INTERFACE:}
\begin{verbatim}       subroutine ESMF_GridSetConfig(grid, config, rc)\end{verbatim}{\em ARGUMENTS:}
\begin{verbatim}       type(ESMF_Grid), intent(in) :: grid
       integer, intent(in) :: config   
       integer, intent(out), optional :: rc             
 \end{verbatim}
{\sf DESCRIPTION:\\ }


       Configures the Grid object with set of resources given.
  
       The arguments are:
       \begin{description}
       \item[grid] 
            Class to be configured.
       \item[config]
            Configuration information.         
       \item[[rc]] 
            Return code; equals {\tt ESMF\_SUCCESS} if there are no errors.
       \end{description}
   
%/////////////////////////////////////////////////////////////
 
\mbox{}\hrulefill\ 
 
\subsubsection{ESMF\_GridGetValue - Get <Value> for a Grid}


 
\bigskip{\sf INTERFACE:}
\begin{verbatim}       subroutine ESMF_GridGetValue(grid, value, rc)\end{verbatim}{\em ARGUMENTS:}
\begin{verbatim}       type(ESMF_Grid), intent(in) :: grid
       integer, intent(out) :: value
       integer, intent(out), optional :: rc             
 \end{verbatim}
{\sf DESCRIPTION:\\ }


       Returns the value of Grid attribute <Value>.
       May be multiple routines, one per attribute.
  
       The arguments are:
       \begin{description}
       \item[grid] 
            Class to be queried.
       \item[value]
            Value to be retrieved.         
       \item[[rc]] 
            Return code; equals {\tt ESMF\_SUCCESS} if there are no errors.
       \end{description}
   
%/////////////////////////////////////////////////////////////
 
\mbox{}\hrulefill\ 
 
\subsubsection{ESMF\_GridSetCoordinateFromArray - Set the coordinates of a Grid from an existing ESMF array}


 
\bigskip{\sf INTERFACE:}
\begin{verbatim}       subroutine ESMF_GridSetCoordinateFromArray(Grid, array, id, rc)\end{verbatim}{\em ARGUMENTS:}
\begin{verbatim}       type(ESMF_Grid), intent(in) :: grid
       type(ESMF_Array), intent(in) :: array
       integer, intent(in) :: id
       integer, intent(out), optional :: rc\end{verbatim}
{\sf DESCRIPTION:\\ }


       This version of set assumes the coordinates exist already and are being
       passed in through an {\tt Array}.
  
       The arguments are:
       \begin{description}
       \item[grid] 
            Pointer to a {\tt Grid} to be modified.
       \item[array]
            ESMF Array of data.         
       \item[[id]]
            Identifier for which set of coordinates are being set:
               1  center\_x
               2  center\_y
               3  corner\_x
               4  corner\_y
               5  face\_x
               6  face\_y 
       \item[[rc]] 
            Return code; equals {\tt ESMF\_SUCCESS} if there are no errors.
       \end{description}
   
%/////////////////////////////////////////////////////////////
 
\mbox{}\hrulefill\ 
 
\subsubsection{ESMF\_GridSetCoordinateFromBuffer - Set the coordinates of a Grid from an existing data buffer}


 
\bigskip{\sf INTERFACE:}
\begin{verbatim}       subroutine ESMF_GridSetCoordinateFromBuffer(Grid, buffer, id, rc)\end{verbatim}{\em ARGUMENTS:}
\begin{verbatim}       type(ESMF_Grid), intent(in) :: grid
       real, dimension (:), pointer :: buffer
       integer, intent(in) :: id            
       integer, intent(out), optional :: rc            \end{verbatim}
{\sf DESCRIPTION:\\ }


       This version of set assumes the coordinates exist already and are being
       passed in as a raw data buffer.
  
       The arguments are:
       \begin{description}
       \item[grid] 
            Pointer to a {\tt Grid} to be modified.
       \item[buffer]
            Raw data buffer.         
       \item[[id]]
            Identifier for which set of coordinates are being set:
               1  center\_x
               2  center\_y
               3  corner\_x
               4  corner\_y
               5  face\_x
               6  face\_y 
       \item[[rc]] 
            Return code; equals {\tt ESMF\_SUCCESS} if there are no errors.
       \end{description}
   
%/////////////////////////////////////////////////////////////
 
\mbox{}\hrulefill\ 
 
\subsubsection{ESMF\_GridSetCoordinateCompute - Compute coordinates for a Grid}


 
\bigskip{\sf INTERFACE:}
\begin{verbatim}       subroutine ESMF_GridSetCoordinateCompute(Grid, id, rc)\end{verbatim}{\em ARGUMENTS:}
\begin{verbatim}       type(ESMF_Grid), intent(in) :: grid
       integer, intent(in) :: id
       integer, intent(out), optional :: rc\end{verbatim}
{\sf DESCRIPTION:\\ }


       This version of set internally computes coordinates for a Grid via a
       prescribed method.
  
       The arguments are:
       \begin{description}
       \item[grid] 
            Pointer to a {\tt Grid} to be modified.
       \item[[id]]
            Identifier for which set of coordinates are being set:
               1  center\_x
               2  center\_y
               3  corner\_x
               4  corner\_y
               5  face\_x
               6  face\_y 
       \item[[rc]] 
            Return code; equals {\tt ESMF\_SUCCESS} if there are no errors.
       \end{description}
  
  TODO: figure out the argument list necessary to completely describe the 
        internal calculation of the coordinates of a simple grid. 
%/////////////////////////////////////////////////////////////
 
\mbox{}\hrulefill\ 
 
\subsubsection{ESMF\_GridSetCoordinateCopy - Copies coordinates from one grid to another}


 
\bigskip{\sf INTERFACE:}
\begin{verbatim}       subroutine ESMF_GridSetCoordinateCopy(Grid, Grid_in, id, rc)\end{verbatim}{\em ARGUMENTS:}
\begin{verbatim}       type(ESMF_Grid), intent(in) :: grid
       type(ESMF_Grid), intent(in) :: grid_in
       integer, intent(in) :: id
       integer, intent(out), optional :: rc\end{verbatim}
{\sf DESCRIPTION:\\ }


       This version of set copies the coordinates of a Grid from another Grid.
  
       The arguments are:
       \begin{description}
       \item[grid] 
            Pointer to a {\tt Grid} to be modified.
       \item[grid\_in] 
            Pointer to a {\tt Grid} whose coordinates are to be copied.
       \item[[id]]
            Identifier for which set of coordinates are being set:
               1  center\_x
               2  center\_y
               3  corner\_x
               4  corner\_y
               5  face\_x
               6  face\_y 
       \item[[rc]] 
            Return code; equals {\tt ESMF\_SUCCESS} if there are no errors.
       \end{description}
   
%/////////////////////////////////////////////////////////////
 
\mbox{}\hrulefill\ 
 
\subsubsection{ESMF\_GridSetLMaskFromArray - Set a logical mask in a Grid from an existing ESMF array}


 
\bigskip{\sf INTERFACE:}
\begin{verbatim}       subroutine ESMF_GridSetLMaskFromArray(Grid, array, name, rc)\end{verbatim}{\em ARGUMENTS:}
\begin{verbatim}       type(ESMF_Grid), intent(in) :: grid
       type(ESMF_Array), intent(in) :: array
       character (len=*), intent(in) :: name  ! TODO: optional?
       integer, intent(out), optional :: rc            \end{verbatim}
{\sf DESCRIPTION:\\ }


       This version of set assumes the logical mask data exists already and is
       being passed in through an {\tt Array}.
  
       The arguments are:
       \begin{description}
       \item[grid] 
            Pointer to a {\tt Grid} to be modified.
       \item[array]
            ESMF Array of data.
       \item [[name]]
             {\tt LMask} name.
       \item[[rc]] 
            Return code; equals {\tt ESMF\_SUCCESS} if there are no errors.
       \end{description}
   
%/////////////////////////////////////////////////////////////
 
\mbox{}\hrulefill\ 
 
\subsubsection{ESMF\_GridSetLMaskFromBuffer - Set a logical mask in a Grid from an existing data buffer}


 
\bigskip{\sf INTERFACE:}
\begin{verbatim}       subroutine ESMF_GridSetLMaskFromBuffer(Grid, buffer, name, rc)\end{verbatim}{\em ARGUMENTS:}
\begin{verbatim}       type(ESMF_Grid), intent(in) :: grid
       real, dimension (:), pointer :: buffer
       character (len=*), intent(in) :: name  ! TODO: optional?
       integer, intent(out), optional :: rc            \end{verbatim}
{\sf DESCRIPTION:\\ }


       This version of set assumes the logical mask data exists already and is
       being passed in as a raw data buffer.
  
       The arguments are:
       \begin{description}
       \item[grid] 
            Pointer to a {\tt Grid} to be modified.
       \item[buffer]
            Raw data buffer.         
       \item [[name]]
             {\tt LMask} name.
       \item[[rc]] 
            Return code; equals {\tt ESMF\_SUCCESS} if there are no errors.
       \end{description}
   
%/////////////////////////////////////////////////////////////
 
\mbox{}\hrulefill\ 
 
\subsubsection{ESMF\_GridSetLMaskCompute - Compute a logical mask for a Grid}


 
\bigskip{\sf INTERFACE:}
\begin{verbatim}       subroutine ESMF_GridSetLMaskCompute(Grid, name, rc)\end{verbatim}{\em ARGUMENTS:}
\begin{verbatim}       type(ESMF_Grid), intent(in) :: grid
       character (len=*), intent(in) :: name  ! TODO: optional?
       integer, intent(out), optional :: rc            \end{verbatim}
{\sf DESCRIPTION:\\ }


       This version of set internally computes a logical mask for a Grid via a
       prescribed method.
  
       The arguments are:
       \begin{description}
       \item[grid] 
            Pointer to a {\tt Grid} to be modified.
       \item [[name]]
             {\tt LMask} name.
       \item[[rc]] 
            Return code; equals {\tt ESMF\_SUCCESS} if there are no errors.
       \end{description}
  
  TODO: figure out the argument list necessary to completely describe the 
        internal calculation of a logical mask for a simple grid. 
%/////////////////////////////////////////////////////////////
 
\mbox{}\hrulefill\ 
 
\subsubsection{ESMF\_GridSetLMaskCopy - Copies a logical mask from one grid to another.}


 
\bigskip{\sf INTERFACE:}
\begin{verbatim}       subroutine ESMF_GridSetLMaskCopy(Grid, name, Grid_in, name_in, rc)\end{verbatim}{\em ARGUMENTS:}
\begin{verbatim}       type(ESMF_Grid), intent(in) :: grid
       character (len=*), intent(in) :: name  ! TODO: optional?
       type(ESMF_Grid), intent(in) :: grid_in
       character (len=*), intent(in) :: name_in  ! TODO: optional?
       integer, intent(out), optional :: rc            \end{verbatim}
{\sf DESCRIPTION:\\ }


       This version of set copies a logical mask for a Grid from another Grid.
  
       The arguments are:
       \begin{description}
       \item[grid] 
            Pointer to a {\tt Grid} to be modified.
       \item [[name]]
             {\tt LMask} name to be set.
       \item[grid\_in] 
            Pointer to a {\tt Grid} whose coordinates are to be copied.
       \item [[name\_in]]
             {\tt LMask} name to be copied.
       \item[[rc]] 
            Return code; equals {\tt ESMF\_SUCCESS} if there are no errors.
       \end{description}
   
%/////////////////////////////////////////////////////////////
 
\mbox{}\hrulefill\ 
 
\subsubsection{ESMF\_GridSetMMaskFromArray - Set a multiplicative mask in a Grid from an existing ESMF array}


 
\bigskip{\sf INTERFACE:}
\begin{verbatim}       subroutine ESMF_GridSetMMaskFromArray(Grid, array, name, rc)\end{verbatim}{\em ARGUMENTS:}
\begin{verbatim}       type(ESMF_Grid), intent(in) :: grid
       type(ESMF_Array), intent(in) :: array
       character (len=*), intent(in) :: name  ! TODO: optional?
       integer, intent(out), optional :: rc            \end{verbatim}
{\sf DESCRIPTION:\\ }


       This version of set assumes the multiplicative mask data exists already
       and is being passed in through an {\tt Array}.
  
       The arguments are:
       \begin{description}
       \item[grid] 
            Pointer to a {\tt Grid} to be modified.
       \item[array]
            ESMF Array of data.
       \item [[name]]
             {\tt MMask} name.
       \item[[rc]] 
            Return code; equals {\tt ESMF\_SUCCESS} if there are no errors.
       \end{description}
   
%/////////////////////////////////////////////////////////////
 
\mbox{}\hrulefill\ 
 
\subsubsection{ESMF\_GridSetMMaskFromBuffer - Set a multiplicative mask in a Grid from an existing data buffer}


 
\bigskip{\sf INTERFACE:}
\begin{verbatim}       subroutine ESMF_GridSetMMaskFromBuffer(Grid, buffer, name, rc)\end{verbatim}{\em ARGUMENTS:}
\begin{verbatim}       type(ESMF_Grid), intent(in) :: grid
       real, dimension (:), pointer :: buffer
       character (len=*), intent(in) :: name  ! TODO: optional?
       integer, intent(out), optional :: rc            \end{verbatim}
{\sf DESCRIPTION:\\ }


       This version of set assumes the multiplicative mask data exists already
       and is being passed in as a raw data buffer.
  
       The arguments are:
       \begin{description}
       \item[grid] 
            Pointer to a {\tt Grid} to be modified.
       \item[buffer]
            Raw data buffer.         
       \item [[name]]
             {\tt MMask} name.
       \item[[rc]] 
            Return code; equals {\tt ESMF\_SUCCESS} if there are no errors.
       \end{description}
   
%/////////////////////////////////////////////////////////////
 
\mbox{}\hrulefill\ 
 
\subsubsection{ESMF\_GridSetMMaskCompute - Compute a multiplicative mask for a Grid}


 
\bigskip{\sf INTERFACE:}
\begin{verbatim}       subroutine ESMF_GridSetMMaskCompute(Grid, name, rc)\end{verbatim}{\em ARGUMENTS:}
\begin{verbatim}       type(ESMF_Grid), intent(in) :: grid
       character (len=*), intent(in) :: name  ! TODO: optional?
       integer, intent(out), optional :: rc            \end{verbatim}
{\sf DESCRIPTION:\\ }


       This version of set internally computes a multiplicative mask for a
       Grid via a prescribed method.
  
       The arguments are:
       \begin{description}
       \item[grid] 
            Pointer to a {\tt Grid} to be modified.
       \item [[name]]
             {\tt MMask} name.
       \item[[rc]] 
            Return code; equals {\tt ESMF\_SUCCESS} if there are no errors.
       \end{description}
  
  TODO: figure out the argument list necessary to completely describe the 
        internal calculation of a multiplicative mask for a simple grid. 
%/////////////////////////////////////////////////////////////
 
\mbox{}\hrulefill\ 
 
\subsubsection{ESMF\_GridSetMMaskCopy - Copies a multiplicative mask from one grid to another.}


 
\bigskip{\sf INTERFACE:}
\begin{verbatim}       subroutine ESMF_GridSetMMaskCopy(Grid, name, Grid_in, name_in, rc)\end{verbatim}{\em ARGUMENTS:}
\begin{verbatim}       type(ESMF_Grid), intent(in) :: grid
       character (len=*), intent(in) :: name  ! TODO: optional?
       type(ESMF_Grid), intent(in) :: grid_in
       character (len=*), intent(in) :: name_in  ! TODO: optional?
       integer, intent(out), optional :: rc            \end{verbatim}
{\sf DESCRIPTION:\\ }


       This version of set copies a multiplicative mask for a Grid from another
       Grid.
  
       The arguments are:
       \begin{description}
       \item[grid] 
            Pointer to a {\tt Grid} to be modified.
       \item [[name]]
             {\tt MMask} name to be set.
       \item[grid\_in] 
            Pointer to a {\tt Grid} whose coordinates are to be copied.
       \item [[name\_in]]
             {\tt MMask} name to be copied.
       \item[[rc]] 
            Return code; equals {\tt ESMF\_SUCCESS} if there are no errors.
       \end{description}
   
%/////////////////////////////////////////////////////////////
 
\mbox{}\hrulefill\ 
 
\subsubsection{ESMF\_GridSetMetricFromArray - Set a metric for a Grid from an existing ESMF array}


 
\bigskip{\sf INTERFACE:}
\begin{verbatim}       subroutine ESMF_GridSetMetricFromArray(Grid, array, name, rc)\end{verbatim}{\em ARGUMENTS:}
\begin{verbatim}       type(ESMF_Grid), intent(in) :: grid
       type(ESMF_Array), intent(in) :: array
       character (len=*), intent(in) :: name  ! TODO: optional?
       integer, intent(out), optional :: rc            \end{verbatim}
{\sf DESCRIPTION:\\ }


       This version of set assumes the metric data exists already and is being
       passed in through an {\tt Array}.
  
       The arguments are:
       \begin{description}
       \item[grid] 
            Pointer to a {\tt Grid} to be modified.
       \item[array]
            ESMF Array of data.
       \item [[name]]
             {\tt Metric} name.
       \item[[rc]] 
            Return code; equals {\tt ESMF\_SUCCESS} if there are no errors.
       \end{description}
   
%/////////////////////////////////////////////////////////////
 
\mbox{}\hrulefill\ 
 
\subsubsection{ESMF\_GridSetMetricFromBuffer - Set a metric for a Grid from an existing data buffer}


 
\bigskip{\sf INTERFACE:}
\begin{verbatim}       subroutine ESMF_GridSetMetricFromBuffer(Grid, buffer, name, rc)\end{verbatim}{\em ARGUMENTS:}
\begin{verbatim}       type(ESMF_Grid), intent(in) :: grid
       real, dimension (:), pointer :: buffer
       character (len=*), intent(in) :: name  ! TODO: optional?
       integer, intent(out), optional :: rc            \end{verbatim}
{\sf DESCRIPTION:\\ }


       This version of set assumes the metric data exists already and is being
       passed in as a raw data buffer.
  
       The arguments are:
       \begin{description}
       \item[grid] 
            Pointer to a {\tt Grid} to be modified.
       \item[buffer]
            Raw data buffer.         
       \item [[name]]
             {\tt Metric} name.
       \item[[rc]] 
            Return code; equals {\tt ESMF\_SUCCESS} if there are no errors.
       \end{description}
   
%/////////////////////////////////////////////////////////////
 
\mbox{}\hrulefill\ 
 
\subsubsection{ESMF\_GridSetMetricCompute - Compute a metric for a Grid}


 
\bigskip{\sf INTERFACE:}
\begin{verbatim}       subroutine ESMF_GridSetMetricCompute(Grid, name, id, rc)\end{verbatim}{\em ARGUMENTS:}
\begin{verbatim}       type(ESMF_Grid), intent(in) :: grid
       character (len=*), intent(in) :: name  ! TODO: optional?
       integer, intent(in) :: id
       integer, intent(out), optional :: rc\end{verbatim}
{\sf DESCRIPTION:\\ }


       This version of set internally computes a metric for a Grid via a
       prescribed method.
  
       The arguments are:
       \begin{description}
       \item[grid] 
            Pointer to a {\tt Grid} to be modified.
       \item [[name]]
             {\tt Metric} name.
       \item[[id]] 
            Identifier for predescribed metrics.  TODO: make list
       \item[[rc]] 
            Return code; equals {\tt ESMF\_SUCCESS} if there are no errors.
       \end{description}
   
%/////////////////////////////////////////////////////////////
 
\mbox{}\hrulefill\ 
 
\subsubsection{ESMF\_GridSetMetricCopy - Copies a metric from one grid to another}


 
\bigskip{\sf INTERFACE:}
\begin{verbatim}       subroutine ESMF_GridSetMetricCopy(Grid, name, Grid_in, name_in, rc)\end{verbatim}{\em ARGUMENTS:}
\begin{verbatim}       type(ESMF_Grid), intent(in) :: grid
       character (len=*), intent(in) :: name  ! TODO: optional?
       type(ESMF_Grid), intent(in) :: grid_in
       character (len=*), intent(in) :: name_in  ! TODO: optional?
       integer, intent(out), optional :: rc            \end{verbatim}
{\sf DESCRIPTION:\\ }


       This version of set copies a metric for a Grid from another Grid.
  
       The arguments are:
       \begin{description}
       \item[grid] 
            Pointer to a {\tt Grid} to be modified.
       \item [[name]]
             {\tt Metric} name to be set.
       \item[grid\_in] 
            Pointer to a {\tt Grid} whose coordinates are to be copied.
       \item [[name\_in]]
             {\tt Metric} name to be copied.
       \item[[rc]] 
            Return code; equals {\tt ESMF\_SUCCESS} if there are no errors.
       \end{description}
   
%/////////////////////////////////////////////////////////////
 
\mbox{}\hrulefill\ 
 
\subsubsection{ESMF\_GridSetRegionIDFromArray - Set a region identifier in a Grid from an existing ESMF array}


 
\bigskip{\sf INTERFACE:}
\begin{verbatim}       subroutine ESMF_GridSetRegionIDFromArray(Grid, array, name, rc)\end{verbatim}{\em ARGUMENTS:}
\begin{verbatim}       type(ESMF_Grid), intent(in) :: grid
       type(ESMF_Array), intent(in) :: array
       character (len=*), intent(in) :: name  ! TODO: optional?
       integer, intent(out), optional :: rc            \end{verbatim}
{\sf DESCRIPTION:\\ }


       This version of set assumes the region identifier data exists already
       and is being passed in through an {\tt Array}.
  
       The arguments are:
       \begin{description}
       \item[grid] 
            Pointer to a {\tt Grid} to be modified.
       \item[array]
            ESMF Array of data.
       \item [[name]]
             {\tt RegionID} name.
       \item[[rc]] 
            Return code; equals {\tt ESMF\_SUCCESS} if there are no errors.
       \end{description}
   
%/////////////////////////////////////////////////////////////
 
\mbox{}\hrulefill\ 
 
\subsubsection{ESMF\_GridSetRegionIDFromBuffer - Set a region identifier in a Grid from an existing data buffer}


 
\bigskip{\sf INTERFACE:}
\begin{verbatim}       subroutine ESMF_GridSetRegionIDFromBuffer(Grid, buffer, name, rc)\end{verbatim}{\em ARGUMENTS:}
\begin{verbatim}       type(ESMF_Grid), intent(in) :: grid
       real, dimension (:), pointer :: buffer
       character (len=*), intent(in) :: name  ! TODO: optional?
       integer, intent(out), optional :: rc            \end{verbatim}
{\sf DESCRIPTION:\\ }


       This version of set assumes the multiplicative mask data exists already
       and is being passed in as a raw data buffer.
  
       The arguments are:
       \begin{description}
       \item[grid] 
            Pointer to a {\tt Grid} to be modified.
       \item[buffer]
            Raw data buffer.         
       \item [[name]]
             {\tt RegionID} name.
       \item[[rc]] 
            Return code; equals {\tt ESMF\_SUCCESS} if there are no errors.
       \end{description}
   
%/////////////////////////////////////////////////////////////
 
\mbox{}\hrulefill\ 
 
\subsubsection{ESMF\_GridSetRegionIDCompute - Compute a region identifier for a Grid}


 
\bigskip{\sf INTERFACE:}
\begin{verbatim}       subroutine ESMF_GridSetRegionIDCompute(Grid, name, rc)\end{verbatim}{\em ARGUMENTS:}
\begin{verbatim}       type(ESMF_Grid), intent(in) :: grid
       character (len=*), intent(in) :: name  ! TODO: optional?
       integer, intent(out), optional :: rc            \end{verbatim}
{\sf DESCRIPTION:\\ }


       This version of set internally computes a region identifier for a
       Grid via a prescribed method.
  
       The arguments are:
       \begin{description}
       \item[grid] 
            Pointer to a {\tt Grid} to be modified.
       \item [[name]]
             {\tt RegionID} name.
       \item[[rc]] 
            Return code; equals {\tt ESMF\_SUCCESS} if there are no errors.
       \end{description}
  
  TODO: figure out the argument list necessary to completely describe the 
        internal calculation of a region identifier for a simple grid. 
%/////////////////////////////////////////////////////////////
 
\mbox{}\hrulefill\ 
 
\subsubsection{ESMF\_GridSetRegionIDCopy - Copies a region identifier from one grid to another}


 
\bigskip{\sf INTERFACE:}
\begin{verbatim}       subroutine ESMF_GridSetRegionIDCopy(Grid, name, Grid_in, name_in, rc)\end{verbatim}{\em ARGUMENTS:}
\begin{verbatim}       type(ESMF_Grid), intent(in) :: grid
       character (len=*), intent(in) :: name  ! TODO: optional?
       type(ESMF_Grid), intent(in) :: grid_in
       character (len=*), intent(in) :: name_in  ! TODO: optional?
       integer, intent(out), optional :: rc            \end{verbatim}
{\sf DESCRIPTION:\\ }


       This version of set copies a region identifier for a Grid from another
       Grid.
  
       The arguments are:
       \begin{description}
       \item[grid] 
            Pointer to a {\tt Grid} to be modified.
       \item [[name]]
             {\tt RegionID} name to be set.
       \item[grid\_in] 
            Pointer to a {\tt Grid} whose coordinates are to be copied.
       \item [[name\_in]]
             {\tt RegionID} name to be copied.
       \item[[rc]] 
            Return code; equals {\tt ESMF\_SUCCESS} if there are no errors.
       \end{description}
   
%/////////////////////////////////////////////////////////////
 
\mbox{}\hrulefill\ 
 
\subsubsection{ESMF\_GridValidate - Check internal consistency of a Grid}


 
\bigskip{\sf INTERFACE:}
\begin{verbatim}       subroutine ESMF_GridValidate(grid, opt, rc)\end{verbatim}{\em ARGUMENTS:}
\begin{verbatim}       type(ESMF_Grid), intent(in) :: grid       
       character (len=*), intent(in), optional :: opt    
       integer, intent(out), optional :: rc            \end{verbatim}
{\sf DESCRIPTION:\\ }


       Validates that a Grid is internally consistent.
  
       The arguments are:
       \begin{description}
       \item[grid] 
            Class to be queried.
       \item[[opt]]
            Validation options.
       \item[[rc]] 
            Return code; equals {\tt ESMF\_SUCCESS} if there are no errors.
       \end{description}
   
%/////////////////////////////////////////////////////////////
 
\mbox{}\hrulefill\ 
 
\subsubsection{ESMF\_GridPrint - Print the contents of a Grid}


 
\bigskip{\sf INTERFACE:}
\begin{verbatim}       subroutine ESMF_GridPrint(grid, opt, rc)\end{verbatim}{\em ARGUMENTS:}
\begin{verbatim}       type(ESMF_Grid), intent(in) :: grid      
       character (len=*), intent(in) :: opt      
       integer, intent(out), optional :: rc           \end{verbatim}
{\sf DESCRIPTION:\\ }


        Print information about a Grid.  
  
       The arguments are:
       \begin{description}
       \item[grid] 
            Class to be queried.
       \item[[opt]]
            Print ptions that control the type of information and level of 
            detail.
       \item[[rc]] 
            Return code; equals {\tt ESMF\_SUCCESS} if there are no errors.
       \end{description}
  
%...............................................................
