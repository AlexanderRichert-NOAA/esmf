% $Id$

%\subsection{Design and Implementation Notes}

\subsubsection{Fortran unit number management}

When ESMF needs to open a Fortran I/O unit, it calls {\tt ESMF\_IOUnitGet()} to find
an unopened unit number.  As delivered, the range of unit numbers that are
searched are between {\tt ESMF\_LOG\_FORTRAN\_UNIT\_NUMBER} (normally set to
50), and {\tt ESMF\_LOG\_UPPER} (normally set to 99.)
Unopened unit numbers are found by using the Fortran {\tt INQUIRE} statement.

When integrating ESMF into an application where there are conflicts with
other uses of the same unit number range, an alternative range can be specified
in the {\tt ESMF\_Initialize()} call by setting the {\tt IOUnitLower} and {\tt IOUnitUpper}
arguments as needed.  {\tt ESMF\_IOUnitGet()} will then search the alternate range
of unit numbers.  Note that {\tt IOUnitUpper} must be set to a value higher than
{\tt IOUnitLower}, and that both must be non-negative.  Otherwise {\tt ESMF\_Initialize}
will return a return code of {\tt ESMF\_FAILURE}.

Fortran unit numbers are not standardized in the Fortran 90 Standard.  The standard
only requires that they be non-negative integers.  But other than that, it is
up to the compiler writers and application developers to provide and
use units which work with the particular implementation.  For example,
units 5 and 6 are a defacto standard for ``standard input'' and
``standard output'' --- even though this is not specified in the actual Fortran
standard.  The Fortran standard also does not specifiy which unit numbers can
be used, nor does it specify how many can be open simultaneously.

Since all current compilers have preconnected unit numbers, and these are
typically found on units lower than 10, it is recommended that applications
use unit numbers 10 and higher.

\subsubsection{Flushing output}

When ESMF needs to flush a Fortran unit, the {\tt ESMF\_IOUnitFlush()} API is used
to centralize the file flushing capability, because Fortran has not historically
had a standard mechanism for flushing output buffers.  Most compilers run-time libraries
support various library extensions to provide this functionality --- though,
being non-standard, the spelling and number of arguments vary between implementations.
Fortran 2003 also provides for a {\tt FLUSH} statement which is built into the
language.  When possible, {\tt ESMF\_IOUnitFlush()} uses the F2003 {\tt FLUSH} statement.
With older compilers, the appropriate library call is made.

\subsubsection{Sorting algorithms}

The {\tt ESMF\_UtilSort()} algorithms are the same as those in the {\tt LAPACK}
sorting procedures {\tt SLASRT()} and {\tt DLASRT()}.  Two algorithms are used.
For small sorts, arrays with 20 or fewer elements, a simple Insertion sort is
used.  For larger sorts, a Quicksort algorithm is used.

Compared to the original {\tt LAPACK} code, a full Fortran 90 style
interface is supported for ease of use and enhanced compile time checking.
Additional support is also provided for integer and character string data
types.
