% $Id$
%
% Earth System Modeling Framework
% Copyright 2002-2017, University Corporation for Atmospheric Research, 
% Massachusetts Institute of Technology, Geophysical Fluid Dynamics 
% Laboratory, University of Michigan, National Centers for Environmental 
% Prediction, Los Alamos National Laboratory, Argonne National Laboratory, 
% NASA Goddard Space Flight Center.
% Licensed under the University of Illinois-NCSA License.

%TODO: This file started as an exact copy of the Fortran version of this file.
%TODO: Changes were made to correctly reflect the current status of the C API.
%TODO: Eventually this file should be removed again and replaced by a single
%TODO: generic version that can be included for both Fortran and C refdocs.

The Array class is an alternative to the Field class for representing 
distributed, structured data.  Unlike Fields, which are built to carry 
grid coordinate information, Arrays can only carry information about the 
{\it indices} associated with grid cells.  Since they do not have coordinate 
information, Arrays cannot be used to calculate interpolation weights.  
However, if the user can supply interpolation weights, the Array sparse 
matrix multiply operation can be used to apply the weights and transfer 
data to the new grid.  Arrays can also perform redistribution, scatter, 
and gather communication operations.

Like Fields, Arrays can be added to a State and used in inter-Component 
data communications.

From a technical standpoint, the ESMF Array class is an index space 
based, distributed data storage class. It provides DE-local memory allocations 
within DE-centric index regions and defines the relationship to the index 
space described by the ESMF DistGrid. The Array class offers common 
communication patterns within the index space formalism.
