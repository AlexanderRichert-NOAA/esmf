% $Id: Array_implnotes.tex,v 1.1 2003/10/22 02:20:18 cdeluca Exp $

%\subsection{Design and Implementation Notes}

\begin{enumerate}

\item{\bf Array design.}
The purpose of the ESMF\_Array type is to fully describe a homogeneous
multidimensional array, possibly strided, so that it can be understood
and manipulated by multiple languages.   It fully describes the
relationship between array indices and the linear form of the array
in physical memory.  It describes all dimensions which are present;
there are no hidden or implied dimensions.  The first dimension specified
is always the one which varies fastest in linearized memory regardless of
interface language used to create or access the array.

The create routine in C++ will require the user to supply all values for
rank, shape, strides, etc because there are no language constructs which
allow a pointer to be queried for this information.

There are two types of create routine in F90.  One mimics the C++
interface and requires the user to supply all information.  
The second
type of routine, which is expected to be more useful and natural, is
the routine is simply passed an existing F90 array pointer.   Many of the
array attributes can be queried using language-defined functions which
should be portable to any F90 compiler.   Other attributes may need to
be discovered using compiler-dependent code.  The implementation approach
will be to write specific platform-dependent code for the most common
compilers and platforms, and then use less efficient or more indirect 
methods for obtaining this information which will be the default if 
no compiler-specific method has been written.

The Array type is defined at the base level because it is used by the
F90 interfaces in Fields, Grids, Route, Regrid to refer to data 
independent of Type/Kind/Rank differences.  This abstraction removes
the need for these other objects to provide
heavily-overloaded interface blocks to hide the number of
different data combinations supported by these routines.

The metadata in this class would be unnecessary for a straight
Fortran implementation since the language provides methods for querying
arrays for this information.  But for interoperability between different
versions of Fortran, different hardware architectures, 
and the C++ interfaces
it is necessary to keep the information in a format which can be
easily managed by the ESMF and not buried in the language layer.

The major challenge for the Array class implementation is that it 
contains user data which can be of many different types, kinds,
and ranks, each of which is a different type in Fortran 90 and
the language is strictly typechecked.

For the C++ interface polymorphism and templates can ease the burden of 
maintaining the interface; in Fortran the interface to the user
is simplified by using interface blocks but the number of internal
routines will be quite large.  Judicious use of the macro preprocessor
will allow generic routines to be expanded on a per-datatype basis.

Another way the data interfaces will be kept down to a manageable
size is to explicitly limit the number of supported user datatypes to:
\begin{description}
\item[integer*1/byte]
\item[integer*2/short]
\item[integer*4/int]
\item[integer*8/long]
\item[real*4/float]
\item[real*8/double]
\end{description}

\end{enumerate}





