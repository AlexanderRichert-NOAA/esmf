% $Id: CommMem_background.tex,v 1.3 2004/04/28 23:10:54 cdeluca Exp $

\section{Background}

The communication and memory kernels ({\bf \shortname}) elements of ESMF 
will provide a set of services to shield the upper levels of ESMF from certain
system level functions. The {\bf \shortname} services serve to help
manage the complexity of targeting a wide range of platforms.
System specific code specializations will largely be confined 
to the {\bf \shortname} layer.
Two particular areas are addressed by {\bf \shortname}

\paragraph {\bf Basic communication primitives} These are the building blocks for 
both the data communication between and within components and for any
control messages that components may need to exchange. Within {\bf
  \shortname} these can bind to different system level libraries (for
example MPI, IPC, mmap, shmem etc..) and include built-in,
low-overhead self-profiling capabilities.  Isolating the basic
communication functions of ESMF abstracts the communication primitives
to allow highly optimized code, targeted to specific platforms and to
specific framework functions, to be easily inserted into ESMF. The
{\bf \shortname} layer allows specialized forms of performance
critical primitives to be employed without impacting application code
or upper layers of ESMF code.  Channeling communication and ESMF
memory allocation through a single layer with built-in monitoring is
also designed to support detailed, application-level monitoring and
provide for partially or fully automated tuning.


\paragraph {\bf Basic memory management} Both user visible and internal
ESMF entities require memory resources. Within {\bf CMK} reservation,
 monitoring and release mechanisms will be provided for securing managing
memory resources. As with the communication primitives, the
ESMF memory primitives will include built-in, low-overhead self-profiling 
capabilities.


\subsection{Location}

The {\bf CMK} set of services is part of the ESMF infrastructure-utilities 
layer. Many ESMF higher level facilities will use {\bf \shortname} facilities.
The {\bf \shortname} services should rarely be directly called from user-level
component code.


\subsection{Scope}

The {\bf CMK} services will be limited to potentially performance
critical functions within ESMF and are intended to be used by internal
ESMF code. In the case of communication this includes messaging
associated with the ESMF {\bf Distributed Grid} {\it Halo} and {\it
  Transpose} operations. In the case of memory resources this includes
providing memory for ESMF {\bf Infrastructure - Fields and Grids} {\it
  field}, {\it field bundle}, {\it physical grid} and {\it distributed
  grid} data structures and objects.  The {\bf \shortname} facility is
not concerned with ESMF {\it partition} based parallelism as this does
not involve any communication between components or among data
domains.

User code containing its own message-passing primitives or shared
memory directives may co-exist with ESMF, but use of this layer
is recommended for fully realizing the benefits of ESMF.

\subsection{Related Material}

