% $Id$

%===============================================================================
% Requirements may be itemized under a main topic:
%===============================================================================
%===============================================================================

\req{Machine model}
For the most part, we consider the machine representation an implementation issue,
and the user will not explicitly create or destroy the machine model.
However, there are some queries specific to the machine model that a user may 
require or find useful.

\sreq{Query homogeneous platform attributes}

Primitives for determining key attributes of the computational
platform are required.
These include
\begin{itemize}
\item Processor, memory and communication groupings across the machine;
\item Shared and distributed memory profiles (rated latency and
  bandwidth for different pipes, size of available memory pools);
\item Availability of low-level APIs.
\end{itemize}

\begin{reqlist}
{\bf Priority:} 1 \\
{\bf Source:} MIT \\
{\bf Status:} Approved-1 \\
{\bf Verification:} Unit tests. \\
\end{reqlist}

\sreq{Query heterogeneous platform attributes}

Platform attributes for multiple platforms must be available for
heterogeneous applications.

\begin{reqlist}
{\bf Priority:} 3 \\
{\bf Source:} MIT \\
{\bf Status:} Proposed \\
{\bf Verification:} Unit tests. \\
\end{reqlist}

\sreq{Query PE attributes}

Primitives for determining key PE attributes are required.
These include
\begin{itemize}
\item Identifier of the PE.
\item Identifiers of the memory and communication groupings to which a 
PE belongs.
\item Identifiers of the physical resources (CPU, memory, storage) associated with a PE.
\item Identifiers of the virtual attributes associated with a PE e.g. coordinate in a layout.
\end{itemize}

In order to determine what resources to use to communicate between data domains 
the communication kernel needs to know information about the PE associated with the
data domains. The binding of data domains to PE's and memory is driven by the application
component and its included components. The communication kernel needs to determine 
what bindings were made.

\ssreq{Query PE ID} 
Each PE within an application shall have a
retrievable unique identifier.

\begin{reqlist}
{\bf Priority:} 1 \\
{\bf Source:}  CCSM-CPL, POP, CICE, PSAS, MIT, GFDL \\
{\bf Status:} Approved-1. \\
{\bf Verification:} Unit test.
\end{reqlist}

\ssreq{Query memory node ID}                         

Each block of PEs with equal flat access to a block of 
memory has a retrievable ID shared by those PEs. 

\begin{reqlist}
{\bf Priority:} 1 \\
{\bf Source:}  CCSM-CPL, POP, CICE, PSAS, MIT, GFDL \\
{\bf Status:} Approved-1. \\
{\bf Verification:} Unit test.
\end{reqlist}

\ssreq{Query addressable node ID}
A set of PEs capable of addressing the same block of physical
memory (e.g., via message passing) has a retrievable ID shared
by those PEs.

\begin{reqlist}
{\bf Priority:} 3 \\
{\bf Source:}  CCSM-CPL (desired), GFDL(desired)             \\
{\bf Status:} Approved-2. \\
{\bf Verification:} Unit test.
\end{reqlist}

\ssreq{Query processing node ID}
A set of PEs to which an operating system scheduler is capable 
of assigning a single job has a retrievable ID shared by those
PEs.
\begin{reqlist}
{\bf Priority:} 3 \\
{\bf Source:}  CCSM-CPL (desired), GFDL(desired) \\
{\bf Status:} Approved-2. \\
{\bf Verification:} Unit test. \\
{\bf Notes:}  Where the OS does not permit this identification, default
  to PE ID.
\end{reqlist}

\sreq{Query other machine attributes}
System communication, memory, and I/O attributes will be retrievable.
\begin{reqlist}
{\bf Priority:} 3 \\
{\bf Source:}  CCSM-CPL (desired), MIT(desired) \\
{\bf Status:} Approved-2. \\
{\bf Verification:} Unit test.
\end{reqlist}

\req{PE list}

\sreq{Creation}

\ssreq{Create from PE IDs}
It shall be possible to create a PE list based on a user-specified set
of PE IDs.

\begin{reqlist}
{\bf Priority:} 1\\
{\bf Source:}  CCSM-CPL, PSAS, MIT, GFDL \\
{\bf Status:} Approved-1. \\
{\bf Verification:} Unit test. \\ 
{\bf Notes:}  For example, ``Component requests PEs 0...31.''
\end{reqlist}

\ssreq{Create from length specification}
It shall be possible to create a PE list of a user-specified length.

\begin{reqlist}
{\bf Priority:} 1 \\
{\bf Source:}  CCSM-CPL, POP, CICE, PSAS, MIT, GFDL \\
{\bf Status:} Approved-1. \\
{\bf Verification:} Unit test.\\ 
{\bf Notes:}  For example, ``Component requests N PEs.'' Exception if N too
  large.
\end{reqlist}

\ssreq{Create list using only free PEs}
It shall be possible to create a PE list of a user-specified length
on which no other components have been scheduled.

\begin{reqlist}
  {\bf Priority:} 3 \\
  {\bf Source:}  CCSM-CPL (desired), PSAS (desired), MIT(desired), GFDL(desired)  \\
  {\bf Status:} Approved-2. \\
  {\bf Verification:} Unit test.\\
  {\bf Notes:} For example, ``Component requests N free PEs.''
  Exception if N too large.  Overlap with Components.
\end{reqlist}

\ssreq{Create based on memory node affinity}

It shall be possible to create PE lists based on memory node affinity.

\begin{reqlist}
{\bf Priority:} 3 \\
{\bf Source:}  CCSM-CPL (desired), MIT(desired), GFDL(desired) \\
{\bf Status:} Approved-2. \\
{\bf Verification:} Unit test. \\ 
{\bf Notes:}  For example, ``component requests N PEs sharing an memory 
node''. Throw exception if length is greater than number of PEs on the memory node. 
Other affinities may be required e.g. scratch space, memory, same CPU,
particular numerical library etc.....
\end{reqlist}

\ssreq{Create based on addressable node affinity}

It is possible to create PE lists based on addressable node affinity.

\begin{reqlist}
{\bf Priority:} 3 \\
{\bf Source:}  CCSM-CPL (desired), MIT(desired), GFDL(desired) \\
{\bf Status:} Approved-2. \\
{\bf Verification:} Unit test. \\ 
{\bf Notes:}  For example, ``component requests N PEs sharing an 
addressable node''. Throw exception if length is greater than number
of PEs on the addressable node.
\end{reqlist}

\ssreq{Create based on processing node affinity}

It shall be possible to create PE lists based on processing node affinity.

\begin{reqlist}
{\bf Priority:} 3 \\
{\bf Source:}  CCSM-CPL (desired), MIT(desired), GFDL(desired) \\
{\bf Status:} Approved-2. \\
{\bf Verification:} Unit test. \\ 
{\bf Notes:}  For example, ``component requests N PEs sharing a processing
node''.  Throw exception if length is greater than number of PEs on 
the processing node.  Overlap with Components.
\end{reqlist}

\ssreq{Create based on component affinity}

It shall be possible to create PE lists based on component affinity. This is
to assign components which are closely linked to be assigned
``nearby'' computational elements.

\begin{reqlist}
{\bf Priority:}  2 \\
{\bf Source:}  CCSM-CPL (desired), POP (desired), CICE (desired), PSAS (desired),
MIT(important - it could be hard to meet the 10\% degradation milestone without
this control somewhere), GFDL \\
{\bf Status:} Approved-2. \\
{\bf Verification:} Unit test. \\ 
{\bf Notes:}  For example, ``atmospheric physics component would like 
to be scheduled next to atmospheric dynamics''.  Overlap with Components.
\end{reqlist}

\ssreq{Create a compact PE list}

It is possible to request a PE list of given length whose associated
memory extent is optimally compact.

\begin{reqlist}
{\bf Priority:} 3 \\
{\bf Source:}  CCSM-CPL (desired), MIT(desired), GFDL(desired) \\
{\bf Status:} Approved-2. \\
{\bf Verification:} Unit test?\\ 
{\bf Notes:}  For example, ``give me N PEs distributed on the fewest possible
  memory nodes''. Other attributes could be optimized in deciding layout.
A general graph and edge optimizer could be used.  Overlap with Components.
\end{reqlist}

\ssreq{Create a subset of a PE list}

It is possible to request a PE list subset of an existing PE list.

\begin{reqlist}
{\bf Priority:} 2 \\
{\bf Source:}  CCSM-CPL (desired), PSAS (desired), MIT, GFDL \\
{\bf Status:} Approved-2. \\
{\bf Verification:} Unit test.\\ 
{\bf Notes:}  ``Of the 80p assigned to ocean+atmosphere, I'd like 32
  for the atmosphere''.  Overlap with Components.
\end{reqlist}

\ssreq{Create a complementary PE lists}

It is possible to request a complementary subset of an existing spawned
PE list with respect to its parent.

\begin{reqlist}
{\bf Priority:}  2 \\
{\bf Source:}  CCSM-CPL (desired), MIT, GFDL \\
{\bf Status:} Approved-2. \\
{\bf Verification:} Unit test.\\ 
{\bf Notes:}  ``...and the other 48 for the ocean''.
  Required for concurrent scheduling.  Overlap with Components.
\end{reqlist}

\sreq{Deletion}
It shall be possible to delete a PE list.

\begin{reqlist}
{\bf Priority:} 1 \\
{\bf Source:}  \\
{\bf Status:} Approved-1. \\
{\bf Verification:} Unit test.\\ 
\end{reqlist}

\sreq{Modify PE list request}

It is possible for a component to make any of the above requests at
any time.

\begin{reqlist}
{\bf Priority:} 3 \\
{\bf Source:}  CCSM-CPL (desired), MIT(desired), GFDL(desired) \\
{\bf Status:} Proposed. \\
{\bf Verification:} System test. \\
{\bf Notes:} overlap with Components.
\end{reqlist}

\req{DELayout}

\sreq{Create layout from PE list}
The ESMF will compute and store a layout given a PE list, number of dimensions, 
and and optional hints about the communication patterns (such as the 
relative frequency of communication for different dimensions).
\begin{reqlist}
  {\bf Priority:} 1 \\
  {\bf Source:}  MIT \\
  {\bf Status:}  \\
  {\bf Verification:} Unit tests, EVA tests \\
\end{reqlist}

\sreq{Create layout from distributed grid}
The ESMF will compute and store a layout given one or more 
distributed grids or other high-level ESMF objects and optional 
hints about communication patterns (such as the relative 
frequency of communication for different dimensions).
\begin{reqlist}
  {\bf Priority:} 1 \\
  {\bf Source:}  MIT \\
  {\bf Status:}  \\
  {\bf Verification:} Unit tests, EVA tests \\
\end{reqlist}

\sreq{Automatic binding of tasks to resources}
Support for automated compute resource binding of tasks should
  be provided.
\begin{reqlist}
{\bf Priority:} 3 \\
{\bf Source:}  CCSM-CPL (desired), MIT (desired), NSIPP, GFDL \\
{\bf Status:} Proposed. \\
{\bf Verification:} System test.\\ 
{\bf Notes:}  Automated binding could be based on estimates and follow
  certain optimality rules that are based of both knowledge of compute
  resources and of the distribution of grids over tasks within
  multiple, interacting components.  Alternatively automated binding
  could be based on runtime measurement.  Overlap with Components.
\end{reqlist}

\sreq{Mapped memory region for data sharing}.

It shall be possible to request a region of shared address space to support
cross-component communication.

\begin{reqlist}
  {\bf Priority:} 1 \\
  {\bf Source:}  MIT \\
  {\bf Status:}  \\
  {\bf Verification:} Unit tests, EVA tests \\
  {\bf Notes:} The optimal ``memory strategy'' might include hints on
  whether to allocate memory from a pool shared between PEs. The
  optimal communication design might include optimal assignment of
  domains to PEs, advocating the use of shared memory where
  appropriate, and a pre-computed and stored buffering pattern where
  message-passing is indicated.  
\end{reqlist}

\sreq{Delete layout}
It shall be possible to delete a layout, freeing the resources that 
are associated with it.

\begin{reqlist}
  {\bf Priority:} 1 \\
  {\bf Source:}  MIT \\
  {\bf Status:}  \\
  {\bf Verification:} Unit tests, EVA tests \\
\end{reqlist}

\req{Communication and Memory Kernel Setup}

Setup functions are run during the initialization phase of an ESMF
application and its components. They are not required to be scalable
or coded for optimal performance. They may require knowledge of (or
the ability to parse) higher-level constructs.

\sreq{Setup communication map for a remapping}

Given a remapping between distributed grids, the ESMF setup
phase will compute and store an optimal memory strategy and
communication pattern. Domains may be reassigned for the individual
distributed grids at this stage to optimize the layouts for remapping.

\begin{reqlist}
{\bf Priority:} 1 \\
{\bf Source:}  MIT \\
{\bf Status:}  \\
{\bf Verification:} Unit tests, EVA tests \\
\end{reqlist}

\sreq{Setup communication map for grid collectives}

Collective operations (e.g global reductions, global gather, global
barrier) have been defined for a distributed grid. Different
buffering and communication strategies may have to be used for
default, performance-critical, and bit-reproducible versions of these
routines.

\ssreq{Setup performance-critical collectives}

Given a distributed grid, the ESMF setup
phase will compute and store a communication map for
performance-critical grid collectives.

\begin{reqlist}
{\bf Priority:} 1 \\
{\bf Source:}  MIT \\
{\bf Status:}  \\
{\bf Verification:} Unit tests, EVA tests \\
\end{reqlist}

\ssreq{Setup bit-reproducible collectives}

Given a distributed grid, the ESMF setup
phase will compute and store a communication map for
bit-reproducible grid collectives.

\begin{reqlist}
{\bf Priority:} Priority 1 \\
{\bf Source:}  MIT \\
{\bf Status:}  \\
{\bf Verification:} Unit tests, EVA tests \\
\end{reqlist}

\sreq{Setup performance profiling}

Communication and memory operations are all instrumented for profiling. 
The setup phase will include mechanisms for
turning on and off the available options for profiling, and assigning
a reporting mechanism (e.g., write to stdout).

\begin{reqlist}
{\bf Priority:} Priority 1 \\
{\bf Source:}  MIT \\
{\bf Status:}  \\
{\bf Verification:} Unit tests, EVA tests \\
\end{reqlist}

\begin{reqlist}
{\bf Priority:} 1 \\
{\bf Source:}  MIT \\
{\bf Status:}  \\
{\bf Verification:} Unit tests, EVA tests \\
\end{reqlist}

\req{Communication primitives}

\sreq{Point-to-point primitives}

\ssreq{Non-blocking send}
A performance-critical non-blocking send operation to a specified
target is required.
\begin{reqlist}
{\bf Priority:} 3 \\
{\bf Source:}  MIT, maybe GFDL. \\
{\bf Status:}  \\
{\bf Verification:} Unit test. \\
{\bf Notes:}  May end up being implemented as private method to for performance
optimization.
\end{reqlist}

\ssreq{Non-blocking receive}
A performance-critical non-blocking receive operation from a specified
source is required.
\begin{reqlist}
{\bf Priority:} 3 \\
{\bf Source:}  MIT, maybe GFDL. \\
{\bf Status:}  \\
{\bf Verification:} Unit test. 
\end{reqlist}

\ssreq{Non-blocking receive from unspecific source}
A performance-critical non-blocking receive operation from an unspecified
source is required.
\begin{reqlist}
{\bf Priority:} 3 \\
{\bf Source:}  MIT, maybe GFDL. \\
{\bf Status:}  \\
{\bf Verification:} Unit test. 
\end{reqlist}

\ssreq{Multiple outstanding messages}
Both send and receive phases must be capable of tolerating multiple
outstanding messages between source-target pairs, up to some specified
queue limit.
\begin{reqlist}
{\bf Priority:} 3 \\
{\bf Source:}  MIT, maybe GFDL. \\
{\bf Status:}  \\
{\bf Verification:} Unit test.
\end{reqlist}

\ssreq{Wait}
A performance-critical wait operation for a specified message must be present.
\begin{reqlist}
{\bf Priority:} 3 \\
{\bf Source:}  MIT, maybe GFDL. \\
{\bf Status:}  \\
{\bf Verification:} Unit test. 
\end{reqlist}

\ssreq{Wait all}
A performance-critical wait operation for all outstanding messages must be present.
\begin{reqlist}
{\bf Priority:} 3 \\
{\bf Source:}  MIT, maybe GFDL. \\
{\bf Status:}  \\
{\bf Verification:} Unit test. 
\end{reqlist}

\ssreq{Barrier}
A point-to-point barrier operation is required.
\begin{reqlist}
{\bf Priority:} 3 \\
{\bf Source:}  MIT, maybe GFDL \\
{\bf Status:}  \\
{\bf Verification:} Unit test.
\end{reqlist}

\sreq{Collective primitives}

\ssreq{All-to-all}
A general, non-blocking all-to-all collective communication operation
shall be the basic ESMF message-passing pathway.
\begin{reqlist}
{\bf Priority:} 1 \\
{\bf Source:}  MIT \\
{\bf Status:}  \\
{\bf Verification:} Unit tests, EVA tests 
\end{reqlist}

\ssreq{Multiple outstanding messages}

Both send and receive phases must be capable of tolerating multiple
outstanding messages between source-target pairs, up to some specified
queue limit.
\begin{reqlist}
{\bf Priority:} 1 \\
{\bf Source:}  MIT, GFDL, CCSM, DAO \\
{\bf Status:}  \\
{\bf Verification:} Unit tests, EVA tests 
\end{reqlist}

\ssreq{Collective wait}

There shall be a wait operation for the basic ESMF collective
communication pathway to ensure all outstanding \emph{local} messages
are completed.
\begin{reqlist}
{\bf Priority:} 1 \\
{\bf Source:}  MIT, GFDL, CCSM, DAO \\
{\bf Status:}  \\
{\bf Verification:} Unit tests, EVA tests 
\end{reqlist}

\ssreq{Broadcast}

A performance-critical broadcast operation is required.
\begin{reqlist}
{\bf Priority:} 1 \\
{\bf Source:}  MIT, GFDL, CCSM, DAO, NSIPP, POP \\
{\bf Status:}  \\
{\bf Verification:} Unit tests, EVA tests 
\end{reqlist}

\ssreq{Sum}
A performance-critical sum operation is required.
\begin{reqlist}
{\bf Priority:} 1 \\
{\bf Source:}  MIT, GDFL, CCSM, DAO, NSIPP, POP \\
{\bf Status:}  \\
{\bf Verification:} Unit tests, EVA tests 
\end{reqlist}

\ssreq{Barrier}
A global barrier operation is required to be provided by ESMF.
Priority:  1
Status:  Approved-1
Notes:  Use in performance-critical code sections is not recommended.

\ssreq{Min-max}
Priority:  1
Status:  Approved-1
Notes:  Other global reduction operations are not required to be 
performance-critical.

\ssreq{All gather}
A performance-critical operation in which every process receives the result 
is required.
\begin{reqlist}
{\bf Priority:} 1 \\
{\bf Source:}  MIT \\
{\bf Status:} Approved-1 \\
{\bf Verification:} Unit tests, EVA tests 
\end{reqlist}

\ssreq{Scatter}
A performance-critical scatter operation is required.

\begin{reqlist}
{\bf Priority:} 1 \\
{\bf Source:}  MIT \\
{\bf Status:} Approved-1 \\
{\bf Verification:} Unit tests, EVA tests 
\end{reqlist}

\sreq{Instrumenting communications for performance profiling}

All the communication operations listed above will be instrumented by
ESMF for performance using very lightweight mechanisms. Various times
will be recorded at runtime, and results collated in a post-execution
phase. Final output will show statistics across PEs, as well as
optionally on a per-message basis (though for climate runs the
internal storage required for this may be prohibitive).

\ssreq{Time spent in communication primitives}
\ssreq{Time spent in collectives}
\ssreq{Time spent waiting for message completion}
\ssreq{Message lengths}
\ssreq{Internal buffer usage}
\ssreq{Use of underlying APIs}
\ssreq{Time spent at global barriers}

\begin{reqlist}
{\bf Priority:} 2 \\
{\bf Source:}  MIT \\
{\bf Status:} Approved-1 \\
{\bf Verification:} Unit tests, EVA tests 
{\bf Notes:} Good candidate for deferral.
\end{reqlist}

\ssreq{Calipers}

It shall be possible during the setup phase to set caliper points
around a code section.  The number of
caliper pairs is limited to a pre-specified value.

\begin{reqlist}
{\bf Priority:} 2 \\
{\bf Source:}  MIT \\
{\bf Status:} Approved-1 \\
{\bf Verification:} Unit tests, EVA tests \\
{\bf Notes:} Good candidate for deferral.
\end{reqlist}

\sreq{Implementation libraries}

\ssreq{Default implementation}

The default implementation for message-passing shall be MPI 1.2 and for
shared memory OpenMP 1.0.  Other bindings will be made available on
platforms where they show advantages in terms of performance, or code
clarity at little cost. The implementation priorities for non-default
bindings will in general be lower (later). ESMF is not required to run
in parallel on platforms which support neither MPI 1.2 nor OpenMP 1.0.

The requirement for other implementations is really asking for a certain
implementation. Arguably the requirement could be stated as ``ESMF
performance and resource utilization on a system which has shmem
pointers available must be competitive to an approach that binds ESMF
communication directly to shmem pointers reads and writes (with
appropriate memory instructions inserted to reconcile memory
inconsistency).''
Notes:  When Open-MP2 is available that will be supported.

\ssreq{System V IPC}

The System V IPC library for message passing shall be optionally
available on appropriate platforms.

\begin{reqlist}
{\bf Priority:} 2 \\
{\bf Source:}  MIT, NSIPP \\
{\bf Status:} Approved-2 \\
{\bf Verification:} Unit test. \\
{\bf Notes:} Need input from MIT and others.
\end{reqlist}

\ssreq{SHMEM}

The SHMEM library for message passing shall be optionally
available on appropriate platforms.

\begin{reqlist}
{\bf Priority:} 1 \\
{\bf Source:}  GFDL, NSIPP \\
{\bf Status:} Approved-2 \\
{\bf Verification:} Unit test. \\
{\bf Notes:} Implementation use/avoid strided shmem? 
\end{reqlist}

\ssreq{Posix threads}

Posix bindings for message passing shall be optionally
available on appropriate platforms.  This implementation
is not guaranteed to work with other shared memory mechanisms.

\begin{reqlist}
{\bf Priority:} 3 \\
{\bf Source:}  MIT \\
{\bf Status:} Approved-2 \\
{\bf Verification:} Unit test. \\ 
\end{reqlist}

\ssreq{MPI 2.0}

MPI 2.0 bindings for message passing shall be optionally
available on appropriate platforms.

\begin{reqlist}
{\bf Priority:} 2 \\
{\bf Source:}  Desired by DAO \\
{\bf Status:} Approved-2 \\
{\bf Verification:} Unit test. \\ 
\end{reqlist}

\ssreq{Extensibility}

Extensibility to allow bindings of performance-critical communication
primitives to other underlying libraries.

\begin{reqlist}
{\bf Priority:} 1 \\
{\bf Source:}  MIT \\
{\bf Status:} Approved-2 \\
{\bf Verification:} Unit test. \\ 
\end{reqlist}

\sreq{Interoperability}

\sreq{Default implementation}

\ssreq{OpenMP and MPI}
We guarantee that the default implementation will interoperate correctly
with codes in which users have inserted their own MPI calls or OpenMP 
calls, or both.  

\begin{reqlist}
{\bf Priority:} 1 \\
{\bf Source:}  MIT \\
{\bf Status:} Approved-1 \\
{\bf Verification:} Unit test. \\ 
\end{reqlist}

\ssreq{Shmem}
The default implementation will work with shmem.

\begin{reqlist}
{\bf Priority:} 2 \\
{\bf Source:}  MIT \\
{\bf Status:} Approved-2 \\
{\bf Verification:} Unit test. \\ 
\end{reqlist}

\ssreq{Other libraries}
The default implementation shall interoperate with other communication
libraries, such as System V.

\begin{reqlist}
{\bf Priority:} 3 \\
{\bf Source:}  MIT \\
{\bf Status:} Proposed. \\
{\bf Verification:} Unit test. \\ 
\end{reqlist}

\sreq{Other ESMF communication library implementations}
Other ESMF communication library implementations shall ineroperate with codes
in which users have inserted call from the library used for the ESMF implementation.

\begin{reqlist}
{\bf Priority:} 1 \\
{\bf Source:}  MIT \\
{\bf Status:} Proposed. \\
{\bf Verification:} Unit test. \\ 
\end{reqlist}

\req{Memory management}

For various operations being implemented for ESMF, there is need to
control where memory is allocated (e.g heap, stack, shared arena,
etc). For this reason we will define some apparatus for managing
ESMF memory which will have the required attributes for optimal
use of ESMF primitives.  

\sreq{Provide ESMF heap memory}

It shall be possible during the setup phase to declare the size of the
ESMF memory pool that is anticipated for use. Subsequent requests for
ESMF memory will operate out of this pool without needing system
calls.
\begin{reqlist}
{\bf Priority:} 3 \\
{\bf Source:}  \\
{\bf Status:} Proposed. \\
{\bf Verification:} Unit test. \\ 
\end{reqlist}

\sreq{Provide ESMF stack memory}

It shall be possible during the setup phase to declare the size of the
ESMF memory pool that is anticipated for use. Subsequent requests for
ESMF memory will operate out of this pool without needing system
calls.
\begin{reqlist}
{\bf Priority:} 3 \\
{\bf Source:} GFDL \\
{\bf Status:} Proposed. \\
{\bf Verification:} Unit test. \\ 
\end{reqlist}

\sreq{Native memory bindings}
Objects whose memory was allocated using language-native allocation
methods (e.g fortran \texttt{allocate}, C \texttt{malloc}) are not
required to deliver the same performance as with ESMF memory.
\begin{reqlist}
{\bf Priority:} 1 \\
{\bf Source:} GFDL \\
{\bf Status:} Proposed. \\
{\bf Verification:} Unit test. \\ 
\end{reqlist}

\req{Heart-beat}

Support and adherence to a heart-beat function is required.

\begin{reqlist}
  {\bf Priority:} 1 \\
  {\bf Source:}  MIT \\
  {\bf Status:}  \\
  {\bf Verification:} Unit tests, EVA tests \\
  {\bf Notes:} Some mechanism should be available to detect the
  failure or dead-lock of one or more communication end-points. This
  can be queried at the application level. When this happens the ESMF
  application should either take a recovery action or shutdown. 
  This should include appropriate freeing of system resources 
  reserved for {\bf \shortname} purposes.
  Need input from Chris -- may be tricky to implement. 
\end{reqlist}

\req{High performance adjoints for communication methods}

The available communication primitives shall not preclude high-performance 
adjoints for distributed grid methods.

\begin{reqlist}
  {\bf Priority:} 1 \\
  {\bf Source:}  MIT \\
  {\bf Status:}  \\
  {\bf Verification:} Unit tests. \\
\end{reqlist}










