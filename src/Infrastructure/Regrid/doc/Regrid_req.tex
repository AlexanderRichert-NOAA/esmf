% $Id: Regrid_req.tex,v 1.10 2002/05/11 01:39:02 mjsuarez Exp $

%===============================================================================
% Requirements may be itemized under a main topic:
%===============================================================================
%===============================================================================
\req{General Regridding Requirements}

%-------------------------------------------------------------------------------

The following are general requirements for regridding operations and are in
addition to the applicable general ESMF requirements (see ESMF General
Requirements document).

%-------------------------------------------------------------------------------
\sreq{Creation}

Components must be able to create a regridding and initialize
various time-independent regridding quantities.

\begin{reqlist}
{\bf Priority:} 1 \\
{\bf Source:} All codes will require this. \\
{\bf Status:} Proposed \\
{\bf Verification:} Unit test \\
{\bf Notes:} This function will, in many cases, be computing
             regridding weights and initializing various
             communication information for performing regridding.
\end{reqlist}

%-------------------------------------------------------------------------------
\sreq{Destruction}

Components must be able to destroy regriddings to free up memory.

\begin{reqlist}
{\bf Priority:} 1 \\
{\bf Source:} POP, CICE, CAM desired, NSIPP \\
{\bf Status:} Proposed \\
{\bf Verification:} Unit test \\
{\bf Notes:} 
\end{reqlist}

%-------------------------------------------------------------------------------
\sreq{Query}

Components must be able to query various properties of a regridding.

\begin{reqlist}
{\bf Priority:} 2 \\
{\bf Source:}  \\
{\bf Status:} Proposed \\
{\bf Verification:} Unit test \\
{\bf Notes:} Ideally, components should not need to access internal
             regridding fields.  But it might be useful to access some
             aspects for error checking, optimization of data layout or
             renormalization.  Exact query functions will be determined
             after design of structure is determined.
\end{reqlist}

%-------------------------------------------------------------------------------
\sreq{Change}

Components should be able to change individual properties of a
regridding.  Examples might include adjusting regridding weights to 
renormalize or to adapt to dynamic area fractions (like ice fraction).

\begin{reqlist}
{\bf Priority:} 3 \\
{\bf Source:}  \\
{\bf Status:} Proposed \\
{\bf Verification:} Unit test \\
{\bf Notes:} In general, this should be strongly discouraged as it may
             affect regridding properties like conservation.  The need
             for such a function will be determined by applications.
\end{reqlist}

%-------------------------------------------------------------------------------
\sreq{Reading}

Components may be able to read regridding information from a file.

\begin{reqlist}
{\bf Priority:} 1 \\
{\bf Source:} POP, CICE, CAM desired, NSIPP \\
{\bf Status:} Proposed \\
{\bf Verification:} Unit test \\
{\bf Notes:} Useful if creating regriddings is time-consuming to avoid
             start-up costs. Also permits off-line computation of regridding.
\end{reqlist}

%-------------------------------------------------------------------------------
\sreq{Writing}

Components may be able to write regridding information to a file.

\begin{reqlist}
{\bf Priority:} 1 \\
{\bf Source:} POP, CICE, CAM desired, NSIPP \\
{\bf Status:} Proposed \\
{\bf Verification:} Unit test \\
{\bf Notes:} Useful if creating regriddings is time-consuming - compute once
             and save for later re-use.  Also useful for any off-line
             use of same regridding info.
\end{reqlist}

%-------------------------------------------------------------------------------
\sreq{Support for ESMF Grids}

Regridding operations must be available for all supported ESMF grids,
including ungridded data.  Not all regridding operations are appropriate for
all grid types; restrictions will be noted in individual requirements.

\begin{reqlist}
{\bf Priority:} 1 \\
{\bf Source:} All codes require this. \\
{\bf Status:} Proposed \\
{\bf Verification:} System test \\
{\bf Notes:} 
\end{reqlist}

%-------------------------------------------------------------------------------
\sreq{Multiple fields}

Regridding of multiple fields (bundles of fields) with
a single call must be supported.

\begin{reqlist}
{\bf Priority:} 1  \\
{\bf Source:} CCSM requires \\
{\bf Status:} Proposed \\
{\bf Verification:} Unit test \\
{\bf Notes:} This should be supplied for efficiency, but may not
             be required to achieve functionality.
\end{reqlist}

\ssreq{Interface requires only data arrays}

An interface requiring only data arrays shall be
provided.  Such an interface would not require the overhead
of a full field structure and is supplied for efficiency.

\begin{reqlist}
{\bf Priority:} 2 \\
{\bf Source:} POP, CICE, CAM desired \\
{\bf Status:} Proposed \\
{\bf Verification:} Unit test \\
{\bf Notes:} Field info may still be used for creation of the
             regrid object.
\end{reqlist}

\ssreq{Consistency of field bundles}

The regridding function must check if the input field bundle and output field
bundle are consistent with each other, particulary in number of fields, name of
fields and grids on which the fields are placed.

\begin{reqlist}
{\bf Priority:} 2 \\
{\bf Source:} POP, CICE, CAM desired, NSIPP \\
{\bf Status:} Proposed \\
{\bf Verification:} Unit test \\
{\bf Notes:} A rudimentary error check, but would probably rely on
             consistent naming convention for fields?
\end{reqlist}

%-------------------------------------------------------------------------------
\sreq{Multiple methods per grid pair}

It shall be possible to create more than one regridding for a given grid
pair.

\begin{reqlist}
{\bf Priority:} 1 \\
{\bf Source:} POP, CICE, CAM required \\
{\bf Status:} Proposed \\
{\bf Verification:} Unit test \\
{\bf Notes:} Both non-conservative and conservative methods will be needed
             between the same two grids.
\end{reqlist}

%-------------------------------------------------------------------------------
\sreq{Consistency of coordinates}

Regridding will assume source and destination grids will be
in compatible coordinate systems.  No knowledge of
the projection used or the physics of the coordinate are
required for performing a regridding.  For example, if a 
horizontal grid is in Caresian coordinates, the second grid 
must also be in Cartesian coordinates with the same origin.
Similarly, if the coordinates of one grid are in spherical 
coordinates, the second grid must also use spherical 
coordinates.  The restriction
also applies to vertical coordinates where the regridding
will not be expected to know how to transform between
two different coordinate choices (eg pressure to isentropic).

\begin{reqlist}
{\bf Priority:} 1 \\
{\bf Source:} Required by all. \\
{\bf Status:} Proposed \\
{\bf Verification:} Code inspection  \\
{\bf Notes:} Exceptions to this are permitted if the user
             supplies the regridding routine (see later
             requirement on user-supplied regridding).
\end{reqlist}

\ssreq{Consistency of coordinates check}

The regridding function must check that the grids are in fact consistent
with each other.

\begin{reqlist}
{\bf Priority:} 2 \\
{\bf Source:} Required by all. \\
{\bf Status:} Proposed \\
{\bf Verification:} Code inspection  \\
{\bf Notes:} Will require some standard nomenclature for grid attributes,
             particularly for vertical grids.
\end{reqlist}

%-------------------------------------------------------------------------------
\sreq{Interpolation adjoints}

Adjoints shall be supplied for regridding methods when possible.  This is 
generally possible for regriddings that are independent of the field being 
regridded (see following requirement) and that can be cast as a linear 
operator (eg matrix multiplication).  Methods where adjoints are absolutely 
required have been so noted within their own respective decriptions.

\begin{reqlist}
{\bf Priority:} 2 \\
{\bf Source:} PSAS, NSIPP \\
{\bf Status:} Proposed \\
{\bf Verification:} Unit test \\
{\bf Notes:} Needed by PSAS, but not milestone
\end{reqlist}

%-------------------------------------------------------------------------------
\sreq{Masked regridding}

It shall be possible to restrict the regridding to parts of a grid through 
the use of a mask.  Note that use of a mask is inappropriate for some methods 
(eg spectral transforms) and will not be supported for those methods.

\begin{reqlist}
{\bf Priority:} 1 \\
{\bf Source:} CCSM required \\
              WRF required, NSIPP \\
{\bf Status:} Proposed \\
{\bf Verification:} Unit test \\
{\bf Notes:} 
\end{reqlist}

\ssreq{Mask consistency}

If masks are supplied for both source and destination grids, a
method for checking consistency of those masks must be supplied.
Alternatively, a convention for resolving mask conflicts must
be determined (eg source grid is ``master'').

\begin{reqlist}
{\bf Priority:} 1 \\
{\bf Source:} CCSM required \\
              WRF required, NSIPP \\
{\bf Status:} Proposed \\
{\bf Verification:} Unit test \\
{\bf Notes:} 
\end{reqlist}

%-------------------------------------------------------------------------------
\sreq{Independence of field}

Whenever possible, regridding should be formulated to be
independent of the field being regridded.  This requirement exists to
aid the creation of an adjoint, to enable pre-computation of regridding
weights and to enable re-use of regridding information for multiple
fields.

\begin{reqlist}
{\bf Priority:} 1 \\
{\bf Source:} Required by all. \\
{\bf Status:} Proposed \\
{\bf Verification:} Code inspection  \\
{\bf Notes:} 
\end{reqlist}

%-------------------------------------------------------------------------------
\sreq{Dependence of field}

For regridding schemes which might require field information,
the required field information can be passed as arguments.
Some higher-order regridding schemes require information on the
gradient or other moments of a field.  In such cases, this 
supplemental field information must be computed by the component 
and passed to the regridding function so that the regridding does 
not require detailed knowledge of operators or grid topology on every
supported grid or field.

\begin{reqlist}
{\bf Priority:} 1 \\
{\bf Source:} Required by all. \\
{\bf Status:} Proposed \\
{\bf Verification:} Code inspection  \\
{\bf Notes:} This requirement could also be satisfied by a later requirement
             for user-supplied regridding routines.
\end{reqlist}

%===============================================================================
\req{Regridding algorithms}
%-------------------------------------------------------------------------------

This section contains requirements on regridding algorithms themselves.

%-------------------------------------------------------------------------------
\sreq{Conservation}

Regridding methods must be supplied which are
\htmlref{conservative}{glos:conservation}.  Where possible,
higher-order conservative methods should also be supplied.  This requirement
applies only to ESMF grids which have an area (2-d), volume (3-d) or
linear region (1-d) associated with them such that conservation is well
defined.

\begin{reqlist}
{\bf Priority:} 1 \\
{\bf Source:} POP,CICE,CAM required, NSIPP \\
{\bf Status:} Proposed \\
{\bf Verification:} Unit test \\
{\bf Notes:} Methods exist for both first and second-order
             conservative schemes in 1-d and 2-d \cite{Jones1999}.
             Conservative methods for 3-d field (eg Monte Carlo
             or 3-d extensions to the above methods) are more difficult
             and may have a lower priority.
             High-order conservative schemes are more expensive and
             no schemes higher than second-order have been implemented.

             MI - Some grids may have a higher-order integration method
             associated with them (overlappping functions as weights),
             potentially making conservative regridding difficult and expensive.
\end{reqlist}

\ssreq{Verification of conservation}

A method for verifying conservation must be supplied.

\begin{reqlist}
{\bf Priority:} 2 \\
{\bf Source:} \\
{\bf Status:} Proposed \\
{\bf Verification:} Unit test \\
{\bf Notes:} For error checking and testing.
\end{reqlist}

%-------------------------------------------------------------------------------
\sreq{Monotonicity}

\htmlref{Monotone}{glos:monotone} regridding methods must be supplied.

\begin{reqlist}
{\bf Priority:} 2 \\
{\bf Source:} CAM-FV \\
{\bf Status:} Proposed \\
{\bf Verification:} Unit test \\
{\bf Notes:} CAM-FV vertical remapping requires this.
	     Biogeochemical models may need this.  First-order
             conservative schemes are generally monotone by
             construction, so this could be satisfied by the
             conservation requirement for gridded data.
             1-d monotone schemes are required for some hybrid
             and Lagrangian vertical coordinate schemes.
\end{reqlist}


%-------------------------------------------------------------------------------
\sreq{Higher-order schemes}

Regridding methods which are higher than first \htmlref{order}{glos:order}
must be supplied.  This is required for preventing
``patchwork'' patterns when regridding from coarse to fine
grids and for preventing discontinuities in gradients of
regridded fields.

\begin{reqlist}
{\bf Priority:} 1 \\
{\bf Source:}  POP, CICE, CAM required. \\
{\bf Status:} Proposed \\
{\bf Verification:} Unit test \\
{\bf Notes:} This will require either internal approximations to
             gradients (eg bilinear, bicubic, trilinear) or will require the
             user to pass gradient information (eg second-order conservative
             methods).  See requirements on field dependence.  Also, this
             requirement can be in conflict with monotonicity requirements.
\end{reqlist}

%-------------------------------------------------------------------------------
\sreq{Vector fields in physical space}

Regridding methods must be available for regridding a horizontal
vector field with components aligned with physical directions
(eg zonal-meridional or x-y), where the physical direction may be
inferred by the grid type or specified by user.

\begin{reqlist}
{\bf Priority:} 1 \\
{\bf Source:}  POP, CICE, CAM required, NSIPP \\
{\bf Status:} Proposed \\
{\bf Verification:} Unit test \\
{\bf Notes:} In spherical coordinates, meridional velocity components
             may be improperly handled except in simple latitude-longitude
             grid combinations.
\end{reqlist}

%-------------------------------------------------------------------------------
\sreq{Vector fields in logical space}

Regridding methods must be available for regridding a horizontal
vector field with components aligned along grid logical directions.
Logical directions here refer to directions parallel and perpendicular
to cell sides.  Such a method would correctly handle flow through 
coordinate singularities such as the poles in spherical coordinates.

\begin{reqlist}
{\bf Priority:} 2 \\
{\bf Source:}  POP, CICE (CCSM) desired, NSIPP \\
{\bf Status:} Proposed \\
{\bf Verification:} Unit test \\
{\bf Notes:} Currently working on whether it is even possible to do this
             in all cases, but willing to make the attempt.  Conversion to
             3-d Cartesian components during the remapping is one option.
\end{reqlist}

%-------------------------------------------------------------------------------
\sreq{Regridding based on index space}

Methods must be available for regridding based
only on logical indices of grid points and thus only on DistGrid information.  
Such a function is useful for nested grid and multi-grid applications where 
no physical grid information is required for creating the regridding.

\begin{reqlist}
{\bf Priority:} 1 \\
{\bf Source:}  WRF required, NSIPP \\
{\bf Status:} Proposed \\
{\bf Verification:} Unit test \\
{\bf Notes:} Will need a general way to specify stencils
\end{reqlist}

\ssreq{Index space changes}

A method must be supplied for rapidly changing the
regridding in cases where indices of one grid shift in relation 
to the other grid (eg as a nested grid moves in relation to its 
parent).  The regridding in this case would be utilizing the
same stencil and weights; only the addresses of the grid points
would shift.  Because of the simple nature of this operation,
this requirement provides an efficient short-cut, avoiding
re-creating a regridding using calls to create or destroy methods.

\begin{reqlist}
{\bf Priority:} 1 \\
{\bf Source:}  WRF required \\
{\bf Status:} Proposed \\
{\bf Verification:} Unit test \\
{\bf Notes:} A requirement for this exists in DistGrid, so
             Regrid would utilize the DistGrid functionality to 
             accomplish this.
\end{reqlist}

%-------------------------------------------------------------------------------
\sreq{Fourier transforms}

Methods shall be supplied for regridding between physical space and
Fourier space.  The adjoints shall also be supplied.  Ordering in
Fourier space will be defined by DistGrid.  This requirement applies only
to grids consistent with the Fourier transform (eg lat/lon grids,
reduced grids, spectral elements, etc.).

\begin{reqlist}
{\bf Priority:} 1 \\
{\bf Source:}  NCEP-GSM, NCEP-SSI (milestone), NSIPP \\
{\bf Status:} Proposed \\
{\bf Verification:} Unit test \\
{\bf Notes:} 
\end{reqlist}

\ssreq{Return types for Fourier modes}

Results of Fourier transforms can be returned as either
complex numbers or as real numbers in a specified order.

\begin{reqlist}
{\bf Priority:} 1 \\
{\bf Source:}  \\
{\bf Status:} Proposed \\
{\bf Verification:} Unit test \\
{\bf Notes:} 
\end{reqlist}

\ssreq{Parallel implementations}

Distributed FFT algorithms will be supported for a limited
number of specific configurations.  Note that a transpose
algorithm (in which a local serial transform is combined
with data transposes to redistribute data) will always
be supported for the general case.

\begin{reqlist}
{\bf Priority:} 1 \\
{\bf Source:}  \\
{\bf Status:} Proposed \\
{\bf Verification:} Unit test \\
{\bf Notes:} Distributed algorithms make assumptions about
             the placement of both input and output data and
             support for all possibilities may be prohibitive.
\end{reqlist}

%-------------------------------------------------------------------------------
\sreq{Legendre transforms}

Methods shall be supplied for regridding between spectral space and
Fourier space.  The adjoints shall also be supplied.  This requirement
applies only to grids consistent with the Legendre transform (the
data must be located at appropriate quadrature points).

\begin{reqlist}
{\bf Priority:} 1 \\
{\bf Source:}  NCEP-GSM, NCEP-SSI (milestone), NSIPP \\
{\bf Status:} Proposed \\
{\bf Verification:} Unit test \\
{\bf Notes:}
\end{reqlist}

\ssreq{Data types for Fourier modes}

Legendre transforms must support Fourier modes stored as either
complex numbers or as real numbers in a specified order.

\begin{reqlist}
{\bf Priority:} 1 \\
{\bf Source:}  \\
{\bf Status:} Proposed \\
{\bf Verification:} Unit test \\
{\bf Notes:}  Companion to the Fourier requirement above.
\end{reqlist}

\ssreq{Parallel implementations}

Distributed Legendre algorithms will be supported for a limited
number of specific configurations.  Note that a transpose
algorithm (in which a local serial transform is combined
with data transposes to redistribute data) will always
be supported for the general case.

\begin{reqlist}
{\bf Priority:} 1 \\
{\bf Source:}  \\
{\bf Status:} Proposed \\
{\bf Verification:} Unit test \\
{\bf Notes:} Similar to the Fourier requirement,
             distributed algorithms make assumptions about
             the placement of both input and output data and
             support for all possibilities may be prohibitive.
\end{reqlist}

%-------------------------------------------------------------------------------
\sreq{Other functional transforms}

Methods shall be supplied for regridding using user-supplied matrices,
particularly between functional space and physical space.
The adjoints shall also be supplied.  The grids again must be consistent
with the functional transform being applied.

\begin{reqlist}
{\bf Priority:} 1 \\
{\bf Source:}  NCEP-SSI (milestone), NSIPP \\
{\bf Status:} Proposed \\
{\bf Verification:} Unit test \\
{\bf Notes:} MI - The NCEP-SSI transforms between vertical EOF space
             and vertical model levels.
\end{reqlist}

%-------------------------------------------------------------------------------
\sreq{Interpolating from gridded data to ungridded data}

All methods shall work for regridding FROM gridded data TO ungridded
data, except that no conservation properties are required.
Adjoints shall be supplied for interpolation TO ungridded data.

\begin{reqlist}
{\bf Priority:} 1 \\
{\bf Source:}  NCEP-SSI (milestone), PSAS (not milestone), NSIPP  \\
{\bf Status:} Proposed \\
{\bf Verification:} Unit test \\
{\bf Notes:} 
\end{reqlist}

%-------------------------------------------------------------------------------
\sreq{Interpolating from ungridded data to gridded data}

Methods for regridding FROM ungridded data TO gridded data may be
supplied (eg nearest-neighbor distance-weighted schemes).

\begin{reqlist}
{\bf Priority:} 1 \\
{\bf Source:} MIT  \\
{\bf Status:} Proposed \\
{\bf Verification:} Unit test \\
{\bf Notes:} Generally, operations like this will be covered by
             data assimilation schemes, but simple methods may be useful
             for other model-data comparisons.
\end{reqlist}

%-------------------------------------------------------------------------------
\sreq{User-supplied regridding methods}

It shall be possible for users to supply their own regridding
routines.  This is especially useful for regriddings that are
strongly dependent on model fields.

\begin{reqlist}
{\bf Priority:} 2 \\
{\bf Source:}  CAM-FV, NSIPP \\
{\bf Status:} Proposed \\
{\bf Verification:} Unit test \\
{\bf Notes:} The implementation report will examine use of
             function pointers or their equivalent, but implementation
             may run into other interface issues.

\end{reqlist}

%===============================================================================
\req{Other utilities}
%-------------------------------------------------------------------------------

The following are utilities related to regridding which should be made
public.

%-------------------------------------------------------------------------------
\sreq{Exchange grid}

A method for constructing a new grid formed by the intersecting
cells of two grids shall be available.

\begin{reqlist}
{\bf Priority:}  \\
{\bf Source:}  NSIPP \\
{\bf Status:} Proposed \\
{\bf Verification:} Unit test \\
{\bf Notes:} 
\end{reqlist}

