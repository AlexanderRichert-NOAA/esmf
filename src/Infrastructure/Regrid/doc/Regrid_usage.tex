% $Id: Regrid_usage.tex,v 1.6 2004/05/07 21:59:33 jwolfe Exp $


Regrid is designed to be called with Field or Bundle
arguments in order to utilize information embedded in
these objects.  For example, Regrid requires knowledge
of underlying grid information (both PhysGrid and DistGrid)
and of the relative location (staggering) of Fields on
the Grid.  In addition, Regrid uses any mask information
that may be associated with a Field.  However, ESMF also
provides an Array interface for users who have gathered all
necessary information.

Regrid is separated into RegridStore functions, a Regrid
function, and a RegridRelease function. The Store functions
compute interpolation weights and initialize communication
requirements for performing a regridding of a Field
from one Grid to another.  The Regrid function uses
a created Regrid object to perform the actual regridding
of Fields or Bundles.  The Release function deletes the
Regrid object and frees all memory associated with a Regrid.
The reason for the separation is that in many cases, the
initial creation is expensive and re-used often throughout
an application.  The Regrid and RegridRelease functions are
also common to all the Regrid methods.

Because many methods are supported for regridding,
the main Store function branches to a specific
creation function based on the regrid method requested
(e.g. bilinear, conservative, spectral).  Each of
these regrid methods are in a separate module to
prevent the main Regrid module from becoming too
large.  The user is unaware of this hierarchy as the
top-level module provides a unified API.

The Regrid object created by the RegridStore function
contains a set of ``links''
which identify how a field at a point on the
destination grid is related to a field at a
point on the source grid.  As such, a ``link''
consists of a source address, a destination address
and a weight.  Because the Grids are generally
distributed very differently, the Regrid object
also contains communication information
for any data motion required for the regridding.

