

%\subsection{Design and Implementation Notes}

Regrid has been designed to be as efficient as possible during its
Run routine.  Although the initial calculation during the Store routines
can be computationally intensive, the {\tt ESMF\_RouteHandle} object
it creates is designed to be reused by similar Fields on the same Grids.
And, as long as the Grids are static, RegridStore can be called once
and reused throughout a simulation.  It leverages internal structures
and methods used throughout ESMF for communication so that algorithmic
and programming improvements can be focused on a single location.

Because many methods are supported for regridding, the main Store function
branches to a specific creation function based on the regrid method requested
(e.g. bilinear, conservative, spectral).  Each of these regrid methods are in
a separate module to prevent the main Regrid module from becoming too
large.  The user is unaware of this hierarchy as the top-level module provides
a unified API.

The RouteHandle object created by the RegridStore function contains a set of
``links'' which identify how a Field at a point on the destination Grid is
related to a Field at a point on the source Grid.  As such, a ``link''
consists of a source address, a destination address and a weight.  The addresses
are stored as indices to allow reuse by different Fields on the same Grids.
Because the Grids are generally distributed very differently, the Regrid object
also contains communication information for any data motion required for the
regridding.

\subsubsection{Parallel Implementation}

On parallel processing platforms, only a piece of each Grid is represented 
locally on a particular DE.  In order to calculate a regrid, the portions of
the decomposed source Grid that overlaps the local destination Grid must be
gathered to its DE.  During the RegridStore routine, ESMF determines the
list of the source DE's and Grid extents that must be collected to cover the
destination Grid and stores them in an {\tt ESMF\_DomainList}.  Currently
ESMF gathers all the data from each DE whose source Grid intersects
the local destination Grid, but there has been some work in identifying and
collecting subsets of distributed data instead.  The communication pattern
to gather the corresponding data is stored in an {\tt ESMF\_Route} for use
during the run routine.  Each DE then stores the Route and list of links
unique to its destination Grid.

During the run routine, the source data is first gathered using the
precomputed Route and stored as a 1-D vector.  The main calculation is then 
a loop over the number of links that effectively sums the product of the
source data and the interpolation weights and loads the result into the
corresponding destination Array address.

\subsubsection{Regrid Objects}
The {\tt ESMF\_Regrid} object itself is relatively small, containing mostly
just pointers to the source and destination Field information and settings
for regrid options.  Regrid methods also create an {\tt ESMF\_TransformValues}
object, which holds information about the ``links'', including the number of
them and Arrays of source addresses, destination addresses, and interpolation
weights.  Both the Regrid and TransformValues objects are private and 
contained by the RouteHandle object.

