% $Id: Regrid_glos.tex,v 1.8 2009/01/21 21:38:01 cdeluca Exp $
%
% Earth System Modeling Framework
% Copyright 2002-2009, University Corporation for Atmospheric Research, 
% Massachusetts Institute of Technology, Geophysical Fluid Dynamics 
% Laboratory, University of Michigan, National Centers for Environmental 
% Prediction, Los Alamos National Laboratory, Argonne National Laboratory, 
% NASA Goddard Space Flight Center.
% Licensed under the University of Illinois-NCSA License.

% USAGE NOTE:
%
% The first use of a term in the text of a document can be 
% linked to the corresponding glossary item using the item label.
% 
% For example,
%
% Original document text: code must include item1
%
% Linked to glossary:     code must include \htmlref{item1}{glos:item1}
%
% The link will appear in the html version of the document.
% The print version of the document will appear unchanged.

\begin{description}

\item [conservation] \label{glos:conservation}
      Conservative regridding requires that the area (or volume)
      integrated field on the source grid is equal to that on
      the destination grid, where the integration is over an
      identical area (volume).

\item [order] \label{glos:order}
      The order of the regridding refers to the order of accuracy
      (in a Taylor series sense) used to approximate a field during
      the regridding process.  First order is thus piecewise constant;
      second order is linear.

\item [monotone] \label{glos:monotone}
      A monotone regridding will not introduce new minima or
      maxima into a regridded field.

\end{description}








