% $Id: Regrid_desc.tex,v 1.6 2005/08/19 23:20:07 jwolfe Exp $
%
% Earth System Modeling Framework
% Copyright 2002-2003, University Corporation for Atmospheric Research, 
% Massachusetts Institute of Technology, Geophysical Fluid Dynamics 
% Laboratory, University of Michigan, National Centers for Environmental 
% Prediction, Los Alamos National Laboratory, Argonne National Laboratory, 
% NASA Goddard Space Flight Center.
% Licensed under the GPL.


Bundle, Field, and Array classes all have regrid methods that transform their
data from one {\tt ESMF\_Grid} to another.  Regrid operations compute addresses
and interpolation weights for remapping between different grids.  All
the information necessary to perform a regridding, including {\tt ESMF\_Routes}
to collect non-local data and the addresses and weights, are contained in the
{\tt ESMF\_RouteHandle} which is returned to the user.  Since interpolation
weights are based solely on the grids' geometries and addresses are stored
as offsets, regrids can be shared by data classes providing
they have the same {\tt ESMF\_Rellocs}.  Some of the algorithms and 
implementation in ESMF's regridding routines are adapted from a software package
called SCRIP that was developed at the Los Alamos National Laboratory by Phil
Jones.  However, SCRIP is a serial code and the ESMF regridding routines have
been parallelized.

\subsubsection{Terminology}

\begin{description}
\item[search]
     {\tt Search} refers to the process of determining which processors must
     exchange data (and how much) when regridding between decomposed grids. 
\item[sweep]
     {\tt Sweep} refers to the process of looping through lists of cells from
     one grid, hunting for interactions with a specified point or subsegment
     from the other grid.  The type of interaction depends on the regrid method
     and is either an intersection with an identified subsegment or containment
     of an identified point.  The limitation of the range of cells that must be
     examined is also considered part of the sweep algorithm.
\end{description}
