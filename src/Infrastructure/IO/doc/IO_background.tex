% $Id: IO_background.tex,v 1.3 2002/04/23 01:16:58 lzaslavsky Exp $

\section{Background}

Applications in Earth system modeling require efficient and robust
tools for input and output of gridded data and observational data streams.

Interfaces and methods provided by ESMF should allow reading and
writing of data in several standard formats as well as support
efficient internal data representations. ESMF IO is supposed to
provide a unified interface to input data from a file with automatic 
detection of file-type in runtime, and to output data in a requested format.

The data files  supported by ESMF are either {\em self-described} or
{\em co-described} files.


{\bf Self-Described Files} contain all metadata
necessary to provide a unique interpretation of the file content
assuming cirtain conventions. The metadata can be easily and
strightforwardly read by a corresponding API.

{\bf Co-Described Files} are accompanied by a metadata file. The
metadata file provides a unique interpretation of the data file content
under cirtain conventions. 


NOTE. Some data files may contain complete description of their
content included in the file itself in a fasion not allowing direct
and strightforward extraction of metadata. A program, called {\em
indexer}, can be applied to extract metadata and put them in a
companion metadata file.


\subsection{Gridded Data Formats}

Several standard formats are currently used in earth science modeling
for input/output of gridded data, including multiple versions of
NetCDF, HDF, GRIB, as well as internal binary format accompanied by
metadata:


\begin{itemize}
\item[\bf NetCDF] Network Common Data Form (netCDF) is an interface for 
array-oriented data access. The netCDF library provides an
implementation of the interface. It also defines a 
machine-independent format for representing scientific data. Together,
the interface, library, and format support the creation, access, and
sharing of scientific data. The netCDF software was developed at the
Unidata Program Center in Boulder, Colorado. See \cite{NetCDF3_UsersGuide_C}.

\item[\bf HDF] The Hierarchical Data Format (HDF) project provides
interface,  software and file formats for scientific data management. 
The HDF software includes I/O libraries and tools for analyzing,
visualizing, and converting scientific data. There are two different
HDF formats, HDF (4.x and previous releases) and HDF5. These formats
are completely different and {\it not} compatible. HDF is developed
and supported at the National Center for Supercomputing Applications,
University of Illinois at Urbana-Champaign. See
\cite{HDF4_tutorials}, \cite{HDF5_tutorial}.

\item[\bf HDF-EOS]  The Hierarchical Data Format - Earth Observing
System (HDF-EOS) is the scientific data format standard selected by
NASA as the baseline standard for the Earth Observing System (EOS). HDF-EOS
is an extension of HDF and uses HDF library calls as underlying
basis. Version 4.1r1 of HDF is used. The library and tools are written
in C language and a Fortran interface is provided. See \cite{HDF-EOS}.

\item[\bf GRIB] GRid in Binary (GRIB) format is a gridded data
standard from the World Meteorological Organization. NCEP uses GRIB
for all the files produced by its analyses. Unfortunately, not all
GRIBs are the same and the ECMWF and EMC use slightly different
formats, though they both claim to use GRIB. See \cite{GRIB_1}.


\item[\bf Binary]
The most important way for a machine to represent data is to use a
native binary data representation. There are  two choices of ordering of 
bytes (so-called, {\it Big Endian} and {\it Little Endian}), and a lot of
ambiguity in representing floating point data. The latter, however, is
specified, if IEEE Floating Point Standard 754 is satisfied 
(\cite{IEEE-Floating-Point}, \cite{Kahan-IEEE-754}). It is
desirable to be able to use efficient native representation,
accompanied by XML metadata file \cite{XML-W3C}. 
\end{itemize}

\subsection{CF conventions} 

The conventions for climate and forecast metadata, ``CF conventions'',
\cite{NetCDF_CF_v1_beta3} promote processing and sharing of files
created with netCDF API. Metadata created according to CF convensions
provide definitive description of what the data represent, and spacial
and temporal properties of data.

We expect that the ESMF gridded data will satisfy CF conventions, or
that an extenstion to CF conventions will be accepted by participants.

Thus, ESMF IO interfaces and software for gridded data will cover only
IO of data covered by the convention.

\subsection{Observational Data}

Observational data are usually transmitted and saved in BUFR format 
\cite{WMO-BUFR-CREX}. 

\begin{itemize}
\item[\bf BUFR] Binary Universal Form of Represantation
of the meteorological data (BUFR) is a self-descriptive format. The
form and content of data contained in a BUFR message are described
within BUFR message itself. In addition, BUFR provides condensation,
or packing of data. 

The BUFR is a table-driven code since the Data Description Section
contains a sequence of data descriptors referring to a set of predefined and 
internationally agreed tables. Thus, instead of writing all detailed
definitions within a message, one will just write a number identifying
a parameter with its descriptions \cite{WMO-BUFR-CREX}.
\end{itemize}

\subsection{IO optimization}

The future development of ESMF IO facility will require further
optimization with an expected increase in IO amount over the next few
years. Increase in amount of satellitte data will result in the
increase of time spent to input observational data. On the other hand,
a customer's desire to download simulation results more frequently leads
to significant increase in the amount of gridded data IO.



\subsection{Location}

Input/Output (IO) is part of the ESMF Infrastructure.  It will provide
efficient utilities to input/output gridded data and observation data
to and from the disk. A standard API will allow manipulate with multiple
standard formats.

\subsection{Scope}

ESMF IO is meant to be used for standard API and underlying
implementation, providing input and outut of gridded data and
observational data streams to and from the disk in multiple standard
formats.














