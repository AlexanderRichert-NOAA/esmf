% $Id: IO_background.tex,v 1.6 2002/05/02 13:07:08 lzaslavsky Exp $

\section{Background}

Earth system modeling applications require efficient and robust tools
for input and output of gridded data associated with
structured and or unstrustured grids, as well as observational data streams.
Interfaces and methods provided by ESMF should allow reading and
writing of data in several standard formats as well as support
efficient internal data representations (see ESMF General
Requirements \cite{ESMFGenReq}, Section 8.1.3). ESMF IO is supposed to
provide a unified interface to input data from a file with automatic 
detection of file-type in runtime, and to output data in a requested format.

The data files  supported by ESMF are either {\em self-described} or
{\em co-described} files. These two types of data files are defined as
follows:


{\bf Self-Described Files} contain all metadata necessary to provide a
unique interpretation of the file content
assuming certain conventions. The metadata can be easily and
straightforwardly read by a corresponding API.

{\bf Co-Described Files} are accompanied by a metadata file. The
metadata file provides a unique interpretation of the data file content
under certain conventions. 


{\bf NOTE} Some data files may contain complete description of their
content, but the way metadata are represented might not allow direct
and straightforward extraction of metadata. Metadata could be extracted
from such a file and allocated to a companion metadata file.

\subsection{Metadata conventions} 

The NetCDF conventions for climate and forecast metadata, ``CF conventions'',
\cite{NetCDF_CF_v1_beta3} promote processing and sharing of files
created with netCDF API. Metadata created according to CF conventions
provide definitive description of what the data represent, and spacial
and temporal properties of data.

We expect ESMF metadata conventions to be defined as extensions to
CF-conventions and be accepted by participants of the ESMF project.

ESMF IO interfaces and software will cover only IO of data covered by
the convention.




\subsection{Gridded Data Formats}

ESMF will extensively support a variety of different discrete grids to
represent continious physical space. The gridded data are supported by
three ESMF classes: {em PhysGrid} class for phyical grids
\cite{ESMF-PhysGrid-Req}, {\em DistGrid} class for distributed grids
\cite{ESMF-DistGrid-Req}, and {em Fields} class for fields
\cite{ESMF-Field-Req}. 

The {\em PhysGrid} class is responsible for maintaining information
about physical coordinate based discretizations of simulation domains. The
distributed grid class, {\em DistGrid}, is used for expressing, and
performing operations that involve, dependencies among data
distributed across processors. {\em Fields} class represents fields. 
Fields within a model component are frequently defined on the same
physical grid and are decomposed in memory in an identical fashion;
that is, they share a distributed grid. They form a {\em bundle of
fields} defined on the same distributed grid. 

Several standard formats are currently used in earth science modeling
for input/output of gridded data, including multiple versions of
NetCDF, HDF, GRIB, as well as internal binary format accompanied by
metadata:


\begin{itemize}
\item[\bf NetCDF] Network Common Data Form (netCDF) is an interface for 
array-oriented data access. The netCDF library provides an
implementation of the interface. It also defines a 
machine-independent format for representing scientific data. Together,
the interface, library, and format support the creation, access, and
sharing of scientific data. The netCDF software was developed at the
Unidata Program Center in Boulder, Colorado. See \cite{NetCDF3_UsersGuide_C}.

\item[\bf HDF] The Hierarchical Data Format (HDF) project provides
interface,  software and file formats for scientific data management. 
The HDF software includes I/O libraries and tools for analyzing,
visualizing, and converting scientific data. There are two different
HDF formats, HDF (4.x and previous releases) and HDF5. These formats
are completely different and {\it not} compatible. HDF is developed
and supported at the National Center for Supercomputing Applications,
University of Illinois at Urbana-Champaign. See
\cite{HDF4_tutorials}, \cite{HDF5_tutorial}.

\item[\bf HDF-EOS]  The Hierarchical Data Format - Earth Observing
System (HDF-EOS) is the scientific data format standard selected by
NASA as the baseline standard for the Earth Observing System (EOS). HDF-EOS
is an extension of HDF and uses HDF library calls as underlying
basis. Version 4.1r1 of HDF is used. The library and tools are written
in C language and a Fortran interface is provided. See \cite{HDF-EOS}.

\item[\bf GRIB] GRid in Binary (GRIB) format is a gridded data
standard from the World Meteorological Organization. NCEP uses GRIB
for all the files produced by its analyses. Unfortunately, not all
GRIBs are the same and the ECMWF and EMC use slightly different
formats, though they both claim to use GRIB. See \cite{GRIB_1}.


\item[\bf Binary]
The most important way for a machine to represent data is to use a
native binary data representation. There are  two choices of ordering of 
bytes (so-called, {\it Big Endian} and {\it Little Endian}), and a lot of
ambiguity in representing floating point data. The latter, however, is
specified, if IEEE Floating Point Standard 754 is satisfied 
(\cite{IEEE-Floating-Point}, \cite{Kahan-IEEE-754}). It is
desirable to be able to use efficient native representation,
accompanied by XML metadata file \cite{XML-W3C}. 
\end{itemize}


\subsection{Unstructural gridded Data}

Unstructural gridded data can be represented using structures that
keep spacial coordinates of each point and topological information
through list of neighbors. 

A possible approach to represent unstructural gridded data in ESMF is
to use {\bf HDF-EOS Point Format} \cite{HDF-EOS} where spacial coordinates
and list of neighbors are included in the hierarchy.

Use of indexing procedures may be necessary to operate with such
representation efficiently. 

\subsection{Observational Data}

Observational data are usually transmitted and saved in BUFR format 
\cite{WMO-BUFR-CREX}. Another internal representation may be used to
manipulate with observational data efficiently.

\begin{itemize}
\item[\bf BUFR] Binary Universal Form of Represantation
of the meteorological data (BUFR) is a self-descriptive format. The
form and content of data contained in a BUFR message are described
within BUFR message itself. In addition, BUFR provides condensation,
or packing of data. 

The BUFR is a table-driven code since the Data Description Section
contains a sequence of data descriptors referring to a set of predefined and 
internationally agreed tables. Thus, instead of writing all detailed
definitions within a message, one will just write a number identifying
a parameter with its descriptions \cite{WMO-BUFR-CREX}.

\item[\bf Relational Database]
The processes of receiving observational data and data assimilation
may require complex access to data. To do it efficiently, a complex
database software can be used. For example ECMWF currently exploits
Observational Data Base (ODB) software based on SQL. It allows
parallel query processing as well as use of {\em virtual data base}
concept -- use of multiple data bases as one big virtual database
\cite{ODB}. 

An optimized indexing can provide very fast access to data in the
process of data assimilation.

\item[\bf HDF-EOS Point Format]
An alternative/complimentary approach may be based on use of {\em
point data structure} in HDF-EOS. In this case, optimized indexing can
significantly accelerate access to data as well.
\end{itemize}


\subsection{Parallel IO}

The future development of ESMF IO facility will require further
optimization with an expected increase in IO amount over the next few
years. Increase in amount of satellitte data will result in the
increase of time spent to input observational data, as well as
efficient manipulation of data in process of data assimilation. On the
other hand, a customer's desire to download simulation results more 
frequently leads to significant increase in the amount of gridded data IO.

There are two aspects of  parallel IO:

\begin{itemize}
\item[-] How dataset distributed among multiple processors can be
written to a single file efficiently;

\item[-] How single file can be distributed accross multiple physical
discs and IO channels.
\end{itemize}
 
There are several possibilities to perform IO in parallel:
\begin{itemize}
\item {\bf Single-threaded IO.} A single process acquires all the data and
                                writes them out; 
\item {\bf Multithreaded/Single-fileset IO.} Many processos write to a
single file;

\item {\bf Multithreaded, multi-fileset IO} Many processors write to
multiple independent files. These files are later assembled. 
\end{itemize}

Multithreaded IO offers a simple way to stripe the data accross as many
IO channels and disk channels as are available \cite{
Balaji_Parallel_IO_1999, Balaji_Parallel_IO_2000}.

Parallel IO implemented in GFDL (\cite{mpp_io}) has features described
above.



*********** WRF IO --- see \cite{WRF-Software}.


\subsection{Location}

Input/Output (IO) is part of the ESMF Infrastructure.  It will provide
efficient utilities to input/output gridded data and observation data
to and from the disk. A standard API will allow manipulation of multiple
standard formats.


\subsection{Scope}

ESMF IO is meant to be used for standard API and underlying
implementation, providing input and outut of gridded data and
observational data streams to and from the disk in multiple standard
formats. Sequential IO and different versions of parallel IO have to
be provided.  


















