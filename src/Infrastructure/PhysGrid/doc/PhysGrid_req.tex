% $Id: PhysGrid_req.tex,v 1.5 2002/04/15 18:36:02 pwjones Exp $
%===============================================================================
\req{Physical locations}
%-------------------------------------------------------------------------------

A mechanism shall be provided for describing physical locations in space in 1,
2, or 3 dimensions, including both specification of points and of ranges.
\begin{reqlist}
{\bf Priority:} 1 \\
{\bf Source:} General Req. 8.0.2 \\
{\bf Status:} Proposed \\
{\bf Verification:} System test\\
{\bf Notes:} 
\end{reqlist}

\sreq{Horizontal locations}

\ssreq{Horizontal coordinates}

Physical domains may use Cartesian, spherical, or cylindrical coordinate
systems in the horizontal directions.  Units for these coordinates are meters
(for Cartesian), degrees of latitude and longitude (for spherical), and meters
and degrees for the radius and angle in cylindrical coordinates.

\begin{reqlist}
{\bf Priority:} 1 \\
{\bf Source:} General Req. 8.0.2 \\
{\bf Status:} Proposed \\
{\bf Verification:} Code inspection\\
{\bf Notes:}  This suggestion follows common practice, but is an explicit
exception to the requirement that MKS units are used by ESMF codes where ever
units must be assumed.
\end{reqlist}

\ssreq{Horizontal locations may be points}

   Horizontal locations may be specified as a pair of real values in the order
(X,Y) or (longitude, latitude).
\begin{reqlist}
{\bf Priority:} 1 \\
{\bf Source:} General Req. 8.0.2 \\
{\bf Status:} Proposed \\
{\bf Verification:} Unit test\\
{\bf Notes:} 
\end{reqlist}


\ssreq{Horizontal locations may be polygonal regions}

  Horizontal Locations may be specified to be regions by providing the number
of vertices and the list of the vertex points.  Vertex points must be specified
either clockwise or counterclockwise around the region.  The vertex points
may be redundant.
\begin{reqlist}
{\bf Priority:} 1 \\
{\bf Source:} General Req. 8.0.3 \\
{\bf Status:} Proposed \\
{\bf Verification:} Unit test\\
{\bf Notes:} Fundamental to allowing conservative interpolation, and for an
precise description of data locations.
\end{reqlist}
 
\ssreq{Horizontal regions may have central points}

  Both a central point and a region may be specified in describing a horizontal
location.  The points may provide a convenient nominal location, even when
a value actually pertains to a region.
\begin{reqlist}
{\bf Priority:} 1 \\
{\bf Source:} General Req. 8.0.2 \\
{\bf Status:} Proposed \\
{\bf Verification:} Unit test\\
{\bf Notes:} Many models mix finite difference and finite volume concepts.
\end{reqlist}

\ssreq{Horizontal locations may have radii of influence}

  A horizontal location may be specified by adding a nominal radius of
influence to the central point.  This may be the radius of a Gaussian
distribution of influence. The exact interpretation of this radius is
the responsibility of user-provided software.
\begin{reqlist}
{\bf Priority:} 2 \\
{\bf Source:} General Req. 8.0.2 \\
{\bf Status:} Proposed \\
{\bf Verification:} Unit test\\
{\bf Notes:} This is necessary for describing certain observational data streams.
\end{reqlist}


\ssreq{Paths between grid locations may be specified} 

A method for determining the path connecting grid locations 
is required.  This path would be used to accurately compute
intersections for regridding, lengths of cell sides, grid
cell areas and a variety of other grid metrics.  A linear 
approximation between points is a proper assumption for 
cartesian grids and for computing sides of latitude/longitude
or reduced grid cells.  A linear approximation is also 
adequate in many other cases and would be a logical default choice. 
The most accurate solution would permit users to pass a 
subroutine which provides analytic or highly-accurate discrete 
forms of the grid Jacobian (the matrix of partial derivatives of 
the physical coordinates with respect to logical coordinates).  An 
additional possibility might internally support analytic forms like 
great circles or higher-order approximations (eg quadratic
approximation to the cell side given a midpoint in addition
to the two endpoints).

\begin{reqlist}
{\bf Priority:} 1 \\
{\bf Source:} General Req. 8.0.3 \\
{\bf Status:} Proposed \\
{\bf Verification:} Unit test\\
{\bf Notes:} Necessary for allowing conservative interpolation.
\end{reqlist}

\sreq{Vertical locations}

\ssreq{Vertical coordinates}

Physical domains may use a variety of vertical coordinates, including pressure,
height, density, isotherms, sigma, other terrain-following, or any other
vertically monotonic quantity.  In addition, a user-interpretable vertical
proxy (such as a satellite measurement channel) may be used.  Units of this
coordinate must be self-consistent.  (See the CF convention for a full
discussion of options for vertical coordinates at
http://www.cgd.ucar.edu/cms/eaton/netcdf/CF-20010629.htm)
\begin{reqlist}
{\bf Priority:} 1 \\
{\bf Source:} General Req. 8.0.2 \\
{\bf Status:} Proposed \\
{\bf Verification:} Code Inspection\\
{\bf Notes:} 
\end{reqlist}

\ssreq{Vertical locations may be points}
\begin{reqlist}
{\bf Priority:} 1 \\
{\bf Source:} General Req. 8.0.2 \\
{\bf Status:} Proposed \\
{\bf Verification:} Unit test\\
{\bf Notes:} 
\end{reqlist}

\ssreq{Vertical locations may be regions}

Vertical locations may be specified by providing the values of the top and
bottom bounding points.  Such regions have the same extent regardless of the
order in which the bounding points are specified.
\begin{reqlist}
{\bf Priority:} 1 \\
{\bf Source:} General Req. 8.0.2 \\
{\bf Status:} Proposed \\
{\bf Verification:} Unit test\\
{\bf Notes:} 
\end{reqlist}

\ssreq{Vertical regions have central points}

  Both a central point and a region may be specified in describing a vertical
location.  The points may provide a convenient nominal location, even when
a value actually pertains to a region.
\begin{reqlist}
{\bf Priority:} 1 \\
{\bf Source:} General Req. 8.0.2 \\
{\bf Status:} Proposed \\
{\bf Verification:} Unit test\\
{\bf Notes:} Many models mix finite difference and finite volume concepts.
\end{reqlist}

\ssreq{Vertical locations may have radii of influence}

  A vertical location may be specified by adding a nominal radius of
influence to the central point.  This may be the radius of a Gaussian
distribution of influence. The exact interpretation of this radius is
the responsibility of user-provided software.
\begin{reqlist}
{\bf Priority:} 2 \\
{\bf Source:} General Req. 8.0.2 \\
{\bf Status:} Proposed \\
{\bf Verification:} Unit test\\
{\bf Notes:} This is necessary for describing certain observational data streams.
\end{reqlist}

\ssreq{Vertical locations may include lopped cells}

  Vertical locations may include a region whose bounds vary between the
horizontal corners of a region.
\begin{reqlist}
{\bf Priority:} 2 \\
{\bf Source:} MIT \\
{\bf Status:} Proposed \\
{\bf Verification:} Unit test\\
{\bf Notes:} 
\end{reqlist}

%===============================================================================
\req{Location streams}
%-------------------------------------------------------------------------------

Streams of locations are used to describe the physical (and potentially temporal)
locations associated with streams of data.  Streams of locations differ from
Physical Grids (see below) in that there are no concept of neighboring values,
topology, covering a space, or of locations being exclusive.
\begin{reqlist}
{\bf Priority:} 1 \\
{\bf Source:} General Req. 8.0.2 \\
{\bf Status:} Proposed \\
{\bf Verification:} System test\\
{\bf Notes:} 
\end{reqlist}

\sreq{Location streams may be created}
\begin{reqlist}
{\bf Priority:} 1 \\
{\bf Source:} \\
{\bf Status:} Proposed \\
{\bf Verification:} Unit test \\
{\bf Notes:} 
\end{reqlist}

\sreq{Location streams may be destroyed}
\begin{reqlist}
{\bf Priority:} 1 \\
{\bf Source:} \\
{\bf Status:} Proposed \\
{\bf Verification:} Unit test \\
{\bf Notes:} 
\end{reqlist}

\sreq{Location streams may be copied}
Given an existing location stream, a new stream may be generated with a new name and
possibly a different length.
\begin{reqlist}
{\bf Priority:} 2 \\
{\bf Source:} \\
{\bf Status:} Proposed \\
{\bf Verification:} Unit test \\
{\bf Notes:} 
\end{reqlist}

\sreq{Reading streams}
Location streams may be read from files.
\begin{reqlist}
{\bf Priority:} 1 \\
{\bf Source:} \\
{\bf Status:} Proposed \\
{\bf Verification:} Unit test \\
{\bf Notes:} 
\end{reqlist}

\sreq{Writing streams}
Location streams may be written to files.
\begin{reqlist}
{\bf Priority:} 1 \\
{\bf Source:} \\
{\bf Status:} Proposed \\
{\bf Verification:} Unit test \\
{\bf Notes:} 
\end{reqlist}

\sreq{Underlying grid}
A location stream may have associated with it an underlying PhysGrid, so that each
location stream element may be uniquely associated with a single grid cell. (A single
grid cell may contain multiple location stream elements.)
\begin{reqlist}
{\bf Priority:} 1 \\
{\bf Source:} \\
{\bf Status:} Proposed \\
{\bf Verification:} Unit test \\
{\bf Notes:} This is necessary for such operations as halo updates on a location
stream.
\end{reqlist}

\sreq{Location stream attributes}

\ssreq{Fixed length location streams}
Location streams may be of fixed length, specified at the time of generation.
\begin{reqlist}
{\bf Priority:} 2 \\
{\bf Source:} \\
{\bf Status:} Proposed \\
{\bf Verification:} Unit test\\
{\bf Notes:} 
\end{reqlist}

\ssreq{Extensible length location streams}
Location streams may be of extensible length (e.g. a linked list),
with an initial length specified at the time of generation.
\begin{reqlist}
{\bf Priority:} 3 \\
{\bf Source:} \\
{\bf Status:} Proposed \\
{\bf Verification:} Unit test\\
{\bf Notes:} 
\end{reqlist}

\ssreq{Global attributes: location stream name}
Each location stream has a unique name by which it can be referred.
\begin{reqlist}
{\bf Priority:} 2 \\
{\bf Source:} \\
{\bf Status:} Proposed \\
{\bf Verification:} Unit test \\
{\bf Notes:} 
\end{reqlist}

\ssreq{Location stream registry}
Upon creation, the name and a pointer to each location stream shall be stored in a
registry.  A pointer to any location stream may be determined given its name.
\begin{reqlist}
{\bf Priority:} 3 \\
{\bf Source:} \\
{\bf Status:} Proposed \\
{\bf Verification:} Unit test \\
{\bf Notes:} 
\end{reqlist}

\ssreq{Global attributes: Number of dimensions}
A location stream may be queried for the number of dimensions, which is
set at the time of creation of the stream.
\begin{reqlist}
{\bf Priority:} 2 \\
{\bf Source:} \\
{\bf Status:} Proposed \\
{\bf Verification:} Code inspection \\
{\bf Notes:} 
\end{reqlist}

\ssreq{Global attributes: dimension names}
Each dimension has a name, which may be set and queried.
\begin{reqlist}
{\bf Priority:} 2 \\
{\bf Source:} \\
{\bf Status:} Proposed \\
{\bf Verification:} Unit test \\
{\bf Notes:} 
\end{reqlist}

\ssreq{Global attributes: dimension units}
A location stream contains the units of each dimension, which may be set and queried.
\begin{reqlist}
{\bf Priority:} 2 \\
{\bf Source:} \\
{\bf Status:} Proposed \\
{\bf Verification:} Unit test \\
{\bf Notes:} 
\end{reqlist}

\ssreq{Global attributes: text attributes}

A location stream may have an arbitrary number of text (???or numeric???) attributes,
which may be added, set and queried.  Each attribute has a text name by which it
can be queried.  Also, a Location Stream can be queried for a list of all global
attribute names.

\begin{reqlist}
{\bf Priority:} 2 \\
{\bf Source:} \\
{\bf Status:} Proposed \\
{\bf Verification:} Unit test \\
{\bf Notes:} 
\end{reqlist}

\ssreq{Global attributes: number of elements}
The number of elements in a location stream is available.  For a fixed length stream,
both the total number of elements and the number of elements before the last valid
element location may be queried.
\begin{reqlist}
{\bf Priority:} 2 \\
{\bf Source:} \\
{\bf Status:} Proposed \\
{\bf Verification:} Unit test \\
{\bf Notes:} 
\end{reqlist}

\ssreq{Global attributes: null element location value}
Each location stream may indicate whether a particular location is valid.
\begin{reqlist}
{\bf Priority:} 2 \\
{\bf Source:} \\
{\bf Status:} Proposed \\
{\bf Verification:} Code inspection \\
{\bf Notes:} 
\end{reqlist}


\ssreq{Elements in stream have similar attributes}
All elements in a location stream will have the same numbers of dimensions, use the
same physical coordinates, the same units, the same element attributes (attributes at
some locations may be missing).
\begin{reqlist}
{\bf Priority:} 2 \\
{\bf Source:} \\
{\bf Status:} Proposed \\
{\bf Verification:} Code inspection \\
{\bf Notes:} 
\end{reqlist}

\ssreq{Elements include values of locations}
Methods shall be provided to set and query each element's location.
\begin{reqlist}
{\bf Priority:} 1 \\
{\bf Source:} \\
{\bf Status:} Proposed \\
{\bf Verification:} Unit test \\
{\bf Notes:} 
\end{reqlist}

\ssreq{Elements may have attributes}
An appropriate text or data attribute may be attached to each element.  These
attributes may be null for any particular element.  Methods shall be provided to set
and query each element's attribute.
\begin{reqlist}
{\bf Priority:} 2 \\
{\bf Source:} \\
{\bf Status:} Proposed \\
{\bf Verification:} Unit test \\
{\bf Notes:} 
\end{reqlist}

\ssreq{Location streams may contain null (discarded) elements}

Some elements within a location stream may be set to be invalid.  This may be a way
to specify the elements that are irrelevant for a particular subdomain.

\begin{reqlist}
{\bf Priority:} 2 \\
{\bf Source:} \\
{\bf Status:} Proposed \\
{\bf Verification:} Unit test \\
{\bf Notes:} 
\end{reqlist}

\ssreq{Location streams may be queried for valid elements}

Location streams may be queried to obtain an ordered list of the indices of (or
pointers to) all valid elements.

\begin{reqlist}
{\bf Priority:} 2 \\
{\bf Source:} \\
{\bf Status:} Proposed \\
{\bf Verification:} Unit test \\
{\bf Notes:} 
\end{reqlist}

%-------------------------------------------------------------------------------
\sreq{Location stream methods requiring registries of dependent data}
If all of the data streams that use a particular location stream are known,
additional methods for manipulating location streams and associated data streams are
possible.
\begin{reqlist}
{\bf Priority:} 3 \\
{\bf Source:} \\
{\bf Status:} Proposed \\
{\bf Verification:} Unit test \\
{\bf Notes:} 
\end{reqlist}

\ssreq{Registry of data streams}
Each location stream includes a registry of all the data streams that rely upon a
location stream.  This is necessary for location streams and data streams to be
manipulated in compatible ways.
\begin{reqlist}
{\bf Priority:} 3 \\
{\bf Source:} \\
{\bf Status:} Proposed \\
{\bf Verification:} Unit test \\
{\bf Notes:} 
\end{reqlist}
\ssreq{Extensible location streams may be extended}
\begin{reqlist}
{\bf Priority:} 3 \\
{\bf Source:} \\
{\bf Status:} Proposed \\
{\bf Verification:} Unit test \\
{\bf Notes:} 
\end{reqlist}
\ssreq{Extensible location streams may be shortened}
\begin{reqlist}
{\bf Priority:} 3 \\
{\bf Source:} \\
{\bf Status:} Proposed \\
{\bf Verification:} Unit test \\
{\bf Notes:} 
\end{reqlist}
\ssreq{Extensible length location streams may be converted to fixed length}
\begin{reqlist}
{\bf Priority:} 3 \\
{\bf Source:} \\
{\bf Status:} Proposed \\
{\bf Verification:} Unit test \\
{\bf Notes:} 
\end{reqlist}
\ssreq{Fixed length location streams may be converted to extensible length}
\begin{reqlist}
{\bf Priority:} 3 \\
{\bf Source:} \\
{\bf Status:} Proposed \\
{\bf Verification:} Unit test \\
{\bf Notes:} 
\end{reqlist}
\ssreq{Fixed length streams may have null elements moved to end}
\begin{reqlist}
{\bf Priority:} 3 \\
{\bf Source:} \\
{\bf Status:} Proposed \\
{\bf Verification:} Unit test \\
{\bf Notes:} 
\end{reqlist}

%===============================================================================
\req{Physical grids}
%-------------------------------------------------------------------------------

Physical grids (PhysGrids) provide the locations of each of the cells/points
associated with the range of indices in a distributed grid.  A PhysGrid is a
distributed object associated with a single DistGrid.  A PhysGrid may have
undistributed dimensions that are not present in the underlying DistGrid. 
Multiple PhysGrids may be derived from the same global, undistributed
PhysGrid.  Physical grids may be purely horizontal, purely vertical, or both. 
Structured grids assume that adjacent locations in index space share boundaries
in a predictable way.  Unstructured grids also have concepts of neighboring
cells, but the relative indices of neighbors are unpredictable.
\begin{reqlist}
{\bf Priority:} 1 \\
{\bf Source:} General Req. 8.0.2 \\
{\bf Status:} Proposed \\
{\bf Verification:} System test\\
{\bf Notes:}
\end{reqlist}

\sreq{Reading grids}
Given a DistGrid, a PhysGrid can be read from a standard file containing a
global PhysGrid.
\begin{reqlist}
{\bf Priority:} 1 \\
{\bf Source:} General Req. 8.0.2 \\
{\bf Status:} Proposed \\
{\bf Verification:} Unit test\\
{\bf Notes:} 
\end{reqlist}

\sreq{Writing grids}
Grids can be output to standard files.
\begin{reqlist}
{\bf Priority:} 1 \\
{\bf Source:} General Req. 8.0.2 \\
{\bf Status:} Proposed \\
{\bf Verification:} Unit test\\
{\bf Notes:} 
\end{reqlist}

\sreq{PhysGrids may be internally generated}
For an arbitrary number of points in the global domain of the associated
DistGrid, it may be possible to specify an algorithm for internally
determining the PhysGrid.
\begin{reqlist}
{\bf Priority:} 1 \\
{\bf Source:} General Req. 8.0.2 \\
{\bf Status:} Proposed \\
{\bf Verification:} Unit test\\
{\bf Notes:} 
\end{reqlist}

\sreq{Cell specification}
PhysGrids shall specify both the locations of cell vertices, and the locations
of cell centers.
\begin{reqlist}
{\bf Priority:} 1 \\
{\bf Source:} General Req. 8.0.2 \\
{\bf Status:} Proposed \\
{\bf Verification:} Unit test\\
{\bf Notes:} Many models mix finite difference and finite volume concepts.
\end{reqlist}

\sreq{Interpolation}
A PhysGrid may be interpolated to generate an equivalent PhysGrid on a coarser
or finer global DistGrid.  Methods should be provided to accomplish such
interpolation, probably via a simple interface that uses the Regrid facility.
\begin{reqlist}
{\bf Priority:} 1 \\
{\bf Source:} General Req. 1.5.2 \\
{\bf Status:} Proposed \\
{\bf Verification:} Unit test\\
{\bf Notes:} Necessary for runtime configurable resolution.  Also, note that 
there may be a Catch-22 here, as Regrid would naturally provide the facility
for regridding, but Regrid will typically require the target PhysGrid for
creating the regridding.
\end{reqlist}

\sreq{Regeneration}
A new PhysGrid may be generated for a given DistGrid from another PhysGrid.
The global domain of the source PhysGrid may be the same as or a superset of
the target global domain.
\begin{reqlist}
{\bf Priority:} 1 \\
{\bf Source:} General Req. 1.5.2 \\
{\bf Status:} Proposed \\
{\bf Verification:} Unit test\\
{\bf Notes:} Necessary for support of transposes, or of moving nests.
\end{reqlist}

\sreq{DistGrid reference}
A PhysGrid may be queried for the DistGrid upon which it is based.
\begin{reqlist}
{\bf Priority:} 2 \\
{\bf Source:} \\
{\bf Status:} Proposed \\
{\bf Verification:} Unit test \\
{\bf Notes:} 
\end{reqlist}

\sreq{Horizontal coordinate independent of vertical}
Horizontal grid locations can be assumed independent of the vertical coordinate.
The horizontal metrics, however, may be function of the vertical coordinate, as
in thick-shell spherical coordinates.
\begin{reqlist}
{\bf Priority:} 2 \\
{\bf Source:} Any Objections? \\
{\bf Status:} Proposed \\
{\bf Verification:} Code inspection\\
{\bf Notes:} If this assumption can be made, it greatly simplifies implementation.
No widely used counterexamples are known.
\end{reqlist}

\sreq{Vertical coordinate potentially dependent on horizontal}
Vertical grid locations may be functions of the horizontal coordinates, or may be
independent of them.
\begin{reqlist}
{\bf Priority:} 2 \\
{\bf Source:} GFDL-MOM4 (required) \\
{\bf Status:} Proposed \\
{\bf Verification:} Code inspection\\
{\bf Notes:} This is necessary to support, for example, partial cells in
Z-coordinate ocean models.
\end{reqlist}

\sreq{Dimension extension}
A new PhysGrid may be generated by adding a dimension to an existing PhysGrid.
The global domain of the source PhysGrid may be the same as or a superset of
the target global domain.  The new dimension will be independent of the underlying
DistGrid, and both PhysGrids share the same DistGrid.  The new dimension be in any
order with respect to existing dimensions.
\begin{reqlist}
{\bf Priority:} 2 \\
{\bf Source:} ? \\
{\bf Status:} Proposed \\
{\bf Verification:} Unit test\\
{\bf Notes:} Valuable for separating generation of vertical and horizontal
coordinates.
\end{reqlist}

\sreq{Dimension reduction}
A new PhysGrid may be generated by removing a dimension from an existing PhysGrid.
If the dimension that is removed is one that is present in the original underlying
DistGrid, an appropriately reduced must also be provided.  Otherwise the new PhysGrid
is based on the same DistGrid as the original PhysGrid.
\begin{reqlist}
{\bf Priority:} 2 \\
{\bf Source:} ? \\
{\bf Status:} Proposed \\
{\bf Verification:} Unit test\\
{\bf Notes:} 
\end{reqlist}

\sreq{Arbitrary dimensional PhysGrids}
PhysGrids may have an arbitrary number of dimensions. 
\begin{reqlist}
{\bf Priority:} 3 \\
{\bf Source:} ? \\
{\bf Status:} Proposed (suggest rejection)\\
{\bf Verification:} Unit test\\
{\bf Notes:} If supported, this facility would dramatically complicate implementation,
without adding much functionality.
\end{reqlist}

\sreq{1- 2- or 3- dimensional PhysGrids}
PhysGrids may have up to dimensions, but must have at least as many dimensions as the
underlying DistGrid. 
\begin{reqlist}
{\bf Priority:} 1 \\
{\bf Source:} General Req. 1.5.2 \\
{\bf Status:} Proposed \\
{\bf Verification:} Unit test\\
{\bf Notes:} If supported, this facility would dramatically complicate implementation,
without adding much functionality.
\end{reqlist}

\sreq{Index order}
A PhysGrid may use any index order (XYZ, XZY, etc.).  Methods shall be provided to
specify the order upon creation and to query the order of a PhysGrid.
\begin{reqlist}
{\bf Priority:} 1 \\
{\bf Source:} Please list required orders in Fortran notation. \\
{\bf Status:} Proposed \\
{\bf Verification:} Unit test\\
{\bf Notes:} Necessary for support of transposes.
\end{reqlist}

\sreq{Dimension reordering}
A new PhysGrid may be generated with reordered dimensions from another PhysGrid.
If the new dimension order is inconsistent with the original DistGrid, a new
consistent PhysGrid must also be provided.  To be consistent, all dimensions present
in a DistGrid must have the same relative order in the PhysGrid.  (i.e. if the
DistGrid uses XY, PhysGrids using XYZ, ZXY, or XZY are all consistent, while one using
ZYX is not.)
\begin{reqlist}
{\bf Priority:} 1 \\
{\bf Source:} General Req. 1.5.2 \\
{\bf Status:} Proposed \\
{\bf Verification:} Unit test\\
{\bf Notes:} Necessary for support of transposes.
\end{reqlist}

\sreq{Location index determination}
A method shall be provided to return the cell index of a location.  An option shall
be provided to either create an exception for any location outside of the valid
range of the coordinate system, or to produce a gracefully treatable return value if
the location is (1) outside of the range of the local PhysGrid, or (2) outside of the
range of the global PhysGrid.  The index locations should be floating point numbers to
facilitate interpolation.
\begin{reqlist}
{\bf Priority:} 1 \\
{\bf Source:} ? \\
{\bf Status:} Proposed \\
{\bf Verification:} Unit test \\
{\bf Notes:} 
\end{reqlist}

\sreq{Horizontal physical grids}

\ssreq{PhysGrids map projections}
PhysGrids may be generated from a number of standard map projections, including
traditional and Mercator grids on a sphere, rotated latitude-longitude,
tripolar, and Gaussian cylindrical grids.  Additional requested grids include 
cubed-sphere, polar stereographic, and Lambert conformal projections.
\begin{reqlist}
{\bf Priority:} 1 \\
{\bf Source:} General Req. 8.0.2 \\
{\bf Status:} Proposed \\
{\bf Verification:} Unit test\\
{\bf Notes:}  Perhaps some of these should be read in from a file, rather than
internally generated.
\end{reqlist}

\ssreq{Unstretched Cartesian internal generation}
A simple interface shall be provided to internally generate a uniform Cartesian
coordinate PhysGrid, given the lengths of the edges of a square domain.
\begin{reqlist}
{\bf Priority:} 1 \\
{\bf Source:} General Req. 8.0.2 \\
{\bf Status:} Proposed \\
{\bf Verification:} Unit test\\
{\bf Notes:} 
\end{reqlist}

\ssreq{Latitude-longitude internal generation}
A simple interface shall be provided to internally generate a uniform (constant
grid-spacing in degrees) latitude-longitude PhysGrid, given the extent of the domain
in latitude and longitude.                                                           
\begin{reqlist}
{\bf Priority:} 1 \\
{\bf Source:} General Req. 8.0.2 \\
{\bf Status:} Proposed \\
{\bf Verification:} Unit test\\
{\bf Notes:} 
\end{reqlist}

\ssreq{Stand-alone PhysGrid generation examples}
Stand-alone software examples shall be provided to demonstrate the generation of a
global PhysGrid file on a stretched latitude-longitude grid, a rotated
latitude-longitude grid and a tripolar grid. 
\begin{reqlist}
{\bf Priority:} 2 \\
{\bf Source:} GFDL, NCAR \\
{\bf Status:} Proposed \\
{\bf Verification:} Unit test\\
{\bf Notes:} These are intended both for real use, and for use as patterns in the
creation of PhysGrid files for more complicated grids.  The above list may be
altered, extended or reduced following discussions.
\end{reqlist}

\ssreq{Supported topologies} Supported horizontal grid topologies will include
logically rectangular grids that are reentrant (periodic) in 0, 1, or 2 directions,
northern and southern tripolar (Murray 1996), sphere, icosahedral, and unstructured
grids.  Unstructured arrays of logically rectangular grids [for cubed-sphere (Rancic
et al. 1996), reduced grids, and arbitrary nesting] will also be supported.
\begin{reqlist}
{\bf Priority:} 1 \\
{\bf Source:} General Req. 8.0.2 \\
{\bf Status:} Proposed \\
{\bf Verification:} Unit test\\
{\bf Notes:}  Topologies are intrinsic to both DistGrid and PhysGrid.  Since
the topology information is so widely used in DistGrid, and since DistGrids
are used to initiate PhysGrids, it is perhaps reasonable to make topology a
property of DistGrid, which is then inherited and checked by a PhysGrid.
\end{reqlist}

\ssreq{PhysGrid topology consistency checking}
A mechanism shall be provided to verify that the locations of the points in
a PhysGrid are consistent with the topology of the underlying DistGrid.  An
exception shall be generated in case of inconsistency.
\begin{reqlist}
{\bf Priority:} 1 \\
{\bf Source:} General Req. 8.0.2 \\
{\bf Status:} Proposed \\
{\bf Verification:} Unit test\\
{\bf Notes:}
\end{reqlist}

\ssreq{Areas tile sphere}
It may be specified that grid areas should be calculated using algorithms that
guarantee that the grid exactly (algorithmically to within 1 part in $10^{12}$) tiles
the sphere (or a portion of it). 
\begin{reqlist}
{\bf Priority:} 2 \\
{\bf Source:} GFDL (required) \\
{\bf Status:} Proposed \\
{\bf Verification:} Unit test\\
{\bf Notes:} Needed to permit exact conservation of fluxes between models.
\end{reqlist}

\ssreq{Available subgrids}
For locally quadrilateral horizontal grids, information shall be available for
each of the 4 related subgrids.  That is if a t-cell is centered at a tracer
point,  cells centered on the east face, north face, and northeast corner of
the t-cell will also be provided in the case of a NorthEast underlying
distributed grid.
\begin{reqlist}
{\bf Priority:} 2 \\
{\bf Source:} GFDL, MIT (required) \\
{\bf Status:} Proposed \\
{\bf Verification:} Code inspection\\
{\bf Notes:} Standard requirement of a staggered grid.
\end{reqlist}

\ssreq{Extensible grid point representations}
It is not anticipated that all possible grids will be included in
ESMF. It must, therefore, be relatively straightforward to add
new grids to the framework and to share those grid "extensions"
amongst the framework community. For example it should be possible
to add an icosahedral grid.
\begin{reqlist}
{\bf Priority:} 1 \\
{\bf Source:} General Req. 8.0.2 \\
{\bf Status:} Proposed \\
{\bf Verification:} Code inspection \\
{\bf Notes:} This is basic to the extensibility of ESMF.
\end{reqlist}

\ssreq{Unstructured grid connectivity}
An unstructured PhysGrid contains information about the intercell connectivity that
can be set and queried.
\begin{reqlist}
{\bf Priority:} 2 \\
{\bf Source:}  \\
{\bf Status:} Proposed \\
{\bf Verification:} Unit Test \\
{\bf Notes:} This requirement should be extended with input from someone familiar with
the practical use of unstructured grids.
\end{reqlist}

\sreq{Horizontal functional representations}
A spectral horizontal description may be used.  More generally, the horizontal
structure of information may be given by specifying functional decompositions.
\begin{reqlist}
{\bf Priority:} 1 \\
{\bf Source:} General Req. 8.0.2 \\
{\bf Status:} Proposed \\
{\bf Verification:} Unit test\\
{\bf Notes:}
\end{reqlist}

\ssreq{Horizontal Fourier grids} 
Cartesian Fourier grids will be supported.  Associated with this grid are the
wavenumbers (in units of $m^{-1}$) of each of the elements on the grid.
\begin{reqlist}
{\bf Priority:} 1 \\
{\bf Source:} General Req. 8.0.2 \\
{\bf Status:} Proposed \\
{\bf Verification:} Unit test\\
{\bf Notes:}
\end{reqlist}

\ssreq{Horizontal spherical harmonics grids} 
Spherical harmonics grids will be supported.  Associated with this grid are the
wavenumbers (nondimensional m,n) of each of the elements on the grid.  At a
minimum, rhomboidal and triangular truncations will be supported.
\begin{reqlist}
{\bf Priority:} 1 \\
{\bf Source:} General Req. 8.0.2 \\
{\bf Status:} Proposed \\
{\bf Verification:} Unit test\\
{\bf Notes:}
\end{reqlist}

\ssreq{Mixed physical and Fourier grids}
Mixed physical and Fourier grids will be supported. In particular, a grid on the
sphere that is latitude in one dimension and Fourier zonal wavenumber
(nondimensional m) in the other dimension will be supported.
\begin{reqlist}
{\bf Priority:} 1 \\
{\bf Source:} NCEP \\
{\bf Status:} Proposed \\
{\bf Verification:} Unit test\\
{\bf Notes:}
\end{reqlist}

\ssreq{Extensible horizontal functional representations}
The PhysGrid design should not preclude the user from using alternative
functional horizontal representations, such as spectral elements.
\begin{reqlist}
{\bf Priority:} 1 \\
{\bf Source:} General Req. 8.0.2 \\
{\bf Status:} Proposed \\
{\bf Verification:} Code inspection \\
{\bf Notes:} This is basic to the extensibility of ESMF.
\end{reqlist}

\sreq{Vertical functional representations}
\ssreq{Vertical user defined functions}
The PhysGrid design should not preclude the user from using 
functional vertical representations, such as EOFs, eigenfunctions,
and finite elements.
(???THIS PART MAY BE MORE APPROPRIATE FOR REGRID...) Support will provided for
accepting a user must supplied matrix that would transform the function into some
vertical physical space. Regridding would then be able to perform the transform, the
inverse transform, and the adjoint transform.
\begin{reqlist}
{\bf Priority:} 1 \\
{\bf Source:} NCEP \\
{\bf Status:} Proposed \\
{\bf Verification:} Code inspection \\
{\bf Notes:} 
\end{reqlist}

\ssreq{Area overlap checking}
A method shall be provided to check that PhysGrid cells do not overlap. 
\begin{reqlist}
{\bf Priority:} 2 \\
{\bf Source:} GFDL, NCAR (required) \\
{\bf Status:} Proposed \\
{\bf Verification:} Unit test\\
{\bf Notes:} Standard self-consistency test.
\end{reqlist}

%-------------------------------------------------------------------------------
\sreq{PhysGrid attributes}

\ssreq{PhysGrid name}
Each PhysGrid has a unique name by which it can be referred.
\begin{reqlist}
{\bf Priority:} 2 \\
{\bf Source:} \\
{\bf Status:} Proposed \\
{\bf Verification:} Unit test \\
{\bf Notes:} 
\end{reqlist}

\ssreq{PhysGrid registry}
Upon creation, the name and a pointer to each PhysGrid shall be stored in a
registry.  A pointer to any PhysGrid may be determined given its name.
\begin{reqlist}
{\bf Priority:} 3 \\
{\bf Source:} \\
{\bf Status:} Proposed \\
{\bf Verification:} Unit test \\
{\bf Notes:} 
\end{reqlist}

\ssreq{Number of dimensions}
A PhysGrid may be queried for the number of dimensions, which is
set at the time of its creation.
\begin{reqlist}
{\bf Priority:} 2 \\
{\bf Source:} \\
{\bf Status:} Proposed \\
{\bf Verification:} Unit test \\
{\bf Notes:} 
\end{reqlist}

\ssreq{Dimension names}
Each dimension has a name, which may be set and queried.
\begin{reqlist}
{\bf Priority:} 2 \\
{\bf Source:} \\
{\bf Status:} Proposed \\
{\bf Verification:} Unit test \\
{\bf Notes:} 
\end{reqlist}

\ssreq{Dimension lengths}
A PhysGrid may be queried for the local or global lengths of each of its dimensions.
\begin{reqlist}
{\bf Priority:} 2 \\
{\bf Source:} \\
{\bf Status:} Proposed \\
{\bf Verification:} Unit test \\
{\bf Notes:} 
\end{reqlist}

\ssreq{Dimension attributes and units}
A PhysGrid contains the units of each dimension, which may be set and queried. 
Dimensions may also have additional named attributes.
\begin{reqlist}
{\bf Priority:} 2 \\
{\bf Source:} \\
{\bf Status:} Proposed \\
{\bf Verification:} Unit test \\
{\bf Notes:} 
\end{reqlist}

\ssreq{Global attributes}
A PhysGrid may have an arbitrary number of text or numeric attributes,
which may be added, set and queried.  Each attribute has a text name by which it
can be queried.  Also, a PhysGrid can be queried for a list of all global
attribute names.

\begin{reqlist}
{\bf Priority:} 2 \\
{\bf Source:} \\
{\bf Status:} Proposed \\
{\bf Verification:} Unit test \\
{\bf Notes:} 
\end{reqlist}

%===============================================================================
\req{Grid metrics}
%-------------------------------------------------------------------------------
Grid metrics are all of the lengths (or partial derivatives of distances with
index number) and related quantities required to do a variety of calculations. 
All metrics are a function of the grid and must be static with time.  Metric-like
fields that vary with time (like layer thicknesses in isopycnal/isentropic coordinates
or node locations in fully Lagrangian codes) are to be handled via the Fields
facility.
\begin{reqlist}
{\bf Priority:} 1 \\
{\bf Source:} General Req. 8.0.2 \\
{\bf Status:} Proposed \\
{\bf Verification:} System test\\
{\bf Notes:} 
\end{reqlist}

\sreq{Calculation of metrics}
All metrics may be calculated from grid locations.
\begin{reqlist}
{\bf Priority:} 1 \\
{\bf Source:} General Req. 8.0.2 \\
{\bf Status:} Proposed \\
{\bf Verification:} Unit test\\
{\bf Notes:} 
\end{reqlist}

\sreq{Reading metrics}
All metrics may be read from a standard grid file.
\begin{reqlist}
{\bf Priority:} 1 \\
{\bf Source:} General Req. 8.0.2 \\
{\bf Status:} Proposed \\
{\bf Verification:} Unit test\\
{\bf Notes:} 
\end{reqlist}

\sreq{MKS metric units}
Metrics have units of m or $m^2$, or other appropriate MKS units.
\begin{reqlist}
{\bf Priority:} 1 \\
{\bf Source:} Standard MKS requirement? \\
{\bf Status:} Proposed \\
{\bf Verification:} Code inspection \\
{\bf Notes:} 
\end{reqlist}

\sreq{Available metrics}
Available metric information includes an extensive list of grid lengths, cell
areas, and the angle between logical and physical north.  (This list will be
developed as a part of the requirement.)
\begin{reqlist}
{\bf Priority:} 1 \\
{\bf Source:} All? (required) \\
{\bf Status:} Proposed \\
{\bf Verification:} Code inspection \\
{\bf Notes:} All models require some subset of this information.
\end{reqlist}

\sreq{On-demand metrics}
In cases where one metric can be generated internally either from grid information
or from other metrics or from another PhysGrid, a method may be provided to create
that metric field only once it is clear that it will be needed.
\begin{reqlist}
{\bf Priority:} ? \\
{\bf Source:} ? (required) \\
{\bf Status:} Proposed \\
{\bf Verification:} Code inspection \\
{\bf Notes:} With extensive metric information, this may be necessary to save space.
\end{reqlist}

\sreq{Dimensionality of metrics}
In cases where metric terms are independent of one or more dimensions, they may be
stored in arrays that omit those dimensions.
\begin{reqlist}
{\bf Priority:} ? \\
{\bf Source:} GFDL/HIM currently uses this. \\
{\bf Status:} Proposed \\
{\bf Verification:} Unit test \\
{\bf Notes:} This may be necessary for adequate cache/register performance.  This may
need to be done at compile time?
\end{reqlist}

\sreq{Available structured horizontal quadrilateral grid metrics}
Available metric information includes an extensive list of grid lengths, cell
areas, and the angle between logical and physical north.
\begin{reqlist}
{\bf Priority:} 1 \\
{\bf Source:} All? (required) \\
{\bf Status:} Proposed \\
{\bf Verification:} Code inspection \\
{\bf Notes:} All using quadrilateral horizontal models require some subset of this information.
\end{reqlist}

\ssreq{Cell areas}
Cell areas shall be available for each of the 4 related subgrids.
\begin{reqlist}
{\bf Priority:} 1 \\
{\bf Source:} GFDL (required) \\
{\bf Status:} Proposed \\
{\bf Verification:} Code inspection \\
{\bf Notes:} 
\end{reqlist}

\ssreq{Half-edge lengths}
Each of the 8 half-edge lengths shall be available for each of the 4 related
subgrids.  Since neighboring cells share edges, it is desirable (although it violates
the proposed CF convention) to include only 4 fields.
\begin{reqlist}
{\bf Priority:} 2 \\
{\bf Source:} GFDL/MOM (required) \\
{\bf Status:} Proposed \\
{\bf Verification:} Code inspection \\
{\bf Notes:} 
\end{reqlist}

\ssreq{Center-to-edge distances}
Each of the 4 center to edge distances shall be available for each of the 4 related
subgrids.
\begin{reqlist}
{\bf Priority:} 2 \\
{\bf Source:} GFDL/MOM (required) \\
{\bf Status:} Proposed \\
{\bf Verification:} Code inspection \\
{\bf Notes:} 
\end{reqlist}

\ssreq{Full-edge lengths}
Each of the 4 edge lengths shall be available for each of the 4 related
subgrids.  Since neighboring cells share edges, it is desirable (although it violates
the proposed CF convention) to include only 2 fields.
\begin{reqlist}
{\bf Priority:} 2 \\
{\bf Source:} GFDL/HIM (required) \\
{\bf Status:} Proposed \\
{\bf Verification:} Code inspection \\
{\bf Notes:} 
\end{reqlist}

\ssreq{Edge-to-edge distances}
Both of the cell edge to edge distances shall be available for each of the 4 related
subgrids.
\begin{reqlist}
{\bf Priority:} 2 \\
{\bf Source:} GFDL/HIM (required) \\
{\bf Status:} Proposed \\
{\bf Verification:} Code inspection \\
{\bf Notes:} 
\end{reqlist}

\ssreq{Center-to-corner distances}
Each of the 4 center to corner distances shall be available for each of the 4
related subgrids.
\begin{reqlist}
{\bf Priority:} 3 \\
{\bf Source:}  \\
{\bf Status:} Proposed \\
{\bf Verification:} Code inspection \\
{\bf Notes:} This is used in some E-grid implementations.
\end{reqlist}

\ssreq{Cell orientation}
The angle at the cell center between logical and physical north shall be available
for each of the 4 related subgrids.
\begin{reqlist}
{\bf Priority:} 2 \\
{\bf Source:} GFDL (required) \\
{\bf Status:} Proposed \\
{\bf Verification:} Code inspection \\
{\bf Notes:} This is required for almost any non-latitude-longitude grid.
\end{reqlist}

\sreq{Available unstructured horizontal grid metrics}
Available metric information includes an cell areas, edge lengths, distances between
adjacent cell centers?
\begin{reqlist}
{\bf Priority:} 2 \\
{\bf Source:} Land Model? \\
{\bf Status:} Proposed \\
{\bf Verification:} Code inspection \\
{\bf Notes:} All using unstructured horizontal grids require some subset of this
information.
\end{reqlist}

\sreq{Vertical metrics}
Available metric information includes spacing between cell centers and faces, in
units consistent with the vertical coordinate.
\begin{reqlist}
{\bf Priority:} 1 \\
{\bf Source:} All? (required) \\
{\bf Status:} Proposed \\
{\bf Verification:} Code inspection \\
{\bf Notes:} All models require some subset of this information.
\end{reqlist}

\sreq{Cell volumes}
Available metric information may include 3-D cell volumes (or masses).  This is not
intended for use with models for which this quantity varies with time.
\begin{reqlist}
{\bf Priority:} 1 \\
{\bf Source:} GFDL/MOM, MIT (required) \\
{\bf Status:} Proposed \\
{\bf Verification:} Code inspection \\
{\bf Notes:} 
\end{reqlist}

\sreq{Methods for calculating metrics}
Metrics may be calculated by either standard or user-provided algorithms. 
\begin{reqlist}
{\bf Priority:} 2 \\
{\bf Source:} ? \\
{\bf Status:} Proposed \\
{\bf Verification:} Unit test \\
{\bf Notes:} This is basic to the extensibility of ESMF. 
\end{reqlist}

\sreq{Additional metrics}
It shall be possible for a user to specify additional metric terms to be associated
with a PhysGrid.
\begin{reqlist}
{\bf Priority:} 2 \\
{\bf Source:} ? \\
{\bf Status:} Proposed \\
{\bf Verification:} Unit test \\
{\bf Notes:} This is basic to the extensibility of ESMF. 
\end{reqlist}


%===============================================================================
\req{Grid masks}
%-------------------------------------------------------------------------------

Grid masks are logical arrays on a grid that indicates whether the various
points on the grid are a part of a physically similar subdomain. For example,
masks are used to indicate which points are a part of the ocean and which are
land.  Masks are also important for nested applications.

\begin{reqlist}
{\bf Priority:} 1 \\
{\bf Source:} General Req. 8.0.2 \\
{\bf Status:} Proposed \\
{\bf Verification:} System test\\
{\bf Notes:} 
\end{reqlist}

\sreq{Arbitrary number of masks}
A PhysGrid may have an arbitrary number of masks associated with it.
\begin{reqlist}
{\bf Priority:} 2 \\
{\bf Source:} ? \\
{\bf Status:} Proposed \\
{\bf Verification:} Code Inspection\\
{\bf Notes:} 
\end{reqlist}

\sreq{Mask names}
Each of the masks associated with a PhysGrid is associated with a
unique name.  A method shall be specified to return a pointer to a mask given
its name.
\begin{reqlist}
{\bf Priority:} 2 \\
{\bf Source:} ? \\
{\bf Status:} Proposed \\
{\bf Verification:} Unit test\\
{\bf Notes:} 
\end{reqlist}

\sreq{Category masks}
Masks may have an arbitrary number of categories. (e.g. 1 for points in the
Atlantic, 2 for the Pacific, 3 for the Mediterranean, etc.)
\begin{reqlist}
{\bf Priority:} 2 \\
{\bf Source:} ? \\
{\bf Status:} Proposed \\
{\bf Verification:} Unit test\\
{\bf Notes:} 
\end{reqlist}

\sreq{Multiplicative masks}
Masks may consist only of the values 0 or 1, for multiplicative masking.
\begin{reqlist}
{\bf Priority:} 2 \\
{\bf Source:} ? \\
{\bf Status:} Proposed \\
{\bf Verification:} Unit test\\
{\bf Notes:} 
\end{reqlist}

\sreq{Mask complement}
A method shall be provided to generate the complement of a mask.
\begin{reqlist}
{\bf Priority:} 3 \\
{\bf Source:} ? \\
{\bf Status:} Proposed \\
{\bf Verification:} Unit test\\
{\bf Notes:} 
\end{reqlist}
