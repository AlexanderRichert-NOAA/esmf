%                **** IMPORTANT NOTICE *****
% This LaTeX file has been automatically produced by ProTeX v. 1.1
% Any changes made to this file will likely be lost next time
% this file is regenerated from its source. Send questions 
% to Arlindo da Silva, dasilva@gsfc.nasa.gov
 
\parskip        0pt
\parindent      0pt
\baselineskip  11pt
 
%--------------------- SHORT-HAND MACROS ----------------------
\def\bv{\begin{verbatim}}
\def\ev{\end{verbatim}}
\def\be{\begin{equation}}
\def\ee{\end{equation}}
\def\bea{\begin{eqnarray}}
\def\eea{\end{eqnarray}}
\def\bi{\begin{itemize}}
\def\ei{\end{itemize}}
\def\bn{\begin{enumerate}}
\def\en{\end{enumerate}}
\def\bd{\begin{description}}
\def\ed{\end{description}}
\def\({\left (}
\def\){\right )}
\def\[{\left [}
\def\]{\right ]}
\def\<{\left  \langle}
\def\>{\right \rangle}
\def\cI{{\cal I}}
\def\diag{\mathop{\rm diag}}
\def\tr{\mathop{\rm tr}}
%-------------------------------------------------------------

\markboth{Left}{Source File: ESMF\_PhysGrid.F90,  Date: Thu Jan 16 09:17:35 MST 2003
}

 
%/////////////////////////////////////////////////////////////
\subsection{Fortran:  Module Interface ESMF\_PhysGridMod - Physical properties of Grid (Source File: ESMF\_PhysGrid.F90)}


  
  
   The code in this file implements the {\tt PhysGrid} class and is responsible
   for computing or initializing physical properties of grids.   Such
   properties include coordinate information necessary for describing grids,
   metric information for grid distances, grid masks and assignment of
   region identifiers to grids.
  
  ------------------------------------------------------------------------------
\bigskip{\em USES:}
\begin{verbatim}       use ESMF_BaseMod    ! ESMF base class
       use ESMF_ArrayMod
       use ESMF_DistGridMod
       implicit none
 
  ------------------------------------------------------------------------------\end{verbatim}{\sf PRIVATE TYPES:}
\begin{verbatim}       private
 
  ------------------------------------------------------------------------------
       !  ESMF_PhysGridType
       !  Description of ESMF_PhysGrid.
 
       type ESMF_PhysGridType
       sequence
       private
 
         type (ESMF_Base) :: base
         integer :: dim_num               ! number of dimensions
         type (ESMF_Array) :: dim_name    ! dimension names
         type (ESMF_Array) :: dim_units   ! dimension units
         integer :: num_corners           ! number of corners for
                                          ! each grid cell
         integer :: num_faces             ! likely assume same as num_corners
                                          ! but might specify storage of only
                                          ! 2 of 4 faces, for example
         integer :: num_metrics           ! a counter for the number of
                                          ! metrics for this subgrid
         integer :: num_lmasks            ! a counter for the number of
                                          ! logical masks for this subgrid
         integer :: num_mmasks            ! a counter for the number of
                                          ! multiplicative masks for this
                                          ! subgrid
         integer :: num_region_ids        ! a counter for the number of
                                          ! region identifiers for this
                                          ! subgrid
         integer :: global_max_i          ! cell count extents in 1st coord
                                          ! direction
         integer :: global_max_j          ! cell count extents in 2nd coord
                                          ! direction
         real :: global_min_coord1        ! coordinate extents in 1st coord
         real :: global_max_coord1        ! direction
         real :: global_min_coord2        ! coordinate extents in 2nd coord
         real :: global_max_coord2        ! direction
         integer :: local_max_i           ! cell count extents in 1st coord
                                          ! direction
         integer :: local_max_j           ! cell count extents in 2nd coord
                                          ! direction
         real :: local_min_coord1         ! coordinate extents in 1st coord
         real :: local_max_coord1         ! direction
         real :: local_min_coord2         ! coordinate extents in 2nd coord
         real :: local_max_coord2         ! direction
         type (ESMF_Array) :: center_coord1   ! coord of center of
                                              ! each cell in 1st direction
         type (ESMF_Array) :: center_coord2   ! coord of center of
                                              ! each cell in 2nd direction
         type (ESMF_Array) :: corner_coord1   ! coord of corner of
                                              ! each cell in 1st direction
         type (ESMF_Array) :: corner_coord2   ! coord of corner of
                                              ! each cell in 2nd direction
         type (ESMF_Array) :: face_coord1     ! coord of face center of
                                              ! each cell in 1st direction
         type (ESMF_Array) :: face_coord2     ! coord of face center of
                                              ! each cell in 2nd direction
         type (ESMF_Array) :: metrics         ! an array of defined metrics
                                              ! for each cell
         character (len=ESMF_MAXSTR), dimension(:), pointer :: metric_names
         type (ESMF_Array) :: lmask     ! an array of defined logical
                                        ! masks for each cell
         type (ESMF_Array) :: mmask     ! an array of defined
                                        ! multiplicative masks for each cell
         type (ESMF_Array) :: region_id ! an array of defined region identifiers
                                        ! for each cell
 
       end type
 
  ------------------------------------------------------------------------------
       !  ESMF_PhysGrid
       !  The PhysGrid data structure that is passed between languages.
 
       type ESMF_PhysGrid
       sequence
       private
         type (ESMF_PhysGridType), pointer :: ptr   ! pointer to a physgrid type
       end type
 
  ------------------------------------------------------------------------------\end{verbatim}{\sf PUBLIC TYPES:}
\begin{verbatim}    public ESMF_PhysGrid, ESMF_PhysGridType
  ------------------------------------------------------------------------------\end{verbatim}{\sf PUBLIC MEMBER FUNCTIONS:}
\begin{verbatim}    public ESMF_PhysGridCreate
    public ESMF_PhysGridDestroy
    public ESMF_PhysGridConstruct
    public ESMF_PhysGridGetInfo
    public ESMF_PhysGridSetInfo
     public ESMF_PhysGridGetCoord
    public ESMF_PhysGridSetCoord
     public ESMF_PhysGridGetMetric
     public ESMF_PhysGridSetMetric
     public ESMF_PhysGridGetLMask
     public ESMF_PhysGridSetLMask
     public ESMF_PhysGridGetMMask
     public ESMF_PhysGridSetMMask
     public ESMF_PhysGridGetRegionID
     public ESMF_PhysGridSetRegionID
    public ESMF_PhysGridValidate
    public ESMF_PhysGridPrint
  
  ------------------------------------------------------------------------------\end{verbatim}{\sf PUBLIC DATA MEMBERS:}
\begin{verbatim} 
    integer, parameter, public ::             &! internally-recognized grid metrics
       ESMF_GridMetric_Unknown          =  0, &! unknown or undefined metric
       ESMF_GridMetric_Area             =  1, &! area of grid cell
       ESMF_GridMetric_Volume           =  2, &! volume of 3-d grid cell
       ESMF_GridMetric_NorthFaceLength  =  3, &! length of horiz grid cell along north face
       ESMF_GridMetric_EastFaceLength   =  4, &! length of horiz grid cell along east  face
       ESMF_GridMetric_WestFaceLength   =  5, &! length of horiz grid cell along west  face
       ESMF_GridMetric_SouthFaceLength  =  6, &! length of horiz grid cell along south face
       ESMF_GridMetric_NNbrDist         =  7, &! cell center to north nbr center dist
       ESMF_GridMetric_ENbrDist         =  8, &! cell center to east  nbr center dist
       ESMF_GridMetric_WNbrDist         =  9, &! cell center to west  nbr center dist
       ESMF_GridMetric_SNbrDist         = 10, &! cell center to south nbr center dist
       ESMF_GridMetric_NFaceDist        = 12, &! cell center to north face distance
       ESMF_GridMetric_EFaceDist        = 13, &! cell center to east  face distance
       ESMF_GridMetric_WFaceDist        = 14, &! cell center to west  face distance
       ESMF_GridMetric_SFaceDist        = 15   ! cell center to south face distance
 
    integer, parameter, public ::             &! internally-recognized cell locations
       ESMF_CellLoc_Unknown             =  0, &! unknown or undefined metric
       ESMF_CellLoc_Center_X            =  1, &! cell center, x-coordinate  
       ESMF_CellLoc_Center_Y            =  2, &! cell center, y-coordinate  
       ESMF_CellLoc_Corner_X            =  3, &! cell vertex, x-coordinate  
       ESMF_CellLoc_Corner_Y            =  4, &! cell vertex, y-coordinate  
       ESMF_CellLoc_Face_X              =  5, &! cell face center, x-coordinate  
       ESMF_CellLoc_Face_Y              =  6   ! cell face center, y-coordinate  
 \end{verbatim}
 
%/////////////////////////////////////////////////////////////
 
\mbox{}\hrulefill\ 
 

\bigskip{\sf INTERFACE:}
\begin{verbatim}       interface ESMF_PhysGridCreate
 \end{verbatim}{\sf PRIVATE MEMBER FUNCTIONS:}
\begin{verbatim}          module procedure ESMF_PhysGridCreateNew
          module procedure ESMF_PhysGridCreateEmpty
          module procedure ESMF_PhysGridCreateInternal
          module procedure ESMF_PhysGridCreateStagger
          module procedure ESMF_PhysGridCreateRead
          module procedure ESMF_PhysGridCreateCopy
          module procedure ESMF_PhysGridCreateCutout
          module procedure ESMF_PhysGridCreateChangeResolution
          module procedure ESMF_PhysGridCreateExchange
 \end{verbatim}
{\sf DESCRIPTION:\\ }


       This interface provides a single entry point for PhysGrid create
       methods.
   
%/////////////////////////////////////////////////////////////
 
\mbox{}\hrulefill\ 
 

\bigskip{\sf INTERFACE:}
\begin{verbatim}       interface ESMF_PhysGridSetCoord
 \end{verbatim}{\sf PRIVATE MEMBER FUNCTIONS:}
\begin{verbatim}          module procedure ESMF_PhysGridSetCoordFromArray
          module procedure ESMF_PhysGridSetCoordInternal
          module procedure ESMF_PhysGridSetCoordStagger
          module procedure ESMF_PhysGridSetCoordRead
 \end{verbatim}
{\sf DESCRIPTION:\\ }


       This interface provides a single function for various methods of
       computing grid coordinates.
   
%/////////////////////////////////////////////////////////////
 
\mbox{}\hrulefill\ 
 

\bigskip{\sf INTERFACE:}
\begin{verbatim}       interface ESMF_PhysGridSetMetric
 \end{verbatim}{\sf PRIVATE MEMBER FUNCTIONS:}
\begin{verbatim}          module procedure ESMF_PhysGridSetMetricFromArray
          module procedure ESMF_PhysGridSetMetricInternal
          module procedure ESMF_PhysGridSetMetricStagger
          module procedure ESMF_PhysGridSetMetricRead
 \end{verbatim}
{\sf DESCRIPTION:\\ }


       This interface provides a single function for computing or initializing
       grid metric information.
   
%/////////////////////////////////////////////////////////////
 
\mbox{}\hrulefill\ 
 

\bigskip{\sf INTERFACE:}
\begin{verbatim}       interface ESMF_PhysGridConstruct
 \end{verbatim}{\sf PRIVATE MEMBER FUNCTIONS:}
\begin{verbatim}          module procedure ESMF_PhysGridConstructNew
          module procedure ESMF_PhysGridConstructInternal
 \end{verbatim}
{\sf DESCRIPTION:\\ }


       This interface provides a single entry point for methods that construct
       a complete {\tt PhysGrid}.
   
%/////////////////////////////////////////////////////////////
 
\mbox{}\hrulefill\ 
 
\subsubsection{ESMF\_PhysGridCreateNew - Create a new PhysGrid}


 
\bigskip{\sf INTERFACE:}
\begin{verbatim}       function ESMF_PhysGridCreateNew(name, rc)\end{verbatim}{\em RETURN VALUE:}
\begin{verbatim}       type(ESMF_PhysGrid) :: ESMF_PhysGridCreateNew\end{verbatim}{\em ARGUMENTS:}
\begin{verbatim}       character (len = *), intent(in), optional :: name  
       integer, intent(out), optional :: rc               
 \end{verbatim}
{\sf DESCRIPTION:\\ }


       Allocates memory for a new {\tt PhysGrid} object and constructs its
       internals.  Returns a pointer to a new {\tt PhysGrid}.
  
       The arguments are:
       \begin{description}
       \item[[name]]
            {\tt PhysGrid} name.
       \item[[rc]] 
            Return code; equals {\tt ESMF\_SUCCESS} if there are no errors.
     \end{description}
  
\bigskip{\sf REQUIREMENTS:}
\begin{verbatim} \end{verbatim}
 
%/////////////////////////////////////////////////////////////
 
\mbox{}\hrulefill\ 
 
\subsubsection{ESMF\_PhysGridCreateInternal - Create a new PhysGrid internally}


\bigskip{\sf INTERFACE:}
\begin{verbatim}       function ESMF_PhysGridCreateInternal(local_min_coord1, &
                                            local_max_coord1, &
                                            local_nmax1, &
                                            local_min_coord2, &
                                            local_max_coord2, &
                                            local_nmax2, &
                                            global_min_coord1, &
                                            global_max_coord1, &
                                            global_nmax1, &
                                            global_min_coord2, &
                                            global_max_coord2, &
                                            global_nmax2, name, rc)\end{verbatim}{\em RETURN VALUE:}
\begin{verbatim}       type(ESMF_PhysGrid) :: ESMF_PhysGridCreateInternal\end{verbatim}{\em ARGUMENTS:}
\begin{verbatim}       real, intent(in) :: local_min_coord1
       real, intent(in) :: local_max_coord1
       integer, intent(in) :: local_nmax1
       real, intent(in) :: local_min_coord2
       real, intent(in) :: local_max_coord2
       integer, intent(in) :: local_nmax2
       real, intent(in) :: global_min_coord1
       real, intent(in) :: global_max_coord1
       integer, intent(in) :: global_nmax1
       real, intent(in) :: global_min_coord2
       real, intent(in) :: global_max_coord2
       integer, intent(in) :: global_nmax2
       character (len = *), intent(in), optional :: name
       integer, intent(out), optional :: rc               \end{verbatim}
{\sf DESCRIPTION:\\ }


       Allocates memory for a new {\tt PhysGrid} object, constructs its
       internals, and internally generates the {\tt PhysGrid}.  Returns a
       pointer to the new {\tt PhysGrid}.
  
       The arguments are:
       \begin{description}
       \item[[local\_min\_coord1]]
            Minimum local physical coordinate in the 1st coordinate direction.
       \item[[local\_max\_coord1]]
            Maximum local physical coordinate in the 1st coordinate direction.
       \item[[local\_nmax1]]
            Number of local grid increments in the 1st coordinate direction.
       \item[[local\_min\_coord2]]
            Minimum local physical coordinate in the 2nd coordinate direction.
       \item[[local\_max\_coord2]]
            Maximum local physical coordinate in the 2nd coordinate direction.
       \item[[local\_nmax2]]
            Number of local grid increments in the 2nd coordinate direction.
       \item[[global\_min\_coord1]]
            Minimum global physical coordinate in the 1st coordinate direction.
       \item[[global\_max\_coord1]]
            Maximum global physical coordinate in the 1st coordinate direction.
       \item[[global\_nmax1]]
            Number of global grid increments in the 1st coordinate direction.
       \item[[global\_min\_coord2]]
            Minimum global physical coordinate in the 2nd coordinate direction.
       \item[[global\_max\_coord2]]
            Maximum global physical coordinate in the 2nd coordinate direction.
       \item[[global\_nmax2]]
            Number of global grid increments in the 2nd coordinate direction.
       \item[[name]]
            {\tt PhysGrid} name.
       \item[[rc]] 
            Return code; equals {\tt ESMF\_SUCCESS} if there are no errors.
     \end{description}
  
\bigskip{\sf REQUIREMENTS:}
\begin{verbatim} \end{verbatim}
 
%/////////////////////////////////////////////////////////////
 
\mbox{}\hrulefill\ 
 
\subsubsection{ESMF\_PhysGridDestroy - Free all resources associated with a PhysGrid }


\bigskip{\sf INTERFACE:}
\begin{verbatim}       subroutine ESMF_PhysGridDestroy(physgrid, rc)\end{verbatim}{\em ARGUMENTS:}
\begin{verbatim}       type(ESMF_PhysGrid), intent(in) :: physgrid   
       integer, intent(out), optional :: rc        \end{verbatim}
{\sf DESCRIPTION:\\ }


       Destroys a {\tt PhysGrid} object previously allocated
       via an {\tt ESMF\_PhysGridCreate routine}.
  
       The arguments are:
       \begin{description}
       \item[physgrid] 
            The class to be destroyed.
       \item[[rc]] 
            Return code; equals {\tt ESMF\_SUCCESS} if there are no errors.
       \end{description}
   
%/////////////////////////////////////////////////////////////
 
\mbox{}\hrulefill\ 
 
\subsubsection{ESMF\_PhysGridConstructNew - Construct the internals of an allocated PhysGrid}


 
\bigskip{\sf INTERFACE:}
\begin{verbatim}       subroutine ESMF_PhysGridConstructNew(physgrid, name, rc)\end{verbatim}{\em ARGUMENTS:}
\begin{verbatim}       type(ESMF_PhysGridType) :: physgrid  
       character (len = *), intent(in), optional :: name  ! name
       integer, intent(out), optional :: rc               ! return code\end{verbatim}
{\sf DESCRIPTION:\\ }


       ESMF routine which fills in the contents of an already
       allocated {\tt PhysGrid} object.  May perform additional allocations
       as needed.  Must call the corresponding ESMF\_PhysGridDestruct
       routine to free the additional memory.  Intended for internal
       ESMF use only; end-users use {\tt ESMF\_PhysGridCreate}, which calls
       {\tt ESMF\_PhysGridConstruct}. 
  
       The arguments are:
       \begin{description}
       \item[physgrid] 
            Pointer to a {\tt PhysGrid}.
       \item[[name]] 
            {\tt PhysGrid} name.
       \item[[rc]] 
            Return code; equals {\tt ESMF\_SUCCESS} if there are no errors.
       \end{description}
  
\bigskip{\sf REQUIREMENTS:}
\begin{verbatim} \end{verbatim}
 
%/////////////////////////////////////////////////////////////
 
\mbox{}\hrulefill\ 
 
\subsubsection{ESMF\_PhysGridConstructInternal - Construct the internals of an allocated PhysGrid}


\bigskip{\sf INTERFACE:}
\begin{verbatim}       subroutine ESMF_PhysGridConstructInternal(physgrid, &
                                                 local_min_coord1, &
                                                 local_max_coord1, &
                                                 local_nmax1, &
                                                 local_min_coord2, &
                                                 local_max_coord2, &
                                                 local_nmax2, &
                                                 global_min_coord1, &
                                                 global_max_coord1, &
                                                 global_nmax1, &
                                                 global_min_coord2, &
                                                 global_max_coord2, &
                                                 global_nmax2, name, rc)\end{verbatim}{\em ARGUMENTS:}
\begin{verbatim}       type(ESMF_PhysGridType) :: physgrid  
       real, intent(in) :: local_min_coord1
       real, intent(in) :: local_max_coord1
       integer, intent(in) :: local_nmax1
       real, intent(in) :: local_min_coord2
       real, intent(in) :: local_max_coord2
       integer, intent(in) :: local_nmax2
       real, intent(in) :: global_min_coord1
       real, intent(in) :: global_max_coord1
       integer, intent(in) :: global_nmax1
       real, intent(in) :: global_min_coord2
       real, intent(in) :: global_max_coord2
       integer, intent(in) :: global_nmax2
       character (len = *), intent(in), optional :: name  ! name
       integer, intent(out), optional :: rc               ! return code\end{verbatim}
{\sf DESCRIPTION:\\ }


       ESMF routine which fills in the contents of an already
       allocated {\tt PhysGrid} object.  May perform additional allocations
       as needed.  Must call the corresponding ESMF\_PhysGridDestruct
       routine to free the additional memory.  Intended for internal
       ESMF use only; end-users use {\tt ESMF\_PhysGridCreate}, which calls
       {\tt ESMF\_PhysGridConstruct}. 
  
       The arguments are:
       \begin{description}
       \item[physgrid] 
            Pointer to a {\tt PhysGrid}.
       \item[[local\_min\_coord1]]
            Minimum local physical coordinate in the 1st coordinate direction.
       \item[[local\_max\_coord1]]
            Maximum local physical coordinate in the 1st coordinate direction.
       \item[[local\_nmax1]]
            Number of local grid increments in the 1st coordinate direction.
       \item[[local\_min\_coord2]]
            Minimum local physical coordinate in the 2nd coordinate direction.
       \item[[local\_max\_coord2]]
            Maximum local physical coordinate in the 2nd coordinate direction.
       \item[[local\_nmax2]]
            Number of local grid increments in the 2nd coordinate direction.
       \item[[global\_min\_coord1]]
            Minimum global physical coordinate in the 1st coordinate direction.
       \item[[global\_max\_coord1]]
            Maximum global physical coordinate in the 1st coordinate direction.
       \item[[global\_nmax1]]
            Number of global grid increments in the 1st coordinate direction.
       \item[[global\_min\_coord2]]
            Minimum global physical coordinate in the 2nd coordinate direction.
       \item[[global\_max\_coord2]]
            Maximum global physical coordinate in the 2nd coordinate direction.
       \item[[global\_nmax2]]
            Number of global grid increments in the 2nd coordinate direction.
       \item[[name]]
            {\tt PhysGrid} name.
       \item[[rc]] 
            Return code; equals {\tt ESMF\_SUCCESS} if there are no errors.
       \end{description}
  
\bigskip{\sf REQUIREMENTS:}
\begin{verbatim} \end{verbatim}
 
%/////////////////////////////////////////////////////////////
 
\mbox{}\hrulefill\ 
 
\subsubsection{ESMF\_PhysGridDestruct - Free any PhysGrid memory allocated internally}


\bigskip{\sf INTERFACE:}
\begin{verbatim}       subroutine ESMF_PhysGridDestruct(physgrid, rc)\end{verbatim}{\em ARGUMENTS:}
\begin{verbatim}       type(ESMF_PhysGrid), intent(in) :: physgrid    
       integer, intent(out), optional :: rc         \end{verbatim}
{\sf DESCRIPTION:\\ }


       ESMF routine which deallocates any space allocated by
      {\tt  ESMF\_PhysGridConstruct}, does any additional cleanup before the
       original PhysGrid object is freed.  Intended for internal ESMF
       use only; end-users use {\tt ESMF\_PhysGridDestroy}, which calls
       {\tt ESMF\_PhysGridDestruct}.  
  
       The arguments are:
       \begin{description}
       \item[physgrid] 
            The class to be destructed.
       \item[[rc]] 
            Return code; equals {\tt ESMF\_SUCCESS} if there are no errors.
       \end{description}
   
%/////////////////////////////////////////////////////////////
 
\mbox{}\hrulefill\ 
 
\subsubsection{ESMF\_PhysGridGetInfo - Get information from a PhysGrid}


 
\bigskip{\sf INTERFACE:}
\begin{verbatim}       subroutine ESMF_PhysGridGetInfo(physgrid, local_min_coord1, &
                                       local_max_coord1, local_nmax1, &
                                       local_min_coord2, local_max_coord2, &
                                       local_nmax2, global_min_coord1, &
                                       global_max_coord1, global_nmax1, &
                                       global_min_coord2, global_max_coord2, &
                                       global_nmax2, rc)\end{verbatim}{\em ARGUMENTS:}
\begin{verbatim}       type(ESMF_PhysGridType) :: physgrid
       real, intent(inout), optional :: local_min_coord1
       real, intent(inout), optional :: local_max_coord1
       integer, intent(inout), optional :: local_nmax1
       real, intent(inout), optional :: local_min_coord2
       real, intent(inout), optional :: local_max_coord2
       integer, intent(inout), optional :: local_nmax2
       real, intent(inout), optional :: global_min_coord1
       real, intent(inout), optional :: global_max_coord1
       integer, intent(inout), optional :: global_nmax1
       real, intent(inout), optional :: global_min_coord2
       real, intent(inout), optional :: global_max_coord2
       integer, intent(inout), optional :: global_nmax2
       integer, intent(out), optional :: rc              \end{verbatim}
{\sf DESCRIPTION:\\ }


       This version gets a variety of information about a physgrid, depending
       on a list of optional arguments.
  
       The arguments are:
       \begin{description}
       \item[physgrid] 
            Pointer to a {\tt PhysGrid}.
       \item[[name]]
            {\tt PhysGrid} name.
       \item[[local\_min\_coord1]]
            Minimum local physical coordinate in the 1st coordinate direction.
       \item[[local\_max\_coord1]]
            Maximum local physical coordinate in the 1st coordinate direction.
       \item[[local\_nmax1]]
            Number of local grid increments in the 1st coordinate direction.
       \item[[local\_min\_coord2]]
            Minimum local physical coordinate in the 2nd coordinate direction.
       \item[[local\_max\_coord2]]
            Maximum local physical coordinate in the 2nd coordinate direction.
       \item[[local\_nmax2]]
            Number of local grid increments in the 2nd coordinate direction.
       \item[[global\_min\_coord1]]
            Minimum global physical coordinate in the 1st coordinate direction.
       \item[[global\_max\_coord1]]
            Maximum global physical coordinate in the 1st coordinate direction.
       \item[[global\_nmax1]]
            Number of global grid increments in the 1st coordinate direction.
       \item[[global\_min\_coord2]]
            Minimum global physical coordinate in the 2nd coordinate direction.
       \item[[global\_max\_coord2]]
            Maximum global physical coordinate in the 2nd coordinate direction.
       \item[[global\_nmax2]]
            Number of global grid increments in the 2nd coordinate direction.
       \item[[rc]] 
            Return code; equals {\tt ESMF\_SUCCESS} if there are no errors.
       \end{description}
   
%/////////////////////////////////////////////////////////////
 
\mbox{}\hrulefill\ 
 
\subsubsection{ESMF\_PhysGridSetInfo - Set information for a PhysGrid}


 
\bigskip{\sf INTERFACE:}
\begin{verbatim}       subroutine ESMF_PhysGridSetInfo(physgrid, local_min_coord1, &
                                       local_max_coord1, local_nmax1, &
                                       local_min_coord2, local_max_coord2, &
                                       local_nmax2, global_min_coord1, &
                                       global_max_coord1, global_nmax1, &
                                       global_min_coord2, global_max_coord2, &
                                       global_nmax2, rc)\end{verbatim}{\em ARGUMENTS:}
\begin{verbatim}       type(ESMF_PhysGridType) :: physgrid
       real, intent(in), optional :: local_min_coord1
       real, intent(in), optional :: local_max_coord1
       integer, intent(in), optional :: local_nmax1
       real, intent(in), optional :: local_min_coord2
       real, intent(in), optional :: local_max_coord2
       integer, intent(in), optional :: local_nmax2
       real, intent(in), optional :: global_min_coord1
       real, intent(in), optional :: global_max_coord1
       integer, intent(in), optional :: global_nmax1
       real, intent(in), optional :: global_min_coord2
       real, intent(in), optional :: global_max_coord2
       integer, intent(in), optional :: global_nmax2
       integer, intent(out), optional :: rc              \end{verbatim}
{\sf DESCRIPTION:\\ }


       This version sets a variety of information about a physgrid, depending
       on a list of optional arguments.
  
       The arguments are:
       \begin{description}
       \item[physgrid] 
            Pointer to a {\tt PhysGrid}.
       \item[[name]]
            {\tt PhysGrid} name.
       \item[[local\_min\_coord1]]
            Minimum local physical coordinate in the 1st coordinate direction.
       \item[[local\_max\_coord1]]
            Maximum local physical coordinate in the 1st coordinate direction.
       \item[[local\_nmax1]]
            Number of local grid increments in the 1st coordinate direction.
       \item[[local\_min\_coord2]]
            Minimum local physical coordinate in the 2nd coordinate direction.
       \item[[local\_max\_coord2]]
            Maximum local physical coordinate in the 2nd coordinate direction.
       \item[[local\_nmax2]]
            Number of local grid increments in the 2nd coordinate direction.
       \item[[global\_min\_coord1]]
            Minimum global physical coordinate in the 1st coordinate direction.
       \item[[global\_max\_coord1]]
            Maximum global physical coordinate in the 1st coordinate direction.
       \item[[global\_nmax1]]
            Number of global grid increments in the 1st coordinate direction.
       \item[[global\_min\_coord2]]
            Minimum global physical coordinate in the 2nd coordinate direction.
       \item[[global\_max\_coord2]]
            Maximum global physical coordinate in the 2nd coordinate direction.
       \item[[global\_nmax2]]
            Number of global grid increments in the 2nd coordinate direction.
       \item[[rc]] 
            Return code; equals {\tt ESMF\_SUCCESS} if there are no errors.
       \end{description}
   
%/////////////////////////////////////////////////////////////
 
\mbox{}\hrulefill\ 
 
\subsubsection{ESMF\_PhysGridGetConfig - Get configuration information from a PhysGrid}


 
\bigskip{\sf INTERFACE:}
\begin{verbatim}       subroutine ESMF_PhysGridGetConfig(physgrid, config, rc)\end{verbatim}{\em ARGUMENTS:}
\begin{verbatim}       type(ESMF_PhysGrid), intent(in) :: physgrid
       integer, intent(out) :: config   
       integer, intent(out), optional :: rc              \end{verbatim}
{\sf DESCRIPTION:\\ }


       Returns the set of resources the PhysGrid object was configured with.
  
       The arguments are:
       \begin{description}
       \item[physgrid] 
            Class to be queried.
       \item[config]
            Configuration information.         
       \item[[rc]] 
            Return code; equals {\tt ESMF\_SUCCESS} if there are no errors.
       \end{description}
   
%/////////////////////////////////////////////////////////////
 
\mbox{}\hrulefill\ 
 
\subsubsection{ESMF\_PhysGridSetConfig - Set configuration information for a PhysGrid}


 
\bigskip{\sf INTERFACE:}
\begin{verbatim}       subroutine ESMF_PhysGridSetConfig(physgrid, config, rc)\end{verbatim}{\em ARGUMENTS:}
\begin{verbatim}       type(ESMF_PhysGrid), intent(in) :: physgrid
       integer, intent(in) :: config   
       integer, intent(out), optional :: rc             
 \end{verbatim}
{\sf DESCRIPTION:\\ }


       Configures the PhysGrid object with set of resources given.
  
       The arguments are:
       \begin{description}
       \item[physgrid] 
            Class to be configured.
       \item[config]
            Configuration information.         
       \item[[rc]] 
            Return code; equals {\tt ESMF\_SUCCESS} if there are no errors.
       \end{description}
   
%/////////////////////////////////////////////////////////////
 
\mbox{}\hrulefill\ 
 
\subsubsection{ESMF\_PhysGridGetValue - Get <Value> for a PhysGrid}


 
\bigskip{\sf INTERFACE:}
\begin{verbatim}       subroutine ESMF_PhysGridGetValue(physgrid, value, rc)\end{verbatim}{\em ARGUMENTS:}
\begin{verbatim}       type(ESMF_PhysGrid), intent(in) :: physgrid
       integer, intent(out) :: value
       integer, intent(out), optional :: rc             
 \end{verbatim}
{\sf DESCRIPTION:\\ }


       Returns the value of PhysGrid attribute <Value>.
       May be multiple routines, one per attribute.
  
       The arguments are:
       \begin{description}
       \item[physgrid] 
            Class to be queried.
       \item[value]
            Value to be retrieved.         
       \item[[rc]] 
            Return code; equals {\tt ESMF\_SUCCESS} if there are no errors.
       \end{description}
   
%/////////////////////////////////////////////////////////////
 
\mbox{}\hrulefill\ 
 
\subsubsection{ESMF\_PhysGridSetValue - Set <Value> for a PhysGrid}


 
\bigskip{\sf INTERFACE:}
\begin{verbatim}       subroutine ESMF_PhysGridSetValue(PhysGrid, value, rc)\end{verbatim}{\em ARGUMENTS:}
\begin{verbatim}       type(ESMF_PhysGrid), intent(in) :: physgrid
       integer, intent(in) :: value
       integer, intent(out), optional :: rc            
 \end{verbatim}
{\sf DESCRIPTION:\\ }


       Set a PhysGrid attribute with the given value.
       May be multiple routines, one per attribute.
  
       The arguments are:
       \begin{description}
       \item[physgrid] 
            Class to be modified.
       \item[value]
            Value to be set.         
       \item[[rc]] 
            Return code; equals {\tt ESMF\_SUCCESS} if there are no errors.
       \end{description}
   
%/////////////////////////////////////////////////////////////
 
\mbox{}\hrulefill\ 
 
\subsubsection{ESMF\_PhysGridSetCoordInternal - Set Coords for a PhysGrid internally}


 
\bigskip{\sf INTERFACE:}
\begin{verbatim}       subroutine ESMF_PhysGridSetCoordInternal(physgrid, ncoord_locs, &
                                                coord_loc, &
                                                global_nmin1, global_nmax1, &
                                                global_nmin2, global_nmax2, &
                                                delta1, delta2, rc)\end{verbatim}{\em ARGUMENTS:}
\begin{verbatim}       type(ESMF_PhysGridType) :: physgrid
       integer, intent(in) :: ncoord_locs
       integer, dimension(:), intent(in) :: coord_loc
       integer, intent(in) :: global_nmin1
       integer, intent(in) :: global_nmax1
       integer, intent(in) :: global_nmin2
       integer, intent(in) :: global_nmax2
       real, intent(in) :: delta1
       real, intent(in) :: delta2
       integer, intent(out), optional :: rc            
 \end{verbatim}
{\sf DESCRIPTION:\\ }


       Compute a PhysGrid's coordinates from a given gridtype and set of
       physical parameters.
  
       The arguments are:
       \begin{description}
       \item[physgrid]
            Class to be modified.
       \item[[ncoord\_locs]]
            Number of coordinate location specifiers
       \item[[coord\_loc]]
            Array of integer specifiers to denote coordinate location relative
            to the cell (center, corner, face)
       \item[[global\_nmin1]]
            Global minimum counter in the 1st direction
       \item[[global\_nmax1]]
            Global maximum counter in the 1st direction
       \item[[global\_nmin2]]
            Global minimum counter in the 2nd direction
       \item[[global\_nmax2]]
            Global maximum counter in the 2nd direction
       \item[[delta1]]
            Grid cell size in the 1st direction (assumed constant for now)
       \item[[delta2]]
            Grid cell size in the 2nd direction (assumed constant for now)
       \item[[rc]]
            Return code; equals {\tt ESMF\_SUCCESS} if there are no errors.
       \end{description}
   
%/////////////////////////////////////////////////////////////
 
\mbox{}\hrulefill\ 
 
\subsubsection{ESMF\_PhysGridValidate - Check internal consistency of a PhysGrid}


 
\bigskip{\sf INTERFACE:}
\begin{verbatim}       subroutine ESMF_PhysGridValidate(physgrid, opt, rc)\end{verbatim}{\em ARGUMENTS:}
\begin{verbatim}       type(ESMF_PhysGrid), intent(in) :: physgrid       
       character (len=*), intent(in), optional :: opt    
       integer, intent(out), optional :: rc            \end{verbatim}
{\sf DESCRIPTION:\\ }


       Validates that a PhysGrid is internally consistent.
  
       The arguments are:
       \begin{description}
       \item[physgrid] 
            Class to be queried.
       \item[[opt]]
            Validation options.
       \item[[rc]] 
            Return code; equals {\tt ESMF\_SUCCESS} if there are no errors.
       \end{description}
   
%/////////////////////////////////////////////////////////////
 
\mbox{}\hrulefill\ 
 
\subsubsection{ESMF\_PhysGridPrint - Print the contents of a PhysGrid}


 
\bigskip{\sf INTERFACE:}
\begin{verbatim}       subroutine ESMF_PhysGridPrint(physgrid, opt, rc)\end{verbatim}{\em ARGUMENTS:}
\begin{verbatim}       type(ESMF_PhysGrid), intent(in) :: physgrid      
       character (len=*), intent(in) :: opt      
       integer, intent(out), optional :: rc           \end{verbatim}
{\sf DESCRIPTION:\\ }


        Print information about a PhysGrid.  
  
       The arguments are:
       \begin{description}
       \item[physgrid] 
            Class to be queried.
       \item[[opt]]
            Print ptions that control the type of information and level of 
            detail.
       \item[[rc]] 
            Return code; equals {\tt ESMF\_SUCCESS} if there are no errors.
       \end{description}
  
%...............................................................
