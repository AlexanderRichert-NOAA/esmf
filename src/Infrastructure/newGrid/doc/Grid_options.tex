% $Id: Grid_options.tex,v 1.5 2007/05/23 17:32:14 cdeluca Exp $

\subsubsection{ESMF\_GridConn}\label{sec:opt:gridconn}
{\sf DESCRIPTION:\\}
The {\tt ESMF\_GridCreateShape} command has three specific arguments
{\tt connDim1}, {\tt connDim2}, and {\tt connDim3}. These can be used
to setup different types of connections at the ends of each dimension
of a Tile.  Each of these parameters is a two element array. The first
element is the connection type at the minimum end of the dimension
and the second is the connection type at the maximum end. The default
value for all the connections is ESMF\_GRIDCONN\_NONE, specifying no
connection.

\medskip
\begin{description}
\item [ESMF\_GRIDCONN\_NONE] No connection.

\item [ESMF\_GRIDCONN\_PERIODIC] Periodic connection.

\item [ESMF\_GRIDCONN\_POLE] Pole.

\item [ESMF\_GRIDCONN\_BIPOLE] Bipole connection.
\end{description}

\subsubsection{ESMF\_GridStatus}\label{sec:opt:gridstatus}

 {\sf DESCRIPTION:\\}
The ESMF Grid class can exist in three states. These states are
present so that the library code can detect if a Grid has been
appropriately setup for the task at hand. The following
are the valid values of ESMF\_GRIDSTATUS.

\medskip
\begin{description}
\item [ESMF\_GRIDSTATUS\_NOT\_READY:] Status after a grid has been created with 
      {\tt ESMF\_GridCreateEmpty}. The only action which may be performed on a
      grid in this state is the setting grid parameters or a commit to move 
      to the next status. 
\item [ESMF\_GRIDSTATUS\_SHAPE\_READY:] Status after a grid has been created with 
      a non-empty create. The grid is complete except that it contains no coordinate
      information. Most actions may be performed on a grid in this state
      except those requiring coordinate values (e.g. regrid).
\item [ESMF\_GRIDSTATUS\_REGRID\_READY:] The grid contains valid coordinate
      values and is now ready to be used in regrid. 
\end{description}


\subsubsection{ESMF\_StaggerLoc}\label{sec:opt:staggerloc}

 {\sf DESCRIPTION:\\}
 In the ESMF Grid class, data can be located at different positions in a
 Grid cell.  When setting or retrieving coordinate data the stagger location is
 specified to tell the Grid method  from where in the cell to get the data. 
 Although the user may define their own custom stagger locations, 
 ESMF provides a set of predefined locations for ease of use. The
following are the valid predefined stagger locations. 

\medskip

\begin{verbatim}
    
                 2----4----2
                 |         |
                 |         |
                 3    1    3
                 |         |
                 |         |
                 2----4----2
 
      Diagram Illustrating 2D Stagger Locations
   (Numbers correspond to bracketed numbers in list.)

\end{verbatim}

The 2D predefined stagger locations are:\\
\begin{description}
\item [ESMF\_STAGGERLOC\_CENTER:] The center of the cell [1].
\item [ESMF\_STAGGERLOC\_CORNER:] The corners of the cell [2].
\item [ESMF\_STAGGERLOC\_EDGE1:] The edges offset from the center in the 1st dimension [3].
\item [ESMF\_STAGGERLOC\_EDGE2:] The edges offset from the center in the 2nd dimension [4].
\end{description}

\medskip

\begin{verbatim}
    
      5----7----5        2----4----2         5----7----5
      |         |        |         |         |         |
      |         |        |         |         |         | 
      6    8    6        3    1    3         6    8    6
      |         |        |         |         |         |
      |         |        |         |         |         |
      5----7----5        2----4----2         5----7----5

       Cell Top          Cell Middle         Cell Bottom
         

           Diagram Illustrating 3D Stagger Locations
       (Numbers correspond to bracketed numbers in list.)

\end{verbatim}

The 3D predefined stagger locations are:\\
\begin{description}
\item [ESMF\_STAGGERLOC\_CENTER\_VCENTER:] The center of the 3D cell [1].
\item [ESMF\_STAGGERLOC\_CORNER\_VCENTER:] Half way up the vertical edges of the cell [2].
\item [ESMF\_STAGGERLOC\_EDGE1\_VCENTER:] The center of the face bounded by edge 1 and the vertical dimension [3].
\item [ESMF\_STAGGERLOC\_EDGE2\_VCENTER:] The center of the face bounded by edge 2 and the vertical dimension [4]. 
\item [ESMF\_STAGGERLOC\_CORNER\_VFACE:] The corners of the 3D cell [5].
\item [ESMF\_STAGGERLOC\_EDGE1\_VFACE:] The center of the edges of the 3D cell parallel offset from the center in the 1st dimension [6].
\item [ESMF\_STAGGERLOC\_EDGE2\_VFACE:] The center of the edges of the 3D cell parallel offset from the center in the 2nd dimension [7].
\item [ESMF\_STAGGERLOC\_CENTER\_VFACE:] The center of the top and bottom face. The face bounded by the 1st and 2nd dimensions [8]. 
\end{description}


