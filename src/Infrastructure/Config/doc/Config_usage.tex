% $Id: Config_usage.tex,v 1.13 2010/10/15 18:10:19 rokuingh Exp $

 \subsubsection{Resource files}

   A {\em Resource File (RF)} is a text file consisting of list of 
   {\em label}-{\em value} pairs. There is a limit of 250 characters 
   per line and the Resource File can contain a maximum of 200 records. 
   Each {\em label} should be followed by some data, the {\em value}. 
   An example Resource File follows.  It is the file used in the example 
   below. 

 \begin{verbatim}
 # This is an example Resource File.  
 # It contains a list of <label,value> pairs.
 # The colon after the label is required. 

 # The values after the label can be an list.
 # Multiple types are authorized.
  
  my_file_names:         jan87.dat jan88.dat jan89.dat  # all strings
  constants:             3.1415   25                    # float and integer
  my_favorite_colors:    green blue 022               


 # Or, the data can be a list of single value pairs. 
 # It is simplier to retrieve data in this format:

  radius_of_the_earth:   6.37E6         
  parameter_1:           89
  parameter_2:           78.2
  input_file_name:       dummy_input.netcdf 


 # Or, the data can be located in a table using the following
 # syntax:

  my_table_name::
   1000     3000     263.0
    925     3000     263.0
    850     3000     263.0
    700     3000     269.0
    500     3000     287.0
    400     3000     295.8
    300     3000     295.8
  ::
 \end{verbatim}

 Note that the colon after the label is required and that the double colon is required
 to declare tabular data. 

 Resource files are intended for random access (except between ::'s in a 
 table definition). This means that order in which a particular 
 {\em label-value} pair is retreived is not dependent upon the original order 
 of the pairs. The only exception to this, however, is when the same {\em label} appears 
 multiple times within the Resource File.




