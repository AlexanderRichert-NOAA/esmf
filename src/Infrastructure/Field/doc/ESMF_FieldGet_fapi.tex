%                **** IMPORTANT NOTICE *****
% This LaTeX file has been automatically produced by ProTeX v. 1.1
% Any changes made to this file will likely be lost next time
% this file is regenerated from its source. Send questions 
% to Arlindo da Silva, dasilva@gsfc.nasa.gov
 
\setlength{\parskip}{0pt}
\setlength{\parindent}{0pt}
\setlength{\baselineskip}{11pt}
 
%--------------------- SHORT-HAND MACROS ----------------------
\def\bv{\begin{verbatim}}
\def\ev{\end{verbatim}}
\def\be{\begin{equation}}
\def\ee{\end{equation}}
\def\bea{\begin{eqnarray}}
\def\eea{\end{eqnarray}}
\def\bi{\begin{itemize}}
\def\ei{\end{itemize}}
\def\bn{\begin{enumerate}}
\def\en{\end{enumerate}}
\def\bd{\begin{description}}
\def\ed{\end{description}}
\def\({\left (}
\def\){\right )}
\def\[{\left [}
\def\]{\right ]}
\def\<{\left  \langle}
\def\>{\right \rangle}
\def\cI{{\cal I}}
\def\diag{\mathop{\rm diag}}
\def\tr{\mathop{\rm tr}}
%-------------------------------------------------------------

\markboth{Left}{Source File: ESMF\_FieldGet.F90,  Date: Mon Mar 15 14:58:14 MST 2004
}

 
%/////////////////////////////////////////////////////////////
\subsubsection [ESMF\_FieldGetDataPointer] {ESMF\_FieldGetDataPointer -- Get an F90 pointer to the data contents}


 
\bigskip{\sf INTERFACE:}
\begin{verbatim}      interface ESMF_FieldGetDataPointer
 \end{verbatim}{\sf PRIVATE MEMBER FUNCTIONS:}
\begin{verbatim}       ! < declarations of interfaces for each T/K/R >
  ------------------------------------------------------------------------------ 
   <This section created by macro - do not edit directly> 
  module procedure ESMF_FieldGetDataPointer1DI2 
  module procedure ESMF_FieldGetDataPointer1DI4 
  module procedure ESMF_FieldGetDataPointer1DI8 
  module procedure ESMF_FieldGetDataPointer2DI2 
  module procedure ESMF_FieldGetDataPointer2DI4 
  module procedure ESMF_FieldGetDataPointer2DI8 
  module procedure ESMF_FieldGetDataPointer3DI2 
  module procedure ESMF_FieldGetDataPointer3DI4 
  module procedure ESMF_FieldGetDataPointer3DI8 
  module procedure ESMF_FieldGetDataPointer4DI2 
  module procedure ESMF_FieldGetDataPointer4DI4 
  module procedure ESMF_FieldGetDataPointer4DI8 
  module procedure ESMF_FieldGetDataPointer5DI2 
  module procedure ESMF_FieldGetDataPointer5DI4 
  module procedure ESMF_FieldGetDataPointer5DI8 
  module procedure ESMF_FieldGetDataPointer1DR4 
  module procedure ESMF_FieldGetDataPointer1DR8 
  module procedure ESMF_FieldGetDataPointer2DR4 
  module procedure ESMF_FieldGetDataPointer2DR8 
  module procedure ESMF_FieldGetDataPointer3DR4 
  module procedure ESMF_FieldGetDataPointer3DR8 
  module procedure ESMF_FieldGetDataPointer4DR4 
  module procedure ESMF_FieldGetDataPointer4DR8 
  module procedure ESMF_FieldGetDataPointer5DR4 
  module procedure ESMF_FieldGetDataPointer5DR8 
   < end macro - do not edit directly > 
  ------------------------------------------------------------------------------ 
 
 \end{verbatim}
{\sf DESCRIPTION:\\ }


   This interface provides a single entry point for the various
   types of {\tt ESMF\_FieldGetDataPointer} subroutines.
   
%/////////////////////////////////////////////////////////////
 
\mbox{}\hrulefill\ 
 
\subsubsection [ESMF\_FieldGetDataPointer] {ESMF\_FieldGetDataPointer - Retrieve F90 pointer directly from a Field }


  
\bigskip{\sf INTERFACE:}
\begin{verbatim}  ! Private name; call using ESMF_FieldGetDataPointer() 
  subroutine ESMF_FieldGetDataPointer<type>(field, f90ptr, copyflag, rc) 
   \end{verbatim}{\em ARGUMENTS:}
\begin{verbatim}  type(ESMF_Field), intent(in) :: field 
  <type> (ESMF_KIND_<kind>), dimension(<rank>), pointer :: f90ptr 
  type(ESMF_CopyFlag), intent(in), optional :: copyflag 
  integer, intent(out), optional :: rc 
   \end{verbatim}
{\sf DESCRIPTION:\\ }

 
   Retrieves data from a field, returning a direct F90 pointer to the start 
   of the actual data array. 
   
   The arguments are: 
   \begin{description} 
   \item[field] 
   The {\tt ESMF\_Field} to query. 
   \item[f90ptr] 
   An unassociated Fortrn 90 pointer of the proper Type, Kind, and Rank as the data 
   in the Field. When this call returns successfully, the pointer will now reference 
   the data in the Field. This is either a reference or a copy, depending on the 
   setting of the following argument. The default is to return a reference. 
   \item[{[copyflag]}] 
   Defaults to {\tt ESMF\_DATA\_REF}. If set to {\tt ESMF\_DATA\_COPY}, a separate 
   copy of the data will be made and the pointer will point at the copy. 
   \item[{[rc]}] 
   Return code; equals {\tt ESMF\_SUCCESS} if there are no errors. 
   \end{description} 
    
%...............................................................
