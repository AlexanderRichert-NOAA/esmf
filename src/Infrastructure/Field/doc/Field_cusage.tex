% $Id$

%TODO: This file started as an exact copy of the Fortran version of this file.
%TODO: Changes were made to correctly reflect the current status of the C API.
%TODO: Eventually this file should be removed again and replaced by a single
%TODO: generic version that can be included for both Fortran and C refdocs.

%\subsection{Use and Examples}

A Field serves as an annotator of data, since it carries 
a description of the grid it is associated with and metadata 
such as name and units.  Fields can be used in this capacity
alone, as convenient, descriptive containers into which arrays 
can be placed and retrieved.  However, for most codes the primary 
use of Fields is in the context of import and export States,
which are the objects that carry coupling information between 
Components.  Fields enable data to be self-describing, and a
State holding ESMF Fields contains data in a standard format
that can be queried and manipulated.  

The sections below go into more detail about Field usage.

\subsubsection{Field create and destroy}

Fields can be created and destroyed at any time during 
application execution.  However, these Field methods require 
some time to complete.  We do not recommend that the user
create or destroy Fields inside performance-critical 
computational loops.

All versions of the {\tt ESMC\_FieldCreate()} 
routines require a Mesh object as input.
The Mesh contains the information needed to know which 
Decomposition Elements (DEs) are participating in 
the processing of this Field, and which subsets of the data
are local to a particular DE.

The details of how the create process happens depends 
on which of the variants of the {\tt ESMC\_FieldCreate()} 
call is used.

When finished with an {\tt ESMC\_Field}, the {\tt ESMC\_FieldDestroy} method
removes it.  However, the objects inside the {\tt ESMC\_Field}
created externally should be destroyed separately, 
since objects can be added to
more than one {\tt ESMC\_Field}.  For example, the same {\tt ESMF\_Mesh}
can be referenced by multiple {\tt ESMC\_Field}s.  In this case the
internal Mesh is not deleted by the {\tt ESMC\_FieldDestroy} call.
