% $Id$

An ESMF Field represents a physical field, such as temperature.
The motivation for including Fields in ESMF is that bundles of 
Fields are the entities that are normally exchanged when coupling
Components.  

The ESMF Field class contains distributed and discretized field data, a reference 
to its associated grid, and metadata.  The Field class stores the grid {\it staggering}
for that physical field.
This is the relationship of how the data array of a field maps onto a grid 
(e.g. one item per
cell located at the cell center, one item per cell located at the NW
corner,  one item per cell vertex, etc.).  This means that different Fields
which are on the same underlying ESMF Grid but have different
staggerings can share the same Grid object without needing to replicate
it multiple times. 

Fields can be added to States for use in inter-Component
data communications.  Fields can also be added to FieldBundles,
which are groups of Fields on the same underlying Grid.  
One motivation for packing Fields into FieldBundles is convenience; 
another is the ability to perform optimized collective data transfers.  

Field communication capabilities include: data redistribution, regridding, scatter,
gather, sparse-matrix multiplication, and halo update.  These are discussed
in more detail in the documentation for the specific method calls.  
ESMF does not currently support vector fields, so the components of 
a vector field must be stored as separate Field objects.  

\subsubsection{Operations}

The Field class allows the user to easily perform a number of operations on 
the data stored in a Field. This section gives a brief summary of the different types of operations
and the range of their capabilities. The operations covered here are: redistribution ({\tt ESMF\_FieldRedistStore()}), sparse matrix multiply ({\tt ESMF\_FieldSMMStore()}), and regridding ({\tt ESMF\_FieldRegridStore()}).

The redistribution operation ({\tt ESMF\_FieldRedistStore()}) allows the user to move data between two Fields with the same size, but different 
distribution. This operation is useful, for example, to move data between two components with different distributions. 
Please see Section~\ref{sec:field:usage:redist_1dptr} for an example of the redistribution capability.

The sparse matrix multiplication operation ({\tt ESMF\_FieldSMMStore()}) allows the user to multiply the data in a Field by a sparse matrix. This operation is useful, for example, if the user has an interpolation matrix and wants to apply it to the data in a Field. Please see Section~\ref{sec:field:usage:smm_1dptr}
for an example of the sparse matrix multiply capability.

The regridding operation ({\tt ESMF\_FieldRegridStore()}) allows the user to move data from one grid to another while maintaining certain properties
of the data. Regridding is also called interpolation or remapping. In the Field regridding operation the grids the data is being moved between
are the grids associated with the Fields storing the data. The regridding operation works on Fields built on 2D or 3D Meshes and 2D or 3D Grids. 
There are three regridding methods available: bilinear, higher-order patch, and first-order conservative. All three of these methods are 
supported on 2D Grids or 2D Meshes. In 3D the situation is more complicated. Bilinear 
is supported on 3D Grid or 3D Meshes composed of hexahedrons. Higher-order patch is not supported in 3D. First-order conservative 
is supported on 3D Grids or 3D Meshes composed of hexahdrons or tetrahedrons. There are also a range of options for what to do at the poles and what to do 
if a point does not map. Please see Section~\ref{sec:fieldregrid} for a full description of the regridding capability. Several sections following Section~\ref{sec:fieldregrid} 
contain examples of using regridding. 







