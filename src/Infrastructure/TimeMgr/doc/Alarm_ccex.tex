% $Id: Alarm_ccex.tex,v 1.3 2008/06/08 03:33:43 rosalind Exp $

A Alarm is used in conjunction with a clock to ring at certain points in time.
The following shows how to create and initialize two in C++, based on the
example shown in Figure 2.  The first is a one-shot and the second is an
interval alarm.

\begin{verbatim}
// use the Alarm class
#include <ESMC_Alarm.h>

// create two Alarms
ESMC_Alarm Alarm1, Alarm2;

// Initialize one to be a one-shot
ESMC_Time AlarmTime;
AlarmTime.Init("YY:MM:DD", 2002, 8, 30, Gregorian);
Alarm1.Init(NULL, &AlarmTime, NULL, true);

// Initialize other to ring on an interval
ESMCI::TimeInterval AlarmInterval;
AlarmInterval.Init("DD", 1);
Alarm2.Init(&AlarmInterval, NULL, NULL, true);

// Associate alarms with clock
ModelTime.AddAlarm(&Alarm1);
ModelTime.AddAlarm(&Alarm2);

// time step, clock reports active alarms in RingingAlarms list
ESMC_Alarm RingingAlarms[2];
ModelTime.Advance(RingingAlarms, NumActiveAlarms);

for(i=0; i<NumActiveAlarms; i++) {
  // process any active alarms
  if (RingingAlarms[i] == Alarm[i])
  {
    // process Alarm1
    ProcessAlarm(i);

    // after processing alarms, turn off interval alarm to prepare for next
    //   ring time
    Alarm[i].TurnOff();
  }
}
\end{verbatim}
