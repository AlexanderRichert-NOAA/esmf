% $Id: TimeInterval_desc.tex,v 1.4 2003/08/12 14:02:07 cdeluca Exp $

A TimeInterval represents a period between time instants.  
It can be either positive or negative.  If the TimeInterval 
is expressed in days or fractions of a day, it is independent of any 
calendar.  When a TimeInterval is expressed in months or years, 
it is dependent on a calendar and can be thought of as a calendar 
interval.  The ESMF recognizes when a TimeInterval needs to be 
associated with a calendar and when it does not.

There are TimeInterval methods defined for incrementing and
decrementing a TimeInterval by another {\tt TimeInterval},
and for multiplication and division of {\tt TimeIntervals} by integers, 
reals, fractions and other {\tt TimeIntervals}.  Methods are also 
defined to take the absolute value and negative absolute value of a 
{\tt TimeInterval}, and for comparing the length of two
{\tt TimeIntervals}.

The class used to represent time instants in ESMF is {\tt Time},
and this class is frequently used in operations along with 
{\tt TimeIntervals}.  For example, the difference between two
{\tt Times} is a {\tt TimeInterval}.  

{\tt Time Intervals} are used by other parts of the ESMF timekeeping
system, such as {\tt Clocks} (Section~\ref{sec:Clock}) and 
{\tt Alarms} (Section~\ref{sec:Alarm}).    





