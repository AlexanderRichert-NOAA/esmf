% $Id: TimeInterval_desc.tex,v 1.7 2003/08/15 17:37:26 cdeluca Exp $

A TimeInterval represents a period between time instants.  
It can be either positive or negative.  Like the Time interface, 
the TimeInterval interface is designed so that you can choose 
one or more options from a list of time units in order 
to specify a TimeInterval.  See Section~\ref{sec:Time}, 
Table~\ref{table:TimeOpts} for the available options.

There are TimeInterval methods defined for setting and getting 
a TimeInterval, for incrementing and decrementing a TimeInterval 
by another TimeInterval, and for multiplying and dividing 
TimeIntervals by integers, reals, fractions and other TimeIntervals.  
Methods are also defined to take the absolute value and negative 
absolute value of a TimeInterval, and for comparing the length of two
TimeIntervals.

When a TimeInterval is expressed as a combination of hours, 
minutes, seconds or fractions of a second, it is independent of a 
standard calendar.  When it is expressed as a longer time unit,
such as a month, the length of the time period that it represents 
depends on the starting time instant of the interval and the 
standard calendar that is associated with that starting instant.

The class used to represent time instants in ESMF is Time,
and this class is frequently used in operations along with 
TimeIntervals.  For example, the difference between two
Times is a TimeInterval.  

TimeIntervals are used by other parts of the ESMF timekeeping
system, such as Clocks (Section~\ref{sec:Clock}) and Alarms 
(Section~\ref{sec:Alarm}).





