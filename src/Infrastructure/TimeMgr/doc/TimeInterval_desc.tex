% $Id: TimeInterval_desc.tex,v 1.5 2003/08/12 14:14:18 cdeluca Exp $

A TimeInterval represents a period between time instants.  
It can be either positive or negative.  If the TimeInterval 
is expressed in days or fractions of a day, it is independent of any 
calendar.  When a TimeInterval is expressed in months or years, 
it is dependent on a calendar and can be thought of as a calendar 
interval.  The ESMF recognizes when a TimeInterval needs to be 
associated with a calendar and when it does not.

There are TimeInterval methods defined for incrementing and
decrementing a TimeInterval by another TimeInterval,
and for multiplication and division of TimeIntervals by integers, 
reals, fractions and other TimeIntervals.  Methods are also 
defined to take the absolute value and negative absolute value of a 
TimeInterval, and for comparing the length of two
TimeIntervals.

The class used to represent time instants in ESMF is Time,
and this class is frequently used in operations along with 
TimeIntervals.  For example, the difference between two
Times is a TimeInterval.  

Time Intervals are used by other parts of the ESMF timekeeping
system, such as Clocks (Section~\ref{sec:Clock}) and Alarms 
(Section~\ref{sec:Alarm}).    





