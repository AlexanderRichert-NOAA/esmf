% $Id: Time_cdesc.tex,v 1.1 2010/09/25 02:44:27 eschwab Exp $
\label{sec:Time}

A Time represents a specific point in time.

There are Time methods defined for setting and getting a Time.

A Time that is specified in hours does not need to be associated with a 
standard calendar; use ESMC\_CAL\_NOCALENDAR.  A Time whose specification 
includes time units of a year must be associated with a standard calendar. 
The ESMF representation of a calendar, the Calendar class, is described in 
Section~\ref{sec:Calendar}.  The {\tt ESMC\_TimeSet} method is used to 
initialize a Time as well as associate it with a Calendar.  If a Time method 
is invoked in which a Calendar is necessary and one has not been set, the 
ESMF method will return an error condition.

In the ESMF the TimeInterval class is used to represent time periods. 
This class is frequently used in combination with the Time class. 
The Clock class, for example, advances model time by incrementing a 
Time with a TimeInterval. 
