% $Id: time_interval_design.tex,v 1.2 2002/10/07 18:56:37 eschwab Exp $

TimeInterval inherits from the base class Time.  As such, it gains the core
representation of time as well as its associated methods.   TimeInterval
further specializes Time by adding shortcut methods to set and get a
TimeInterval in natural way with appropriate unit combinations, as per the
requirements.  The largest unit of time for a TimeInterval is a day, so a
TimeInterval is independent of any calendar.  This is in contrast with a
TimeInstant, which is calendar-dependent, since its largest units of time
are months and years.  TimeInterval also defines methods for multiplication
and division of TimeIntervals by integers, reals, fractions and other
TimeIntervals.  TimeInterval defines methods for absolute value and negative
absolute value for use with both positive or negative time intervals.  
TimeInterval does not add any new attributes to Time.

Calendar intervals are dependent on a calendar and so represent a specialized
case of a TimeInterval.  A derived class CalendarInterval will be defined to
inherit from TimeInterval and specialize it for use with Calendars.
