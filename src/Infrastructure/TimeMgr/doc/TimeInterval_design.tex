% $Id: TimeInterval_design.tex,v 1.4 2003/08/12 02:47:48 cdeluca Exp $

{\tt TimeInterval} inherits from the base class {\tt BaseTime}.  As such,
it gains the core representation of time as well as its associated methods.
{\tt TimeInterval} further specializes {\tt BaseTime} by adding shortcut
methods to set and get a {\tt TimeInterval} in a natural way with
appropriate unit combinations, as per the requirements.  Usually, the
largest resolution of time for a {\tt TimeInterval} is in days, making it
independent of any calendar.  A {\tt TimeInterval} can also be used as a
{\tt Calendar} interval by associating it with a calendar type.  Then it
becomes calendar-dependent, since its largest resolution of time will be
in months and years.  In order to support calendar intervals,
{\tt TimeInterval} adds a calendar type attribute to {\tt BaseTime}.
{\tt TimeInterval} also defines methods for multiplication and division
of {\tt TimeIntervals} by integers, reals, fractions and other
{\tt TimeIntervals}.  {\tt TimeInterval} defines methods for absolute
value and negative absolute value for use with both positive or
negative time intervals.