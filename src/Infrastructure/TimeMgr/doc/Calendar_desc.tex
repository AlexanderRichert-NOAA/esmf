% $Id: Calendar_desc.tex,v 1.8 2003/08/15 17:55:11 eschwab Exp $

\label{sec:Calendar}
The Calendar class represents the standard calendars used in 
geophysical modeling:  Gregorian, Julian, no-leap, 360-day, and 
no-calendar.  It also supports a user-customized calendar.  Brief 
descriptions are provided for each calendar below.  For more information 
on standard calendars, see ~\cite{seidelman}.

\begin{description}            
\item[Gregorian calendar]
{\it Valid range: 2/29/4800 BC to 10/30/292,277,019,914 } \newline
The Gregorian calendar is the calendar currently in use throughout Western
countries.  Named after Pope Gregory, it is a minor 
correction to the older Julian calendar. In the Gregorian calendar every
fourth year is a leap year in which February has 29 and not 28 days;
however, years divisible by 100 are not leap years unless they are also 
divisible  by 400.  As in the Julian calendar, days begin at midnight.

\item[Julian calendar] 
{\it Valid range: 4713 BC onward} \newline
The Julian calendar was introduced by Julius Caesar in 46 B.C., and 
reached its final form in 8 B.C.  The Julian calendar differs from the 
Gregorian only in the determination of leap years, lacking the correction 
for years divisible
by 100 and 400 in the Gregorian calendar. In the Julian calendar, any positive 
year is a leap year if divisible by 4. (Negative years are leap years if, when 
divided by 4, a remainder of 3 results.) Days are considered to begin at 
midnight.

\item[Julian days]
{\it Valid range: -32045 to 106751991167301} \newline
Julian days simply enumerate the days and fraction of a day which have elapsed 
since the start of the Julian era, defined as beginning at noon on Monday, 
1st January of year 4713 B.C. in the Julian calendar.  Julian days, 
unlike the dates in the Julian and Gregorian calendars, begin at noon.

\item[No-leap calendar]
{\it Valid range: machine limits} \newline
The no-leap calendar is the Gregorian calendar with no leap years - 
February is always assumed to have 28 days.  Modelers sometimes use this 
calendar as a simple, close approximation to the Gregorian calendar.

\item[360-day calendar]
{\it Valid range: machine limits} \newline
In the 360-day calendar, there are 12 months, each of which has 30 days.  
Like the no-leap calendar, this is a simple approximation to the Gregorian
calendar sometimes used by modelers.

\item[Generic calendar]
{\it Valid range: machine limits} \newline 
The user can set calendar parameters in the generic calendar.

\item[No-calendar]
{\it Valid range: machine limits} \newline 
The no-calendar option simply tracks the numer of elapsed timesteps.

\end{description}



