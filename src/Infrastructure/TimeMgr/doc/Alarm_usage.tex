% $Id: Alarm_usage.tex,v 1.3 2003/08/15 19:04:13 cdeluca Exp $


After advancing a time step, a Clock iterates over its internal
Alarm list and returns the number of ringing alarms.  The user can then check
if the number of ringing Alarms is greater than zero.  If so, the user
goes into a loop to query the Clock for each ringing Alarm, one at a time.
Each ringing Alarm can then be processed using Alarm methods for identifying,
turning off, disabling or resetting the Alarm.

The clock will pass a parameter telling the alarm check method whether 
the ringer is
to be set upon crossing the ring time in the positive or negative direction.
This is to handle both positive and negative clock timesteps.  After the
ringer is set for interval alarms, the check method will recalculate the
next ring time.  This can be in the positive or negative direction, again
depending on the parameter passed in by the clock.

Other methods are defined for getting the ringing state, turning the
ringer on/off, enabling/disabling the alarm, and getting/setting the
time attributes defined above.  These will be typically be used to check for
and process ringing alarms after a clock's time step.
