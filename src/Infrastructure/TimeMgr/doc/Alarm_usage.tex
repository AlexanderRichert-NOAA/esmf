% $Id: Alarm_usage.tex,v 1.1 2003/05/07 06:58:56 cdeluca Exp $

The {\tt Alarm} class contains {\tt Time} instants and a {\tt TimeInterval}
to perform one-shot and interval alarming.  A single {\tt TimeInterval}
holds the alarm interval if used.  A {\tt Time} instant is defined for the
ring time, used for either the one-shot alarm time or for the next interval
alarm time.  A {\tt Time} instant is also defined for the previous ring
time to keep track of alarm intervals.  A {\tt Time} instant for stop time
defines when alarm intervals end.  If a one-shot alarm is defined, only
the ring time attribute is used, the others are not.  To keep track of
alarm state, two logical attributes are defined, one for ringing on/off,
and the other for alarm enabled/disabled.  An alarm is enabled by default;
if disabled by the user, it does not function at all.

The primary method is to check whether it is time to set the ringer, which
is called by the associated clock after performing a time step.  The clock
will pass a parameter telling the alarm check method whether the ringer is
to be set upon crossing the ring time in the positive or negative direction.
This is to handle both positive and negative clock timesteps.  After the
ringer is set for interval alarms, the check method will recalculate the
next ring time.  This can be in the positve or negative direction, again
depending on the parameter passed in by the clock.

Other methods are defined for getting the ringing state, turning the
ringer on/off, enabling/disabling the alarm, and getting/setting the
time attributes defined above.
