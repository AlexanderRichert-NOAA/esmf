% $Id: TimeMgr_desc.tex,v 1.1 2003/08/12 02:47:48 cdeluca Exp $

The ESMF Time Manager utility includes software for time and date 
representation and calculations, model time advancement, and the 
identification of unique and periodic events.  Since components
in multi-component applications are often time-synchronized, the 
Time Manager's standard calendars and consistent time representation 
promote component interoperability.  

Key features include:
\begin{itemize}
\item Drift-free timekeeping through purely integer internal time 
representation. 
\item Support for rational fraction time representation.
\item Support for many calendar types, including user-customized calendars.
\item Support for both concurrent and sequential modes of component execution.
\item Support for varying and negative time steps.
\end{itemize}

\subsection{Time Manager Classes}
There are five ESMF classes that represent time concepts:
\begin{itemize}
\item {\bf Calendar}  Standard calendars (such as Gregorian 
and 360-day) and user-specified calendars are supported.  
Calendars may be queried for quantities such as seconds per day, 
days per month, and days per year.  A Calendar can be 
used to keep track of the date as a Gridded Component advances 
in time. 
\item {\bf Time} A Time represents a time instant, such as 
November 28, 1964, at 7:31pm EST.  The Time class can be used 
to represent the start and stop time of a time integration.
\item {\bf TimeInterval} TimeIntervals represent a period 
of time, such as 300 milliseconds.  Time steps can be represented 
using TimeIntervals.  
\item {\bf Clock} Clocks collect the parameters and 
methods used for model time advancement into a convenient 
package.  A Clock can be queried for quantities such
as start time, stop time, current time, and time step.  Clock
methods include incrementing the current time, and determining
if it is time to stop.  
\item {\bf Alarm} Alarms identify unique or periodic events
by ``ringing'' - returning a true value - at specified times.  
For example, an Alarm might be set to ring on the day of the 
year when leaves start falling from the trees in a climate model.
\end{itemize}

In the remainder of this section, we briefly summarizes the 
functionality that the Time Manager classes provide.  Detailed 
descriptions and examples of use precede the API for each class.

\subsection{Calendar}
An ESMF Calendar can be queried for seconds per day, days per month 
and days per year.  This flexible definition allows Calendars to be 
defined for planetary bodies other than Earth.  The set of supported 
calendars includes:
\begin{itemize}
\item {\bf Gregorian} The standard Gregorian calendar.
\item {\bf no-leap} The Gregorian calendar with no leap years.
\item {\bf Julian} The standard Julian calendar.
\item {\bf 360-day} A 30-day-per-month, 12-month-per-year calendar
\item {\bf no calendar} Tracks only timesteps.
\end{itemize}
The Time Manager also allows you to specify your own {\tt Calendar}.
See Section~\ref{sec:Calendar} for more details on supported calendars, 
and how to create a custom ESMF Calendar.

\subsection{Time Instants and Time Intervals}
Since climate modeling, numerical weather prediction and other 
Earth and space applications have widely varying time scales and require 
different sorts of calendars, the Time Manager must provide a wide range 
of time specifiers, spanning nanoseconds to years.  

Time intervals and time instants are the computational building blocks of the
Time Manager library.  Time intervals, which are time periods independent 
of a calendar, support operations such as add, subtract, compare size, 
reset value, copy value, and subdivide by a scalar.  Time instants, which 
are moments in time associated with specific calendars, can be incremented 
or decremented by time intervals, compared to see which of two time instants 
is later, differenced to obtain the interval between two time instants, 
copied, reset, and manipulated in other useful ways.  Time instants support 
a host of different queries, both for values of individual time instant 
components such as year, month, day, and second, and for derived values such 
as day of year, middle of current month and Julian day.  It is also possible 
to retrieve the value of the hardware realtime clock in the form of a time 
instant.  See Sections~\ref{sec:Time} and {sec:TimeInterval}, respectively,
for use and examples of Times and TimeIntervals.

\subsection{Clocks and Alarms}
Although it is possible to repeatedly step a Time forward by a 
TimeInterval using arithmetic on these basic types, it is useful to 
identify a higher-level concept to represent this function.  We refer to 
this capability as a Clock, and include in its required features the 
ability to store reference times such as the start and stop time 
instants of a model run, to check when time advancement should cease, 
and to query the value of quantities such as the previous and current 
time instants.  The Time Manager includes a class with methods that 
return a true value when a periodic or unique event has taken place; 
we refer to these as Alarms.  Applications may require temporary 
or multiple Clocks and Alarms.  










