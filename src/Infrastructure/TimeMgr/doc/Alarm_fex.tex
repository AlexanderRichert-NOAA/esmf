% $Id: Alarm_fex.tex,v 1.2 2003/02/11 18:58:54 eschwab Exp $

A Alarm is used in conjunction with a clock to ring at certain points in time.
The following shows how to create and initialize two in F90, based on the
example shown in Figure 2.  The first is a one-shot and the second is an
interval alarm.

\begin{verbatim}
! use the Alarm Module
use ESMF_AlarmMod

! create two Alarms
type(ESMF_Alarm) :: Alarm1, Alarm2

! Initialize one to be a one-shot
type(ESMF_Time) :: AlarmTime
call ESMF_TimeInit(AlarmTime, YY=2002, MM=8, DD=30, Gregorian)
call ESMF_AlarmInit(Alarm1, RingTime=AlarmTime)

! Initialize other to ring on an interval
type(ESMF_TimeInterval) :: AlarmInterval
call ESMF_TimeIntervalInit(AlarmInterval, DD=1)
call ESMF_AlarmInit(Alarm2, RingInterval=AlarmInterval)

! Associate alarms with clock
call ESMF_ClockAddAlarm(ModelTime, Alarm1)
call ESMF_ClockAddAlarm(ModelTime, Alarm2)

! time step, clock reports active alarms in RingingAlarms list
call ESMF_ClockAdvance(ModelTime, RingingAlarms, NumActiveAlarms)

! process any active alarms
do i=1,NumActiveAlarms
  if (RingingAlarms(i) == Alarm(i)) then
    ! process Alarm(i)
    call ProcessAlarm(i)

   ! after processing alarms, turn off interval alarm to prepare for next
   !   ring time
   ESMF_AlarmTurnOff(Alarm(i))
end do

end if
\end{verbatim}
