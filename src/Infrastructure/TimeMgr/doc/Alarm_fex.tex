% $Id: Alarm_fex.tex,v 1.3 2003/07/25 23:11:44 eschwab Exp $

A Alarm is used in conjunction with a clock to ring at certain points in time.
The following shows how to create and initialize two in F90, based on the
example shown in Figure 2.  The first is a one-shot and the second is an
interval alarm.

\begin{verbatim}
use ESMF_Mod

! create two Alarms
type(ESMF_Alarm) :: alarm1, alarm2

! Initialize one to be a one-shot
type(ESMF_Time) :: alarmTime
call ESMF_TimeSet(alarmTime, yr_i4=2002, mm_i4=8, dd_i4=30, calendar=gregorian)
call ESMF_AlarmSet(alarm1, ringTime=alarmTime)

! Initialize other to ring on an interval
type(ESMF_TimeInterval) :: alarmInterval
call ESMF_TimeIntervalInit(alarmInterval, dd_i4=1)
call ESMF_AlarmInit(alarm2, ringInterval=alarmInterval)

! Associate alarms with clock
call ESMF_ClockAddAlarm(modelTime, alarm1)
call ESMF_ClockAddAlarm(modelTime, alarm2)

! time step, clock reports active alarms in RingingAlarms list
call ESMF_ClockAdvance(modelTime, ringingAlarms, numActiveAlarms)

! process any active alarms
do i=1,numActiveAlarms
  if (ringingAlarms(i) == alarm(i)) then
    ! process alarm(i)
    call ProcessAlarm(i)

   ! after processing alarms, turn off interval alarm to prepare for next
   !   ring time
   call ESMF_AlarmTurnOff(alarm(i))
end do

end if
\end{verbatim}
