% $Id: Clock_options.tex,v 1.1 2005/06/22 22:08:48 eschwab Exp $

\subsubsection{ESMF\_Direction}

{\sf DESCRIPTION:\\}
Specifies the time-stepping direction of a clock.  Use as argument to
methods {\tt ESMF\_ClockSet()} and {\tt ESMF\_ClockGet()}.  Cannot be used
with method {\tt ESMF\_ClockCreate()}, since it only initializes a clock in
the default forward mode; a clock must be advanced (timestepped) at least once
before reversing direction via {\tt ESMF\_ClockSet()}.  This also holds true 
for negative timestep clocks which are initialized (created) with
stopTime < startTime, since "forward" means timestepping from startTime
towards stopTime (see ESMF\_MODE\_FORWARD below).

"Forward" and "reverse" directions are distinct from postive and negative
timesteps.  "Forward" means timestepping in the direction established at
{\tt ESMF\_ClockCreate()}, from startTime towards stopTime, regardless
of the timestep sign.  "Reverse" means timestepping in the opposite direction,
back towards the clock's startTime, regardless of the timestep sign.

Valid values are:
\begin{description}

\item [ESMF\_MODE\_FORWARD] 
      Upon calling {\tt ESMF\_ClockAdvance()}, the clock will timestep from
its startTime toward its stopTime.  This is the default direction.  A user
can use either {\tt ESMF\_ClockIsStopTime()} or {\tt ESMF\_ClockIsDone()}
methods to determine when stopTime is reached.  This forward behavior also
holds for negative timestep clocks which are initialized (created) with
stopTime < startTime.

\item [ESMF\_MODE\_REVERSE] 
      Upon calling {\tt ESMF\_ClockAdvance()}, the clock will timestep backwards
toward its startTime.  Use method {\tt ESMF\_ClockIsDone()} to determine when
startTime is reached.  This reverse behavior also holds for negative timestep
clocks which are initialized (created) with stopTime < startTime.

\end{description}

