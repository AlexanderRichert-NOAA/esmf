% $Id: Clock_options.tex,v 1.9 2011/06/30 15:34:30 rokuingh Exp $

\subsubsection{ESMF\_DIRECTION}
\label{const:direction}

{\sf DESCRIPTION:\\}
\begin{sloppypar}
Specifies the time-stepping direction of a clock.  Use with "direction"
argument to methods {\tt ESMF\_ClockSet()} and {\tt ESMF\_ClockGet()}.
Cannot be used with method {\tt ESMF\_ClockCreate()}, since it only
initializes a clock in the default forward mode; a clock must be advanced
(timestepped) at least once before reversing direction via
{\tt ESMF\_ClockSet()}.  This also holds true for negative timestep clocks
which are initialized (created) with stopTime < startTime, since "forward"
means timestepping from startTime towards stopTime
(see {\tt ESMF\_DIRECTION\_FORWARD} below).
\end{sloppypar}

"Forward" and "reverse" directions are distinct from postive and negative
timesteps.  "Forward" means timestepping in the direction established at
{\tt ESMF\_ClockCreate()}, from startTime towards stopTime, regardless
of the timestep sign.  "Reverse" means timestepping in the opposite direction,
back towards the clock's startTime, regardless of the timestep sign.

Clocks and alarms run in reverse in such a way that the state of a clock and
its alarms after each time step is precisely replicated as it was in forward
time-stepping mode.  All methods which query clock and alarm state will
return the same result for a given timeStep, regardless of the direction of
arrival.

The type of this flag is:

{\tt type(ESMF\_Direction\_Flag)}

The valid values are:
\begin{description}

\item [ESMF\_DIRECTION\_FORWARD] 
      Upon calling {\tt ESMF\_ClockAdvance()}, the clock will timestep from
its startTime toward its stopTime.  This is the default direction.  A user
can use either {\tt ESMF\_ClockIsStopTime()} or {\tt ESMF\_ClockIsDone()}
methods to determine when stopTime is reached.  This forward behavior also
holds for negative timestep clocks which are initialized (created) with
stopTime < startTime.

\item [ESMF\_DIRECTION\_REVERSE] 
      Upon calling {\tt ESMF\_ClockAdvance()}, the clock will timestep backwards
toward its startTime.  Use method {\tt ESMF\_ClockIsDone()} to determine when
startTime is reached.  This reverse behavior also holds for negative timestep
clocks which are initialized (created) with stopTime < startTime.

\end{description}

