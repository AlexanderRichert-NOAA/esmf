% $Id$

\subsection{Design and Implementation Notes}

The Alarm class is designed as a deep, dynamically allocatable class,
based on a pointer type.  This allows for both indirect and direct
manipulation of alarms.  Indirect alarm manipulation is where ESMF\_Alarm API
methods, such as ESMF\_AlarmRingerOff(), are invoked on alarm references
(pointers) returned from ESMF\_Clock queries such as "return ringing alarms."
Since the method is performed on an alarm reference, the actual alarm held
by the clock is affected, not just a user's local copy.  Direct alarm
manipulation is the more common case where alarm API methods are invoked on
the original alarm objects created by the user.

For consistency, the ESMF\_Clock class is also designed as a deep, dynamically
allocatable class. 

An additional benefit from this approach is that Clocks and Alarms can be
created and used from anywhere in a user's code without regard to the scope
in which they were created.  In constrast, statically created Alarms and
Clocks would disappear if created within a user's routine that returns,
whereas dynamically allocated Alarms and Clocks will persist until explicitly
destroyed by the user.
