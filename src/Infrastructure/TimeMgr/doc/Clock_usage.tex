% $Id: Clock_usage.tex,v 1.3 2003/08/15 17:17:53 eschwab Exp $

The following is a typical sequence for using a Clock in a 
geophysical model.

\noindent {\bf At initialization:}
\begin{itemize}
\item Create a Calendar.
\item Set start time, stop time and time step as Times and 
Time Intervals.
\item Create a Clock using the start time, stop time and time
step to initialize it.
\item Define Times and Time Intervals associated with special
events, and use these to create Alarms.
\end{itemize}

\noindent {\bf At run:}
\begin{itemize}
\item If Advance the Clock, checking for ringing alarms as needed.
\item Check if it is time to stop.
\end{itemize}

\noindent {\bf At finalize:}
\begin{itemize}
\item Destroy the Clock 
\item Destroy the Calendar
\end{itemize}

The following code example illustrates Clock usage.




The {\tt Clock} class contains {\tt Time} instants and a {\tt TimeInterval}
to track and time step model time.  For tracking, {\tt Time} instants are
instantiated for the current time, stop time, start time, reference time,
and previous time.  For time stepping, a single {\tt TimeInterval} is
instantiated.  There is also an integer counter for keeping track of the
number of timesteps, and an array of associated alarms.  Methods are
defined for advancing the clock (perform a time step), checking if the
stop time is reached, synchronizing with a real-time clock, and getting
values of the class attributes defined above.

After performing the time step, the advance method will iterate over the
alarm list and return the number of ringing alarms.  The user can then check
if the the number of ringing alarms is greater than zero.  If so, the user
goes into a loop to query the clock for each ringing alarm, one at a time.
Each ringing alarm can then be processed using alarm methods for identifying,
turning off, disabling or resetting the alarm.
