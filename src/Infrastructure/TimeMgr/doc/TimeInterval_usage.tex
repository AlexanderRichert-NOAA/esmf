% $Id: TimeInterval_usage.tex,v 1.2 2003/07/24 15:50:50 cdeluca Exp $

A typical use for a {\tt TimeInterval} in a geophysical model 
based on time-dependent differential equations is to represent 
the time step by which the model is advanced.  Some models change 
the size of their time step as the model run progresses; this could
be done by incrementing or decrementing the original time 
step by another {\tt TimeInterval}, or by dividing or multiplying
the time step by an integer value.  An example of advancing 
model time using a {\tt TimeInterval} representation of a time
step is shown in Section~\ref{sec:ClockEx}.



