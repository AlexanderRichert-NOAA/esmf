% $Id$

A typical use for a TimeInterval in a geophysical model 
is representation of the time step by which the model is 
advanced.  Some models change the size of their time step as 
the model run progresses; this could
be done by incrementing or decrementing the original time 
step by another TimeInterval, or by dividing or multiplying
the time step by an integer value.  An example of advancing 
model time using a TimeInterval representation of a time
step is shown in Section~\ref{sec:Clock}.

The following brief example shows how to create, initialize 
and manipulate {\tt TimeInterval}.


