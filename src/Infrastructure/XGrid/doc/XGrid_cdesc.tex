% $Id$

\label{sec:xgrid:desc}
An exchange grid represents the 2D boundary layer usually between the
atmosphere on one side and ocean and land on the other in an Earth
system model. There are dynamical and thermodynamical processes on
either side of the boundary layer and on the boundary layer itself.
The boundary layer exchanges fluxes from either side and adjusts
boundary conditions for the model components involved. For climate modeling,
it is critical that the fluxes transferred by the boundary layer are
conservative.

The ESMF exchange grid is implemented as the {\tt ESMC\_XGrid} class. 
Internally it's represented by a collection of the intersected cells
between atmosphere and ocean/land\cite{BalajiXGrid} grids. 
These polygonal cells can have irregular shapes
and can be broken down into triangles facilitating a finite element
approach. 

Through the C API there is one way to create an {\tt ESMC\_XGrid} object from
user supplied information. The {\tt ESMC\_XGrid} takes
two lists of {\tt ESMC\_Grid} or {\tt ESMC\_Mesh} that represent the model component grids on
either side of the exchange grid. From the two lists of {\tt ESMC\_Grid} or {\tt ESMC\_Mesh},
information required for flux exchange calculation between any pair of the 
model components from either side of the exchange grid is computed. In addition, the
internal representation of the {\tt ESMC\_XGrid} is computed and can be optionally stored
as an {\tt ESMC\_Mesh}. This internal representation is the collection of the intersected
polygonal cells as a result of merged {\tt ESMC\_Mesh}es from both sides of the exchange grid.
{\tt ESMC\_Field} can be created on the {\tt ESMC\_XGrid} and used for weight generation
and regridding as the internal representation in the {\tt ESMC\_XGrid} has
a complete geometrical description of the exchange grid.

Once an {\tt ESMC\_XGrid} has been created, information describing it (e.g. cell areas, mesh representation, etc.)
can be retrieved from the object using a set of get calls (e.g. {\tt ESMC\_XGridGetElementArea()}). The full extent of these
can be found in the API section below.  
