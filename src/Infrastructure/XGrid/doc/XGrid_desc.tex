% $Id: XGrid_desc.tex,v 1.1 2010/07/16 14:15:08 feiliu Exp $

An exchange grid represents the 2D boundary layer usually between the
atmosphere on one side and ocean and land on the other in an Earth
system model. There are dynamical and thermodynamical processes on
either side of the boundary layer and on the boundary layer itself.
The boundary layer exchanges fluxes from either side and adjusts
boundary condition for model components involved. For climate modeling,
it's critical that the fluxes transferred by the boundary layer are
conservative.

The exchange grid is implemented as a collection of the intersected cells
between atmosphere and ocean/land. These cells can have irregular shapes
and can be broken down into triangles facilitating a finite element
approach. In practice, there is a threshold of minimum cell area below
which intersections are discarded.

(see V. Balaji, Jeff Anderson, Isaac Held, Michael Winton, Jeff Durachta,
Sergey Malyshev and Ronald J. Stouffer, 2006:  The Exchange Grid: a
mechanism for data exchange between Earth System components on independent
grids, Parallel Computational Fluid Dynamics: Theory and Applications,
Proceedings of the 2005 International Conference on Parallel Computational
Fluid Dynamics, Elsevier (2006).) 
