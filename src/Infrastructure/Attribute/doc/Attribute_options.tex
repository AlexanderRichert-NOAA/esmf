% $Id: Attribute_options.tex,v 1.1 2011/07/01 20:34:00 rokuingh Exp $

\subsubsection{ESMF\_ATTGETCOUNT}
\label{const:attgetcount}

{\sf DESCRIPTION:\\}
Indicates whether or not to descend the Attribute hierarchy for the present operation.

The type of this flag is:

{\tt type(ESMF\_AttGetCountFlag)}

The valid values are:
\begin{description}
	\item[ESMF\_ATTGETCOUNT\_ATTRIBUTE]
	This option will allow the routine to return the number of single Attributes.
	\item[ESMF\_ATTGETCOUNT\_ATTPACK]
	This option will allow the routine to return the number of Attribute packages.
	\item[ESMF\_ATTGETCOUNT\_ATTLINK]
	This option will allow the routine to return the number of Attribute links.
	\item[ESMF\_ATTGETCOUNT\_TOTAL]
	This option will allow the routine to return the total number of Attributes.
\end{description}

\subsubsection{ESMF\_ATTTREE}
\label{const:atttree}

{\sf DESCRIPTION:\\}
Indicates whether or not to descend the Attribute hierarchy for the present operation.

The type of this flag is:

{\tt type(ESMF\_AttTreeFlag)}

The valid values are:
\begin{description}
	\item[ESMF\_ATTTREE\_OFF]
	This option will allow the routine to only descend the first base level of the Attribute hierarchy.
	\item[ESMF\_ATTTREE\_ON]
	This option will allow the routine to descend the entire Attribute hierarchy.
\end{description}

\subsubsection{ESMF\_ATTWRITE}
\label{const:attwrite}

{\sf DESCRIPTION:\\}
Indicates which file format to use in the write operation.

The type of this flag is:

{\tt type(ESMF\_AttWriteFlag)}

The valid values are:
\begin{description}
	\item[ESMF\_ATTWRITE\_XML]
	This option will allow the routine to write in xml format.
	\item[ESMF\_ATTWRITE\_TAB]
	This option will allow the routine to write in tab-delimited format.
\end{description}

