% $Id: Attribute_desc.tex,v 1.19 2009/02/10 18:52:24 rokuingh Exp $
%
% Earth System Modeling Framework
% Copyright 2002-2009, University Corporation for Atmospheric Research,
% Massachusetts Institute of Technology, Geophysical Fluid Dynamics
% Laboratory, University of Michigan, National Centers for Environmental
% Prediction, Los Alamos National Laboratory, Argonne National Laboratory,
% NASA Goddard Space Flight Center.
% Licensed under the University of Illinois-NCSA License.

The {\tt Attribute} class is used to hold the metadata for other ESMF objects.  This class can be used to build {\tt Attribute} hierarchies which connect the {\tt Attribute}s of different ESMF classes.  The class is also capable of allowing the representation of standard {\tt Attribute} packages for a more unified description of an object.  All {\tt Attribute}s are consistent across the entire virtual machine of the object to which they are attached.  This class is only partially implemented in this release.

\subsubsection{{\tt Attribute} Representation in ESMF}

{\tt Attribute}s are meant to be used as a tool for the user to help internally document their project.   Several ESMF objects are allowed to have {\tt Attribute}s associated with them, these objects are the following:

\begin{itemize}
\item Array
\item CplComp
\item GridComp
\item FieldBundle
\item Field
\item Grid
\item State
\end{itemize}

Each {\tt Attribute} contains a name value pair in which the value can be any of several numeric, character, and logical types.  See Figure \ref{fig:AttributeClassUML} for the available {\tt Attribute} value types.  All {\tt Attribute}s also contain character strings specifying the convention, type, and object of the {\tt Attribute} for identification purposes.  All {\tt Attribute}s contain a list of pointers to other {\tt Attribute}s, which is empty until specified otherwise.  This list of {\tt Attribute} pointers is what allows {\tt Attribute} hierarchies to be built.  Using this feature an {\tt Attribute} for one ESMF class can be linked to an {\tt Attribute} in another ESMF class.

\subsubsection{{\tt Attribute} Hierarchies}

Of the ESMF objects with {\tt Attribute}s, only some can link their {\tt Attribute}s together in an {\tt Attribute} hierarchy.  These objects are:

\begin{itemize}
\item CplComp
\item GridComp
\item Field
\item FieldBundle
\item State
\end{itemize}

The most common use for this capability is for linking the {\tt Attribute}s of a Field to the FieldBundle which holds it, which is then linked to the State that is used to transport all of the data for a Component.  All of these links, with the exception of the link between the Component and the State, are automatically handled by ESMF.  

{\tt Attribute} hierarchies are linked in a "shallow" manner, meaning that the {\tt Attribute}s belonging to an external object are not copied, they are merely referenced by a pointer.  This allows for the certainty that the {\tt Attribute} hierarchy has a one-to-one correspondence with the object hierarchy.  

\subsubsection{{\tt Attribute} Packages}

At this time, only select ESMF objects are enabled to contain {\tt Attribute} packages.  Every {\tt Attribute} package is specified by a {\bf convention} and a {\bf purpose}, hereafter called {\bf specifiers}, such as "CF" (see below) and "general".  These specifiers are used to enforce compliance with several existing metadata standards.  One can use an ESMF supplied standard {\tt Attribute} package, specify their own {\tt Attribute} packages, or add customized {\tt Attribute}s to the ESMF supplied {\tt Attribute} packages.  Working with {\tt Attribute} packages is quite involved, at present, but significant effort in automation of these metadata capabilities in ESMF is anticipated in future releases. 

The standard {\tt Attribute} packages supplied by ESMF exist for the following ESMF objects:

\begin{itemize}
\item CplComp
\item GridComp
\item Field
\end{itemize}

The ESMF standard {\tt Attribute} packages are based on the Climate and Forecast (CF) convention.  When additional {\tt Attribute}s beyond the ESMF supplied {\tt Attribute} packages are desired, these can be generated with the {\tt Attribute} package nesting capabilities.  User supplied {\tt Attribute} packages will also be an option in future releases.  An example of some more standardized user supplied {\tt Attribute} packages are in Tables \ref{ComponentAttributePackages} - \ref{FieldAttributePackages}, which summarize the {\tt Attribute} packages available at this point in time.

\vspace{18pt}

\begin{tabular}{|p{4cm}|p{4cm}|p{6cm}|}
\hline
\multicolumn{3}{|c|}{{\bf \large Component {\tt Attribute} Packages}} \\
\hline\hline
{\bf Convention} & {\bf Purpose} & {\bf Name} \\
\hline\hline
{\tt ESG} & {\tt general} & {\bf Component CF general +} \\
     & & \\
     & & {\tt agency} \\
     & & {\tt author} \\
     & & {\tt coding\_language}  \\
     & & {\tt discipline}  \\
     & & {\tt full\_name} \\
     & & {\tt model\_component\_framework} \\
     & & {\tt name} \\
     & & {\tt physical\_domain}\\
     & & {\tt version} \\ 
\hline
{\tt CF} & {\tt general} & {\tt comment}\\
     & & {\tt references} \\
\hline
\end{tabular}
\label{ComponentAttributePackages}

\vspace{18pt}

\begin{tabular}{|p{4cm}|p{4cm}|p{6cm}|}
\hline
\multicolumn{3}{|c|}{{\bf \large State {\tt Attribute} Packages}} \\
\hline\hline
{\bf Convention} & {\bf Purpose} & {\bf Name} \\
\hline\hline
{\tt ESMF} & {\tt general} & {\tt export}  \\
 & & {\tt import} \\ 
\hline
\end{tabular}
\label{StateAttributePackages}

\vspace{18pt}

\begin{tabular}{|p{4cm}|p{4cm}|p{6cm}|}
\hline
\multicolumn{3}{|c|}{{\bf \large Grid {\tt Attribute} Packages}} \\
\hline\hline
{\bf Convention} & {\bf Purpose} & {\bf Name} \\
\hline\hline
{\tt GridSpec} & {\tt general} & {\tt area}  \\
 & & {\tt dimension\_order} \\ 
 & & {\tt discretization\_type} \\ 
 & & {\tt geometry\_type} \\ 
 & & {\tt grid\_type} \\ 
 & & {\tt horizontal\_dimension\_value} \\ 
 & & {\tt IsConformal} \\ 
 & & {\tt IsRegular} \\ 
 & & {\tt IsUniform} \\ 
 & & {\tt north\_pole\_location} \\ 
 & & {\tt number\_of\_cells} \\ 
 & & {\tt number\_of\_dimensions} \\ 
 & & {\tt pole\_covered} \\ 
 & & {\tt vertical\_dimension\_value} \\ 
\hline
\end{tabular}
\label{GridAttributePackages}

\vspace{18pt}

\begin{tabular}{|p{4cm}|p{4cm}|p{6cm}|}
\hline
\multicolumn{3}{|c|}{{\bf \large Field {\tt Attribute} Packages}} \\
\hline\hline
{\bf Convention} & {\bf Purpose} & {\bf Name} \\
\hline\hline
{\tt ESG} & {\tt general} & {\bf Field CF extended +} \\
& & \\
& & {\tt export}  \\
 & & {\tt import} \\ 
\hline
{\tt CF} & {\tt extended} & {\bf Field CF general +} \\
& & \\
& & {\tt standard\_name}\\ 
\hline
{\tt CF} & {\tt general} & {\tt long\_name}\\
     & & {\tt name} \\
     & & {\tt units}  \\
\hline
\end{tabular}
\label{FieldAttributePackages}

\vspace{18pt}

ESMF also allows nesting of {\tt Attribute} packages.  This capability is meant to be used to organize different metadata compliance levels, such as CF and ESG.  The nesting of {\tt Attribute} packages is also very helpful when adding customized {\tt Attribute}s to a package.  The main use of the nesting capabilities of {\tt Attribute} packages is geared towards organizing different metadata compliance levels.  For instance, the CF metadata standard for Fields requires that there be {\tt Attribute}s to track the {\it name}, {\it long\_name}, {\it standard\_name}, and {\it units} of the Field.  The ESG standard, on the other hand, requires two additional {\tt Attribute}s called {\it import} and {\it export}.  In this case the ESMF representation of the ESG compliant {\tt Attribute} package for a Field would involve a nested {\tt Attribute} package structure.  This would involve the ESG-specific {\tt Attribute} package, containing the {\tt Attribute}s {\it import} and {\it export} containing a nested version of the CF-specific {\tt Attribute} package, with the {\tt Attribute}s {\it name}, {\it long\_name}, {\it standard\_name}, and {\it units}.  An {\tt Attribute} package can be nested by passing in a value of {\bf ESMF\_ATTPACKNEST\_ON} for the {\it attpacknestflag} argument of the {\tt ESMF\_AttributeAdd()} interface call.  The default value of the optional {\it attpacknestflag} argument is {\bf ESMF\_ATTPACKNEST\_OFF}.

The nesting capabilities of {\tt Attribute} packages are also very useful for organizing the customized metadata supplied by a user.  For example, if a user was not satisfied with the metadata support required in the ESG convention for Field they could supply a list of {\tt Attribute}s they would like to support.  This new {\tt Attribute} package would then be used as an additional layer, inside which the {\tt Attribute} package of ESG would be nested, inside which the CF {\tt Attribute} package would be nested.  This capability of nesting {\tt Attribute} packages allows for a multitude of metadata organization strategies, and will also aid in the enforcement of metadata compliance when it is implemented in a future ESMF release.



