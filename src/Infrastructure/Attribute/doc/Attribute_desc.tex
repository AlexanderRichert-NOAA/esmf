% $Id: Attribute_desc.tex,v 1.32 2010/07/30 06:00:00 eschwab Exp $
%
% Earth System Modeling Framework
% Copyright 2002-2010, University Corporation for Atmospheric Research,
% Massachusetts Institute of Technology, Geophysical Fluid Dynamics
% Laboratory, University of Michigan, National Centers for Environmental
% Prediction, Los Alamos National Laboratory, Argonne National Laboratory,
% NASA Goddard Space Flight Center.
% Licensed under the University of Illinois-NCSA License.

The Attribute class is used to hold the metadata for other ESMF objects.  This class can be used to build Attribute hierarchies which connect the Attributes of different ESMF classes.  The class is also capable of allowing the representation of standard Attribute packages for a more unified description of an object.  All Attributes are consistent across the entire virtual machine of the object to which they are attached.  This class is only partially implemented in this release.

\subsubsection{Attribute Representation in ESMF}

Attributes are meant to be used as a tool for the user to help internally document their project.   Several ESMF objects are allowed to have Attributes associated with them, these objects are the following:

\begin{itemize}
\item Array
\item ArrayBundle
\item CplComp
\item GridComp
\item DistGrid
\item FieldBundle
\item Field
\item Grid
\item State
\end{itemize}

Each Attribute contains a name-value pair in which the value can be any of several numeric, character, and logical types.  See Figure \ref{fig:AttributeClassUML} for the available Attribute value types.  All Attributes also contain character strings specifying the convention, purpose, and object type of the Attribute for identification purposes - each of which is initialized as an empty string until specified otherwise.  Each Attribute can be uniquely identified by its name, convention, purpose within any one ESMF object.  

All Attributes contain three vectors of pointers to other Attributes, which are empty until specified otherwise.  These vectors of Attribute pointers hold the Attributes, Attribute packages, and Attribute links.  This feature is what allows the Attribute class to self assemble complex structures for representing and organizing the metadata of an ESMF object hierarchy.

\subsubsection{Attribute Hierarchies}

Of the ESMF objects with Attributes, only some can link their Attributes together in an Attribute hierarchy.  These objects are:

\begin{itemize}
\item CplComp
\item GridComp
\item State
\item Field
\item FieldBundle
\item Array
\item ArrayBundle
\end{itemize}

The most common use for this capability is for linking the Attributes of a Field to the FieldBundle which holds it, which is then linked to the State that is used to transport all of the data for a Component.  All of these links, with the exception of the link between the Component and the State, are automatically handled by ESMF.  In addition, the State will automatically set the {\it import} and {\it export} boolean valued Attributes that are part of the ESMF supplied standard Attribute package for Field when that Field is added to the State. 

Attribute hierarchies are linked in a ``shallow" manner, meaning that the Attributes belonging to an external object are not copied, they are merely referenced by a pointer.  This is important to ensure that the Attribute hierarchy has a one-to-one correspondence with the object hierarchy.  

\subsubsection{Attribute Packages}
 \label{desc:AttPacks}

At this time, all ESMF objects which are enabled to contain Attributes can also contain Attribute packages.  Every Attribute package is specified by a {\bf convention} and a {\bf purpose}, hereafter called {\bf specifiers}, such as ``CF" (see below) and ``General".  These specifiers are used to validate ESMF Attribute packages against existing metadata conventions.  One can use an ESMF supplied standard Attribute package, specify their own Attribute packages, or add customized Attributes to the ESMF supplied Attribute packages.  Currently, working with Attribute packages is quite involved, but future development with IO will allow for a more automated approach to populating Attribute packages from a file.

The standard Attribute packages supplied by ESMF exist for the following ESMF objects:

\begin{itemize}
\item CplComp
\item GridComp
\item State
\item Field
\item Grid
\item Array
\end{itemize}

The standard Attribute packages supplied by ESMF are based on 

\begin{itemize}
\item a blend of the Earth System Grid (ESG) and Climate and Forecast (CF) conventions, and
\item a blend of the Metafor Common Information Model (CIM) and Climate and Forecast (CF) conventions.
\end{itemize}

When additional Attributes beyond the ESMF supplied Attribute packages are desired, these can be generated with the Attribute package nesting capabilities.  User supplied Attribute packages will also be an option in future releases.  The ESMF supplied Attribute packages are shown in Tables \ref{ComponentAttributePackages} - \ref{CIMFieldAttributePackages}, which summarize the Attribute packages available at this point in time.

\vspace{18pt}

{\bf \large ESG/CF Attribute Packages}

These Attribute packages are designed to hold ESG and CF conforming Attributes on ESMF objects. The Attributes can be written to an XML file, or in the case of the Field package, also to a tab-delimited file.  The file layout for both XML and tab-delimited is prescribed by ESMF.  Tab-delimited format is not currently supported for packages other than Field.

\vspace{8pt}

\label{ComponentAttributePackages}
\begin{tabular}{|p{4cm}|p{4cm}|p{6cm}|}
\hline
\multicolumn{3}{|c|}{{\bf \large ESG/CF Component Attribute Packages}} \\
\hline\hline
{\bf Convention} & {\bf Purpose} & {\bf Name} \\
\hline\hline
{\tt ESMF} & {\tt General} & {\bf Component ESG General +} \\
{\tt ESG} & {\tt General} & {\bf Component CF General +} \\
     & & \\
     & & {\tt Agency} \\
     & & {\tt Author} \\
     & & {\tt CodingLanguage}  \\
     & & {\tt Discipline}  \\
     & & {\tt FullName} \\
     & & {\tt Institution} \\
     & & {\tt ModelComponentFramework} \\
     & & {\tt Name} \\
     & & {\tt PhysicalDomain}\\
     & & {\tt Version} \\ 
\hline
{\tt CF} & {\tt General} & {\tt Comment}\\
     & & {\tt References} \\
\hline
\end{tabular}

\vspace{18pt}

\label{StateAttributePackages}
\begin{tabular}{|p{4cm}|p{4cm}|p{6cm}|}
\hline
\multicolumn{3}{|c|}{{\bf \large ESG/CF State Attribute Packages}} \\
\hline\hline
{\bf Convention} & {\bf Purpose} & {\bf Name} \\
\hline\hline
{\tt ESMF} & {\tt General} & {\tt Export}  \\
 & & {\tt Import} \\ 
\hline
\end{tabular}

\vspace{18pt}

\label{FieldAttributePackages}
\begin{tabular}{|p{4cm}|p{4cm}|p{6cm}|}
\hline
\multicolumn{3}{|c|}{{\bf \large ESG/CF Field Attribute Packages}} \\
\hline\hline
{\bf Convention} & {\bf Purpose} & {\bf Name} \\
\hline\hline
{\tt ESMF} & {\tt General} & {\bf Field ESG General +} \\
{\tt ESG} & {\tt General} & {\bf Field CF Extended +} \\
& & \\
& & {\tt Export}  \\
 & & {\tt Import} \\ 
\hline
{\tt CF} & {\tt Extended} & {\bf Field CF General +} \\
& & \\
& & {\tt StandardName}\\ 
\hline
{\tt CF} & {\tt General} & {\tt LongName}\\
     & & {\tt Name} \\
     & & {\tt Units}  \\
\hline
\end{tabular}

\label{ArrayAttributePackages}
\begin{tabular}{|p{4cm}|p{4cm}|p{6cm}|}
\hline
\multicolumn{3}{|c|}{{\bf \large ESG/CF Array Attribute Packages}} \\
\hline\hline
{\bf Convention} & {\bf Purpose} & {\bf Name} \\
\hline\hline
{\tt ESMF} & {\tt General} & {\bf Array ESG General +} \\
{\tt ESG} & {\tt General} & {\bf Array CF Extended +} \\
& & \\
& & {\tt Export}  \\
 & & {\tt Import} \\ 
\hline
{\tt CF} & {\tt Extended} & {\bf Array CF General +} \\
& & \\
& & {\tt StandardName}\\ 
\hline
{\tt CF} & {\tt General} & {\tt LongName}\\
     & & {\tt Name} \\
     & & {\tt Units}  \\
\hline
\end{tabular}

\label{GridAttributePackages}
\begin{tabular}{|p{4cm}|p{4cm}|p{6cm}|}
\hline
\multicolumn{3}{|c|}{{\bf \large ESG/CF Grid Attribute Packages}} \\
\hline\hline
{\bf Convention} & {\bf Purpose} & {\bf Name} \\
\hline\hline
{\tt ESMF} & {\tt General} & {\bf Grid GridSpec General +} \\
{\tt GridSpec} & {\tt General} & {\tt CongruentTiles} \\
 & & {\tt GridType} \\ 
 & & {\tt DimOrder} \\ 
 & & {\tt DiscretizationType} \\ 
 & & {\tt GeometryType} \\ 
 & & {\tt IsConformal} \\ 
 & & {\tt IsPoleCovered} \\ 
 & & {\tt IsRegular} \\ 
 & & {\tt IsUniform} \\ 
 & & {\tt NorthPoleLocation} \\ 
 & & {\tt NumberOfCells} \\ 
 & & {\tt NumDims} \\ 
 & & {\tt NX} \\ 
 & & {\tt NY} \\ 
 & & {\tt NZ} \\ 
 & & {\tt Resolution} \\ 
\hline
\end{tabular}

\vspace{18pt}

{\bf \large CIM/CF Attribute Packages}

These Attribute packages are designed to hold CIM Attributes on ESMF Component and Field objects. The Attributes can be written to an XML file whose layout is specified by the CIM convention, as promulgated by Metafor, http://metaforclimate.eu.  Writing CIM Attributes to a tab-delimited file is currently not supported.

The CIM XML output file is divided into major sections called <CIMRecord>s.  The Attributes in the CIM Component package below are output to the <modelComponent>, <simulationRun>, and <platform> <CIMRecord>s.  The <modelComponet> <CIMRecord> can contain a tree of <childComponents>, reflecting a hierarchical ESMF-based model.  Each <childComponent> is derived from its own instance of a CIM Component Attribute package, attached to a corresponding ESMF Component object.

The Attributes in the CIM Field package below, attached to ESMF Field objects, appear in the <modelComponent> <CIMRecord> of the output XML file, within a subsection called <composition>.  In contrast to the hierarchical <childComponent>s, the Field Attributes appearing in <composition> are flat in structure, representing a single listing of Fields from across all the Components in an ESMF Component tree.

See the example "Advanced Attribute usage: CIM Attribute Packages."

\vspace{8pt}

\label{CIMComponentAttributePackages}
\begin{tabular}{|p{4cm}|p{4cm}|p{8cm}|}
\hline
\multicolumn{3}{|c|}{{\bf \large CIM Component Attribute Packages}} \\
\hline\hline
{\bf Convention} & {\bf Purpose} & {\bf Name} \\
\hline\hline
{\tt CIM 1.0} & {\tt Model Component Simulation Description} & {\bf Component CIM 1.0 Platform Description, \newline\newline \bf Component ISO 19115 Responsible Party Description +} \\
     & & \\
     & & {\tt ComponentShortName} \\
     & & {\tt ComponentLongName} \\
     & & {\tt ComponentDescription}  \\
     & & {\tt Version}  \\
     & & {\tt YearReleased} \\
     & & {\tt ModelType} \\
     & & {\tt SimulationShortName}\\
     & & {\tt SimulationLongName} \\ 
     & & {\tt SimulationRationale} \\
     & & {\tt SimulationStartDate} \\ 
     & & {\tt SimulationDuration} \\ 
     & & {\tt PreviousVersion} \\ 
     & & {\tt PreviousVersionDescription} \\ 
\hline
{\tt CIM 1.0} & {\tt Platform Description} & {\bf Child of Component CIM 1.0 Model Component Simulation Description}\\
     & & {\tt MachineName} \\
     & & {\tt MachineDescription} \\
     & & {\tt MachineHardwareType} \\
     & & {\tt MachineOperatingSystem} \\
     & & {\tt MachineVendor} \\
     & & {\tt MachineInterconnectType} \\
     & & {\tt MachineMaxProcessors} \\
     & & {\tt MachineCoresPerProcessor} \\
     & & {\tt MachineProcessor} \\
     & & {\tt MachineCompiler} \\
     & & {\tt MachineCompilerVersion} \\
\hline
{\tt ISO 19115} & {\tt Responsible Party Description} & {\bf Child of Component CIM 1.0 Model Component Simulation Description}\\
     & & {\tt IndividualName} \\
     & & {\tt IndividualPhysicalAddress} \\
     & & {\tt IndividualEmailAddress} \\
     & & {\tt IndividualURL} \\
     & & {\tt ResponsiblePartyRole} \\
\hline
\end{tabular}

\vspace{18pt}

\label{CIMFieldAttributePackages}
\begin{tabular}{|p{4cm}|p{4cm}|p{6cm}|}
\hline
\multicolumn{3}{|c|}{{\bf \large CIM/CF Field Attribute Packages}} \\
\hline\hline
{\bf Convention} & {\bf Purpose} & {\bf Name} \\
\hline\hline
{\tt CIM 1.0} & {\tt Model Component Simulation Description} & {\bf Field CF Extended +} \\
& & \\
& & {\tt InputType}  \\
 & & {\tt InputSourceComponent} \\ 
 & & {\tt InputTargetComponent} \\ 
 & & {\tt InputTechnique} \\ 
 & & {\tt InputSpatialRegriddingMethod} \\ 
 & & {\tt InputSpatialRegriddingType} \\ 
 & & {\tt InputFrequency} \\ 
 & & {\tt InputTimeTransformationType} \\ 
\hline
{\tt CF} & {\tt Extended} & {\bf Field CF General +} \\
& & \\
& & {\tt StandardName}\\ 
\hline
{\tt CF} & {\tt General} & {\tt LongName}\\
     & & {\tt Name} \\
     & & {\tt Units}  \\
\hline
\end{tabular}

\vspace{18pt}

ESMF also allows nesting of Attribute packages.  This capability is intended to help organize different metadata compliance levels, such as CF and ESG.  The nesting of Attribute packages is also very helpful when adding customized Attributes to a package.  The main use of the nesting capabilities of Attribute packages is geared towards organizing different metadata compliance levels.  For instance, the CF metadata standard for Fields requires that there be Attributes to track the {\it name}, {\it long\_name}, {\it standard\_name}, and {\it units} of the Field.  The ESG standard, on the other hand, requires two additional Attributes called {\it import} and {\it export}.  In this case the ESMF representation of the ESG compliant Attribute package for a Field would involve a nested Attribute package structure.  This would involve the ESG-specific Attribute package, containing the Attributes {\it import} and {\it export} containing a nested version of the CF-specific Attribute package, with the Attributes {\it name}, {\it long\_name}, {\it standard\_name}, and {\it units}.  An Attribute package can be nested by including the specifiers of both packages in the {\tt ESMF\_AttributeAdd()} interface call.  

The nesting capabilities of Attribute packages are also very useful for organizing the customized metadata supplied by a user.  For example, if a user was not satisfied with the metadata support required in the ESG convention for Field they could supply a list of Attributes they would like to support.  This new Attribute package would then be used as an additional layer, inside which the Attribute package of ESG would be nested, inside which the CF Attribute package would be nested.  One important thing to remember when working with nested Attribute packages is that naming two Attributes the same in the same nested structure can yield undefined behavior.

An explanation of the specifiers is in order at this point.  The purpose 
specifier is really just meant as an additional means, beyond the use of 
"convention", to specify Attribute packages.  One could imagine that the 
CF convention would want to be able to have Attribute packages divided 
up in some fashion, which ESMF could then keep track of with the purpose 
specifier.  It was added with the intention of allowing Attributes, and 
packages, maximum flexibility.  Take the Field's ESMF standard Attribute 
package for example.  This 
package is made up of three nested Attribute packages.  The lowest one 
is made up of three Attributes with convention=CF and purpose=General.  
The next level contains one Attribute with convention=CF but 
purpose=Extended.  On top of this is the convention=ESG package, also 
with purpose=General.
