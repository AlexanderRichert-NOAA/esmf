% $Id: Attribute_implnotes.tex,v 1.11 2009/02/10 18:52:24 rokuingh Exp $
%
% Earth System Modeling Framework
% Copyright 2002-2009, University Corporation for Atmospheric Research,
% Massachusetts Institute of Technology, Geophysical Fluid Dynamics
% Laboratory, University of Michigan, National Centers for Environmental
% Prediction, Los Alamos National Laboratory, Argonne National Laboratory,
% NASA Goddard Space Flight Center.
% Licensed under the University of Illinois-NCSA License.

This section covers {\tt Attribute} memory deallocation, the use of {\tt ESMF\_AttributeGet()}, {\tt Attribute} package nesting capabilities, and issues with {\tt Attribute}s in a distributed environment.  Issues and procedures dealing with {\tt Attribute} memory deallocation using {\tt ESMF\_AttributeGet()} to retrieve {\tt Attribute} lists, and nested {\tt Attribute} package capabilities are discussed to help avoid misuse.  The limitations with {\tt Attribute}s in a distributed environment are also discussed, with an outline of the future work to be done in this area. 

\subsubsection{{\tt Attribute} Memory Deallocation}

The {\tt ESMF\_Attribute} class presents a somewhat different paradigm with respect to memory deallocation than other ESMF objects.  The {\tt ESMF\_AttributeRemove()} call can be issued to remove any {\tt Attribute} from an ESMF object or an {\tt Attribute} package on an ESMF object.  This call is also enabled to remove entire {\tt Attribute} packages with one call, which would remove any nested {\tt Attribute} packages as well.  The user is {\bf not} required to remove all {\tt Attribute}s that are used in a model run.  The entire {\tt Attribute} hierarchy will be removed automatically by ESMF, provided the ESMF objects which contain them are properly destroyed.  

\subsubsection{Using {\tt ESMF\_AttributeGet()} to retrieve {\tt Attribute} lists}

The behavior of the {\tt ESMF\_AttributeGet()} routine,when retrieving an {\tt Attribute} containing a value list, follows a slightly different convention than other similar ESMF routines.  This routine requires the input of a Fortran array as a place to store the retrieved values of the {\tt Attribute} list.  If the array that is given is longer that the list of {\tt Attribute} values, the first part of the array will be filled, leaving the extra space untouched.  If, however the array passed in, is shorter than the number of {\tt Attribute} values, the routine will exit with an {\bf ESMF\_FAILURE} return code.  It is suggested that if it is required by the user to use a Fortran array that is longer than the number of {\tt Attribute} values returned, only the indices of the array which the user desires to be filleds should be passed into the routine.  
  
\subsubsection{Using {\tt Attribute} package nesting capabilites}

There is a recommended practice when using nested {\tt Attribute} packages to organize metadata conventions.  The most general {\tt Attribute} packages should always be added first, followed by the more specific ones.  For instance, when adding {\tt Attribute} packages to a Field, it is recommend that the CF convention be added first, followed by the ESG convention, followed by any additional customized {\tt Attribute} packages.  In a future release the standardized {\tt Attribute} packages will be added automatically by ESMF, but until that time this recommendation stands.

Another consideration when using nested {\tt Attribute} packages is to remember to specify that the nesting capability is desired by setting the {\it attpacknestflag} argument to {\bf ESMF\_ATTPACKNEST\_ON}.  In addition, when a nested {\tt Attribute} package is removed, every nested {\tt Attribute} package below the point of removal will also be removed.  Thus, by removing the CF {\tt Attribute} package on a Field, the ESG and any other customized {\tt Attribute} packages will also be removed.

\subsubsection{{\tt Attribute}s in a Distributed Environment}
\label{sec:Att:Dist}

The  {\tt ESMF\_Attribute} class is slightly different than other ESMF objects in the context of building a consistent view of the metadata across the VM.  To better explain the ESMF capabilities for ensuring the integrity of {\tt Attribute}s in a distributed environment, a small working vocabulary of ESMF {\tt Attribute}s will be presented.  Three types of changes to an {\tt Attribute} hierarchy need to be specified, these are: 1. {\bf link changes} are structural links created when two separate {\tt Attribute} hierarchies are linked, 2. {\bf structural changes} are changes which occur when {\tt Attribute}s or {\tt Attribute} packages are added or removed within a single level of an {\tt Attribute} hierarchy, and 3. {\bf value changes} occur when the value portion of any single {\tt Attribute} is modified.  These definitions will help to describe how {\tt ESMF\_StateReconcile()} and {\tt ESMF\_AttributeUpdate()} can be effectively used to ensure a consistent view of metadata throughout a model run.

The {\tt ESMF\_StateReconcile()} call is used to create a consistent view of ESMF objects over the entire VM in the initialization phase of a model run.  All {\tt Attribute}s that are attached to an ESMF object contained in the State, i.e. an object that is being reconciled, are also reconciled.  This means that, at the conclusion of {\tt ESMF\_StateReconcile()} there is a one-to-one correspondence between {\tt Attribute} hierarchies and the objects to which they belong.  This is the only place where link changes in an {\tt Attribute} hierarchy can be resolved.

The {\tt ESMF\_AttributeUpdate()} call can be used any time during the run phase of a model to insure that either structural or value changes made to an {\tt Attribute} hierarchy on a subset of the VM are consistently represented across the remainder of the VM.  This call is similar to {\tt ESMF\_StateReconcile()} in that it must be called from a location that has a view of the entire VM across which to update the {\tt Attribute} hierarchy, such as a coupler Component.  The main difference is that {\tt ESMF\_AttributeUpdate()} operates only on the underlying {\tt Attribute} hierarchy of the given ESMF object.  The {\tt Attribute} hierarchy may be updated as many times as necessary, this call is much more efficient than {\tt ESMF\_StateReconcile()} for this reason.  The specification of a list of PETs that are to be used as the basis for the update is one such efficiency boost, as this allows a many-to-many communication.

\subsubsection{Copying {\tt Attribute} hierarchies}

The ability to copy an {\tt Attribute} hierarchy is limited at this time.  The {\tt ESMF\_AttributeCopy()} routine can be used to {\it locally} copy an {\tt Attribute} hierarchy between States.  It is important to note that this is a local copy, and no inter-PET communication is carried out.  Another feature of note is that this functionality is based on a shallow copy, and therefore further changes made to some portions of the original {\tt Attribute} hierarchy will also affect the new {\tt Attribute} hierarchy.

It must be noted that this is not a true shallow copy, it is more of a hybrid approach of shallow and deep copies.  In this case the {\tt Attribute}s which {\it belong} to the object being copied are actually copied in full, while the {\tt Attribute}s which are linked to the object being copied are referenced by a pointer.  This means that after copying an {\tt Attribute} hierarchy from ESMF object A to ESMF object B, the changes made to the lower portion of either A or B's {\tt Attribute} hierarchy will be reflected on {\it both} object A and object B.





