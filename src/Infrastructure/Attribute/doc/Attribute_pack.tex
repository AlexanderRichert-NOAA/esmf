% $Id: Attribute_pack.tex,v 1.9 2010/09/13 05:50:47 eschwab Exp $
%
% Earth System Modeling Framework
% Copyright 2002-2010, University Corporation for Atmospheric Research,
% Massachusetts Institute of Technology, Geophysical Fluid Dynamics
% Laboratory, University of Michigan, National Centers for Environmental
% Prediction, Los Alamos National Laboratory, Argonne National Laboratory,
% NASA Goddard Space Flight Center.
% Licensed under the University of Illinois-NCSA License.


\label{sec:AttPacks}

At this time, all ESMF objects which are enabled to contain Attributes can also contain Attribute packages.  Every Attribute package is specified by a {\bf convention} and a {\bf purpose}, hereafter called {\bf specifiers}, such as "CF" (see below) and "General".  These specifiers are used to validate ESMF Attribute packages against existing metadata conventions.  The user can choose to use an ESMF pre-defined Attribute package, specify their own Attribute package, or add customized Attributes to any of the ESMF pre-defined Attribute packages. Currently, the creation and setting of Attribute packages is quite involved, but future development with IO will allow for a more automated approach to populating Attribute packages from a file.  This is already possible via {\tt ESMF\_AttributeRead()} for the ESMF/ESG/CF Attribute packages supplied by ESMF, as well as for custom individual Attributes not in a package.

The standard Attribute packages supplied by ESMF exist for the following ESMF objects:

\begin{itemize}
    \item CplComp
    \item GridComp
    \item State
    \item Field
    \item Array
    \item Grid
\end{itemize}

The packages described in this section are grouped by the ESMF object they apply to. The creation of custom attributes and custom attribute packages is also possible and is discussed in Section X. In some cases it is possible to nest custom packages on top of ESMF packages. Attribute package nesting is described separately in the following section.  

Some Attributes come with a controlled vocabulary. A controlled vocabulary is a list of options that can be selected as the value of the attribute. The controlled vocabularies listed in this documentation represent those chosen by the community. They are not exhaustive and users may set these Attributes to a different value if they so choose. The primary consequence of doing so is that the resulting output may not be recognized by any of the online tools being developed with respect to this controlled vocabulary.  

\subsubsection{Component Attribute Packages}
\label{ComponentAttributePackages}

There are many attributes that are used to describe components. There are currently 5 predefined component-level Attribute packages:

\begin{enumerate}
    \item Earth System Grid (ESG) Basic
    \item Common Information Model (CIM) Main
    \item Common Information Model (CIM) Platform
    \item Common Information Model (CIM) Responsible Party
    \item Common Information Model (CIM) Citation
\end{enumerate}

\vspace{.20in}

{\bf 1. Earth System Grid (ESG) Basic Attribute Package}

\begin{itemize}
    \item Convention: ESG or ESMF
    \item Purpose: General
    \item Output Options:
    \begin{itemize}
        \item{Simple XML file. This requires the Xerces 3rd party library, see \ref{Att:Xerces}}
    \end{itemize} 
    \item Description: This package contains several Attributes used to describe model components within the Earth System Grid (ESG) ontology. 
\end{itemize}


\begin{tabular}{|p{8cm}|p{20cm}|p{10cm}}
     \hline\hline
     {\bf Name} & {\bf Definition} & {\bf Controlled Vocabulary}\\
     \hline\hline
     {\tt Agency} & An administrative unit of government.& DoD, DOE, DOI, NASA, NOAA, NSF\\
     {\tt Author} & The person who created the content of a book, article, or other source. & N/A\\
     {\tt CodingLanguage} & The computer language in which a unit of software is written. & C, C++, F77, F90, Java\\
     {\tt Discipline} & A subject, theme, category, or general area of interest.& Aerosol, Fisheries, Climate, Carbon Cycle, Hydrology, Land, Ocean, Polar, Sediment, Storm Surge, Turbulence, Weather, Wave, Weather Prediction \\
     {\tt FullName} & The name of a model, model component, simulation, experiment, or dataset with all acronyms spelled out.& N/A\\
     {\tt Institution} & An organization associated with a model component, simulation, or dataset.& N/A\\
     {\tt ModelComponentFramework} & The software package or mechanism used to transfer and transform data between model components.& CCA, ESMF, Flume, FMS, OASIS, SWMF \\
     {\tt Name} & \\
     {\tt PhysicalDomain} & A description of the geographic range being simulated. & Atmosphere, Earth System, Ice, Lake, Land Ocean, River\\
     {\tt Version} & A specific form or variation of an artifact, i.e. a unit of software or metadata. & N/A\\ 
\end{tabular}



\vspace{.20in}


{\bf 2. Common Information Model (CIM) Main Attribute Package}

\begin{itemize}
    \item Convention: CIM 1.0
    \item Purpose: Model Component Simulation Description
    \item Output Options: 
    \begin{itemize}
        \item CIM XML
    \end{itemize}  
    \item Description: The CIM is a formal model of the climate modeling process being developed by the European Union's METAFOR project (http://metaforclimate.eu/). ``It includes descriptions of the experiments being undertaken, the simulations being run in support of these experiments, the software models and tools being used to implement the simulations and the data generated by the software.'' The CIM divides up the climate modeling process into 6 sections. The CIM also contains other standards. It is the primary metadata representation for the fifth Climate Model Intercomparison Project (CMIP5). ESMF is currently implementing only a subset of the CIM. The representation is expected to grow. The CIM Main Package contains several standalone properties used to describe components. It also serves as the anchor to which other CIM packages are nested. Presently, these additional CIM Packages (described below) can only be created if the CIM Main Package is created. In the future, these packages will be decoupled, so that users may select subsections of the CIM to create and use. Nesting is described in Section \ref{sec:AttPackNesting}.
    \begin{itemize}
       \item Shared: Contains those elements that are used in many different packages. 
       \item Quality: Contains elements used to express diverse quality metrics for CIM Metadata or for artifacts that the CIM metadata describes.
       \item Grids: Provides a complete description of the horizontal and vertical discretization of modeling elements. This may refer to grids that output data is mapped onto, software adheres to, as well as activity constraints. 
       \item Activity: Specifies the experimental design including the experimental requirements and description of how simulations conform to these requirements.
       \item Software: Specifies all the modeling software components used within the experiment process. 
       \item Data: Describes the data output from the climate modeling process as well as for input data. 
    \end{itemize}
\end{itemize}

\begin{tabular}{|p{8cm}|p{20cm}|p{10cm}}
     {\bf Name} & {\bf Definition} & {\bf Controlled Vocabulary}\\
     \hline\hline
     {\tt ComponentShortName} & A version of the component name that contains acronyms. & N/A \\
     {\tt ComponentLongName} & A version of the component name with all acronyms spelled out. & N/A\\
     {\tt ComponentDescription} & A multi-line description of the component. & N/A \\
     {\tt Version} &  A specific form or variation of an artifact i.e.a unit of software or metadata. & N/A\\
     {\tt YearReleased} & The year a model component was issued. & N/A \\
     {\tt ModelType} & A short string describing the discipline of a model component. & Advection, Aerosol3D-Sources etc.\\
     {\tt SimulationShortName} & The name of the simulation. & N/A \\
     {\tt SimulationLongName} & The name of the simulation with any acronyms spelled out. & N/A \\ 
     {\tt SimulationRationale} & The reason for performing an experiment. & N/A\\
     {\tt SimulationStartDate} & The date in simulated time of the start of a model simulation. & N/A\\ 
     {\tt SimulationDuration} & The length of time a simulation runs.&  N/A\\ 
     {\tt PreviousVersion} & Name of the previous version of a model or model component. & N/A\\ 
     {\tt PreviousVersionDescription} &  A short note about the previous version of the model or model component. & N/A \\ 
\end{tabular}




\vspace{.20in}


{\bf 3. CIM Platform Attribute Package}

\begin{itemize}
    \item Convention: CIM 1.0
    \item Purpose: Platform Description
    \item Output Options:
    \begin{itemize}
       \item CIM XML 
    \end{itemize} 
    \item Description: This package describes the platform a particular simulation is run on. It must be created in conjunction with the CIM Main Package (see above) 
\end{itemize}


\begin{tabular}{|p{8cm}|p{20cm}|p{10cm}|}
     {\bf Name} & {\bf Definition} & {\bf Controlled Vocabulary} \\
     \hline\hline
     {\tt MachineName} & The name given to a computer by its system administrators. This is not the brand name of the system.& N/A \\
     {\tt MachineDescription} & A short note about the machine. & N/A \\
     {\tt MachineHardwareType} & The type of computer system (e.g. vector, parallel, cluster, etc.).& Beowulf, Parallel, Vector \\
     {\tt MachineOperatingSystem} & The software that is responsible for the management and coordination of activities and the sharing of resources of a computer. & Aix, Darwin, Irix64, Linux, SUNOS, Unicos\\
     {\tt MachineVendor} & The brand name of a computer system. & ACS, Action, Appro International, Bull SA, Cray Inc, Dalco AG Switzerland, Dawning, Dell, Fujitsu, Hitachi, HP, IMB, Intel, Koi Computers, Lenovo, Mac, NEC, NEC SUN, NUDT, PC, Pyramid Computer, Rahtyon-Aspen Systems, Self Made, SGI, Sun Microsystems, T-platorms\\
     {\tt MachineInterconnectType} & The technology used to associate each node in a supercomputer with every other node. & Cray Interconnect, Fat Tree, Gigabit Ethernet, Infiniband, Mixed, Myrinet, Numalink, Quadrics, SP Switch\\
     {\tt MachineMaximumProcessors} & The highest number of computer chips on a computer system. & N/A\\
     {\tt MachineCoresPerProcessor} & The number of sub-divided elements or mini-chips on a computer chip. &  N/A\\
     {\tt MachineProcessor} & The type of computer chip used in a particular computer platform. & Altix, AMD x86-64, Bluegene, G4, G5, Intel em64t, intel IA-64, Itanium, NEC, operteron, Origin3800, Pentium 3, Pentium 4, SP, SPARC, X1, Xeon, XT3-4, ZX6000\\
     {\tt MachineCompiler} & The brand of the software that takes source code and turns it into an executable.& Absoft, Default, Intel, Lahey, NAG, Pathscale, PGI, PGIGCC, XLF, XLFGCC\\
     {\tt MachineCompilerVersion} & The specific configuration value of the software used to take source code and turn it into executable code. & N/A \\
\end{tabular}


\vspace{.20in}

{\bf 4. CIM Responsible Party Attribute Package}

\begin{itemize}
    \item Convention: ISO 19115
    \item Purpose: Responsible Party Description
    \item Output Options: 
    \begin{itemize}
        \item CIM XML 
    \end{itemize} 
    \item Descriptipn: This package is used to describe contacts, authors, institutions, and funding agencies. 
\end{itemize}


\begin{tabular}{|p{8cm}|p{20cm}|p{10cm}|}
     {\bf Name } & {\bf Definition} & {\bf Controlled Vocabulary} \\
     \hline\hline
     {\tt IndividualName} & The person designated to provide information about a model component. & N/A \\
     {\tt IndividualPhysicalAddress} & The address of the person designated to provide information about a model component. & N/A \\
     {\tt IndividualEmailAddress} & The email address that others can use to ask questions about a model component. & N/A \\
     {\tt IndividualURL} & A URL of a contact or institution. & N/A \\
     {\tt ResponsiblePartyRole} & A flag to define the Responsible Party block. & Author, Principal Investigator, Contact, Center, Funder\\
\end{tabular}



\vspace{.20in}
\subsubsection{State Attribute Packages}
\label{StateAttributePackages}

There is currently only 1 predefined State-level Attribute package:

\begin{enumerate}
    \item ESMF Basic
\end{enumerate}



\vspace{.20in}
{\bf 1.ESMF Basic State Attribute Package}

\begin{itemize}
    \item Convention: ESMF
    \item Purpose: General
    \item Output Options
    \begin{itemize}
        \item Tab-delimited
        \item Simple XML 
    \end{itemize}
    \item Description: This package is used to define whether an ESMF State object is an Import State or Export State.   
\end{itemize}

\begin{tabular}{|p{8cm}|p{20cm}|p{10cm}|}
    {\bf Name } & {\bf Definition} & {\bf Controlled Vocabulary} \\
    \hline\hline
    {\tt StateIntent} & An indication of whether a field is imported into or exported from a particular model component. This refers to coupling, and not history outout. & Export,Import \\
\end{tabular}


\vspace{.20in}
\subsubsection{Field Attribute Packages}
\label{FieldAttributePackages}

Several standards exist to describe fields. There are currently 4 predefined Field-level Attribute packages:

\begin{enumerate}
    \item Climate Forecast (CF) Convention General
    \item Climate Forecast (CF) Convention Extended
    \item Common Information Model (CIM) Inputs
    \item Earth System Grid General

\end{enumerate}

\vspace{.20in}
{\bf 1. Climate Forecast (CF) Convention General}

\begin{itemize}
    \item Convention: CF
    \item Purpose: General
    \item Output Options:
    \begin{itemize}
        \item Tab-delimited
        \item Simple XML
        \item CIM XML (when part of the CIM package)
    \end{itemize} 
    \item: Description: The climate and forecast (CF) convention contains metadata that is designed to promote the processing and sharing of files created with the NetCDF API. The CF conventions are increasingly gaining acceptance and have been adopted by a number of projects and groups as a primary standard. The conventions define metadata that provide a definitive description of what the data in each variable represents, and the spatial and temporal properties of the data. This enables users of data from different sources to decide which quantities are comparable, and facilitates building applications with powerful extraction, regridding, and display capabilities. The ESMF CF Attribute package contains the three mandatory Attributes required to describe fields.  
\end{itemize}

\begin{tabular}{|p{8cm}|p{20cm}|p{10cm}|}
    {\bf Name } & {\bf Definition} & {\bf Controlled Vocabulary} \\
    \hline\hline
    {\tt VariableLongName} & An ad-hoc long descriptive name which may, for example, be used for labeling plots & N/A\\
    {\tt VariableShortName}  & The short\_name is technically not part of the CF stanard but is commonly the name of the variable on the output file and so is distinct from the long\_name & N/A \\
    {\tt VariableUnits}  & The value of the units attribute is a string that can be recognized by UNIDATA"s Udunits package & None\\
\end{tabular}




\vspace{.20in}
{\bf 2. Climate Forecast (CF) Convention Extended}

\begin{itemize}
    \item Convention: CF
    \item Purpose: Extended
    \item Output Options: 
    \begin{itemize}
        \item Tab-delimited
        \item Simple XML
        \item CIM XML (when part of the CIM package)
    \end{itemize} 
    \item The CF standard for fields contains an optional standard\_name Attribute. Standard names are controlled vocabularies and not every variable in the earth system sciences contains a standard name. Because of this, ESMF implemented this optional Attribute in its own package.
\end{itemize}

\begin{tabular}{|p{8cm}|p{20cm}|p{10cm}|}
    {\bf Name } & {\bf Definition} & {\bf Controlled Vocabulary} \\
    \hline\hline
    {\tt VariableStandardName} & The approved CF standard name for a variable if it exists &  N/A\\
\end{tabular}



\vspace{.20in}
{\bf 3. Earth System Grid (ESG) Field}

\begin{itemize}
    \item Convention: ESG or ESMF
    \item Purpose : General
    \item Output Options: 
    \begin{itemize}
        \item Tab-delimited
        \item Simple XML
    \end{itemize}
    \item Description: ESG has the ability to list variables as either import or export variables. This should not be confused with the ESMF State Attribute Package, which has similiar attributes. This attribute is assigned to individual variables.  
\end{itemize}

\begin{tabular}{|p{8cm}|p{20cm}|p{10cm}|}
     {\bf Name } & {\bf Definition} & {\bf Controlled Vocabulary} \\
     \hline\hline
     {\tt VariableIntent} & An indication of whether a variable is exported or imported. This refers to coupling and not history output. & Export,Import\\
\end{tabular}




\vspace{.20in}
{\bf 4. Common Information Model (CIM) Inputs Package}
\label{CIMFieldAttributePackages}

\begin{itemize}
    \item Convention: CIM 1.0
    \item Purpose: Inputs Description
    \item Output Options
    \begin{itemize}
        \item CIM XML 
    \end{itemize}
    \item Description: This package is used to describe a simulation and the input (initial and boundary) conditions used in that simulation. It is also used to describe any ancillary data sets that contain input condition variables. This package should not be used to describe the variables in an unconfigured model component. A pre-defined Attribute package for that case will be implemented in a future release of ESMF.    
\end{itemize}

\begin{tabular}{|p{8cm}|p{20cm}|p{10cm}|}
    {\bf Name} & {\bf Definition} & {\bf Controlled Vocabulary} \\
    \hline\hline
    {\tt InputType} & The form of the input condition (e.g. initial condition or boundary condition). &  Ancillary, Boundary, Initial\\
    {\tt InputSourceComponent} & The component the input condition is coming from. & N/A \\ 
    {\tt InputTargetComponent} & The component the input condition is going to. & N/A \\ 
    {\tt InputTechnique} & The software package or mechanism used to transfer and transform data between model components. & CCSM Flux Coupler, ESMF, Files, FMS, MCT, OASIS3, Shared\\ 
    {\tt InputSpatialRegriddingMethod} & Method used to interpolate a field from one grid (source grid) to another (target grid). & Conservative, Non-conservative, None\\ 
    {\tt InputSpatialRegriddingType} & Characteristics of the scheme used to interpolate a field from one grid (source grid) to another (target grid).& 2D-FirstOrder, 2D-SecondOrder, 3D-FirstOrder, 3D-SecondOrder\\ 
    {\tt InputFrequency} & The frequency (e.g. months, days) that a field from one component is input
 to another. & N/A\\ 
    {\tt InputTimeTransformationType} & Temporal transformation performed on the input field before or after regridding onto the target grid.& Exact, None, Time accumulation, Time averaged\\ 
\end{tabular}




\vspace{.20in}
\subsubsection{Array Attribute Packages}
\label{ArrayAttributePackages}

At this time the Array packages are the same as the Field packages.


\vspace{.20in}
\subsubsection{Grid Attribute Packages}
\label{GridAttributePackages}

The GFDL Gridspec Standard is currently the only Grid pre-defined Attribute package in ESMF.

\begin{enumerate}
    \item GFDL Gridspec
\end{enumerate}



\vspace{.20in}
{\bf 1. Gridspec}

\label{CIMFieldAttributePackages}

\begin{itemize}
    \item Convention: GridSpec or ESMF
    \item Purpose: General
    \item Output Options
    \begin{itemize}
        \item Simple XML
        \item Can be used to create a Grid object by reading in an XML file with these attributes. This is currently limited to 2D regularly distributed rectinlinear grids. See Section \ref{example:GridCrFromFile} for details. 
    \end{itemize}
    \item Description: This package contains the Attributes developed as part of GFDL's Gridspec standard.
\end{itemize}

\begin{tabular}{|p{8cm}|p{20cm}|p{10cm}|}
{\bf Name} & {\bf Definition} & {\bf Controlled Vocabulary} \\
\hline\hline
{\tt GridType} & A text description of the grid that uses common terminology. & Gnomonic,Cubed Sphere,Displaced Pole, Icosahedral geodesic, Reduced gaussian, Regular lat lon, Spectral gaussian, Tripolar, Yin Yang\\ 
{\tt DimensionOrder} & The order in which latitude and longitude appear within the two dimensional grid array. & None \\ 
{\tt DiscretizationType} & Specifies the method by which a two-dimensional coordinate system is sampled to form a computational grid. &  Logically rectangular, Pixel-based catchment, Structured triangular, Unstructured triangular\\ 
{\tt GeometryType} & Indicates the geometric figure used to approximate the shape of the Earth, e.g. "sphere".& Ellipsoid, Plane, Sphere\\ 
{\tt IsConformal} & Indicates if the grid tile is angle-preserving. If so, angles measured on the grid are equal to the equivalent angles on the Earth. & True, False\\ 
{\tt IsRegular} & Indicates whether or not the horizontal coordinates of the grid can be defined using 1D arrays (vectors). This means that grid node locations are defined by the cartesian product of the X/Lon and Y/Lat coordinate vectors. It also means that grid cells are logically rectangular (they may also be physically rectangular in the case of projected coordinates). & True, False\\ 
{\tt IsUniform} &Indicates whether or not the horizontal coordinates of a grid have fixed offsets in the X and Y directions. If the offset is the same in both directions then the grid is logically square, otherwise it is logically rectangular. & True, False\\ 
{\tt NorthPoleLocation} & Defines the lat-long position of the 'north pole' used by the grid tile in the case of rotated/displaced pole grids.& N/A \\ 
{\tt NumberOfCells} & The number of cells in an unstructured grid. & N/A\\ 
{\tt NX} & Specifies the length of the X, or longitude, dimension of the grid tile. & N/A \\ 
{\tt NY} & Specifies the length of the Y, or latitude, dimension of the grid tile. & N/A \\ 
\end{tabular}


\vspace{.20in}
\subsubsection{Table of Available Attributes}

The following is an alphabetical list of all the attributes impletmented in ESMF, their definitions, and which packages they are contained within. A list of attributes by package exists in the following section. 


\begin{tabular}{|p{8cm}|p{20cm}|p{10cm}}
     \hline\hline
     {\bf Name} & {\bf Definition} & {\bf Attribute Package}\\
     \hline\hline
     {\tt Agency} & An administrative unit of government.& ESG Basic Component\\
     {\tt Author} & The person who created the content of a book, article, or other source. & ESG Basic Component\\
     {\tt CodingLanguage} & The computer language in which a unit of software is written. & ESG Basic Component\\
     {\tt ComponentShortName} & A version of the component name that contains acronyms. & CIM Main\\
     {\tt ComponentLongName} & A version of the component name with all acronyms spelled out. & CIM Main\\
     {\tt ComponentDescription} & A multi-line description of the component. & CIM Main \\
     {\tt Discipline} & A subject, theme, category, or general area of interest.& ESG Basic Component \\
     {\tt DimensionOrder} & The order in which latitude and longitude appear within the two dimensional grid array. & Gridspec\\ 
     {\tt DiscretizationType} & Specifies the method by which a two-dimensional coordinate system is sampled to form a computational grid.& Gridspec \\
     {\tt FullName} & The name of a model, model component, simulation, experiment, or dataset with all acronyms spelled out.& ESG Basic Component\\
     {\tt GeometryType} & Indicates the geometric figure used to approximate the shape of the Earth, e.g. "sphere". & Gridspec \\
     {\tt GridType} & A text description of the grid that uses common terminology. & Gridspec \\
     {\tt IndividualName} & The person designated to provide information about a model component. & CIM Responsible Party\\
     {\tt IndividualPhysicalAddress} & The address of the person designated to provide information about a model component. & CIM Responsible Party\\
     {\tt IndividualEmailAddress} & The email address that others can use to ask questions about a model component. & CIM Responsible Party\\
     {\tt IndividualURL} & A URL of a contact or institution. & CIM Responsible Party\\
     {\tt InputType} & The form of the input condition (e.g. initial condition or boundary condition). &  CIM Inputs \\
     {\tt InputSourceComponent} & The component the input condition is coming from. & CIM Inputs\\ 
     {\tt InputTargetComponent} & The component the input condition is going to. & CIM Inputs\\ 
     {\tt InputTechnique} & The software package or mechanism used to transfer and transform data between model components. & CIM Inputs\\ 
     {\tt InputSpatialRegriddingMethod} & Method used to interpolate a field from one grid (source grid) to another (target grid). & CIM Inputs\\ 
     {\tt InputSpatialRegriddingType} & Characteristics of the scheme used to interpolate a field from one grid (source grid) to another (target grid).& CIM Inputs\\ 
     {\tt InputFrequency} & The frequency (e.g. months, days) that a field from one component is input to another. & CIM Inputs\\ 
     {\tt InputTimeTransformationType} & Temporal transformation performed on the input field before or after regridding onto the target grid.& CIM Inputs\\ 
     {\tt Institution} & An organization associated with a model component, simulation, or dataset.& ESG Basic Component\\
     {\tt IsConformal} & Indicates if the grid tile is angle-preserving. If so, angles measured on the grid are equal to the equivalent angles on the Earth. & Gridspec\\ 
     {\tt IsRegular} & Indicates whether or not the horizontal coordinates of the grid can be defined using 1D arrays (vectors). This means that grid node locations are defined by the cartesian product of the X/Lon and Y/Lat coordinate vectors. It also means that grid cells are logically rectangular (they may also be physically rectangular in the case of projected coordinates). & Gridspec\\ 
     {\tt IsUniform} &Indicates whether or not the horizontal coordinates of a grid have fixed offsets in the X and Y directions. If the offset is the same in both directions then the grid is logically square, otherwise it is logically rectangular. & Gridspec \\
     {\tt MachineName} & The name given to a computer by its system administrators. This is not the brand name of the system.& CIM Platform \\
     {\tt MachineDescription} & A short note about the machine. & CIM Platform\\
     {\tt MachineHardwareType} & The type of computer system (e.g. vector, parallel, cluster, etc.).& CIM Platform\\
     {\tt MachineOperatingSystem} & The software that is responsible for the management and coordination of activities and the sharing of resources of a computer. & CIM Platform\\
     {\tt MachineVendor} & The brand name of a computer system. & CIM Platform \\
     {\tt MachineInterconnectType} & The technology used to associate each node in a supercomputer with every other node. & CIM Platform\\
     {\tt MachineMaximumProcessors} & The highest number of computer chips on a computer system. & CIM Platform\\
     {\tt MachineCoresPerProcessor} & The number of sub-divided elements or mini-chips on a computer chip. &  CIM Platform\\
     {\tt MachineProcessor} & The type of computer chip used in a particular computer platform. & CIM Platform\\
     {\tt MachineCompiler} & The brand of the software that takes source code and turns it into an executable.& CIM Platform\\
     {\tt MachineCompilerVersion} & The specific configuration value of the software used to take source code and turn it into executable code. & CIM Platform\\
     {\tt ModelComponentFramework} & The software package or mechanism used to transfer and transform data between model components.& ESG Basic Component \\
     {\tt ModelType} & A short string describing the discipline of a model component. & CIM Main\\
     {\tt Name} & \\
     {\tt NorthPoleLocation} & Defines the lat-long position of the 'north pole' used by the grid tile in the case of rotated/displaced pole grids.& N/A \\ 
     {\tt NumberOfCells} & The number of cells in an unstructured grid. & \\ 
     {\tt NumDims} & & N/A\\ 
     {\tt NX} & Specifies the length of the X, or longitude, dimension of the grid tile. & Gridspec \\ 
     {\tt NY} & Specifies the length of the Y, or latitude, dimension of the grid tile. & Gridspec\\ 
     {\tt PhysicalDomain} & A description of the geographic range being simulated. & ESG Basic Component \\
     {\tt PreviousVersion} & Name of the previous version of a model or model component. & CIM Main\\ 
     {\tt PreviousVersionDescription} &  A short note about the previous version of the model or model component. & CIM Main \\
     {\tt ResponsiblePartyRole} & A flag to define the Responsible Party block. &  CIM Responsible Party\\ 
     {\tt SimulationShortName} & The name of the simulation. & CIM Main\\
     {\tt SimulationLongName} & The name of the simulation with any acronyms spelled out. & CIM Main\\ 
     {\tt SimulationRationale} & The reason for performing an experiment. & CIM Main\\
     {\tt SimulationStartDate} & The date in simulated time of the start of a model simulation. & CIM Main\\ 
     {\tt SimulationDuration} & The length of time a simulation runs.&  CIM Main\\ 
     {\tt StateIntent} & An indication of whether a field is imported into or exported from a particular model component. This refers to coupling, and not history outout. & ESMF State\\
     {\tt YearReleased} & The year a model component was issued. & CIM Main\\
     {\tt VariableLongName} & An ad-hoc long descriptive name which may, for example, be used for labeling plots & CF General\\
     {\tt VariableShortName}  & The short\_name is technically not part of the CF stanard but is commonly the name of the variable on the output file and so is distinct from the long\_name & CF General\\
     {\tt VariableStandardName} & The approved CF standard name for a variable if it exists &  CF Extended\\
     {\tt VariableIntentState} & An indication of whether a variable is exported or imported. This refers to coupling and not history output. & ESG Field\\
     {\tt VariableUnits}  & The value of the units attribute is a string that can be recognized by UNIDATA"s Udunits package & CF General\\
     {\tt Version} & A specific form or variation of an artifact i.e. a unit of software or metadata. & ESG Basic Component, CIM Main\\ 

\end{tabular}

\vspace{.20in}




\subsubsection{Custom Attribute Packages}
\label{sec:CustomAttPacks}

ESMF allows for the creation of custom attribute packages. This can be done to augment one of the pre-defined packages (via package nesting \ref{sec:AttPackNesting}) or to create a suite of 
attributes unique to the user. An example of how to create a custom package is contained in Section XX (link to the proposed custom att pack example).
