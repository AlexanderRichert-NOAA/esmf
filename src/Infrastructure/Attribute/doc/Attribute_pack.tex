% $Id: Attribute_pack.tex,v 1.1 2010/08/18 16:55:56 murphysj Exp $
%
% Earth System Modeling Framework
% Copyright 2002-2010, University Corporation for Atmospheric Research,
% Massachusetts Institute of Technology, Geophysical Fluid Dynamics
% Laboratory, University of Michigan, National Centers for Environmental
% Prediction, Los Alamos National Laboratory, Argonne National Laboratory,
% NASA Goddard Space Flight Center.
% Licensed under the University of Illinois-NCSA License.


\label{desc:AttPacks}

At this time, all ESMF objects which are enabled to contain Attributes can also contain Attribute packages.  Every Attribute package is specified by a {\bf convention} and a {\bf purpose}, hereafter called {\bf specifiers}, such as "CF" (see below) and "General".  These specifiers are used to validate ESMF Attribute packages against existing metadata conventions.  The user can choose to use an ESMF pre-defined Attribute package, specify their own Attribute package, or add customized Attributes to any of the ESMF pre-defined Attribute packages. Currently, the creation and setting of Attribute packages is quite involved, but future development with IO will allow for a more automated approach to populating Attribute packages from a file.  This is already possible via {\tt ESMF\_AttributeRead()} for the ESMF/ESG/CF Attribute packages supplied by ESMF, as well as for custom individual Attributes not in a package.

The standard Attribute packages supplied by ESMF exist for the following ESMF objects:

\begin{itemize}
    \item CplComp
    \item GridComp
    \item State
    \item Field
    \item Grid
    \item Array
\end{itemize}

The packages described in this section are grouped by the ESMF object they apply to. The creation of custom attributes and custom attribute packages is also possible and is discussed in Section X. In some cases it is possible to nest custom packages to ESMF packages. Attribute package nesting is described separately in the following section.  

Some Attributes come with a controlled vocabulary. A controlled vocabulary is a list of options that can be selected as the value of the attribute. The controlled vocabularies listed in this documentation represent those chosen by the community. They are not exhaustive and users may set these Attributes to a different value if they so choose. The primary consequence of doing so is that the resulting output may not be recognized by any of the online tools being developed around this controlled vocabulary.  

\subsubsection{CplComp and GridComp Packages}
\label{ComponentAttributePackages}

There many attributes that are used to describe components. There are currently 2 predefined component-level Attribute packages:

\begin{itemize}
    \item Earth System Grid (ESG) Simple
    \item Common Information Model (CIM) 
\end{itemize}


{\bf ESG Attribute Package}

\begin{itemize}
    \item Convention = General
    \item Purpose = ESG Simple
    \item Output Options
    \begin{itemize}
        \item{Simple XML file. This requires the Xerces 3rd party library, see \ref{Att:Xerces}}
    \end{itemize}  
\end{itemize}

This package contains several simple Earth System Grid (ESG) Attributes. 

\begin{tabular}{|p{8cm}|p{20cm}|}
     \hline\hline
     {\bf Name} & {\bf Definition} \\
     \hline\hline
     {\tt Agency} & An administrative unit of government.\\
     {\tt Author} & The person who created the content of a book, article, or other source.\\
     {\tt CodingLanguage} & The computer language in which a unit of software is written. \\
     {\tt Discipline} & A subject, theme, category, or general area of interest.\\
     {\tt FullName} & The name of a model, model component, simulation, experiment, or dataset with all acronyms spelled out.\\
     {\tt Institution} & An organization associated with a model component, simulation, or dataset.\\
     {\tt ModelComponentFramework} & The software package or mechanism used to transfer and transform data between model components. \\
     {\tt Name} & \\
     {\tt PhysicalDomain} & A description of the geographic range being simulated.\\
     {\tt Version} & A specific form or variation of an artifact i.e. a unit of software or metadata.\\ 
\end{tabular}

{\bf Common Information Model (CIM) Main Attribute Package}

\begin{itemize}
    \item Convention = CIM 1.0
    \item Purpose = Model Component Description
    \item Output Options
    \begin{itemize}
        \item CIM XML
    \end{itemize}  
\end{itemize}

The CIM is a formal model of the climate modelling process being developed by the European Union's METAFOR project (http://metaforclimate.eu/). ``It includes descriptions of the experiments being undertaken, the simulations being run in support of these experiments, the software models and tools being used to implement the simulations and the data generated by the software.'' The CIM divides up the climate modeling process into 6 sections:
 
\begin{itemize}
    \item Shared: Contains those elements that are used in many different packages. 
    \item Quality: Contains elements used to express diverse quality metrics for CIM Metadata or for artifacts that the CIM metadata describes.
    \item Grids: Provides a complete description of the horizontal and vertical discretization of modeling elements. This may refer to grids that output data is mapped onto, software adheres to, as well as activity constraints. 
    \item Activity: Specifies the experimental design including the experimental requirements and description of how simulations conform to these requirements.
    \item Software: Specifies all the modeling software components used within the experiment process. 
    \item Data: Describes the data output from the climate modelling process as well as for input data. 
\end{itemize}

The CIM also contains other standards. It is the primary metadata representation for the fifth Climate Model Intercomparison Project (CMIP5). 

ESMF is currently implementing only a subset of the CIM. The representation is expected to grow.  

The CIM Main Package contains several standalone properties used to describe components. It also serves as the anchor to which other CIM packages are nested. Presently, these additional CIM Packages (described below) can only be created if the CIM Main Package is created. In the future, these packages will be decoupled, so that users may select subsections of the CIM to create and use. Nesting is described in Section XX.



\begin{tabular}{|p{8cm}|p{20cm}|p{10cm}}
     {\bf Name} & {\bf Definition} & {\bf Controlled Vocabulary}\\
     \hline\hline
     {\tt ComponentShortName} & A version of the component name that contains acronyms. & None \\
     {\tt ComponentLongName} & A version of the component name with all acronyms spelled out. & None\\
     {\tt ComponentDescription} & A multi-line description of the component. & None \\
     {\tt Version} &  A specific form or variation of an artifact i.e.a unit of software or metadata. & None\\
     {\tt YearReleased} & The year a model component was issued. & None \\
     {\tt ModelType} & A short string describing the discipline of a model component. & Advection, Aerosol3D-Sources etc.\\
     {\tt SimulationShortName} & The name of the simulation. & None \\
     {\tt SimulationLongName} & The name of the simulation with any acronyms spelled out. & None. \\ 
     {\tt SimulationRationale} & The reason for performing an experiment. & None\\
     {\tt SimulationStartDate} & The date in simulated time of the start of a model simulation. & None\\ 
     {\tt SimulationDuration} & The length of time a simulation runs.&  None\\ 
     {\tt PreviousVersion} & Name of the previous version of a model or model component. & None\\ 
     {\tt PreviousVersionDescription} &  A short note about the previous version of the model or model component. & None \\ 
\end{tabular}


{\bf CIM Platform Attribute Package}

This package describes the platform a particular simulation is run on. It must be created in conjunction with the CIM Main Package (see above)

\begin{itemize}
    \item Convention = CIM 1.0
    \item Purpose = Platform Description
    \item Output Options
    \begin{itemize}
       \item CIM XML 
    \end{itemize}  
\end{itemize}


\begin{tabular}{|p{8cm}|p{20cm}|p{10cm}|}
     {\bf Name} & {\bf Definition} & {\bf Controlled Vocabulary} \\
     \hline\hline
     {\tt MachineName} & The name given to a computer by its system administrators. This is not the brand name of the system.& None \\
     {\tt MachineDescription} & A short note about the machine. & None \\
     {\tt MachineHardwareType} & The type of computer system (e.g. vector, parallel, cluster, etc.).& None \\
     {\tt MachineOperatingSystem} & The software that is responsible for the management and coordination of activities and the sharing of resources of a computer. & None\\
     {\tt MachineVendor} & The brand name of a computer system. & None \\
     {\tt MachineInterconnectType} & The technology used to associate each node in a supercomputer with every other node. & \\
     {\tt MachineMaxProcessors} & The highest number of computer chips on a computer system. & \\
     {\tt MachineCoresPerProcessor} & The number of sub-divided elements or mini-chips on a computer chip. &  \\
     {\tt MachineProcessor} & The type of computer chip used in a particular computer platform. & \\
     {\tt MachineCompiler} & The brand of the software that takes source code and turns it into an executable.& \\
     {\tt MachineCompilerVersion} & The specific configuration value of the software used to take source code and turn it into executable code. & None \\
\end{tabular}


{\bf CIM Responsible Party Attribute Package}

\begin{itemize}
    \item Convention = ISO 19115
    \item Purpose = Responsible Party Description
    \item Output Options
    \begin{itemize}
        \item CIM XML 
    \end{itemize} 
\end{itemize}


\begin{tabular}{|p{8cm}|p{20cm}|p{10cm}|}
     {\bf Name } & {\bf Definition} & {\bf Controlled Vocabulary} \\
     \hline\hline
     {\tt IndividualName} & The person designated to provide information about a model component. & None \\
     {\tt IndividualPhysicalAddress} & The address of the person designated to provide information about a model component. & None \\
     {\tt IndividualEmailAddress} & The email address that others can use to ask questions about a model component. & None \\
     {\tt IndividualURL} & A URL of a contact or institution. & None \\
     {\tt ResponsiblePartyRole} & A flag to define the Responsible Party block. & Author, Principal Investigator, Contact, Center, Funder\\
\end{tabular}


\subsection{State}
\label{StateAttributePackages}

There is currently only 1 predefined State-level Attribute package:

\begin{itemize}
    \item Earth System Grid (ESG) Simple
\end{itemize}

{\bf ESMF}

\begin{itemize}
    \item Convention = ESMF
    \item Purpose = General
    \item Output Options
    \begin{itemize}
        \item Tab-delimited
        \item Simple XML 
    \end{itemize} 
\end{itemize}

\begin{tabular}{|p{8cm}|p{20cm}|p{10cm}|}
    {\bf Name } & {\bf Definition} & {\bf Controlled Vocabulary} \\
    \hline\hline
    {\tt VariableState} & An indication of whether a field is imported into or exported from a particular model component. This refers to coupling, and not history outout. & Import, Export \\
\end{tabular}



\subsection{Field}
\label{FieldAttributePackages}

Several standards exist to describe fields. There are currently 4 predefined Field-level Attribute packages:

\begin{itemize}
    \item Climate Forecast (CF) Convention General
    \item Climate Forecast (CF) Convention Extended
    \item Common Information Model (CIM) Inputs
    \item Earth System Grid General

\end{itemize}


{\bf Climate Forecast (CF) Convention General}

\begin{itemize}
    \item Convention = CF
    \item Purpose = General
    \item Output Options
    \begin{itemize}
        \item Tab-delimited
        \item Simple XML
        \item CIM XML (when part of the CIM package)
    \end{itemize} 
\end{itemize}

\begin{tabular}{|p{8cm}|p{20cm}|p{10cm}|}
    {\bf Name } & {\bf Definition} & {\bf Controlled Vocabulary} \\
    \hline\hline
    {\tt VariableLongName} & & \\
    {\tt VariableShortName}  & & \\
    {\tt VariableUnits}  & & \\
\end{tabular}


{\bf Climate Forecast (CF) Convention Extended}

\begin{itemize}
    \item Convention = CF
    \item Purpose = Extended
    \item Output Options
    \begin{itemize}
        \item Tab-delimited
        \item Simple XML
        \item CIM XML (when part of the CIM package)
    \end{itemize} 
\end{itemize}

\begin{tabular}{|p{8cm}|p{20cm}|p{10cm}|}
    {\bf Name } & {\bf Definition} & {\bf Controlled Vocabulary} \\
    \hline\hline
    {\tt VariableStandardName} & & \\
\end{tabular}


{\bf Earth System Grid (ESG) Field}

\begin{itemize}
    \item Convention = ESG
    \item Purpose = General
    \item Output Options
    \begin{itemize}
        \item Tab-delimited
        \item Simple XML
    \end{itemize} 
\end{itemize}

\begin{tabular}{|p{8cm}|p{20cm}|p{10cm}|}
     {\bf Name } & {\bf Definition} & {\bf Controlled Vocabulary} \\
     \hline\hline
     {\tt VariableState} 7  & Export,Import\\
\end{tabular}


{\bf CIM Field}
\label{CIMFieldAttributePackages}

\begin{itemize}
    \item Convention = CIM 1.0
    \item Purpose = Inputs Description
\item Output Options
    \begin{itemize}
        \item CIM XML 
    \end{itemize} 
\end{itemize}

\begin{tabular}{|p{8cm}|p{20cm}|p{10cm}|}
    {\bf Name} & {\bf Definition} & {\bf Controlled Vocabulary} \\
    \hline\hline
    {\tt InputType} & The form of the input condition (e.g. initial condition or boundary condition). &  \\
    {\tt InputSourceComponent} & The component the input condition is coming from. & \\ 
    {\tt InputTargetComponent} & The component the input condition is going to. & \\ 
    {\tt InputTechnique} & The software package or mechanism used to transfer and transform data between model components. & \\ 
    {\tt InputSpatialRegriddingMethod} & Method used to interpolate a field from one grid (source grid) to another (target grid). & \\ 
    {\tt InputSpatialRegriddingType} & Characteristics of the scheme used to interpolate a field from one grid (source grid) to another (target grid).& \\ 
    {\tt InputFrequency} & The frequency (e.g. months, days) that a field from one component is input
 to another. & \\ 
    {\tt InputTimeTransformationType} & Temporal transformation performed on the input field before or after regridding onto the target grid.& \\ 
\end{tabular}


\subsection{Array}
\label{ArrayAttributePackages}

At this time the Array packages are the same as the field packages.



\subsection{Grid}
\label{GridAttributePackages}

The GFDL Gridspec Standard is currently the only Grid pre-defined Attribute package in ESMF.

\begin{itemize}
    \item GFDL Gridspec
\end{itemize}

{\bf Gridspec}

\begin{tabular}{|p{8cm}|p{20cm}|p{10cm}|}
{\bf Name} & {\bf Definition} & {\bf Controlled Vocabulary} \\
\hline\hline
{\tt GridType} & A text description of the grid that uses common terminology. & \\ 
{\tt DimOrder} & The order in which latitude and longitude appear within the two dimensional grid array. & \\ 
{\tt DiscretizationType} & Specifies the method by which a two-dimensional coordinate system is sampled to form a computational grid. &  \\ 
{\tt GeometryType} & Indicates the geometric figure used to approximate the shape of the Earth, e.g. "sphere".& \\ 
{\tt IsConformal} & Indicates if the grid tile is angle-preserving. If so, angles measured on the grid are equal to the equivalent angles on the Earth. & True, False\\ 
{\tt IsPoleCovered} & & \\ 
{\tt IsRegular} & Indicates whether or not the horizontal coordinates of the grid can be defined using 1D arrays (vectors). This means that grid node locations are defined by the cartesian product of the X/Lon and Y/Lat coordinate vectors. It also means that grid cells are logically rectangular (they may also be physically rectangular in the case of projected coordinates). & True, False\\ 
{\tt IsUniform} &Indicates whether or not the horizontal coordinates of a grid have fixed offsets in the X and Y directions. If the offset is the same in both directions then the grid is logically square, otherwise it is logically rectangular. & \\ 
{\tt NorthPoleLocation} & Defines the lat-long position of the 'north pole' used by the grid tile in the case of rotated/displaced pole grids.& None \\ 
{\tt NumberOfCells} & The number of cells in an unstructured grid. & \\ 
{\tt NumDims} & & \\ 
{\tt NX} & Specifies the length of the X, or longitude, dimension of the grid tile. & None \\ 
{\tt NY} & Specifies the length of the Y, or latitude, dimension of the grid tile. & None \\ 
{\tt NZ} & Specifies the length of the Z, or vertical deminsion of a grid. & None \\ 
{\tt Resolution} & & \\ 
\end{tabular}


ESMF also allows nesting of Attribute packages.  This capability is intended to help organize different metadata compliance levels, such as CF and ESG.  The nesting of Attribute packages is also very helpful when adding customized Attributes to a package.  A primary use of the nesting capabilities of Attribute packages is geared towards organizing different metadata compliance levels.  For instance, the CF metadata standard for Fields requires that there be Attributes to track the {\it name}, {\it long\_name}, {\it standard\_name}, and {\it units} of the Field.  The ESG standard, on the other hand, requires two additional Attributes called {\it import} and {\it export}.  In this case the ESMF representation of the ESG compliant Attribute package for a Field would involve a nested Attribute package structure.  This would involve the ESG-specific Attribute package, containing the Attributes {\it import} and {\it export} containing a nested version of the CF-specific Attribute package, with the Attributes {\it name}, {\it long\_name}, {\it standard\_name}, and {\it units}.  An Attribute package can be nested by including the specifiers of both packages in the {\tt ESMF\_AttributeAdd()} interface call.  

In addition to the single-child nesting described above, multiple-child nesting is also possible, allowing for full tree structures of Attribute packages.  An example of this can be seen in the CIM Component Attribute package above.  It has two child sub-packages, one with purpose "Platform Description" and the other with purpose "Responsible Party Description".  These are two separate packages; one does not contain the other, however the CIM Component Attribute package contains them both at the same level.  An Attribute package can include multiple children via arrays of specifiers for the nested packages in the {\tt ESMF\_AttributeAdd()} interface call.

The nesting capabilities of Attribute packages are also very useful for organizing the customized metadata supplied by a user.  For example, if a user was not satisfied with the metadata support required in the ESG convention for Field they could supply a list of Attributes they would like to support.  This new Attribute package would then be used as an additional layer, inside which the Attribute package of ESG would be nested, inside which the CF Attribute package would be nested.  One important thing to remember when working with nested Attribute packages is that naming two Attributes the same in the same nested structure can yield undefined behavior.

An explanation of the specifiers is in order at this point.  The purpose 
specifier is really just meant as an additional means, beyond the use of 
"convention", to specify Attribute packages.  One could imagine that the 
CF convention would want to be able to have Attribute packages divided 
up in some fashion, which ESMF could then keep track of with the purpose 
specifier.  It was added with the intention of allowing Attributes, and 
packages, maximum flexibility.  Take the Field's ESMF standard Attribute 
package for example.  This 
package is made up of three nested Attribute packages.  The lowest one 
is made up of three Attributes with convention=CF and purpose=General.  
The next level contains one Attribute with convention=CF but 
purpose=Extended.  On top of this is the convention=ESG package, also 
with purpose=General.
