% $Id$

Unstructured grids are commonly used in the computational solution of partial differential 
equations.  These are especially useful for problems that involve complex geometry, where 
using the less flexible structured grids can result in grid representation of regions 
where no computation is needed.  Finite element and finite volume methods map naturally 
to unstructured grids and are used commonly in hydrology, ocean modeling, and many other 
applications.

In order to provide support for application codes using unstructured grids, the ESMF library 
provides a class for representing unstructured grids called the {\bf Mesh}. Fields can be 
created on a Mesh to hold data. Fields created on a Mesh can also be used as either the 
source or destination or both of an interpolaton (i.e. an {\tt ESMF\_FieldRegridStore()} call) 
which allows data to be moved between unstructured grids. This section describes the Mesh 
and how to create and use them in ESMF. 

\subsubsection{Mesh representation in ESMF}\label{sec:meshrep}

A Mesh in ESMF is constructed of {\bf nodes} and {\bf elements}.

A {\bf node}, also known as a vertex or corner, is a part of a Mesh which represents a single point. Coordinate information is
set in a node.

An {\bf element}, also known as a cell, is a part of a mesh which represents a small 
region of space. Elements are described in terms of a connected set of nodes which represent locations along their boundaries.

Field data may be located on either the nodes or elements of a Mesh. 

\medskip

The dimension of a Mesh in ESMF is specified with two parameters: the {\bf parametric dimension} and the {\bf spatial dimension}.

The {\bf parametric dimension} of a Mesh is the dimension of the topology of the Mesh. This can be thought of as the dimension of 
the elements which make up the Mesh. For example, a Mesh composed of triangles would have a parametric dimension of 2, whereas
a Mesh composed of tetrahedra would have a parametric dimension of 3. 

The {\bf spatial dimension} of a Mesh is the dimension of the space the Mesh is embedded in. In other words, it is the number of coordinate dimensions needed to describe the location of the nodes making up the Mesh. 

For example, a Mesh constructed of squares on a plane would have a parametric dimension of 2 and a spatial dimension of 2. 
If that same Mesh were used to represent the 2D surface of a sphere, then the Mesh would still have a parametric dimension 
of 2, but now its spatial dimension would be 3. 

\subsubsection{Supported Meshes}

The range of Meshes supported by ESMF are defined by several factors: dimension, element 
types, and distribution.

ESMF currently only supports Meshes whose number of coordinate dimensions (spatial dimension) 
is 2 or 3. The dimension of the elements in a Mesh (parametric dimension) must be less than 
or equal to the spatial dimension, but also must be either 2 or 3. This means that a Mesh may 
be either 2D elements in 2D space, 3D elements in 3D space, or a manifold constructed of 2D 
elements embedded in 3D space. 

ESMF currently supports two types of elements for each Mesh parametric dimension. For a 
parametric dimension of 2, the supported element types are triangles or quadrilaterals. For 
a parametric dimension of 3, the supported element types are tetrahedrons
and hexahedrons. See Section~\ref{const:meshelemtype} for diagrams of these. The Mesh 
supports any combination of element types within a particular dimension, but types from 
different dimensions may not be mixed.  For example, a Mesh cannot be constructed of both 
quadrilaterals and tetrahedra.

ESMF currently only supports distributions where every node on a PET must be a part of an 
element on that PET. In other words, there must not be nodes without a corresponding element 
on any PET.
