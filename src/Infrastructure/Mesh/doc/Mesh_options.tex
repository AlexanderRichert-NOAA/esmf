% $Id$

\subsubsection{ESMF\_MESHELEMTYPE}
\label{const:meshelemtype}

 {\sf DESCRIPTION:\\}
 An ESMF Mesh can be constructed from a combination of different elements. The type of elements that can
be used in a Mesh depends on the Mesh's parameteric dimension, which is set during Mesh creation. The
following are the valid Mesh element types for each valid Mesh parametric dimension (2D or 3D) .

\medskip

\begin{verbatim}

                     3                          4 ---------- 3
                    / \                         |            |  
                   /   \                        |            |
                  /     \                       |            |
                 /       \                      |            |
                /         \                     |            |
               1 --------- 2                    1 ---------- 2

           ESMF_MESHELEMTYPE_TRI            ESMF_MESHELEMTYPE_QUAD

     2D element types (numbers are the order for elementConn during Mesh create)

\end{verbatim}

For a Mesh with parametric dimension of 2 the valid element types (illustrated above) are:

\smallskip

\begin{tabular}{|l|c|l|}
\hline
Element Type &  Number of Nodes  & Description \\
\hline
ESMF\_MESHELEMTYPE\_TRI  & 3 & A triangle \\
ESMF\_MESHELEMTYPE\_QUAD & 4 & A quadrilateral (e.g. a rectangle) \\
\hline
\end{tabular}

\medskip
\medskip

\begin{verbatim}
                                            
                 3                               8---------------7
                /|\                             /|              /|
               / | \                           / |             / |
              /  |  \                         /  |            /  |
             /   |   \                       /   |           /   |
            /    |    \                     5---------------6    |
           4-----|-----2                    |    |          |    |
            \    |    /                     |    4----------|----3
             \   |   /                      |   /           |   /
              \  |  /                       |  /            |  /
               \ | /                        | /             | /
                \|/                         |/              |/
                 1                          1---------------2

       ESMF_MESHELEMTYPE_TETRA             ESMF_MESHELEMTYPE_HEX  

  3D element types (numbers are the order for elementConn during Mesh create)

\end{verbatim}

For a Mesh with parametric dimension of 3 the valid element types (illustrated above) are:

\smallskip

\begin{tabular}{|l|c|l|}
\hline
Element Type & Number of Nodes & Description \\
\hline                                         
ESMF\_MESHELEMTYPE\_TETRA & 4 & A tetrahedron (NOT VALID IN BILINEAR OR PATCH REGRID)\\
ESMF\_MESHELEMTYPE\_HEX  & 8 & A hexahedron (e.g. a cube) \\
\hline
\end{tabular}

\subsubsection{ESMF\_FILEFORMAT}
\label{const:mesh:fileformat}

{\sf DESCRIPTION:\\}
This option is used by {\tt ESMF\_MeshCreate} to specify the type of the input grid file.  See 
sections~\ref{sec:fileformat:scrip}, \ref{sec:fileformat:esmf} and \ref{sec:fileformat:ugrid} for more 
detailed description of the supported file formats.

The type of this flag is:

{\tt type(ESMF\_FileFormat\_Flag)}

The valid values are:
\begin{description}
\item [ESMF\_FILEFORMAT\_SCRIP] SCRIP format grid file. The SCRIP format is the format accepted by the SCRIP regridding tool~\cite{ref:SCRIP}.   For Mesh creation, files of this type only work when the {\tt grid\_rank} in the file is equal to 1.

\item [ESMF\_FILEFORMAT\_ESMFMESH] ESMF unstructured grid file format. This format was developed by the ESMF team to match the capabilities of the Mesh class and to be efficient to convert to that class. 

\item [ESMF\_FILEFORMAT\_UGRID] CF-convention unstructured grid file format. This format is a proposed extention to the 
CF-conventions for unstructured grid data model. Currently, only the 2D flexible mesh topology is supported in ESMF.
\end{description}

