% $Id$

The following examples demonstrate how a user typically interacts with the
HConfig API. HConfig objects can be created explicitly by the user, or they
can be accessed from an existing Config object, e.g. queried from a Component.

The {\tt ESMF\_HConfig} object plays a number of roles when interacting with
a HConfig hierarchy:
\begin{enumerate}
\item It can be the root node of the entire hierarchy. In YAML terminology, this
refers to a {\em document}.
\item It can be any node within the hierarchy.
\item It can be an iterator, {\em referencing} a specific node within the
hierarchy. The iterator approach allows convenient sequential access to all
the child nodes of a particular location in the hierarchy. There are two flavors
of iterators in HConfig: {\em sequence} and {\em map} iterators.
\item It can be a collection of hierarchies, i.e. a set of YAML {\em documents}.
\end{enumerate}

There are some redundancies built into the HConfig API, where different ways
are available to achieve the same goal. This is mostly done for convenience,
allowing the user to pick the approach most suitable to their needs.

For instance, while it can be convenient to use iterators in some cases, in
others, it might be easier to access each element directly by {\em index}
(for sequences) or {\em key} (for maps).
