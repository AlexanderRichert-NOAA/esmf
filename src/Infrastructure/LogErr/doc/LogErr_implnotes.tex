% $Id: LogErr_implnotes.tex,v 1.8 2004/06/22 20:07:32 cpboulder Exp $
%
% Earth System Modeling Framework
% Copyright 2002-2003, University Corporation for Atmospheric Research, 
% Massachusetts Institute of Technology, Geophysical Fluid Dynamics 
% Laboratory, University of Michigan, National Centers for Environmental 
% Prediction, Los Alamos National Laboratory, Argonne National Laboratory, 
% NASA Goddard Space Flight Center.
% Licensed under the GPL.
%\subsection{Implementation Notes}
\begin{enumerate}
\item The Log class was implemented in C/C++, but uses the Fortran I/O 
libraries when the class methods are called from Fortran. We forced the C/C++
methods to use the Fortran I/O library by creating utility functions that are
written in Fortran, but callable from Log's C++ methods.  These utility 
functions call the standard Fortran write, open and close functions.  If you
call the Log methods from C/C++ code, you bypass the utility functions and 
all I/O is done with the C I/O libraries.
At initialization a LogArray is created.  This array is a container that holds
multiple Logs.  Within the LogArray, one or more Logs may be created.  Each Log
stores information for a specific Log file.   When working with more than one 
Log file, multiple Logs are required (one Log for each Log file).  For each 
Log, a handle is returned through the {\tt ESMF\_LogOpen()} method so that the
user can manage the various Logs.  The properties of a specific Log are set 
with the {\tt ESMF\_LogSet()} method.  Additionally, buffering occurs within a
Log.  Every time the {\tt ESMF\_LogWrite()} method is called, LogEntry element
is populated with the {\tt ESMF\_LogWrite()} information.  The user can set the
number of LogEntry elements in the Log buffer.  When the buffer is full (i.e.,
when all the LogEntry elements are populated), the buffer will be flushed and
all the contents will be written to file.  If buffering is not needed 
({\tt autoflush=false}), the {\tt ESMF\_LogWrite()} method will immediately 
write to the Log files. 

Buffering allows ESMF to manage output data streams in a desired way.  For 
instance, a solution to clean streaming PE\_ordered output would be to gather
the buffers on root PE when flushing and order and stream them as desired by
the user.  Buffering the output provides a layer where the output can be
managed.  Writing to the buffer is transparent to the user because all the Log
entries are handled automatically by the {\tt ESMF\_LogWrite()} method. 
\end{enumerate}




