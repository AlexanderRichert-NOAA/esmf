% $Id: LogErr_implnotes.tex,v 1.3 2003/10/21 03:56:51 cdeluca Exp $
%
% Earth System Modeling Framework
% Copyright 2002-2003, University Corporation for Atmospheric Research, 
% Massachusetts Institute of Technology, Geophysical Fluid Dynamics 
% Laboratory, University of Michigan, National Centers for Environmental 
% Prediction, Los Alamos National Laboratory, Argonne National Laboratory, 
% NASA Goddard Space Flight Center.
% Licensed under the GPL.

%\subsection{Implementation Notes}

\begin{enumerate}

\item The Log class was implemented in C/C++, but uses the Fortran I/O libraries when
the class methods are called from Fortran. We forced the C/C++ methods 
to use the Fortran I/O library by creating 
utility functions that are written in Fortran, but callable from Log's C++ methods.
These utility functions call the standard Fortran write, open and close functions.
If you call the Log methods from C/C++ code, you bypass the utility functions
and all I/O is done with the C I/O libraries.

\item There are a few restrictions to keep in mind.  First, the error handling and
warning routines, {\tt ESMF\_LogErr}, {\tt ESMF\_LogErrMsg}, {\tt ESMF\_LogWarn},
and {\tt ESMF\_LogWarnErr},
are expanded using a macro that adds the
predefined symbolic constants \_\_LINE\_\_ and \_\_FILE\_\_ to the argument list of the above
error handling routines. Using these constants, we can determine the line number and file that ther
error occurred in.  If
your preprocessor is not working properly, this expansion will not be done properly and
the line and file names will not appear correctly when writing out error and warning
information. We also rely on the C and the Fortran preprocessor
to set a variety of symbolic constants (defined in ESMF\_LogConstants.inc and 
ESMF\_ErrConstants.inc).  Again, if your preprocessor is not working properly, these
constants will not be set properly. 

\end{enumerate}



