%                **** IMPORTANT NOTICE *****
% This LaTeX file has been automatically produced by ProTeX v. 1.1
% Any changes made to this file will likely be lost next time
% this file is regenerated from its source. Send questions 
% to Arlindo da Silva, dasilva@gsfc.nasa.gov
 
\setlength{\parskip}{0pt}
\setlength{\parindent}{0pt}
\setlength{\baselineskip}{11pt}
 
%--------------------- SHORT-HAND MACROS ----------------------
\def\bv{\begin{verbatim}}
\def\ev{\end{verbatim}}
\def\be{\begin{equation}}
\def\ee{\end{equation}}
\def\bea{\begin{eqnarray}}
\def\eea{\end{eqnarray}}
\def\bi{\begin{itemize}}
\def\ei{\end{itemize}}
\def\bn{\begin{enumerate}}
\def\en{\end{enumerate}}
\def\bd{\begin{description}}
\def\ed{\end{description}}
\def\({\left (}
\def\){\right )}
\def\[{\left [}
\def\]{\right ]}
\def\<{\left  \langle}
\def\>{\right \rangle}
\def\cI{{\cal I}}
\def\diag{\mathop{\rm diag}}
\def\tr{\mathop{\rm tr}}
%-------------------------------------------------------------

\markboth{Left}{Source File: ESMC\_LogErr.h,  Date: Fri Mar 28 13:54:35 EST 2003
}

 
%/////////////////////////////////////////////////////////////

  !CLASS: ESMC\_Log - C++ interface to Log
  
  
   The code in this file defines the C++ Log members and declares all class
   data and methods.  All methods, except for the SetVerbosity method, which is
   inlined are defined in the companion file ESMC\_LogErr.C
  
\bigskip{\em USES:}
\begin{verbatim} #include <stdarg.h>
 #include <stdio.h>
 #include <stdlib.h>
 #include <mpi.h>
 #include <mpi++.h>
 #include <time.h>
 #include "ftn.h"
 #include <ctype.h>
 #include "ESMF_LogConstants.inc"
 #include "ESMF_ErrConstants.inc"
 #include "ESMC_UtilityFunctions.h"
 
 
 class ESMC_Log {
   private:
   !PRIVATE MEMBER FUNCIONS
     void ESMC_LogFormName();
     void ESMC_LogPrintHeader(int fortIO);
     void ESMC_LogPrint(int fortIO, int errCode, int line, char file[],
     char dir[], char msg[]=NULL);
     void ESMC_LogGetErrMsg(int errCode, char msg[]) const;
     bool ESMC_LogNameValid(char name[]);\end{verbatim}{\sf PRIVATE TYPES:}
\begin{verbatim} 
     int oneLogErrFile;      // if log data written to one log file,
                             // this is set to
                             // true.  Otherwise set to false.
 			    // ESMC_OpenFile can override
 			    // this value
 
     int standardOut;        // if log data written to standard out, this variable
                             // is set to true. Otherwise set to false.
 			    // ESMC_OpenFile
 			    // can over-ride this value.
 
     int fortIsOpen;         // used to to a file with Fortran I/O libraries 
     
     int unitNumber;         // fortran unit number for log/err file when
                             // ESMC\_LogWrite
                             // is used  Can be overwritten by
 			    // ESMC_OpenFileFortran
 
 
     int numFilePtr;         // index into global array of File pointers
 			    // for C++ I/O.
 
 
     int numFileFort;        // index into global array of unit numbers for 
                             // Fortran I/O
 
     int verbose;            // integer used to control which log
                             // messages written out.
                             // value can be over written by ESMC_Init
 
     int flush;              // if true, all output is flushed
 			    // value can be overwritten by ESMC_INIT
 
     int haltOnWarn;          // Code will stop executing on
                              // encountering a warning
 			    
     int haltOnErr;           // Code will stop executing on
                              // encountering an error
 
 
     char nameLogErrFile[32]; // name of logfile.
                              // Specified by user when LogInit called.  If
 			     // multiple files are written out,
 			     // PE rank is appended to name.
   public:
   !PUBLIC MEMBER FUNCTIONS (see ESMC\_LogErr.C for a description of these methods)
     void ESMC_LogInfo(char* fmt,...);   
     void ESMC_LogInfoFortran(char fmt[],
     char charData[],char strData[][32],int intData[], double floatData[]);
     void ESMC_LogOpenFile(int numLogFile,char name[]);
     void ESMC_LogOpenFileForWrite(int numLogFile, char name[]);
     void ESMC_LogInit(int verbosity, int flush,
 	 int haltOnError, int haltOnWarning);
     int ESMC_LogWrite();
     void ESMC_LogCloseFile();
     void ESMC_LogCloseFileForWrite();
     void ESMC_LogFlush();
     void ESMC_LogNotFlush();
     void ESMC_LogVerbose();
     void ESMC_LogNotVerbose();
     void ESMC_LogHaltOnErr();
     void ESMC_LogNotHaltOnErr();
     void ESMC_LogHaltOnWarn();
     void ESMC_LogNotHaltOnWarn();
     void ESMC_LogWarningMsg(int errCode, int line, char file[],
                      char dir[], char msg[]);
     void ESMC_LogWarning(int errCode, int line, char file[],
                      char dir[]);
     void ESMC_LogWarningFortran(int errCode, int line, char file[],
          char dir[], char msg[]);
     void ESMC_LogErr(int errCode, int line, char file[], char dir[]);
     void ESMC_LogErrMsg(int errCode, int line, char file[],
                      char dir[], char msg[]);
     void ESMC_LogErrFortran(int errCode,int line,char file[],char dir[],char msg[]);
     void ESMC_LogExit();
     ESMC_Log();
 };
  
  --------------------------------------------------------------------------- 
%/////////////////////////////////////////////////////////////
 
\mbox{}\hrulefill\ 
 

  \subsubsection [ESMC\_LogFlush()] {ESMC\_LogFlush() - set the flushSet variable.}


\bigskip{\sf INTERFACE:}
\begin{verbatim} 
 inline void ESMC_Log::ESMC_LogFlush(
 
   !ARGUMENTS
     none
 
    ) 
 \end{verbatim}
{\sf DESCRIPTION:\\ }

 
   Causes output to be flushed. 
%/////////////////////////////////////////////////////////////
 
\mbox{}\hrulefill\ 
 

                     \subsubsection [ESMC\_LogNotFlush()] {ESMC\_LogNotFlush() - output not flushed}


\bigskip{\sf INTERFACE:}
\begin{verbatim} 	     
 inline void ESMC_Log::ESMC_LogNotFlush(
 				  
   !ARGUMENTS        
     none   
 
  )    
 	     \end{verbatim}
{\sf DESCRIPTION:\\ }


   Causes output not to be flushed. 
%/////////////////////////////////////////////////////////////
 
\mbox{}\hrulefill\ 
 

  \subsubsection [ESMC\_LogVerbose] {ESMC\_LogVerbose - output verbose }


  
\bigskip{\sf INTERFACE:}
\begin{verbatim} 
 inline void ESMC_Log::ESMC_LogVerbose(
 \end{verbatim}{\em ARGUMENTS:}
\begin{verbatim}     none
   )
 \end{verbatim}
{\sf DESCRIPTION:\\ }


   If theVerbosity is set to ESMF\_LOG\_TRUE, messages are printed out. 
    
%/////////////////////////////////////////////////////////////
 
\mbox{}\hrulefill\ 
 

  \subsubsection [ESMC\_LogNotVerbose] {ESMC\_LogNotVerbose - output not verbose }


  
\bigskip{\sf INTERFACE:}
\begin{verbatim} 
 inline void ESMC_Log::ESMC_LogNotVerbose(
 \end{verbatim}{\em ARGUMENTS:}
\begin{verbatim}     none
   )
 \end{verbatim}
{\sf DESCRIPTION:\\ }


   If theVerbosity is set to ESMC\_LOG\_FALSE, no messages are printed out. 
    
%/////////////////////////////////////////////////////////////
 
\mbox{}\hrulefill\ 
 

  \subsubsection [ESMC\_LogHaltOnErr] {ESMC\_LogHaltOnErr - code will stop on encountering an error }


  
\bigskip{\sf INTERFACE:}
\begin{verbatim} 
 inline void ESMC_Log::ESMC_LogHaltOnErr(
 
   !ARGUMENTS
     none
   )
 \end{verbatim}
{\sf DESCRIPTION:\\ }


   If haltOnErr is set to ESMC\_LOG\_TRUE, code will stop executing when
   encountering an error.
    
%/////////////////////////////////////////////////////////////
 
\mbox{}\hrulefill\ 
 

  \subsubsection [ESMC\_LogNotHaltOnErr] {ESMC\_LogNotHaltOnErr - code will not stop on encountering}


   an error  
  
\bigskip{\sf INTERFACE:}
\begin{verbatim} 
 inline void ESMC_Log::ESMC_LogNotHaltOnErr(
 \end{verbatim}{\em ARGUMENTS:}
\begin{verbatim}     none
   )
 \end{verbatim}
{\sf DESCRIPTION:\\ }


   If haltOnErr is set to ESMC\_LOG\_FALSE, code will not stop executing when
   encountering an error.
    
%/////////////////////////////////////////////////////////////
 
\mbox{}\hrulefill\ 
 

  \subsubsection [ESMC\_LogHaltOnWarn] {ESMC\_LogHaltOnWarn - code will stop on encountering}


   a warning
  
\bigskip{\sf INTERFACE:}
\begin{verbatim} 
 inline void ESMC_Log::ESMC_LogHaltOnWarn(
 
   !ARGUMENTS
     none
   )
 \end{verbatim}
{\sf DESCRIPTION:\\ }


   If haltOnWarn is set to ESMC\_LOG\_TRUE, code will stop executing when
   encountering an error.
    
%/////////////////////////////////////////////////////////////
 
\mbox{}\hrulefill\ 
 

  \subsubsection [ESMC\_LogNotHaltOnWarn] {ESMC\_LogNotHaltOnWarn - code will not stop on encountering}


   a warning
  
\bigskip{\sf INTERFACE:}
\begin{verbatim} 
 inline void ESMC_Log::ESMC_LogNotHaltOnWarn(
 
   !ARGUMENTS
     none
   )
 \end{verbatim}
{\sf DESCRIPTION:\\ }


   If haltOnWarn is set to ESMC\_LOG\_FALSE, code will not stop executing when
   encountering an error.
   
%...............................................................
