%                **** IMPORTANT NOTICE *****
% This LaTeX file has been automatically produced by ProTeX v. 1.1
% Any changes made to this file will likely be lost next time
% this file is regenerated from its source. Send questions 
% to Arlindo da Silva, dasilva@gsfc.nasa.gov
 
\setlength{\parskip}{0pt}
\setlength{\parindent}{0pt}
\setlength{\baselineskip}{11pt}
 
%--------------------- SHORT-HAND MACROS ----------------------
\def\bv{\begin{verbatim}}
\def\ev{\end{verbatim}}
\def\be{\begin{equation}}
\def\ee{\end{equation}}
\def\bea{\begin{eqnarray}}
\def\eea{\end{eqnarray}}
\def\bi{\begin{itemize}}
\def\ei{\end{itemize}}
\def\bn{\begin{enumerate}}
\def\en{\end{enumerate}}
\def\bd{\begin{description}}
\def\ed{\end{description}}
\def\({\left (}
\def\){\right )}
\def\[{\left [}
\def\]{\right ]}
\def\<{\left  \langle}
\def\>{\right \rangle}
\def\cI{{\cal I}}
\def\diag{\mathop{\rm diag}}
\def\tr{\mathop{\rm tr}}
%-------------------------------------------------------------

\markboth{Left}{Source File: ESMC\_LogErr.C,  Date: Thu Aug  7 13:36:32 EDT 2003
}

 
%/////////////////////////////////////////////////////////////
\subsubsection [ESMC\_LogOpenFile] {ESMC\_LogOpenFile - opens a Log object}


  
\bigskip{\sf INTERFACE:}
\begin{verbatim} 
 void ESMC_Log::ESMC_LogOpenFile(\end{verbatim}{\em RETURN VALUE:}
\begin{verbatim}     none\end{verbatim}{\em ARGUMENTS:}
\begin{verbatim} 
      int numLogFile, //number of log files written (input). Set
 		     // to either ESMF_SINGLE_LOG_FILE or
 		     // ESMF_MULT_LOG_FILE .
 
      char name[]     //string to form name of log file (input)
 
    )
   !DESCRIPTION
   {\tt ESMC\_LogErrOpenFile} takes two
   arguments.  The first should be set to ESMF\_SINGLE\_LOG\_FILE or
   ESMF\_MULT\_LOG\_FILE. These are symbolic constants, defined in
   ESMF\_LogConstants.h, set whether one file should be written for all 
   processes (ESMF_SINGLE\_LOG\_FILE), or whether one file per process should
   be written (ESMF\_MULT\_LOG\_FILE).
   The second argument is a string and is used to form the name of the
   logfile.
   This routine is called from native C or C++ code. C I/O libraries are used.\end{verbatim}
 
%/////////////////////////////////////////////////////////////
 
\mbox{}\hrulefill\ 
 
\subsubsection [ESMC\_LogOpenFortFile] {ESMC\_LogOpenFortFile - opens a Log object}


  
\bigskip{\sf INTERFACE:}
\begin{verbatim} 
 void ESMC_Log::ESMC_LogOpenFortFile(\end{verbatim}{\em RETURN VALUE:}
\begin{verbatim}     none\end{verbatim}{\em ARGUMENTS:}
\begin{verbatim} 
      int numLogFile, //number of log files written (input). Set
 		     // to either ESMF_SINGLE_LOG_FILE or
 		     // ESMF_MULT_LOG_FILE .
 
      char name[]     //string to form name of log file (input)
 
    )
   !DESCRIPTION
   {\tt ESMC\_LogErrOpenFile} takes two
   arguments.  The first should be set to ESMF\_SINGLE\_LOG\_FILE or
   ESMF\_MULT\_LOG\_FILE. These are symbolic constants, defined in
   ESMF\_LogConstants.h, set whether one file should be written for all 
   processes (ESMF_SINGLE\_LOG\_FILE), or whether one file per process should
   be written (ESMF\_MULT\_LOG\_FILE).
   The second argument is a string and is used to form the name of the
   logfile.
   This routine is called from native Fortran code. Fortran I/O libraries are
   used.\end{verbatim}
 
%/////////////////////////////////////////////////////////////
 
\mbox{}\hrulefill\ 
 
\subsubsection [ESMC\_LogNameValid] {ESMC\_LogNameValid - checks to see if a name has been used before}


  
\bigskip{\sf INTERFACE:}
\begin{verbatim} bool ESMC_Log::ESMC_LogNameValid(
   !RETURN VALUE
      none
   !ARGUMENTS
       char name[],          // name of file
       int FortIO            //are we doing FortIO?
       )
   !DESCRIPTION
      Checks to see if a file of the name name has been opened by {\tt ESMC\_Log}.
      If it has the function returns a false value.  Note: this function
      use a global array that all {\tt ESMC\_Log} objects have access to.
   EOP
 {
   int i;
   if (FortIO == ESMF_TF_FALSE) {
     for(i=0; i< numCFiles; i++)
      if (strcmp(name,listOfCFileNames[i])  == 0) 
 	return false;
     strcpy(listOfCFileNames[i-1],name);
     return true;
   } else {
     for(i=0; i< numFortFiles;i++)
       if (strcmp(name,listOfFortFileNames[i])  == 0)
           return false;
     strcpy(listOfFortFileNames[i-1],name);
     return true;
   }
 }
  ----------------------------------------------------------------------------- 
%/////////////////////////////////////////////////////////////
 
\mbox{}\hrulefill\ 
 
\subsubsection [ESMC\_LogInfo] {ESMC\_LogInfo - print contents of a Log}


  
\bigskip{\sf INTERFACE:}
\begin{verbatim} 
 void ESMC_Log::ESMC_LogInfo(
 \end{verbatim}{\em RETURN VALUE:}
\begin{verbatim}      none\end{verbatim}{\em ARGUMENTS:}
\begin{verbatim} 
      char* fmt, // c-style character format; subsequent arguments
      ...
     )\end{verbatim}
{\sf DESCRIPTION:\\ }


   {\tt ESMC\_Log\_Info} works similar to C's printf statement.
   The first argument is a character string that can include text to be
   written as well as a description of the number and kind of characters
   to be printed out.
   The format of this string follows the C format description convention,
   though not every feature is supported.  The current conversion specifiers
   are supported: d (signed decimal integer), f (double values),
   and s (string).
  
   Any number of data items may be passed to {\tt ESMC\_Log\_Print}.
   
   The items are printed on a single line.  Widths, precision, and
   escape sequences are not supported.  If you specify these, the code
   ignores them.
   
%/////////////////////////////////////////////////////////////
 
\mbox{}\hrulefill\ 
 
\subsubsection [ESMC\_LogInfoFortran] {ESMC\_LogInfoFortran - print contents of a Log from Fortran}


  
\bigskip{\sf INTERFACE:}
\begin{verbatim} 
 void ESMC_Log::ESMC_LogInfoFortran(
 \end{verbatim}{\em RETURN VALUE:}
\begin{verbatim}      none\end{verbatim}{\em ARGUMENTS:}
\begin{verbatim}      
      char fmt[],           // array for c-style character format (input)
 
      char charData[],      // holds character data
      char strData[][32],   // two dimensional array for string data;
                            // strDat[i][j] is the jth
 			   // character of string i. (input)
 
      int intData[],        // array storing integer data to be printed (input)
 
      double floatData[]    // array storing double data to be printed (input) 
 
     )\end{verbatim}
{\sf DESCRIPTION:\\ }


   {\tt ESMC\_LogInfoFortran} is the version of {\tt ESMC\_LogInfo} used for the Fortran
   interface.  Instead of printing the data from the stack as {\tt ESMC\_LogInfo} 
   does, {\tt ESMC\_LogInfoFortran} prints the data stored in the "Data" arrays. 
   The routine is called from {\tt ESMF\_LogPrint} which is  defined in
   ESMF\_LogInterface.C.  {\tt ESMF\_LogPrint} is callable from the user's
   from Fortran code.  It is this routine that takes the data
   from the stack and stores it in the "Data" arrays.
  
   
%/////////////////////////////////////////////////////////////
 
\mbox{}\hrulefill\ 
 
\subsubsection [ESMC\_LogPrintHeader] {ESMC\_LogPrintHeader - prints header data}


  
\bigskip{\sf INTERFACE:}
\begin{verbatim}    void ESMC_Log::ESMC_LogPrintHeader(\end{verbatim}{\em RETURN VALUE:}
\begin{verbatim}      none \end{verbatim}{\em ARGUMENTS:}
\begin{verbatim}      int fortIO       //if set to ESMF_TF_TRUE, use fortran io libraries
 
    )
   !Description:
   This is a private method, used by LogErr's various
   print methods to print header
   data to the log file.  This data consists of a time stamp, an ascii string 
   holding the day, month, time, and year. If the code is running in parallel
   with MPI, the header also consists of the PE number.\end{verbatim}
 
%/////////////////////////////////////////////////////////////
 
\mbox{}\hrulefill\ 
 
\subsubsection [ESMC\_LogGet] {ESMC\_LogGet - get value of verbose, flush, haltOnErr, and/or haltOnWarn}


  
\bigskip{\sf INTERFACE:}
\begin{verbatim}    void ESMC_LogGet(\end{verbatim}{\em RETURN VALUE:}
\begin{verbatim}    none\end{verbatim}{\em ARGUMENTS:}
\begin{verbatim}    An arbitrary number of "option, value" pairs
 
    char* option, ESMC_Logical value,...)
   \end{verbatim}
{\sf DESCRIPTION:\\ }


    This method returns the value of the argument option in the variable value.
    Option may be set tot he string verbose, flush,haltOnErr or haltOnWarn 
%/////////////////////////////////////////////////////////////
 
\mbox{}\hrulefill\ 
 
\subsubsection [ESMC\_LogSet] {ESMC\_LogSet - sets the value of verbose, flush, haltOnWarn and/or haltOnErr}


  
\bigskip{\sf INTERFACE:}
\begin{verbatim}    void ESMC_LogSet(
 
   !RETURN VALUE 
     none
   !ARGUMENTS
    An arbitrary number of "option, value" pairs
     char* option, ESMC_Logical value,...) \end{verbatim}
{\sf DESCRIPTION:\\ }


   This method returns the value of the argument option in the variable value.
   Option may be set tot he string verbose, flush,haltOnErr or haltOnWarn 
%/////////////////////////////////////////////////////////////
 
\mbox{}\hrulefill\ 
 
\subsubsection [ESMC\_GetFileHandle] {ESMC\_GetFileHandle - used in conjunction with fprintf to write}


   data to the log file.
  
\bigskip{\sf INTERFACE:}
\begin{verbatim}    FILE* ESMC_Log::ESMC_GetFileHandle(\end{verbatim}{\em RETURN VALUE:}
\begin{verbatim}    none\end{verbatim}{\em ARGUMENTS:}
\begin{verbatim}    none
    )\end{verbatim}
{\sf DESCRIPTION:\\ }


   This method is used in conjunction with fprintf 
   to write data to the log file, eg. fprintf(aLog.ESMC\_GetUnit(),"Hi")
   The routine writes header data, and
   then returns a point to a file structure to fprintf
   
%/////////////////////////////////////////////////////////////
 
\mbox{}\hrulefill\ 
 
\subsubsection [ESMC\_LogGetUnit] {ESMC\_LogGetUnit - used in conjunction with standard Fortran write}


   to write data to log file. 
  
\bigskip{\sf INTERFACE:}
\begin{verbatim} int ESMC_Log::ESMC_LogGetUnit(\end{verbatim}{\em RETURN VALUE:}
\begin{verbatim}      unit number \end{verbatim}{\em ARGUMENTS:}
\begin{verbatim}       none.
 
    )\end{verbatim}
{\sf DESCRIPTION:\\ }


   This is method is
   used with the standard Fortran write function, eg.
   write(ESMC\_LogGetUnitNumber(aLog),*) 'Hi '. 
   The routine writes header data, and
   then returns a unit number to the Fortran write function.
   
%/////////////////////////////////////////////////////////////
 
\mbox{}\hrulefill\ 
 
\subsubsection [ESMC\_LogFormName] {ESMC\_LogFormName - private method that forms}


    the name of the LogErr file to be written when multiple LogErr
    files are to be written.
  
\bigskip{\sf INTERFACE:}
\begin{verbatim} 
 void ESMC_Log::ESMC_LogFormName(
 \end{verbatim}{\em RETURN VALUE:}
\begin{verbatim}      none\end{verbatim}{\em ARGUMENTS:}
\begin{verbatim}       none.
 
    )\end{verbatim}
{\sf DESCRIPTION:\\ }


   This routine forms the names the LogErr files by appending the
   PE number to base name specified in the Open method.  This method
   is used only when running in parallel on multiple PEs.
   
%/////////////////////////////////////////////////////////////
 
\mbox{}\hrulefill\ 
 
\subsubsection [ESMC\_LogCloseFortFile] {ESMC\_LogCloseFortFile - closes log file. }


  
\bigskip{\sf INTERFACE:}
\begin{verbatim} void ESMC_Log::ESMC_LogCloseFortFile(
   ! RETURN VALUE:
      none\end{verbatim}{\em ARGUMENTS:}
\begin{verbatim}     none
 
    )
   ! DESCRIPTION:
   This routine simply closes the log file(s).  It also removes
   file from the global file array. The routine is called from native
   Fortran code (File is closed with Fortran I/O libraries.)\end{verbatim}
 
%/////////////////////////////////////////////////////////////
 
\mbox{}\hrulefill\ 
 
\subsubsection [ESMC\_LogCloseFile] {ESMC\_LogCloseFile - closes log file. }


  
\bigskip{\sf INTERFACE:}
\begin{verbatim} void ESMC_Log::ESMC_LogCloseFile(
   ! RETURN VALUE:
      none\end{verbatim}{\em ARGUMENTS:}
\begin{verbatim}     none
 
    )
   ! DESCRIPTION:
   This routine simply closes the log file(s).  It also removes
   file from the global file array. The routine is called from native
   C/C++ code (File is closed with C I/O libraries.)\end{verbatim}
 
%/////////////////////////////////////////////////////////////
 
\mbox{}\hrulefill\ 
 
\subsubsection [ESMC\_Log] {ESMC\_Log - native C++ constructor for log class}


  
\bigskip{\sf INTERFACE:}
\begin{verbatim} 
 ESMC_Log::ESMC_Log(
 \end{verbatim}{\em RETURN VALUE:}
\begin{verbatim}    none\end{verbatim}{\em ARGUMENTS:}
\begin{verbatim}    none
 
   )
 \end{verbatim}
{\sf DESCRIPTION:\\ }


     Native C++ constructor 
%/////////////////////////////////////////////////////////////
 
\mbox{}\hrulefill\ 
 
\subsubsection [ESMC\_LogErr] {ESMC\_LogErr - write error message to log file}


  
\bigskip{\sf INTERFACE:}
\begin{verbatim} 
 void ESMC_Log::ESMC_LogErr_(
 \end{verbatim}{\em RETURN VALUE:}
\begin{verbatim}    none\end{verbatim}{\em ARGUMENTS:}
\begin{verbatim} 
    int errCode,     // Error code
    int line,        // Line number
    char file[],     // Filename
    char dir[]      // Directory
   )\end{verbatim}
{\sf DESCRIPTION:\\ }


   Prints error code, corresponding message, line number, file, directory
   that error occurred at.  
%/////////////////////////////////////////////////////////////
 
\mbox{}\hrulefill\ 
 
\subsubsection [ESMC\_LogErrMsg] {ESMC\_LogErrMsg - write error message to log file}


  
\bigskip{\sf INTERFACE:}
\begin{verbatim} 
 void ESMC_Log::ESMC_LogErrMsg_(
 \end{verbatim}{\em RETURN VALUE:}
\begin{verbatim}    none\end{verbatim}{\em ARGUMENTS:}
\begin{verbatim} 
    int errCode,     // Error code
    int line,        // Line number
    char file[],     // Filename
    char dir[],     // Directory
    char msg[]
   )\end{verbatim}
{\sf DESCRIPTION:\\ }


   Prints error code, corresponding message, line number, file, directory
   that error occurred at.  Can also print a message. 
%/////////////////////////////////////////////////////////////
 
\mbox{}\hrulefill\ 
 
\subsubsection [ESMC\_LogWarn] {ESMC\_LogWarn -- Print warning message}


  
\bigskip{\sf INTERFACE:}
\begin{verbatim}       void ESMC_Log::ESMC_LogWarn_(\end{verbatim}{\em ARGUMENTS:}
\begin{verbatim}       int errCode,     // Error code
       int line,        // Line number
       char file[],     // Filename
       char dir[]      // Directory
     )\end{verbatim}
{\sf DESCRIPTION:\\ }


    Same as {\tt ESMC\_LogErr}, except execution is not stopped after
    printing message, except when haltOnWarn set to true
   
%/////////////////////////////////////////////////////////////
 
\mbox{}\hrulefill\ 
 
\subsubsection [ESMC\_LogWarnMsg] {ESMC\_LogWarnMsg -- Print warning message}


  
\bigskip{\sf INTERFACE:}
\begin{verbatim}       void ESMC_Log::ESMC_LogWarnMsg_(\end{verbatim}{\em ARGUMENTS:}
\begin{verbatim}       int errCode,     // Error code
       int line,        // Line number
       char file[],     // Filename
       char dir[],      // Directory
       char msg[]
     )\end{verbatim}
{\sf DESCRIPTION:\\ }


    Same as {\tt ESMC\_LogErr}, except execution is not stopped after
    printing message, except when haltOnWarn set to true
   
%/////////////////////////////////////////////////////////////
 
\mbox{}\hrulefill\ 
 

  \subsubsection [ESMC\_LogExit()] {ESMC\_LogExit() - private routine uses by Log to stop program}


  
\bigskip{\sf INTERFACE:}
\begin{verbatim} void ESMC_Log::ESMC_LogExit(\end{verbatim}{\em RETURN VALUE:}
\begin{verbatim}    none\end{verbatim}{\em ARGUMENTS:}
\begin{verbatim}    none
   )\end{verbatim}
{\sf DESCRIPTION:\\ }


   Used by {\tt ESMC\_Log} to exit program.
   
%/////////////////////////////////////////////////////////////
 
\mbox{}\hrulefill\ 
 

  \subsubsection [ESMC\_LogErrFortran] {ESMC\_LogErrFortran - called by fortran wrapper to write error msg}


  
\bigskip{\sf INTERFACE:}
\begin{verbatim} 
 void ESMC_Log::ESMC_LogErrFortran(
 \end{verbatim}{\em ARGUMENTS:}
\begin{verbatim} 
     int errCode,         //error code
     int line,            //line number
     char file[],         //file error occurred in
     char dir[],          //directory error occurred in
     char msg[]           //msg
   )\end{verbatim}
{\sf DESCRIPTION:\\ }


  Similar to {\tt ESMC\_LogErr}, except called by the fortran wrapper
  esmf\_logerr which is defined in
  ESMC\_Interface.C.  The major difference between this routine
  and {\tt ESMC\_LogErr} is that this
  routine prints the printf style data from the Data arrays not the stack. 
%/////////////////////////////////////////////////////////////
 
\mbox{}\hrulefill\ 
 
\subsubsection [ESMC\_LogWarnFortran] {ESMC\_LogWarnFortran - called by fortran wrapper to write warnings}


  
\bigskip{\sf INTERFACE:}
\begin{verbatim} 
 void ESMC_Log::ESMC_LogWarnFortran(\end{verbatim}{\em RETURN VALUE:}
\begin{verbatim}    none\end{verbatim}{\em ARGUMENTS:}
\begin{verbatim} 
      int errCode,
      int line,
      char file[],
      char dir[],
      char msg[]
 
 )\end{verbatim}
{\sf DESCRIPTION:\\ }


   Similar to {\tt ESMC\_LogWarn}, except called by the fortran
   wrapper esmf\_logerr which is
   defined in ESMC\_Interface.C 
%/////////////////////////////////////////////////////////////
 
\mbox{}\hrulefill\ 
 

  \subsubsection [ESMC\_LogPrint] {ESMC\_LogPrint - prints to the log file}


   
\bigskip{\sf INTERFACE:}
\begin{verbatim} 
 void ESMC_Log:: ESMC_LogPrint(
 \end{verbatim}{\em ARGUMENTS:}
\begin{verbatim} 
    int fortIO,          //if set to ESMF_TF_TRUE use Fortran IO libraries
    int errCode,
    int line,            // see LogErr for a definition
    char file[],         // of these variables
    char dir[],           
    char msg[]            // optional msg
 
   )
 \end{verbatim}
{\sf DESCRIPTION:\\ }

This is a private routine, used by many methods of 
   {\tt ESMC\_Log} to print data to the log file. If used from Fortran, then
   Fortran I/O libraries are used.  Otherwise C I/O libraries are used. 
%/////////////////////////////////////////////////////////////
 
\mbox{}\hrulefill\ 
 
\subsubsection [ESMC\_LogGetErrMsg] {ESMC\_LogGetErrMsg -- Return error message for given error code.}


  
\bigskip{\sf INTERFACE:}
\begin{verbatim}       void ESMC_Log::ESMC_LogGetErrMsg(\end{verbatim}{\em RETURN VALUE:}
\begin{verbatim}    none\end{verbatim}{\em ARGUMENTS:}
\begin{verbatim}      int errCode,
      char msg[]
      ) const\end{verbatim}
{\sf DESCRIPTION:\\ }


   {\tt ESMC\_GetErrMsg} returns a string corresponding to the error code
  
%...............................................................
