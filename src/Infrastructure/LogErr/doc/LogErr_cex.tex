% $Id: LogErr_cex.tex,v 1.2 2003/08/07 17:54:18 shep_smith Exp $
%
% Earth System Modeling Framework
% Copyright 2002-2003, University Corporation for Atmospheric Research, 
% Massachusetts Institute of Technology, Geophysical Fluid Dynamics 
% Laboratory, University of Michigan, National Centers for Environmental 
% Prediction, Los Alamos National Laboratory, Argonne National Laboratory, 
% NASA Goddard Space Flight Center.
% Licensed under the GPL.

%\subsection{C++ Use and Examples}

%<Detailed examples of F90 usage of the class.>
\subsubsection{Example 1. Simple Example.}

In this example we use the default ESMF\_Log\_World object and illustrate
the ESMF\_LogGetUnit()to write to the log file aLog.txt.

{\tt
\begin{verbatim}
#include "ESMC_LogErr.inc"
ESMC_Log_World.ESMC_LogOpen(ESMF_Log_World,ESMF_SINGLE_LOG_FILE,"aLog.txt")
fprintf(ESMC_Log_Wold.ESMC_GetFileHandle(),"This is a test");
ESMC_Log_World.ESMC_LogClose();
\end{verbatim}
\tt}

\subsubsection{Example 2. Simple Error Handling.}

Here we illustrate simple error handling.  We define an error handler - anErr -
and error output goes to the
file anErr.txt.  There are four options that can be set with ESMF\_LogSet()
and we set them all here: verbose (whether an
error message should be written for anErr), flush (whether output should be
flushed), haltOnErr and haltOnWarn (whether the code should halt on
an error or warning).  While the values set here are the default
values, so they need not be set with ESMC\_Set(), the example illustrates how
to use ESMC\_SET().  Note that verbose, flush, haltOnErr, and haltOnWarn are symbolic
constants defined in ESMC\_LogErr.inc.
{\tt
\begin{verbatim}
#include "ESMC_LogErr.inc"
ESMC_Log anErr;
int returnCode;
anErr.ESMC_LogSet(verbose,ESMF_TF_TRUE,flush,ESMF_TF_TRUE,haltOnErr,
                  ESMF_TF_TRUE,haltOnWarn,ESMF_TF_FALSE);
anErr.ESMC_LogOpenFile(ESMF_SINGLE_LOG_FILE,"anErr.txt");
returnCode=foo();
if (returnCode == ESMF_FATAL) anErr.ESMC_LogErr(ESMF_FATAL); 
anErr.ESMC_LogCloseFile();
\end{verbatim}
\tt}

\subsubsection{Example 3. More error handling. }

In this example, we illustrate more complex error handling.  We use
ESMF\_LogGet() to get the value of verbose and if it's set to ESMF\_TF\_TRUE
then turn off verbose midway through the code.  This allows you to put in lots
of debugging code, and then if you want to turn them off, you can just call
an ESMF\_LogSet() rather than commenting out all the error handling calls.
{\tt
\begin{verbatim}
#include "ESMC_LogErr.inc"

\begin{verbatim}
#include "ESMC_LogErr.inc" 
ESMC_Log anErr;
ESMC_Logical myVerbosity;
char myMsg[32]="This will be written out.";
int returnCode; 
anErr.ESMC_LogSet(verbose,ESMF_TF_TRUE,flush,ESMF_TF_FALSE,
                 haltOnErr,ESMF_TF_TRUE, haltOnWarn,ESMF_TF_FALSE)
anErr.ESMC_LogOpenFile(ESMF_SINGLE_LOG_FILE,"anErr.txt")
returnCode=foo();
if (returnCode == ESMF_FATAL) anErr.ESMC_LogErrMsg(anErr, returnCode,myMsg) 
anErr.ESMF_LogGet(verbose,myVerbosity)
if (myVerbosity==ESMF_TF_TRUE) anErr.ESMF_LogSet(verbose=ESMF_TF_FALSE)
myMsg="This will not be written out."
returnCode=foo()
if (returnCode == ESMF_FATAL) anErr.ESMF_LogErrMsg(anErr, returnCode,myMsg)
anErr.ESMF_LogCloseFile()
\end{verbatim}
\tt}

