% $Id: LogErr_design.tex,v 1.2 2003/08/07 17:54:18 shep_smith Exp $
%
% Earth System Modeling Framework
% Copyright 2002-2003, University Corporation for Atmospheric Research, 
% Massachusetts Institute of Technology, Geophysical Fluid Dynamics 
% Laboratory, University of Michigan, National Centers for Environmental 
% Prediction, Los Alamos National Laboratory, Argonne National Laboratory, 
% NASA Goddard Space Flight Center.
% Licensed under the GPL.

%\subsection{Design/Implementation}

Although the class contains a large number of methods, the most Fortran/C++ important methods
are ESMF\_LogSet()/ESMC\_LogSet() and ESMF\_LogGet()/ESMC\_LogGet(),
for setting and getting data; ESMF\_LogOpenFile()/ESMC\_LogOpenFile() and ESMF\_CloseFile()/ESMC\_LogCloseFile(),
for opening and closing output files; and ESMF\_LogErr()/ESMC\_LogErr(), and ESMF\_LogWarn()/ESMC\_LogWarn(), for
writing error and warning information. In addition, there are
a variety of other methods that allow the user to control other behavior such whether
or not to stop execution on encountering errors or warnings, whether output
should be flushed from buffers to files automatically, or whether output should be verbose
or not. 

The Log class was implemented in C/C++, but uses the Fortran I/O libraries when
the class methods are called from Fortran. We forced the C/C++ methods 
to use the Fortran I/O library by creating 
utility functions that are written in Fortran, but callable from Log's C++ methods.
These utility functions call the standard Fortran write, open and close functions.
If you call the Log methods from C/C++ code, you bypass the utility functions
and all I/O is done with the C I/O libraries.




