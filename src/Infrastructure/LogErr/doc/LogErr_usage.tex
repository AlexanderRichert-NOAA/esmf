% $Id: LogErr_usage.tex,v 1.2 2004/03/17 22:42:30 cdeluca Exp $

%\subsection{Use and Examples}

A typical sequence of tasks associated with using the Log utility would begin
with the user calling {\tt ESMF\_LogOpen()} to open a Log file and retrieve
an active Log object.  The user can then set or get options on how the Log
should be used with the {\tt ESMF\_LogSet()} and {\tt ESMF\_LogGet()} methods.
Depending on how the options are set, {\tt ESMF\_LogWrite()} either writes
user messages direstly to a Log file or writes to a buffer that can be flushed
when full or by using the {\tt ESMF\_LogFlush()} method.  When done writing
messages, the Log is closed by calling {\tt ESMF\_LogClose}.  

Some of the assumptions implicit in using the Log utility are:

\begin{enumerate}

\item Prior to closing, ESMF_LogFlush is always executed.
\item Prior to halting, ESMF_LogFlush is always executed.
\item If the number of elements in an ESMF_Log exceeds max_elements, a message 
   is written to file and the buffer flushes automatically.

\end{enumerate}
