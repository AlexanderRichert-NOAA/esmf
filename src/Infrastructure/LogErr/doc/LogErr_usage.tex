% $Id: LogErr_usage.tex,v 1.27 2011/01/19 02:13:18 svasquez Exp $

%\subsection{Use and Examples}

By default {\tt ESMF\_Initialize()} opens a default Log in 
{\tt ESMF\_LOG\_MULTI} mode. ESMF handles the initialization and finalization
of the default Log so the user can immediately start using it. If additional Log
objects are desired, they must be explicitly created or opened using
{\tt ESMF\_LogOpen()}.

{\tt ESMF\_LogOpen()} requires a Log object and filename argument. Additionally,
the user can specify single or multi Logs by setting the {\tt logtype} property
to {\tt ESMF\_LOG\_SINGLE} or {\tt ESMF\_LOG\_MULTI}.
This is useful as the PET numbers are automatically added to the Log entries.
A single Log will put all entries, regardless of PET number, into a single
log while a multi Log will create multiple Logs with the PET number prepended
to the filename and all entries will be written to their corresponding Log 
by their PET number.
 
By default, the Log file is not truncated at the start of a new run; it just
gets appended each time.  Future functionality may include an option to
either truncate or append to the Log file. 

In all cases where a Log is opened, a Fortran unit number is assigned to a specific
Log.  A Log is assigned an unused unit number using the algorithm described in
the {\tt ESMF\_IOUnitGet()} method.

The user can then set or get options on how the Log should be used 
with the {\tt ESMF\_LogSet()} and {\tt ESMF\_LogGet()} methods.  These are 
partially implemented at this time. 

Depending on how the options are set, {\tt ESMF\_LogWrite()} either writes user
messages directly to a Log file or writes to a buffer that can be flushed when 
full or by using the {\tt ESMF\_LogFlush()} method.  The default is to flush 
after every ten entries because {\tt maxElements} is initialized to ten 
(which means the buffer reaches its full state after every ten writes and then
flushes).

A message filtering option may be set with {\tt ESMF\_LogSet()} so
that only selected message types are actually written to the log.  One key
use of this feature is to allow placing informational log write requests
into the code for debugging or tracing.  Then, when the informational entries
are not needed, the messages at that level may be turned off --- leaving only
warning and error messages in the logs. 

For every {\tt ESMF\_LogWrite()}, a time and date stamp is prepended to the
Log entry.  The time is given in microsecond precision.  The user can call 
other methods to write to the Log.  In every case, all methods eventually make 
a call implicitly to {\tt ESMF\_LogWrite()} even though the user may never 
explicitly call it.

When calling {\tt ESMF\_LogWrite()}, the user can supply an optional line,
file and method.  These arguments can be passed in explicitly or with the help
of cpp macros.  In the latter case, a define for an {\tt ESMF\_FILENAME} must 
be placed at the beginning of a file and a define for {\tt ESMF\_METHOD} must
be placed at the beginning of each method.  The user can then use the
{\tt ESMF\_CONTEXT} cpp macro in place of line, file and method to insert the 
parameters into the method.  The user does not have to specify line number as
it is a value supplied by cpp.

An example of Log output is given below running with {\tt logtype} 
property set to {\tt ESMF\_LOG\_MULTI} (default) using the default Log:

(Log file {\tt PET0.ESMF\_LogFile})
\begin{verbatim}
20041105 163418.472210 INFO      PET0     Running with ESMF Version 2.2.1   
\end{verbatim}

(Log file {\tt PET1.ESMF\_LogFile})
\begin{verbatim}
20041105 163419.186153 ERROR     PET1     ESMF_Field.F90             812  
ESMF_FieldGet No Grid or Bad Grid attached to Field
\end{verbatim}

The first entry shows date and time stamp.  The time is given in microsecond 
precision.  The next item shown is the type of message (INFO in this case).  
Next, the PET number is added.  Lastly, the content is written.

\begin{sloppypar}
The second entry shows something slightly different.  In this case, we have
an ERROR.  The method name (ESMF\_Field.F90) is automatically provided from 
the cpp macros as well as the line number (812).  Then the content of the 
message is written.
\end{sloppypar}
 
When done writing messages, the default Log is closed by calling 
{\tt ESMF\_LogFinalize()}  or {\tt ESMF\_LogClose()} for user created Logs.  
Both methods will release the assigned unit number.
