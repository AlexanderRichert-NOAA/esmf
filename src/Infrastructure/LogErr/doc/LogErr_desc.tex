% $Id: LogErr_desc.tex,v 1.12 2004/10/14 20:39:35 cpboulder Exp $

% Earth System Modeling Framework
% Copyright 2002-2003, University Corporation for Atmospheric Research, 
% Massachusetts Institute of Technology, Geophysical Fluid Dynamics 
% Laboratory, University of Michigan, National Centers for Environmental 
% Prediction, Los Alamos National Laboratory, Argonne National Laboratory, 
% NASA Goddard Space Flight Center.
% Licensed under the GPL.


%\subsection{Description}

The Log class consists of a variety of methods for writing error, warning, and
informational messages to files.  A default Log is created at ESMF
initialization.  Other Logs can be created later in the code by the user.  A set
of standard return codes and associated messages are provided for error 
handling.  

Log provides capabilities to write to a file immediately or store entries in a 
buffer for writing either when the buffer is full, or when the user calls an 
{\tt ESMF\_LogFlush()} command.  Currently the default is set so the Log flushes
after every write command but can easily set to higher values using the 
{\tt ESMF\_LogSet} command and setting the {\tt maxElements} property to a value
greater than one.  The {\tt ESMF\_LogFlush()} command is automatically
called when the program exits (either by finishing or by using the {\tt halt}
property in the {\tt ESMF\_LogSet} command or when the Log is closed.)

The user has the capability to halt the program on an error or on a warning.  By 
using the {\tt ESMF\_LogSet} command and setting the {\tt halt} property to
user can halt the program when the halt property is set to 
{\tt ESMF\_LOG\_HALTWARNING}, the program will stop on any waning or errors.  By
setting the {\tt halt} property to {\tt ESMF\_LOG\_HALTERROR}, the program will 
only halt on errors.  Lastly, the user can choose to never halt by setting the 
{\tt halt} property to {\tt ESMF\_LOG\_HALTNEVER}.

Many options are planned for Log, including the ability to write to a single
file or multiple files, to write from a root PE or from all active PEs,to adjust
the verbosity of ouput, and to optionally write to {\tt stdout} instead of 
file(s).




