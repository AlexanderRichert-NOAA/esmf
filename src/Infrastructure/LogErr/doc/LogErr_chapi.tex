%                **** IMPORTANT NOTICE *****
% This LaTeX file has been automatically produced by ProTeX v. 1.1
% Any changes made to this file will likely be lost next time
% this file is regenerated from its source. Send questions 
% to Arlindo da Silva, dasilva@gsfc.nasa.gov
 
\setlength{\parskip}{0pt}
\setlength{\parindent}{0pt}
\setlength{\baselineskip}{11pt}
 
%--------------------- SHORT-HAND MACROS ----------------------
\def\bv{\begin{verbatim}}
\def\ev{\end{verbatim}}
\def\be{\begin{equation}}
\def\ee{\end{equation}}
\def\bea{\begin{eqnarray}}
\def\eea{\end{eqnarray}}
\def\bi{\begin{itemize}}
\def\ei{\end{itemize}}
\def\bn{\begin{enumerate}}
\def\en{\end{enumerate}}
\def\bd{\begin{description}}
\def\ed{\end{description}}
\def\({\left (}
\def\){\right )}
\def\[{\left [}
\def\]{\right ]}
\def\<{\left  \langle}
\def\>{\right \rangle}
\def\cI{{\cal I}}
\def\diag{\mathop{\rm diag}}
\def\tr{\mathop{\rm tr}}
%-------------------------------------------------------------

\markboth{Left}{Source File: ESMC\_LogErr.h,  Date: Thu Aug  7 13:09:19 EDT 2003
}

 
%/////////////////////////////////////////////////////////////
\subsection{C++:  Class Interface ESMC\_Log - C++ interface to Log (Source File: ESMC\_LogErr.h)}


  
  
   The code in this file defines the C++ Log members and declares all class
   data and methods.  All methods, except for the Set and Get methods, which
   are inlined, are defined in the companion file ESMC\_LogErr.C
  
\bigskip{\em USES:}
\begin{verbatim} 
 
 #include <ESMC_Base.h>
 #include <stdarg.h>
 #include <stdio.h>
 #include <stdlib.h>
 #include <mpi.h>
 #include <time.h>
 #include <ctype.h>
 
 #include "ESMF_LogConstants.inc"
 #include "ESMF_ErrConstants.inc"
 #include "ESMC_UtilityFunctions.h"
 
 class ESMC_Log {
   private:
   !PRIVATE MEMBER FUNCIONS:
     void ESMC_LogFormName();
     void ESMC_LogPrintHeader(int fortIO);
     void ESMC_LogPrint(int fortIO, int errCode, int line, char file[],
                        char dir[], char msg[]=NULL);
     void ESMC_LogGetErrMsg(int errCode, char msg[]) const;
     bool ESMC_LogNameValid(char name[], int FortIO);\end{verbatim}{\sf PRIVATE TYPES:}
\begin{verbatim} 
     ESMC_Logical oneLogErrFile;
                                 // If log data written to one log file,
                                 // this is set to true.  Otherwise set to false.
                                 // ESMC_OpenFile can override
                                 // this value
 
     ESMC_Logical  standardOut;  // if log data written to standard out,
                                 // this variable
                                 // is set to true. Otherwise set to false.
                                 // ESMC_OpenFile
                                 // can over-ride this value.
 
     ESMC_Logical fortIsOpen;    // used to to a file with Fortran
                                 // I/O libraries 
     
     int unitNumber;             // fortran unit number for log/err file when
                                 // ESMC\_LogGetUnit
                                 // is used  Can be overwritten by
                                 // ESMC_OpenFileFortran
 
 
     int numFilePtr;             // index into global array of File pointers
                                 // for C++ I/O.
 
 
     int numFileFort;            // index into global array of unit numbers for 
                                 // Fortran I/O
 
     ESMC_Logical verbose;       // If set to ESMC_TF_TRUE, log
                                 // messages written out.
 
     ESMC_Logical flush;         // If true, all output is flushed
 
     ESMC_Logical haltOnWarn;    // Code will stop executing on
                                 // encountering a warning
 			    
     ESMC_Logical haltOnErr;     // Code will stop executing on
                                 // encountering an error
 
 
     char nameLogErrFile[32];    // name of logfile.
                                 // If multiple files are written out,
                                 // PE rank is automatically
                                 // appended to name.
   public:\end{verbatim}{\sf PUBLIC MEMBER FUNCTIONS:}
\begin{verbatim}   (see ESMC\_LogErr.C for a description of these methods)
     void ESMC_LogInfo(char* fmt,...);   
     void ESMC_LogInfoFortran(char fmt[],
     char charData[],char strData[][32],int intData[], double floatData[]);
     void ESMC_LogOpenFile(int numLogFile,char name[]);
     void ESMC_LogOpenFortFile(int numLogFile, char name[]);
     int ESMC_LogGetUnit();
     void ESMC_LogCloseFile();
     void ESMC_LogCloseFortFile();
     void ESMC_LogSetFlush();
     ESMC_Logical ESMC_LogGetFlush() const;
     void ESMC_LogSetNotFlush();
     void ESMC_LogSetVerbose();
     ESMC_Logical ESMC_LogGetVerbose() const;
     void ESMC_LogSetNotVerbose();
     void ESMC_LogSetHaltOnErr();
     ESMC_Logical ESMC_LogGetHaltOnErr() const;
     void ESMC_LogSetNotHaltOnErr();
     void ESMC_LogSetHaltOnWarn();
     ESMC_Logical ESMC_LogGetHaltOnWarn() const;
     void ESMC_LogSetNotHaltOnWarn();
     void ESMC_LogWarnMsg_(int errCode, int line, char file[],
                      char dir[], char msg[]);
     void ESMC_LogWarn_(int errCode, int line, char file[],
                      char dir[]);
     void ESMC_LogWarnFortran(int errCode, int line, char file[],
          char dir[], char msg[]);
     void ESMC_LogErr_(int errCode, int line, char file[], char dir[]);
     void ESMC_LogErrMsg_(int errCode, int line, char file[],
                      char dir[], char msg[]);
     void ESMC_LogErrFortran(int errCode,int line,char file[],char dir[],char msg[]);
     void ESMC_LogExit();
     void ESMC_LogSet(char* option, ESMC_Logical value, ...);
     void ESMC_LogGet(char* option, ESMC_Logical value, ...);
     FILE* ESMC_Log::ESMC_GetFileHandle();
     ESMC_Log();
 };\end{verbatim}
 
%/////////////////////////////////////////////////////////////
 
\mbox{}\hrulefill\ 
 

  \subsubsection [ESMC\_Log()] {ESMC\_Log() - constructor}


\bigskip{\sf INTERFACE:}
\begin{verbatim} 
 inline ESMC_Log::ESMC_Log(\end{verbatim}{\em RETURN VALUE:}
\begin{verbatim}    none\end{verbatim}{\em ARGUMENTS:}
\begin{verbatim}   none
    
   )
 \end{verbatim}
{\sf DESCRIPTION:\\ }


   This is the constructor.  Sets verbose, flush, haltOnErr and haltOnWarn
   to their default values.
   
%/////////////////////////////////////////////////////////////
 
\mbox{}\hrulefill\ 
 

  \subsubsection [ESMC\_LogSetFlush()] {ESMC\_LogSetFlush() - set the flushSet variable.}


\bigskip{\sf INTERFACE:}
\begin{verbatim} 
 inline void ESMC_Log::ESMC_LogSetFlush(\end{verbatim}{\em RETURN VALUE:}
\begin{verbatim}    none\end{verbatim}{\em ARGUMENTS:}
\begin{verbatim}     none
 
    ) 
 \end{verbatim}
{\sf DESCRIPTION:\\ }

 
   Causes output to be flushed.
    
%/////////////////////////////////////////////////////////////
 
\mbox{}\hrulefill\ 
 

  \subsubsection [ESMC\_LogGetFlush()] {ESMC\_LogGetFlush() - returns the flush variable }


\bigskip{\sf INTERFACE:}
\begin{verbatim} 
 ESMC_Logical ESMC_Log::ESMC_LogGetFlush (
   !RETURN VALUE
    Value of flush\end{verbatim}{\em ARGUMENTS:}
\begin{verbatim}     none
 
    ) const 
 \end{verbatim}
{\sf DESCRIPTION:\\ }

 
   Returns the flush variable 
    
%/////////////////////////////////////////////////////////////
 
\mbox{}\hrulefill\ 
 

                     \subsubsection [ESMC\_LogSetNotFlush()] {ESMC\_LogSetNotFlush() - output not flushed}


\bigskip{\sf INTERFACE:}
\begin{verbatim} 	     
 inline void ESMC_Log::ESMC_LogSetNotFlush(
   !RETUN VALUE:
    none
   !ARGUMENTS        
     none   
 
  )    
 	     \end{verbatim}
{\sf DESCRIPTION:\\ }


   Causes output not to be flushed. 
%/////////////////////////////////////////////////////////////
 
\mbox{}\hrulefill\ 
 

  \subsubsection [ESMC\_LogSetVerbose] {ESMC\_LogSetVerbose - make output verbose }


  
\bigskip{\sf INTERFACE:}
\begin{verbatim} 
 inline void ESMC_Log::ESMC_LogSetVerbose(\end{verbatim}{\em RETURN VALUE:}
\begin{verbatim}    none
   !ARGUMENTS
     none
   )
 \end{verbatim}
{\sf DESCRIPTION:\\ }


   If theVerbosity is set to ESMF\_TF\_TRUE, messages are printed out. 
    
%/////////////////////////////////////////////////////////////
 
\mbox{}\hrulefill\ 
 

  \subsubsection [ESMC\_LogGetVerbose] {ESMC\_LogGetVerbose - return verbose }


  
\bigskip{\sf INTERFACE:}
\begin{verbatim} 
 ESMC_Logical ESMC_Log::ESMC_LogGetVerbose(\end{verbatim}{\em RETURN VALUE:}
\begin{verbatim}    value of verbose
   !ARGUMENTS
     none
   ) const
 \end{verbatim}
{\sf DESCRIPTION:\\ }


   Returns  verbose value 
    
%/////////////////////////////////////////////////////////////
 
\mbox{}\hrulefill\ 
 

  \subsubsection [ESMC\_LogSetNotVerbose] {ESMC\_LogSetNotVerbose - output not verbose }


  
\bigskip{\sf INTERFACE:}
\begin{verbatim} 
 inline void ESMC_Log::ESMC_LogSetNotVerbose(
   RETURN VALUE:
    none
   !ARGUMENTS
     none
   )
 \end{verbatim}
{\sf DESCRIPTION:\\ }


   If theVerbosity is set to ESMC\_TF\_FALSE, no messages are printed out. 
    
%/////////////////////////////////////////////////////////////
 
\mbox{}\hrulefill\ 
 

  \subsubsection [ESMC\_LogSetHaltOnErr] {ESMC\_LogSetHaltOnErr - code will stop on encountering an error }


  
\bigskip{\sf INTERFACE:}
\begin{verbatim} 
 inline void ESMC_Log::ESMC_LogSetHaltOnErr(
   RETURN VALUE:
    none
   !ARGUMENTS
     none
   )
 \end{verbatim}
{\sf DESCRIPTION:\\ }


   If haltOnErr is set to ESMC\_TF\_TRUE, code will stop executing when
   encountering an error.
    
%/////////////////////////////////////////////////////////////
 
\mbox{}\hrulefill\ 
 

  \subsubsection [ESMC\_LogGetHaltOnErr] {ESMC\_LogGetHaltOnErr - returns haltOnErr}


  
\bigskip{\sf INTERFACE:}
\begin{verbatim} 
 ESMC_Logical ESMC_Log::ESMC_LogGetHaltOnErr(
   !RETURN VALUE
    haltOnErr
   !ARGUMENTS
     none
   ) const
 \end{verbatim}
{\sf DESCRIPTION:\\ }


   Returns haltOnErr
    
%/////////////////////////////////////////////////////////////
 
\mbox{}\hrulefill\ 
 

  \subsubsection [ESMC\_LogSetNotHaltOnErr] {ESMC\_LogSetNotHaltOnErr - code will not stop on encountering}


   an error  
  
\bigskip{\sf INTERFACE:}
\begin{verbatim} 
 inline void ESMC_Log::ESMC_LogSetNotHaltOnErr(\end{verbatim}{\em RETURN VALUE:}
\begin{verbatim}    none
   !ARGUMENTS
     none
   )
 \end{verbatim}
{\sf DESCRIPTION:\\ }


   If haltOnErr is set to ESMC\_TF\_FALSE, code will not stop executing when
   encountering an error.
    
%/////////////////////////////////////////////////////////////
 
\mbox{}\hrulefill\ 
 

  \subsubsection [ESMC\_LogSetHaltOnWarn] {ESMC\_LogSetHaltOnWarn - code will stop on encountering}


   a warning
  
\bigskip{\sf INTERFACE:}
\begin{verbatim} 
 inline void ESMC_Log::ESMC_LogSetHaltOnWarn(\end{verbatim}{\em RETURN VALUE:}
\begin{verbatim}   none
   !ARGUMENTS
     none
   )
 \end{verbatim}
{\sf DESCRIPTION:\\ }


   If haltOnWarn is set to ESMC\_TF\_TRUE, code will stop executing when
   encountering an error.
    
%/////////////////////////////////////////////////////////////
 
\mbox{}\hrulefill\ 
 

  \subsubsection [ESMC\_LogGetHaltOnWarn] {ESMC\_LogGetHaltOnWarn - returns haltOnwarn value}


   a warning
  
\bigskip{\sf INTERFACE:}
\begin{verbatim} 
 ESMC_Logical ESMC_Log::ESMC_LogGetHaltOnWarn(
   !RETURN VALUE
    Value of HaltOnWarn
   !ARGUMENTS
     none
   ) const
 \end{verbatim}
{\sf DESCRIPTION:\\ }


   Returns haltOnWarn
    
%/////////////////////////////////////////////////////////////
 
\mbox{}\hrulefill\ 
 

  \subsubsection [ESMC\_LogSetNotHaltOnWarn] {ESMC\_LogSetNotHaltOnWarn - code will not stop on encountering}


   a warning
  
\bigskip{\sf INTERFACE:}
\begin{verbatim} 
 inline void ESMC_Log::ESMC_LogSetNotHaltOnWarn(\end{verbatim}{\em RETURN VALUE:}
\begin{verbatim}    none
   !ARGUMENTS
     none
   )
 \end{verbatim}
{\sf DESCRIPTION:\\ }


   If haltOnWarn is set to ESMC\_Tf\_FALSE, code will not stop executing when
   encountering an error.
   
%...............................................................
