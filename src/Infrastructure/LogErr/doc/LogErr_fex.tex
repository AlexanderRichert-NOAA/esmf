% $Id: LogErr_fex.tex,v 1.2 2003/08/07 17:54:18 shep_smith Exp $
%
% Earth System Modeling Framework
% Copyright 2002-2003, University Corporation for Atmospheric Research, 
% Massachusetts Institute of Technology, Geophysical Fluid Dynamics 
% Laboratory, University of Michigan, National Centers for Environmental 
% Prediction, Los Alamos National Laboratory, Argonne National Laboratory, 
% NASA Goddard Space Flight Center.
% Licensed under the GPL.

%\subsection{F90 Use and Examples}

%<Detailed examples of F90 usage of the class.>


\subsubsection{Example 1. Simple Example.}

In this example we use the default ESMF\_Log\_World object and illustrate
the ESMF\_LogGetUnit()to write to
the log file aLog.txt.

{\tt
\begin{verbatim}
program test_log
  use ESMF_Mod
  implicit none
#include "ESMC_LogErr.inc"
  call ESMF_LogOpenFile(ESMF_Log_World,ESMF_SINGLE_LOG_FILE,"aLog.txt")
  write(ESMF_LogGetUnit(ESMF_Log_World),*)"This is a test."
  call ESMF_LogCloseFile(ESMF_Log_World)
end program
\end{verbatim}
\tt}

\subsubsection{Example 2. Simple Error Handling.}

Here we illustrate simple error handling.  We define an error handler - anErr -
and error output goes to the
file anErr.txt.  There are four options that can be set with ESMF\_LogSet()
and we set them all here: verbose (whether an
error message should be written for anErr), flush (whether output should be 
flushed), haltOnErr and haltOnWarn (whether the code should halt on
an error or warning).  While the values set here are the default
values, so they need not be set with ESMF\_Set(), the example illustrates how
to ESMF\_SET().

{\tt
\begin{verbatim}
#include "ESMC_LogErr.inc" 
type(ESMF_Log) :: anErr
integer returnCode 
call ESMF_LogSet(anErr, verbose=ESMF_TF_TRUE, flush=ESMF_TF_FALSE,   &
                 haltOnErr=ESMF_TF_TRUE, haltOnWarn=ESMF_TF_FALSE)
call ESMF_LogOpenFile(anErr,ESMF_SINGLE_LOG_FILE,"anErr.txt")
returnCode=foo()
if (returnCode == ESMF_FATAL) call ESMF_LogErr(anErr, returnCode) 
call ESMF_LogCloseFile(anErr)
\end{verbatim}
\tt}

\subsubsection{Example 3. More error handling.}

In this example, we illustrate more complex error handling.  We use
ESMF\_LogGet() to get the value of verbose and if it's set to ESMF\_TF\_TRUE
then turn off verbose midway through the code.  This allows you to put in lots
of debugging code, and then if you want to turn them off, you can just call
an ESMF\_LogSet() rather than commenting out all the error handling calls.
We also illustrate the use 
of ESMF\_LogErrMsg() which writes an additional message to the err file.

{\tt
\begin{verbatim}
#include "ESMC_LogErr.inc" 
type(ESMF_Log) :: anErr
type(ESMF_Logical) :: myVerbosity
character(len=32) :: myMsg
integer returnCode 
call ESMF_LogSet(anErr, verbose=ESMF_TF_TRUE, flush=ESMF_TF_FALSE,   &
                 haltOnErr=ESMF_TF_TRUE, haltOnWarn=ESMF_TF_FALSE)
call ESMF_LogOpenFile(anErr,ESMF_SINGLE_LOG_FILE,"anErr.txt")
returnCode=foo()
myMsg="This will be written out."
if (returnCode == ESMF_FATAL) call ESMF_LogErrMsg(anErr, returnCode,myMsg) 
ESMF_LogGet(anErr,verbose=myVerbosity)
if (myVerbosity==ESMF_TF_TRUE) call ESMF_LogSet(verbose=ESMF_TF_FALSE)
myMsg="This will not be written out."
returnCode=foo()
if (returnCode == ESMF_FATAL) call ESMF_LogErrMsg(anErr, returnCode,myMsg)
call ESMF_LogCloseFile(anErr)
\end{verbatim}
\tt}




