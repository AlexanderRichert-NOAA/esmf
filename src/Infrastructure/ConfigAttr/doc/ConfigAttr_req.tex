% $Id: ConfigAttr_req.tex,v 1.4 2002/06/11 17:05:27 dneckels Exp $

%===============================================================================
% Requirements may be itemized under a main topic:
%===============================================================================
%===============================================================================
\req{A human friendly format for parameter specification should be provided}
%-------------------------------------------------------------------------------


\sreq{Text files specifying parameters should allow comments}

A simple form of \# at the beginning of a comment line
should suffice.

\begin{reqlist}
{\bf Priority:} <Priority 1-3> \\
{\bf Source:} MITgcm \\
{\bf Status:} <Proposed, Approved-1, Approved-2, Rejected, Implemented, Verified> \\
{\bf Verification:} <e.g., Code inspection, Unit test, System test> \\
{\bf Notes:} <Background, comments on design, implementation, etc.> 
\end{reqlist}

\sreq{Text file parsing errors should at the least provide}
file name, line of file, meaningful error message logged in an easy
to find location

\begin{reqlist}
{\bf Priority:} <Priority 1-3> \\
{\bf Source:} MITgcm \\
{\bf Status:} <Proposed, Approved-1, Approved-2, Rejected, Implemented, Verified> \\
{\bf Verification:} <e.g., Code inspection, Unit test, System test> \\
{\bf Notes:} <Background, comments on design, implementation, etc.> 
\end{reqlist}

\sreq{Text file syntax should be backwards compatible with NAMELIST formats.}
The existing NAMELIST formats used by the JMC codes need to be parseable
as Configuration Attributes. This could be through conversion or through
a CA format that only extends the NAMELIST format i.e. that includes the
NAMELIST format as a subset.
The full CA format does not need to be compatible with the NAMELIST standard.
\begin{reqlist}
{\bf Priority:} <Priority 1-3> \\
{\bf Source:} MITgcm \\
{\bf Status:} <Proposed, Approved-1, Approved-2, Rejected, Implemented, Verified> \\
{\bf Verification:} <e.g., Code inspection, Unit test, System test> \\
{\bf Notes:} <Background, comments on design, implementation, etc.> 
\end{reqlist}

\req{Attributes can be sub-classified}

A system for grouping attributes into sub-categories within an ESMF application
would be useful. 

\begin{reqlist}
{\bf Priority:} <Priority 1-3> \\
{\bf Source:} MITgcm \\
{\bf Status:} <Proposed, Approved-1, Approved-2, Rejected, Implemented, Verified> \\
{\bf Verification:} <e.g., Code inspection, Unit test, System test> \\
{\bf Notes:} <Background, comments on design, implementation, etc.> 
\end{reqlist}

\req{Attributes can be optional and could be introduced at runtime}
There should be no need to specify all attributes. It should also be possible
to introduce a new attribute e.g. {\it fred=7} without having to
modify the attribute parser code.The Fortran NAMELIST requires that all
attributes be listed in the parsing routine. This should not be mandatory in CA.

\begin{reqlist}
{\bf Priority:} <Priority 1-3> \\
{\bf Source:} MITgcm \\
{\bf Status:} <Proposed, Approved-1, Approved-2, Rejected, Implemented, Verified> \\
{\bf Verification:} <e.g., Code inspection, Unit test, System test> \\
{\bf Notes:} <Background, comments on design, implementation, etc.> 
\end{reqlist}

\req{A system for defaults and overrides should be supported}
It should be possible to have a set of default attribute settings and
only require overrides to be specified.

\begin{reqlist}
{\bf Priority:} <Priority 1-3> \\
{\bf Source:} MITgcm \\
{\bf Status:} <Proposed, Approved-1, Approved-2, Rejected, Implemented, Verified> \\
{\bf Verification:} <e.g., Code inspection, Unit test, System test> \\
{\bf Notes:} <Background, comments on design, implementation, etc.> 
\end{reqlist}

\req{A system that is compatible with unique naming should be devised}
For example it should be possible to refer to an attribute with a
name such as {\it gov.gfdl.mom4.timestepping.deltat}. It should
be possible to determine whether the name matches uniquely or has
multiple matches (e.g. for the above example an application
with an ensemble of mom4 components would have multiple matches).

\begin{reqlist}
{\bf Priority:} <Priority 1-3> \\
{\bf Source:} MITgcm \\
{\bf Status:} <Proposed, Approved-1, Approved-2, Rejected, Implemented, Verified> \\
{\bf Verification:} <e.g., Code inspection, Unit test, System test> \\
{\bf Notes:} <Background, comments on design, implementation, etc.> 
\end{reqlist}

\req{Components should be able to query the attributes of other components}
This will be the main benefit of CA to ESMF.

\begin{reqlist}
{\bf Priority:} <Priority 1-3> \\
{\bf Source:} MITgcm \\
{\bf Status:} <Proposed, Approved-1, Approved-2, Rejected, Implemented, Verified> \\
{\bf Verification:} <e.g., Code inspection, Unit test, System test> \\
{\bf Notes:} <Background, comments on design, implementation, etc.> 
\end{reqlist}

\req{Components should be able to set attributes to be writable by other components}
Queries will be read-only, however it could also be useful to ultimately allow 
one component to set a value of another components attributes. For example in
ensemble simulations or computational steering a {\it driver} component could
control other components.

\begin{reqlist}
{\bf Priority:} <Priority 1-3> \\
{\bf Source:} MITgcm \\
{\bf Status:} <Proposed, Approved-1, Approved-2, Rejected, Implemented, Verified> \\
{\bf Verification:} <e.g., Code inspection, Unit test, System test> \\
{\bf Notes:} <Background, comments on design, implementation, etc.> 
\end{reqlist}
