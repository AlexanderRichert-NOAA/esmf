%
% Earth System Modeling Framework
% Copyright 2002-2015, University Corporation for Atmospheric Research, 
% Massachusetts Institute of Technology, Geophysical Fluid Dynamics 
% Laboratory, University of Michigan, National Centers for Environmental 
% Prediction, Los Alamos National Laboratory, Argonne National Laboratory, 
% NASA Goddard Space Flight Center.
% Licensed under the University of Illinois-NCSA License.

A location stream (LocStream) is used to represent the locations of
a set of data points (for example, in the data assimilation world, 
LocStreams can be thought of as a set of observations).  The values 
of the data points are stored within a Field or FieldBundle created 
using the LocStream.

The locations are generally
described using Cartesian (x, y, z), or (lat, lon, radius) coordinates.
There is no assumption of any regularity in the positions of the points.
To make the concept more general, the locations for each data point are
represented using a construct called {\it keys}. Keys are essentially 
vectors containing point descriptors.  They may include 
other information besides location (a mask, for example) or a second 
set of coordinates.  Each key must contain the same number of elements 
as there are locations in the LocStream.

LocStreams can be very large. Data assimilation systems might use
LocStreams with up to $10^{8}$ observations, so efficiency is critical.
LocStreams can be created from file, see \ref{locstream:createfromfile}.

Common operations involving LocStreams are similar to those involving Grids.
For example, LocStreams allow users to:

\begin{enumerate}
\item Create a Field or FieldBundle on a LocStream
\item Regrid data in Fields defined on LocStreams
\item Redistribute data between Fields defined on LocStreams
\item Gather or scatter a FieldBundle defined on a LocStream from/to a root DE
\item Extract Fortran array from Field which was defined by a LocStream
\end{enumerate}


%The operations on the Fortran arrays underlying LocStreams are usually simple numerical ones. However,
%it is necessary to sort them in place, and access only portions of the them. It would
%not be efficient to continually create new LocStreams to reflect this sorting. Instead,
%the sorting is managed by the application through permutation arrays while keeping
%the data in place. Locations can become inactive, e.g., if the quality control asserts that
%observation is invalid. This can be managed again by the application through masks.

A LocStream differs from a Grid in that no topological structure is
maintained between the points
(e.g. the class contains no information about which point is the neighbor
of which other point).

A LocStream is similar to a Mesh in that both are collections of irregularly positioned 
points.  However, the two structures differ because a Mesh also has connectivity: 
each data point has a set of neighboring data points. There is no requirement that the
points in a LocStream have connectivity, in fact there is no requirement that any two points 
have any particular spatial relationship at all.
