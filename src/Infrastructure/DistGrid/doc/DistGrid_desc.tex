% $Id: DistGrid_desc.tex,v 1.9 2010/10/13 22:49:06 rokuingh Exp $


\label{sec:DistGrid}
The {\tt ESMF\_DistGrid} class sits on top of the DELayout class and holds domain 
information in index space. A DistGrid object captures the index space topology and 
describes its decomposition in terms of DEs. Combined with DELayout and VM the DistGrid 
defines the data distribution of a domain decomposition across the computational 
resources of an ESMF Component.

The global domain is defined as the union or ``patchwork'' of logically rectangular (LR) 
sub-domains or {\em patches}. The DistGrid create methods allow the specification of such 
a patchwork global domain and its decomposition into exclusive, DE-local LR regions 
according to various degrees of user specified constraints. Complex index space topologies 
can be constructed by specifying connection relationships between patches during creation.

The DistGrid class holds domain information for all DEs. Each DE is associated with a local 
LR region. No overlap of the regions is allowed. The DistGrid offers query methods that 
allow DE-local topology information to be extracted, e.g. for the construction of halos by 
higher classes.

A DistGrid object only contains decomposable dimensions. The minimum rank for a DistGrid 
object is 1. A maximum rank does not exist for DistGrid objects, however, ranks greater than 
7 may lead to difficulties with respect to the Fortran API of higher classes based on 
DistGrid. The rank of a DELayout object contained within a DistGrid object must be equal to 
the DistGrid rank. Higher class objects that use the DistGrid, such as an Array object, may 
be of different rank than the associated DistGrid object. The higher class object will hold 
the mapping information between its dimensions and the DistGrid dimensions.
