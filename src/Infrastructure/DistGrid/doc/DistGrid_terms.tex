% $Id: DistGrid_terms.tex,v 1.11 2003/03/10 05:40:47 cdeluca Exp $

\begin{description}
\item[PEList] a list of processor IDs associated with a component. (We
  use PE as shorthand for processor, or ``processing element'').
\item[DELayout] distribution of elements of a PEList across multiple
  dimensions for the purpose of domain decomposition. All members of
  the PEList may not necessarily be assigned to the layout, but me
  set aside for the pool of PEs available for multi-threading.
\item[Node] ESMF recognizes a machine model based on a multi-level node
  hierarchy. An \emph{mnode} or memory node, is a set of processors
  sharing equal flat access to a block of physical memory. An
  \emph{anode}, or addressable node, is a set of processors that are
  capable of addressing the same set of blocks of physical memory. A
  \emph{pnode}, or processing node, is a set of processors to which an
  operating system scheduler is capable of assigning to a single job.

  An anode is at least the size of an mnode, and a pnode is at least
  the size of an anode.
  
  Examples: SGI Origin: the mnode is a ``C-brick'' of 4p, the anode is
  the full processor set operating as a single system image, the pnode
  is the cluster of anodes across which a single job may be
  submitted. Add examples on other hardware.
\item[Partition] in a multi-threaded application, the subset of a
  computational domain that is associated with a logically independent
  sequence of operations. The logical independence requirement is so
  that partitions may be scheduled as separable concurrent tasks.
\item[Scalar array] an array of scalar data (e.g temperature) on a
  grid.
\item[Vector array] vector data (e.g velocity components) on a
  grid. These may be arrays where one of the dimensions represents the
  vector component axis, or there may be separate data arrays for each
  vector component. Vector arrays may be 2D or 3D vectors.
\item[Grid staggering, grid offsets] a descriptor of relative locations
  of scalar and vector data on a structured grid. On different
  staggered grids, vector data may lie at cell faces or vertices,
  while scalar data may lie in the interior. The staggered locations
  are often written in a notation like $(i+\frac12,j+\frac12)$ to
  describe the offset of a corner with respect to the cell $(i,j)$.
\item[Data dependency] the property of the numerics that expresses the
  requirement of data values in order to perform a
  computation at a data point. For instance, a forward differencing
  operation in X at $(i,j)$ has a dependency on $(i+1,j)$.
\item[Distributed grid, or DistGrid] Given the global range of indices
  associated with a component, this entity associates index subsets
  with PEs, and describes the the topology and connectivity of the
  layout.
\item[Domain decomposition] the act of grid distribution: creating a
  layout; and associating gridpoints with PEs. The dimensionality
  of the domain decomposition is the dimensionality of the associated
  layout.
\item[Regular distribution] the distribution along any axis is
  independent of position on any other axis. An \emph{irregular
    distribution} tiles the global domain with irregularly-sized
  domains.
\item[Global domain] the global range of indices of data points.
\item[Computational domain] the set of points whose data is
  exclusively associated with a PE, and must be updated by it.
\item[Memory domain] A specification of the index
  range used to allocated data arrays on a PE.
\item[Data domain] This includes the computational domain, as well as
  the points with whom the computational points have data
  dependencies. The data domain may be the same as the memory domain,
  or the memory domain may be padded beyond the data domain.
\item[Halo] the points in the allocated domain outside the
  computational domain. Halo points are associated with other PEs'
  computational domains, and the halo update operation involves
  synchronization of halo points with other PEs.
\item[Grid topology] data connectivities on the pelist for scheduling
  data-sharing operations.
\item[Data transpose] Transfer of data arrays between two distributed
  grids sharing the same global domain.
\item[Global reduction] Reduction operations (sum, max, min, etc.) on
  data on a distributed grid.
\item[Background grid] A structured grid that is associated with an
  ungridded location stream, so that each index in the location stream
  is uniquely associated with a cell on the grid. (But a single grid
  cell may contain multiple location stream points).
  
  One may then apply a distribution to the underlying grid to generate
  a \textbf{PhysGrid} and a \textbf{DistGrid}. So the \textbf{DistGrid}
  could then have "halos" and associated operations. This approach
  also permits blocking and domain decomposition on unstructured grids
  and ungridded location stream, by applying those to the background
  grid.

% \item [<item1>] \label<glos:item1> <Description item 1.>

% \item [<item2>] \label<glos:item2> <Description item 2.>

\end{description}



