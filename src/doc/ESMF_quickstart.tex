\section{Quick Start}
\label{sec:QuickStart}

This section gives a brief
description of to how to get the ESMF software, build it, 
and run the self-tests to verify the installation was successful.
More detailed information on each of these steps, as well as
information on running a demonstration application and linking
the ESMF with your own code is found in Sections \ref{sec:TechOver} 
and \ref{sec:TechOver2}.  

\subsection{Downloading ESMF}

ESMF is distributed as a source-code tar file.  The tar file, release notes, 
known bugs, supported platforms, documentation, and other related information 
can be found on the ESMF web site, under the "Downloads \& Documentation" link:
\begin{verbatim}
    http://www.esmf.ucar.edu -> Downloads & Documentation
\end{verbatim}
ESMF can also be downloaded from the Sourceforge web site
from the "Files" link:
\begin{verbatim}
    http://sourceforge.net/projects/esmf -> Files
\end{verbatim}
Follow the directions on that web page to download a tar file.  
The Sourceforge web site also contains 
the bugs list, support requests, and core development group task
lists for the ESMF project.
See the {\tt Bugs}, {\tt Support}, and {\tt Tasks} links, respectively,
for more information.  Archives of the ESMF mailing lists can be
found under the {\tt Lists} link.

Contributions from ESMF users are available at a second Sourceforge web site:
\begin{verbatim}
    http://sourceforge.net/projects/esmfcontrib -> CVS
\end{verbatim}
In case of problems or questions, contact the ESMF support mailing
list at {\tt esmf\_support@ucar.edu}.


\subsection{Building ESMF}

After downloading and unpacking the ESMF tar file, the build procedure is:
\begin{enumerate}
\item Set the required environment variables. 
\item Type {\tt gmake } to build the library.
\item Type {\tt gmake check } to run self-tests to verify
the build was successful.
\end{enumerate}
See the following sections for more information on each of these steps.

\subsubsection{GNU make}
The ESMF build system uses the GNU make program; it is normally named 
{\tt gmake} but may also be simply {\tt make} or {\tt gnumake} on some 
platforms.  ESMF does not use configure or autoconf;  the selection of 
various options is done by
setting environment variables before building the framework. 


\subsubsection{Environment Variables}

The syntax for setting environment variables depends on which shell
you are running.  Examples of the two most common ways to set 
an environment variable are:
\begin{description}
\item[ksh] {\tt  export ESMF\_DIR=/home/joeuser/esmf}
\item[csh] {\tt  setenv ESMF\_DIR /home/joeuser/esmf}
\end{description}

The shell environment variables listed below are the ones most
frequently used.  There are others which address needs on specific
platforms or are needed under more unusual circumstances; 
see Section \ref{sec:TechOver} for the full list.  
\begin{description}

\item[ESMF\_DIR]
The environment variable {\tt ESMF\_DIR} must be set to the full pathname 
of the top level ESMF directory before building the framework.  This is the 
only environment variable which is required to be set on all platforms under 
all conditions.

\item[ESMF\_COMPILER]
On any platform which does not come with a single vendor-supplied Fortran 
compiler (e.g. Linux or Mac OS X) {\tt ESMF\_COMPILER} must be set to select
which Fortran compiler has been installed.  To see valid values for 
{\tt ESMF\_COMPILER}, look in the {\tt build\_config} directory in the
unpacked source tree.  List the subdirectories there.
The first part of each subdirectory name corresponds to the output 
of 'uname -s' for each platform.  The second part contains valid values 
for {\tt ESMF\_COMPILER}.

\item[ESMF\_COMM]
On multiprocessor systems with a vendor-supplied MPI communications library,
that is the default for communications and {\tt ESMF\_COMM} should not be set.  
For other systems, ESMF comes with with a single-processor MPI-bypass 
library which is the default.   To run multiprocessor applications
select an MPI implementation by setting {\tt ESMF\_COMM} to {\tt mpich} or
{\tt lam} before building ESMF.  Additional valid values may exist; check the
release notes for the specific platform on the web site for more details.
If the MPI files are not installed in a system directory (e.g. {\tt /usr/lib})
also set {\tt MPI\_HOME} to the directory containing the include, lib, 
and bin dirs.

\item[ESMF\_SITE]
The Sourceforge {\tt esmfcontrib} repository contains makefiles which have 
already been customized for certain machines.  If one exists for your site 
and you wish to use it, download the corresponding files into the 
{\tt build\_contrib} directory
and set {\tt ESMF\_SITE} to your location (which corresponds to the last
part of the directory name).  See the Sourceforge site
{\tt http://sourceforge.net/projects/esmfcontrib} for more information.

\item[ESMF\_BOPT]
To make a debuggable version of the library set {\tt ESMF\_BOPT} to 
{\tt g} before building.  The default is {\tt O} (capital oh) which builds an 
optimized library.
If {\tt ESMF\_BOPT} is unset or is {\tt O}, {\tt ESMF\_OPTLEVEL} can be set 
to a numeric value between 0 and 4 to select a specific optimization level.
(Note that level 0 corresponds to no optimization on most compilers.)

\item[ESMF\_PREC]
If this system supports the building of both 32-bit and 64-bit executables
(pointer wordsize), this variable can be set to select which format to use.
Valid values are {\tt 32} or {\tt 64}.

\end{description}


\subsubsection{Build makefile Targets}

The makefiles follow the GNU target standards where possible.
The most frequently used targets for building are listed below:
\begin{description}
\item[lib] build the ESMF libraries only (default)
\item[all] build the libraries, unit and system tests, examples, and demos
\item[doc] build the documentation (requires specific latex macros packages
and additional utilities; see Section \ref{sec:TechOver} for more details
on the requirements).  The ESMF web site 
contains pre-built pdf and html versions of all documentation
under the "Downloads \& Documentation" link.
\item[info] print out extensive system configuration information about what
           compilers, libraries, paths, flags, etc are being used
\item[clean] remove all files built for this platform/compiler/wordsize.
\item[clobber] remove all files built for all architectures
\end{description}

Note: The normal {\tt install} target is {\it not} currently supported.  
Neither is the {\tt uninstall} target.  
They will be added in a future release of ESMF.



\subsubsection{Testing makefile Targets}

To build and run the unit and system tests in non-exhaustive mode, type:
\begin{verbatim}
gmake check
\end{verbatim}
A summary report of success and failures will be printed out at the end.

\noindent Other test-related targets are:
\begin{description}
\item[all\_tests] build and run all available tests and demos
\item[build\_all\_tests] build tests only, do not execute
\item[run\_all\_tests] run tests without rebuilding and print a
summary of the results
\item[check\_all\_tests] 
print out the results summary without re-executing the tests again
\item[clean\_all\_tests] remove all test and demo executables 
\end{description}

For all the targets listed above, the string {\tt all\_tests} can be
replaced with one of the strings listed below to select a
specific type of test:
\begin{description}
\item[unit\_tests] unit tests exercise a single part of the system
\item[system\_tests] system tests combine functions across the system
\item[examples] examples contain code illustrating a single type of function
\item[demos] demos are example applications showing the use of the system
\end{description}
For example, {\tt gmake build\_examples} recompiles the example programs but 
does not execute them.  {\tt gmake clean\_system\_tests} removes all
executables and files associated with the system tests.

For the unit tests only, there is an additional environment variable
which affects how the tests are built:
\begin{description}
\item[ESMF\_EXHAUSTIVE]
If this variable is set to {\tt ON} before compiling the unit tests,
longer and more exhaustive unit tests will be run.  Note that this is a
compile-time and not run-time option.
\end{description}

