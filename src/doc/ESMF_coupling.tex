\section{Coupling}

\subsection{Introduction}

One of the main goals of the ESMF project is to increase interoperability
across a range of Earth system modeling components.  Initially these 
components will be large scale, such as land models, ocean models, 
and other systems representing physical domains; data assimilation systems
and diagnostics packages; and computational tools such as couplers and 
I/O packages.  An efficient implementation of a coupling system can
be used within models themselves for tasks such
as the transformation of data between the physics and dynamics in a 
spectral atmsopheric model, or the creation of nested higher resolution 
regions within a coarser grid.  For all of these tasks, the coupling and 
underlying 
communication mechanism of the framework is central.  In this section, we 
explain the classes involved in ESMF coupling, their relationships, and 
the sequence of actions given a number of different coupling 
scenarios.  These scenarios include a single executable configuration in 
which application components execute sequentially, and a multiple executable
configuration in which model components execute concurrently.

\subsection{Classes and Scoping}

The most general view of ESMF coupling involves an application component
({\tt ESMF\_App})
containing two or more gridded components ({\tt ESMF\_GComp}s) that require an 
inter-component data exchange, and a coupling component ({\tt ESMF\_Coupler}).

There are restrictions on how objects within an application may be scoped
on a parallel platform.  The application is required to exist on all 
decomposition elements ({\tt ESMF\_DE}s) in its layout ({\tt ESMF\_Layout}), 
no matter whether the application contains a single executable or multiple 
executables.

A coupler or gridded component may also exist across all {\tt ESMF\_DE}s.  When 
a set of gridded  components and a coupler all reside on the same set of 
{\tt ESMF\_DE}s and are contained within an application running as a single 
executable we have an SPMD, sequential execution model.

Within an application, a coupler or gridded component may also reside on 
a subset of {\tt ESMF\_DE}s.  Either of these may be wholly coincident with, 
wholly 
contained within or wholly contain another component.  A coupler or gridded
component may be defined as a separate executable.  An MPMD, concurrent
execution configuration is one in which gridded components and 
a coupler are defined on separate ESMF\_DEs running simultaneously as multiple 
executables.

It is possible for ESMF applications to be a combination of the SPMD, 
sequential 
execution model and the MPMD, concurrent execution model.  We might have,
for example, atmosphere and land components defined on the same set of 
{\tt ESMF\_DE}s and running as one executable, and ocean and sea ice 
components defined on a separate set of {\tt ESMF\_DE}s and running as 
another executable.  As required, the application component is defined
on all sets of {\tt ESMF\_DE}s.

\subsection{Flow of Control and Data}

Gridded components have few responsibilities with respect to coupling
to other components.  They do not need to keep track of the data needed
by other components, compute fluxes, perform grid transformations, or 
perform inter-component transfers of data.  They {\bf are} responsible for 
providing a description of all of the fields and data they
can export to other components, and for similarly providing a description 
of the fields they must import in order to run.  These descriptions are
embodied in the {\tt ESMF\_State} class, and the kind of data that they 
represent is identified by the type attribute of that class, which can
have an {\tt ESMF\_IMPORT} or {\tt ESMF\_EXPORT} value.  In addition to 
providing an import state, gridded components are required to check that 
initial or imported data is valid and complete.

The coupler component performs all the computations necessary for effecting
communication between components, both scientific and computational.  
It is responsible for computing fluxes, time averages, and interpolation 
weights for regridding.  The coupler is not responsible for transferring
data between components or for managing the synchronization of components;
this is handled by the application.  The coupler simply returns a transform 
specification that directs exactly how sequencing and data transfers should 
occur.  The coupler is also responsible for checking the correctness of 
the transfer, through means such as computing conserved quantities.

The application component manages the sequencing and synchronization of
components, and activates the transforms specified by the coupler.  The
coupler may specify that transformations occur on either the data sending 
or receiving end (or both), that data be sent directly between components,
or that data be sent to the coupler for intermediate processing before 
it is sent to its destination.

\subsection{Examples}

In our first set of examples, involving just two components, we'll call 
the instance of the application {\tt app}, the gridded components {\tt atm} 
and {\tt ocn}, and the coupler {\tt coupler}.









