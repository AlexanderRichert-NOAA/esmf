
This glossary defines terms used in Earth system modeling to describe 
parallel computer architectures, grids and grid decompositions, and 
numerical and computational methods.  While some of the concepts in 
the glossary may eventually appear as computational objects, many 
will not.  The goal here is not to define a framework design or an 
object model but simply to achieve a common language.

\begin{description}

\item[Accumulator] \label{glos:Accumulator} A facility for collecting and 
  averaging data values.  Generally accumulators are associated with 
  temporal averaging, although they might be associated with 
  other weighted averaging operations.    
  
\item[Address space] \label{glos:ASP}A standard term to refer to the memory
  seen by a computer program that it can write to directly using
  simple language primitives. 

\item[Alarm] \label{glos:Alarm} An event 
  that occurs at a particular time (or set of times).  It is like an
   alarm on a real alarm clock except that in order to determine whether 
it is "ringing", an alarm is "read" by an explicit application action.

\item[Addressable node] \label{glos:Anode} A set of processors that are
  capable of addressing the same set of blocks of physical memory.

\item[Application] \label{glos:Application} A coherent computational 
  entity run 
  as a single executable or set of communicating executables.  It 
  typically consists of a set of interacting components.

\item[Background grid] \label{glos:BackGrid} 
  A background grid associates each point in a location stream with a 
  location on a grid. A single grid cell may contain zero or more location 
  stream points.  

\item[Bundle] \label{glos:Bundle} A bundle refers to a set of fields that 
  are associated with the same physical grid and distributed in a similar 
  fashion across the same physical axes.  Fields within a bundle may be
  staggered differently and may have different dimensions.

\item[Calendar interval] \label{glos:CalInt} A period of time specified
in calendar-based units that may be used to increment or decrement time instants.  
One year and three months is an example of a calendar interval.  Since 
mathematical operations involving calendar intervals may be ambiguously 
defined -- for example, incrementing January 31 in the Gregorian calendar by 
one month -- default behavior must be carefully specified.  

\item[Cell] \label{glos:Cell} A physical location that is specified by both 
  its extent (vertices) and nominal central location, and is associated with 
  a single integer index value or a set of integer index values ( e.g.
  (i) for 1-d, (i,j) for 2-d, (i,j,k) for 3d ).

\item[Clock] \label{glos:Clock} A clock tracks the passage of time and 
reports the current time instant, like a real clock.  However, most clocks 
used in ESMF components have a key difference to a real clock. Clocks 
in an ESMF component are generally stepped forward by the component, as an 
explicitly coded time step within the overall component.

\item[Component] \label{glos:Component} A large-scale computational entity 
  associated with a particular physical process or computational function, 
  such as a land model.  Components may be generic or user-supplied.  
  See also gridded component, coupler component.

\item[Compute resource] \label{glos:CompResource} Something that appears as a
  physical or virtual computer resource. Example of compute resources
  are a CPU, a network connection, a communication API, a protocol, a 
  particular network fabric or a piece of computer memory. 

\item[Concurrent execution] \label{glos:ConcurrentExecution} 
Concurrent execution of model components occurs when two components,
whether in the same or different executables, run simultaneously.
Components executing concurrently may be in the same or different 
executables and may have coincident or non-overlapping memory 
distributions.  See \htmladdnormallink{sequential execution}
{glos:SequentialExecution}.

\item[Coupler component] \label{glos:Coupler}
  A component that includes all data and actions needed to enable 
  communication between two or more other components.

\item[Data dependency] \label{glos:DataDep} The property of a computational
  operator that defines the data indices required to perform
  the computation at a point.  For instance, a forward differencing
  operation in X at $(i,j)$ has a dependency on $(i+1,j)$.

\item[Data parallel] \label{glos:DataParallel} In a data parallel operation,
roughly the same calculation is performed by all processors at the same 
time on the same data set, which is partitioned among multiple memory 
locations.  Operations within many model components are essentially data 
parallel.  See \htmladdnormallink{task parallel}{glos:TaskParallel}, 
\htmladdnormallink{SPMD}{glos:SPMD}, \htmladdnormallink{MPMD}{glos:MPMD}.

\item[Data transpose] \label{glos:DataTranspose} Rearrangement of data arrays 
  between two distributed grids sharing the same global domain.

\item[Day of year] \label{glos:DayOfYear} The day number in the calendar year. 
January 1 is day 1 of the year. Day of year expressed in a floating point 
format is used to express the day number plus the time of day. 
For example, assuming a Gregorian calendar:

\begin{tabular}{ll}
{\bf date}              & {\bf day of year} \\
\hline 
10 January 2000, 6Z     & 10.25 \\
31 December 2000, 18Z   & 366.75 
\end{tabular}

\item[DE] \label{glos:DE} 
Short for decomposition element.

\item[Decomposition element (DE)] \label{glos:Decomp_Element}
A decomposition element is a virtual portion of a computer 
associated with a processing element and an address space.  A DE may 
be associated with an MPI process or a thread.  Layouts 
assign a topology to decomposition elements.

\item[Distributed grid] \label{glos:DistGrid}
  A distributed grid defines the decomposition of the global index space 
  across the layout and methods on the indexed data.

\item[Distribution] \label{glos:Distribution} The function that expresses
the relationship between the indices in a distributed grid and the elements 
in a layout.  

\item[Domain decomposition] \label{glos:DomainDecomp} The act of grid 
  distribution: creating a layout; and associating gridpoints with the layout. 
  The dimensionality of the domain decomposition is the dimensionality of 
  the associated layout.

\item [Exact] \label{glos:Exact} The word exact is used
to denote entities, such as time instants and time intervals, for which truncation-free arithmetic is required. 

\item[Exchange grid] \label{glos:ExchangeGrid} A grid whose vertices are
formed by the intersection of the vertices of two overlying grids.  Each 
cell in the exchange grid overlies exactly one cell in each grid of the 
exhange.

\item[Exchange packets] \label{glos:EP} The data exchanged by components.  
  Exchange packets may or may not contain contiguous data, and may contain 
  both field and other forms of data.

\item[Exclusive domain] \label{glos:ExcDomain} The set of indices whose 
  data is exclusively and definitively updated by a particular PE.

\item[Executable] \label{glos:Exec} 
  A parallel program that is under independent control by the operating 
  system.

\item[Export state] \label{glos:ExportState} The data and 
  metadata that a component can make available for exchange 
  with other components. This may be data at a physical boundary 
  (e.g land-atmosphere interface) or in other cases, it might be the 
  entire model state.  See also restart state, import state.

\item[Field] \label{glos:Field} A field is a physical quantity
  defined within a region of space.  A field includes a grid 
  and any metadata necessary for a full description of the field data.

\item[Framework] \label{glos:Framework} We use the term framework to 
refer to a structured collection of software building blocks that can be used 
and customized to compose, couple, and run applications.

\item[Functionality Class] \label{glos:FuncClass}
A functionality class is a body that accomplishes a given function, such
as I/O. It may contain several different classes or extend over multiple
framework layers. Functionality classes are described in the ESMF Architecture
Document.

\item[Generic component] \label{glos:GenericTrans} A generic component
  is one supplied by the framework.  The user is not expected to 
  customize or otherwise modify it.  See also user component.

\item[Generic transform] \label{glos:GenericTrans} A generic transform 
  is a operation supplied by the framework, for example, a method 
  that converts gridded data from one supported physical grid and/or 
  decomposition to another using a specified technique.  See also user 
  transform.

\item[Global physical grid] \label{glos:GlobPhysGrid} 
  A global physical grid contains physical information about the entire, 
  undecomposed domain.  No distributed grid need be associated with a global 
  physical grid.  

\item[Global domain] \label{glos:GlobDomain}
  The global range of indices of data points.

\item[Global reduction] \label{glos:GlobReduction} 
  Reduction operations (sum, max, min, etc.) on
  data defined on a distributed grid.  See also global broadcast.

\item[Global broadcast] \label{glos:GlobBroadcast}
  Scatter operations on data defined on a distributed grid.
  See also global reduction.

\item[Grid] \label{glos:Grid} The discrete division of space associated with
  a particular coordinate system.  A grid contains all physical grid and memory 
  organization information (via distributed grid and layout) required to manipulate 
  fields, as well as to create and execute grid transforms. 

\item[Grid metrics] \label{glos:GridMetrics} Terms relating measurements 
  in index space to physical grid quantities like distances and areas.

\item[Grid staggering] \label{glos:GridStagger} 
  A descriptor of relative locations
  of scalar and vector data on a structured grid. On different
  staggered grids, vector data may lie at cell faces or vertices,
  while scalar data may lie in the interior. The staggered locations
  are often written in a notation like $(i+\frac12,j+\frac12)$ to
  describe the offset of a corner with respect to the cell $(i,j)$.

\item[Grid topology] \label{glos:GridTopo} Description of data 
  connectivities in index space.

\item[Grid union] \label{glos:GridUnion} The formation of a new grid
  by taking the union of the vertices of two input grids. 

\item[Gridded component] \label{glos:GridComp}
  A component that is associated with one or more grids.  No requirements 
  may be placed on the physical content of a gridded component's data or 
  on the nature of its computations. 

\item[Halo] \label{glos:Halo} 
  The points in the data domain outside the local domain. 

\item[Halo update] \label{glos:HaloUpdate}
  Halo points are associated with other PEs'
  local domains, and the halo update operation involves
  synchronization of some or all halo points with other PEs. 

\item[Import state] \label{glos:ImportState} The data and metadata 
  that a component requires from other components in order to run.  
  See also export state, restart state.

\item[Index] \label{glos:Index} An integer value associated with a set
  of coordinates that describe a cell or location in physical space.

\item[Index space] \label{glos:IndexSpace} The space implied 
  by a set of indices.  An index space has a defined dimensionality and 
  connectivity.

\item[Index space location] \label{glos:IndexSpaceloc} 
  A location within index space.  A index space location may be fractional.
  See also physical location.

\item[Layout] \label{glos:Layout} A layout specifies a PE list, 
  decomposition strategy (thread and process), and the dimensionality 
  and connectivity of the decomposition.  Multiple distributed 
  grids may be defined per layout.

\item[Local domain] \label{glos:LocalDomain} This includes the exclusive 
  domain, as well as the points with whom the exclusive points have data 
  dependencies.

\item[Local physical grid] \label{glos:LocPhysGrid} The portion of a 
  physical grid associated with a local domain.  

\item[Location stream] \label{glos:LocStream} A list of
  locations with no assumed relationship between these locations.  The
  elements of a location stream are assumed to share the same data
  items and metadata, though some elements may have blank entries for
  particular data or attributes.

\item[Logically rectangular grid] \label{glos:RecGrid} A grid in 
  which sequential indices are physically adjacent, and in which the 
  extent of each index is independent of the other indices.

\item[Loose bundle] \label{glos:LooseBundle} A loose bundle consists of 
  fields whose data is not contiguous in memory.

\item[Machine model] A generic representation of the computing 
  platform architecture.

\item[Mask] \label{glos:Mask} A field marking a span within a larger grid.

\item[Memory domain] \label{glos:MemDomain} The portion of memory 
  associated with an local domain.  The memory domain is always at least 
  as large as the local domain.

\item[Memory node] \label{glos:Mnode} A set of processors
  sharing equal flat access to a block of physical memory.

\item[MPMD] \label{glos:MPMD} Multiple Program Multiple Datastream.
  Multiple executables, any of which could itself be an SPMD
  executable, executing independently within an application.

\item [No-leap calendar] \label{glos:NoLeap} Every year uses the same months 
and days per month as in a non-leap year of a Gregorian calendar.

\item[Packed bundle] \label{glos:PackedBundle} A packed bundle is arranged
  so that field data is contiguous in memory.

\item[Partition] \label{glos:Partition} In a multi-threaded application, the subset of a
  computational domain that is associated with a logically independent
  sequence of operations. The logical independence requirement is so
  that partitions may be scheduled as separable concurrent tasks.

\item[PE] \label{glos:PE} Short for processing element.

\item[PE list] \label{glos:PElist} A list of processor IDs associated 
  with a component.  See also layout.

\item[Physical grid] \label{term:PhysGrid} 
  A physical grid contains a variety of information
  on the location in physical space and physical metrics (area,
  grid lengths, etc.) of various grid points.

\item[Physical location] \label{glos:PhysLoc} The point in physical space 
  to which data pertain. 

\item[Platform] \label{glos:Platform} 
  The processor hardware, operating system, compiler and
  parallel library that together form a unique compilation target.

\item[Processing element (PE)] \label{glos:Processing_Element}
A processing element is associated with a single hardware processor.  It may
have a framework ID that is different than its vendor-assigned ID.  

\item[Processing node] \label{glos:Pnode} A set of processors to which an
  operating system scheduler is capable of assigning to a single job.

\item[Restart state] \label{glos:RestartState} The component 
  data that 
  is needed for an exact restart. This can include, in addition to 
  a physical state,  time information, static field data,
  metadata and control information. 

\item[Scheduler] \label{glos:Scheduler} An operating system component 
  that assigns system
  resources (processors, memory, CPU time, I/O channels, etc.) to
  executables.

\item[Sequential execution] \label{glos:SequentialExecution}
Sequential execution of model components describes the case in which 
one component waits for the other to finish before it begins
to run.  Components executing sequentially may be in the same or 
different executables and may have conincident or non-overlapping 
memory distributions.  See \htmladdnormallink{concurrent execution}
{glos:ConcurrentExecution}.

\item[Span] \label{glos:Span} The physical extent associated with a grid.

\item[SPMD] \label{glos:SPMD} Single Program Multiple Datastream. 
  A single executable, possibly with many 
  components (representing for example the atmosphere, the ocean, 
  land surface) executing serially or concurrently.

\item [System time] \label{glos:SysTime}Time spent doing system tasks such as I/O or in system calls.  May also
include time spent running other processes on a multiprocessor system.

\item[Task parallel] \label{glos:TaskParallel}  In a task parallel operation,
different calculations are performed by different processors at the same time
on what are usually different data sets.  Operations on different model 
components running within either a SPMD or MPMD application may be task 
parallel.  See \htmladdnormallink{data parallel}{glos:DataParallel}, 
\htmladdnormallink{SPMD}{glos:SPMD}, \htmladdnormallink{MPMD}{glos:MPMD}.

\item [Time instant] \label{glos:TimeInstant}
Generic name for an absolute time and date specification. A time instant is made 
up of a time and date and an associated calendar. It may include a time zone.
``Jan 3rd 1999, 03:30:24.56s, UTC'' is one example of a time instant.

\item [Time interval] \label{glos:TimeInterval} A time interval is the
period between any two time instants, measured in units, such as days, 
seconds, and fractions of a second, that are not associated with a specific
calendar.  Time intervals may be negative.  The periods 2 days and 10 seconds, 
86400 and 1/3 seconds and 31104000.75 seconds are all examples of time intervals.  
Mathematical operations such as addition, multiplication and subdivision 
can be applied to time intervals.

\item [User component] \label{UserComp} A component that is customized or
written by the user.  See also generic component.

\item [User time] \label{UserTime} Processor time actually spent executing a process's code.

\item[User transform] \label{glos:UserTrans} A user-supplied 
  method that is used to extend framework capabilities beyond generic 
  transforms.  

\item [Wall clock time] \label{WallClockTime} Elapsed real-world time (i.e. difference between start time minus
stop time).

\end{description}








































