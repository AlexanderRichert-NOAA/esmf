% $Id: ESMF_options.tex,v 1.21 2007/03/31 05:18:30 cdeluca Exp $

\section{Global Options and Parameters}

\subsection{Flags}

\subsubsection{ESMF\_AllocFlag}
\label{opt:allocflag}
{\sf DESCRIPTION:\\}  
Indicates whether to allocate data or not.

Valid values are:
\begin{description}
\item [ESMF\_ALLOC]
      Allocate data. 
\item [ESMF\_NO\_ALLOC]
      Do not allocate data at this time. 
\end{description}

\subsubsection{ESMF\_BlockingFlag}
\label{opt:blockingflag}
{\sf DESCRIPTION:\\}  
Indicates blocking behavior and PET synchronization.

Valid values are:
\begin{description}

\item [ESMF\_BLOCKING]
         {\em Communication calls:} The called method will block until all
         (PET-)local operations are complete. After the return of a blocking
         communication method it is safe to modify or use all participating
         local data.
         
         {\em Component calls:} The called method will block until all PETs in
         all virtual address spaces (VASs) have completed the operation.

\item [ESMF\_VASBLOCKING]
         {\em Communication calls:} Not available for communication calls.
         
         {\em Component calls:} The called method will block until all PETs that
         operate in the PET-local VAS have completed the operation. For PETs
         that run as single threaded processes this means that the method does
         {\em not} synchronize PETs on return.
         
\item [ESMF\_NONBLOCKING]
         {\em Communication calls:} The called method will not block but 
         returns immediately after initiating the requested operation. It is
         unsafe to modify or use participating local data before all local
         operations have completed.

         {\em Component calls:} The called method will not block but 
         returns immediately after initiating the requested operation. It is
         unsafe to modify or use participating local data before all local
         operations have completed.

\end{description}

\subsubsection{ESMF\_ContextFlag}
\label{opt:contextflag}
{\sf DESCRIPTION:\\}  
Indicates the type of VM context in which a component is executing.

Valid values are:
\begin{description}

\item [ESMF\_CHILD\_IN\_NEW\_VM]
         The component is running in its own, separate VM context. Resources
         are inherited from the parent but can be arranged to fit the
         component's requirements.
\item [ESMF\_CHILD\_IN\_PARENT\_VM]
         The component uses the parent's VM for resource management. Compared
         to components that use their own VM context components that run in the
         parent's VM context are more light-weight with respect to the overhead
         of calling into their Initialize, Run and Finalize methods.
         Furthermore, VM-specific properties remain unchanged when going from
         the parent component to the child component. These properties include
         the MPI communicator, the number of PETs, the PET labeling, 
         communication attributes, threading-level.

\end{description}

\subsubsection{ESMF\_CopyFlag}
\label{opt:copyflag}
{\sf DESCRIPTION:\\}
Indicates whether to reference a data item or make a copy of it.

Valid values are:
\begin{description}
\item [ESMF\_DATA\_COPY]
      Copy the data item to another buffer.
\item [ESMF\_DATA\_REF]
      Reference the data item.
\end{description}

\subsubsection{ESMF\_IndexFlag}
\label{opt:indexflag}
{\sf DESCRIPTION:\\}
Indicates whether index is local (per DE) or global (per object).

Valid values are:
\begin{description}
\item [ESMF\_INDEX\_DELOCAL]
      Refers to indices on the local DE.
\item [ESMF\_INDEX\_GLOBAL]
      Refers to object-wide indices.
\end{description}

\subsubsection{ESMF\_InterleaveFlag}
\label{opt:interleave}
{\sf DESCRIPTION:\\}
Interleave is used when there are multiple variables or
if individual data items are vectors.  Used in {\tt ESMF\_FieldDataMap}
and {\tt ESMF\_BundleDataMap}.
(The interleave option is not yet implemented.)


Valid values are:
\begin{description}
   \item [ESMF\_INTERLEAVE\_BY\_BLOCK]
         Items are listed in blocks, all items of one type followed
         by all items of the next type.
   \item [ESMF\_INTERLEAVE\_BY\_ITEM]
         Items are interleaved item by item.
\end{description}


\subsubsection{ESMF\_NeededFlag}
\label{opt:neededflag}
{\sf DESCRIPTION:\\}
Specifies whether or not a data item is needed for a 
particular application configuration.  Used in {\tt ESMF\_State}.

Valid values are:
\begin{description}
   \item [ESMF\_NEEDED] 
         Data is needed.
   \item [ESMF\_NOTNEEDED]
         Data is not needed.
\end{description}

\subsubsection{ESMF\_ReadyFlag}
\label{opt:readyflag}
{\sf DESCRIPTION:\\}
Specifies whether a data item is ready to read or write.

Valid values are:
\begin{description}
   \item [ESMF\_READYTOREAD] 
         Data is ready to read.
   \item [ESMF\_READYTOWRITE]
         Data is ready to write.
   \item [ESMF\_NOTREADY]
         Data is not ready.
\end{description}

\subsubsection{ESMF\_ReduceFlag}
\label{opt:reduceflag}
{\sf DESCRIPTION:\\}
Indicates reduce operation to a {\tt Reduce()} method.

Valid values are:
\begin{description}
   \item [ESMF\_SUM]
         Use arithmetic sum to add all data elements.
   \item [ESMF\_MIN]
         Determine the minimum of all data elements.
   \item [ESMF\_MAX]
         Determine the maximum of all data elements.
\end{description}

\subsubsection{ESMF\_ReqForRestartFlag}
\label{opt:reqforrestartflag}
{\sf DESCRIPTION:\\}
Specifies whether a data item is necessary for restart.

Valid values are:
\begin{description}
   \item [ESMF\_REQUIRED\_FOR\_RESTART] 
         Data is required for restart.
   \item [ESMF\_NOTREQUIRED\_FOR\_RESTART]
         Data is not required for restart.
\end{description}

\subsubsection{ESMF\_ValidFlag}
\label{opt:validflag}
{\sf DESCRIPTION:\\}
Specifies whether a data item contains valid data.

Valid values are:
\begin{description}
   \item [ESMF\_VALID] 
         Data is ready to read.
   \item [ESMF\_INVALID]
         Data is ready to write.
   \item [ESMF\_NOTREADY]
         Data is not ready.
\end{description}


\subsection{Parameters}

\subsubsection{ESMF\_TypeKind}
\label{opt:typekind}

{\sf DESCRIPTION:\\}
Supported ESMF type and kind combinations.   
This is an ESMF derived type used for arguments to subroutines 
and functions that specify or query a data precision and type.
These values cannot be used when declaring variables; see the next 
section on Fortran 90 Kinds for that.

Valid values are:
\begin{description}
\item [ESMF\_TYPEKIND\_I1]
      1 byte integer.
\item [ESMF\_TYPEKIND\_I2]
      2 byte integer.
\item [ESMF\_TYPEKIND\_I4]
      4 byte integer.
\item [ESMF\_TYPEKIND\_I8]
      8 byte integer.
\item [ESMF\_TYPEKIND\_R4]
      4 byte real.
\item [ESMF\_TYPEKIND\_R8]
      8 byte real.
\end{description}

\subsubsection{Fortran 90 Kinds}

{\sf DESCRIPTION:\\}
These are integer parameters of the proper type to be
used when declaring variables with a specific precision 
in Fortran 90 syntax.  For example:
\begin{verbatim}
 integer(ESMF_KIND_I4) :: myintegervariable
 real(ESMF_KIND_R4) :: myrealvariable
\end{verbatim}
The Fortran 90 standard does not mandate what numeric values
correspond to actual number of bytes allocated for the
various kinds, so these are defined by ESMF to be correct across
the different supported Fortran 90 compilers.   Note that not
all compilers support every kind listed below; in particular
1 and 2 byte integers can be problematic.

Valid values are:
\begin{description}
\item [ESMF\_KIND\_I1]
      1 byte integer.
\item [ESMF\_KIND\_I2]
      2 byte integer.
\item [ESMF\_KIND\_I4]
      4 byte integer.
\item [ESMF\_KIND\_I8]
      8 byte integer.
\item [ESMF\_KIND\_R4]
      4 byte real.
\item [ESMF\_KIND\_R8]
      8 byte real.
\item [ESMF\_KIND\_C8]
      8 byte character.
\item [ESMF\_KIND\_C16]
      16 byte character.
\end{description}









