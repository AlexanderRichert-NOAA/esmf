
\section{What is the Earth System Modeling Framework?}

The Earth System Modeling Framework (ESMF) is a suite of software
tools that can be used to develop 
Earth system model components and assemble them into applications.  ESMF
consists of an {\it infrastructure} of utilities and data structures
useful for building components and a {\it superstructure} for coupling them.  
User code sits between these two layers, making calls to the infrastructure
libraries beneath it and being scheduled and synchronized by the 
superstructure above it.  The configuration resembles a sandwich, as
shown in Figure~\ref{fig:TheESMFwich}.

The ESMF architecture is scalable, flexible paradigm for building highly 
complex climate, weather, and related applications from components such
as atmospheric models, land models, and data assimilation systems.  The 
ESMF is not a single master application into which all components must fit; 
rather it is a way of developing components so that they can be used 
in many different user-written applications.  Model components that adopt 
ESMF are designed to be usable in different contexts with minimal code
modification, and may be
incorporated into other ESMF-based modeling systems within the Earth 
science community.  In addition to high-level organization, ESMF provides 
a set of robust, portable, performance optimized libraries for regridding, 
data transfers, time management, and other common modeling functions.  
ESMF users may choose to extensively rewrite their codes to take advantage 
of the ESMF infrastructure, or they may decide to simply wrap user-written 
components in ESMF interfaces in order to adopt the ESMF architecture and 
utilize framework coupling services.

