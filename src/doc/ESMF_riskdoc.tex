% $ Id: $

\documentclass[english]{article}
\usepackage{babel}
\usepackage{supertabular}
\usepackage{html}
\usepackage{times}
\usepackage{graphicx}
\usepackage[T1]{fontenc}

\newcommand{\docmttype}{Risk Report}
\newcommand{\req}[1]{\section{\hspace{.2in}#1}}
\newcommand{\sreq}[1]{\subsection{\hspace{.2in}#1}}
\newcommand{\ssreq}[1]{\subsubsection{\hspace{.2in}#1}}
\newcommand{\mytitle}{DRAFT \longname \docmttype ~~}

\newenvironment
{reqlist}
{\begin{list} {} {} \rm \item[]}
{\end{list}}

%===============================================================================
% User-defined commands
%-------------------------------------------------------------------------------
\newcommand{\longname}{ESMF }
\newcommand{\funcname}{ESMF }
\newcommand{\shortname}{RSK}
\newcommand{\myauthors}{ESMF Technical Leads:  V. Balaji, Cecelia DeLuca, Chris Hill}
%===============================================================================
\setlength{\textwidth}{6.5truein}
\setlength{\textheight}{8.5truein}
\setlength{\oddsidemargin}{0in}
\setlength{\unitlength}{1truecm}

\begin{document}

\bodytext{BGCOLOR=white LINK=#083194 VLINK=#21004A}

% Title page

% $Id: title_alldoc.tex,v 1.14 2012/01/06 20:15:12 svasquez Exp $
%
% Earth System Modeling Framework
% Copyright 2002-2013, University Corporation for Atmospheric Research, 
% Massachusetts Institute of Technology, Geophysical Fluid Dynamics 
% Laboratory, University of Michigan, National Centers for Environmental 
% Prediction, Los Alamos National Laboratory, Argonne National Laboratory, 
% NASA Goddard Space Flight Center.
% Licensed under the University of Illinois-NCSA License.


\begin{titlepage}

\begin{center}
{\Large Earth System Modeling Framework } \\
\vspace{.25in}
{\Large {\bf \mytitle}} \\
\vspace{.75in}
{\large {\it \myauthors}} \\
\vspace{.25in}
{\large {\today}}
\vspace{.25in}
\end{center}

\begin{latexonly}
\vspace{4.5in}
\begin{tabular}{p{5in}p{.9in}}
\hrulefill \\
\noindent {\bf NASA Earth Science Technology Office} \\
\noindent Computational Technologies Project \\
\noindent CAN 00-OES-01 \\
\noindent http://www.earthsystemmodeling.org \\
\end{tabular}
\end{latexonly}

\end{titlepage}
















\newpage
\tableofcontents

\newpage
\section{Synopsis}

This document lists a series of risks that the Earth System Modeling Framework project faces
in delivering a system that meets the requirements of a community standard with

\begin{enumerate}
\item adequate performance
\item general present and future applicability to a range of applications and a broad mix of hardware platforms
\item reasonable ease of adoption
\item inbuilt support for ongoing extensibility
\end{enumerate}

Risks have been categorized as {\bf Very High}, {\bf High}, {\bf Medium}, {\bf Fairly Low} and {\bf Low}.
This classification is used for both measuring impact on the ESMF project ( the {\it Level of Impact On Project})
category and for quantifying the currently perceived exposure to the risk within
the project ( the {\it Problem Likelihood} attribute), so that the
status of the risk is a pair of values, such as ({\bf Low,Low}), that expresses
both the risk seriousness (the first element) and the risk impact (the second element) on
ESMF.  

ESMF is funded on a strict budget and "hard" milestone basis. Special attention needs to be
paid to risks that affect the milestone schedule.
This document is reviewed on a monthly basis by the technical lead team and managerial adjustments
made as appropriate. Typical adjustments are expected to include updates to notes on specific risks,
occaisional risk status adjustments and occaisional addition of new risks.
Preventitive measures are to be taken for any risk that has a status greater than {\bf Medium, Medium}.
Fix actions are taken for risks with a ({\bf High, Medium}) status or higher.
The technical lead team will keep the NASA CT {\it milestone validation team} informed of 
risk issue status at their regular monthly telecons and will also involve project PI's where 
appropriate.

\subsection{Identifying Risks}
ESMF project members will be responsible for
\begin{itemize}
\item Proposing and assessing new risk issues.
\item Assessing progress in mitigating risks.
\item Reviewing and commenting on risk level.
\end{itemize}
These issues will be communicated via the standard 
project mechanisms of teleconferences and email to the technical lead team who will incorporate updates
into the document as appropriate.

\section{ESMF Development Risks}

This section deals with risks that are related to the overall goal of the ESMF project and
to specific technical issues that ESMF faces.

\req{Community Adoption Issues}
ESMF aims to be a widely used community standard. As such it faces a number of challenges that are
related to a mix of technical and organizational factors.
\sreq{Inadequate community dialog}
\begin{reqlist}
{\bf Level of Impact On Project:}  High \\
{\bf Nature of Impact:} ESMF is aimed to a community standard. It won't be one if it isn't used.\\
{\bf Problem Likelihood:} Fairly Low\\
{\bf Preventive Measures Being Taken:} Community meetings. Active participation by project
members in related community efforts. Deployment of wide number of community codes
onto the framework. Attendance by project participants at related project
meetings.\\
{\bf Fix:} Increase community pariticpation in project. Improve features to make
adoption more compeling.\\
{\bf Notes:} \\
\end{reqlist}

\sreq{Unable to respond to all community needs}
\begin{reqlist}
{\bf Level of Impact On Project:}  Medium \\
{\bf Nature of Impact:} If ESMF is perceived to be unresponsive to needs and unable to adapt to needs community adoption
will be in jeopardy.\\
{\bf Problem Likelihood:} Medium \\
{\bf Preventitive Measures Being Taken:} Ongoing dialog to identify an prioritize
community needs. Critical requirements will be adopted to the extent
that contractual obligactions admit. Track requirements that cannot be accomodate during the
project timeframe. Plan and design for extensibility and future community contributions.\\
{\bf Fix:} Seek resources for requirements that cannot be accomodated within current
contract obligations. Discuss contract modifications with NASA if truly critical
need (as determined by advisory board members) can not be addressed. \\
{\bf Notes:} \\
\end{reqlist}

\sreq{Overcommitment to support community}
\begin{reqlist}
{\bf Level of Impact On Project:} High \\
{\bf Nature of Impact:} Devoting resources to community support at the expense of contractual obligations
could impact milestone achievement.\\
{\bf Problem Likelihood:} Medium \\
{\bf Preventitive Measures Being Taken:} Clear emphasis of project milestones. Implementation
team only accountable to technical leads.\\
{\bf Fix:} Down priortize community needs that do not align with contractual obligations.
Discuss with NASA sponsors possible amends to contract terms if appropriate.\\
{\bf Notes:} \\
\end{reqlist}

\sreq{Unable to achieve sufficient community adoption in project time frame}
\begin{reqlist}
{\bf Level of Impact On Project:}  High \\
{\bf Nature of Impact:} The long-term viability of ESMF will rest on its broad use and a perception of
its value to a large user base. This will take time, possibly longer
than the three-year project window.\\
{\bf Problem Likelihood:} Fairly low \\
{\bf Preventitive Measures Being Taken:} Broad range of dialog and visibility 
forums being maintained. Example codes, extensive documentation and tutorial
workshops are being planned.
Close monitoring of experiences of collaborator groups.  \\
{\bf Fix:} Solicit non-adopter input on what is missing or why framework is not working for them.\\
{\bf Notes:} \\
\end{reqlist}

\sreq{Framework too difficult and too complex for users.}
\begin{reqlist}
{\bf Level of Impact On Project:}  High \\
{\bf Nature of Impact:} Community adoption will not occur at sufficient level.\\
{\bf Problem Likelihood:} Fairly low \\
{\bf Preventitive Measures Being Taken:}
Incremental development, review of prototype designs and feedback from users
will allow discovery of issues. Feedback will also be solicited through the
project web site. Independent integration by the Hauser and Mechoso investigator
teams and one independent adopter will provide additional validation.
The feedback will be used to focus the teams design and development efforts on 
specific parts of the framework that require clean, flexible and easy to use design.
\\
{\bf Fix:} Solicit non-adopter input on what is missing or why framework is not working for them.\\
{\bf Notes:} \\
\end{reqlist}

\req{Superstructure Level Issues}
\sreq{Arbitrary SPMD and MPMD support}
ESMF is required to manage and direct arbitrary collections of SPMD and MPMD programs 
running across a potentially heterogeneous mix of platforms.
\begin{reqlist}
{\bf Level of Impact On Project:} High \\
{\bf Nature of Impact:} Inability to achieve a sufficiently robust and maintainable
solution would limit ramge of code architectures that could work with ESMF.\\
{\bf Problem Likelihood:} Low \\
{\bf Preventitive Measures Being Taken:} Detailed prototype studies to demostrate all required
combinations.\\
{\bf Fix:} Reduce scope of support to a tractable level.\\
{\bf Notes:}
\end{reqlist}

\req{System Issues}
\sreq{Interoperation between languages}
ESMF will use a mix of Fortran, C and C++ languages. Several interlanguage binary binding
issues need to be addressed covering both normal execution and exceptional
behavior.
\begin{reqlist}
{\bf Level of Impact On Project:} High \\
{\bf Nature of Impact:} Multi-language support could create complex development
issues that draw resources away from other areas.\\
{\bf Problem Likelihood:} Low\\
{\bf Preventitive Measures Being Taken:} Significant prototyping in being undertaken and
the language boundaries are being very clearly delineated.\\
{\bf Fix:} Entire framework would need to be in C/C++.\\
{\bf Notes:} Reimplementation in C/C++ would be time-consuming but
otherwise straightforward.
\end{reqlist}

\sreq{F90 Continued Availability}
Fortran 90/95 remain {\it fringe} languages, even though most of the Earth system
community uses them. Two consequences of this are that ESMF Fortran 90 code will
rely on vendor supplied compilers that have an ever decreasing user base and
that there is no true open-source compiler for Fortran 90/95.
This impacts ESMF directly because some of its code will use F90 and
indirectly because many of its components exist in F90 only.
\begin{reqlist}
{\bf Level of Impact On Project:} High \\
{\bf Nature of Impact:} Code that is written in F90 would no longer be portable
and components would only be available on certain platforms. \\
{\bf Problem Likelihood:} Low\\
{\bf Preventitive Measures Being Taken:} Some framework development is being pursued in C/C++.
Availability and quality of Fortran 90/95 compilers
is being tracked.\\
{\bf Fix:} Entire framework would need to be in C/C++.\\
{\bf Notes:} Reimplementation in C/C++ would be time-consuming but
otherwise straightforward. For components the impact would be far more
serious, but this is outside the scope of ESMF.
Components in Fortran 77, C and C++ would not be affected by lack of 
compiler availability as compilers exist almost all hardware for these languages.
\end{reqlist}

\sreq{Vector Platform Availability of ESMF}
\begin{reqlist}
{\bf Level of Impact On Project:} Medium \\
{\bf Nature of Impact:} The framework might not build and install "out of the box"
on a vector system.\\
{\bf Problem Likelihood:} Medium\\
{\bf Preventitive Measures Being Taken:} ESMF being tested and built on a range of *NIX systems. This will
ensure fairly easy protability to other systems, however, effort will still be needed 
to validate ESMF code and installation on a new platform. \\
{\bf Fix:} Test and modify to build on apropriate system\\
{\bf Notes:} This would require some resources but otherwise would not be a major challenge.
\end{reqlist}

\sreq{MS Windows Availability of ESMF}
\begin{reqlist}
{\bf Level of Impact On Project:} Fairly Low \\
{\bf Nature of Impact:} The framework might not build and install "out of the box"
on a Windows system.\\
{\bf Problem Likelihood:} High \\
{\bf Preventitive Measures Being Taken:}  ESMF being tested and built on a range of *NIX systems.
No specific tests of build actions are envisaged to ensure Windows compatability.\\
{\bf Fix:} Test and modify to build on a Windows system\\
{\bf Notes:} This would require resources but otherwise would not be a major challenge.
\end{reqlist}

\sreq{Other OS Validation}
\begin{reqlist}
{\bf Level of Impact On Project:} Low \\
{\bf Nature of Impact:} he framework might not build and install "out of the box"
on a non-validated OS.\\
{\bf Problem Likelihood:} High \\
{\bf Preventitive Measures Being Taken:} ESMF being tested and built on a range of *NIX systems\\
{\bf Fix:}  Test and modify to build on appropriate system\\
{\bf Notes:} This could require resources but otherwise would not be a major challenge.
However, for systems where no F90 compiler is available resource requirements could be substantial.
\end{reqlist}

\req{Infrastructure - Fields and Grid Level Issues}
\sreq{Too many complex features}
Many different grids will be supported under ESMF. A manageable and scalable approach
to development and testing is required to enable robust support for different grids.
\begin{reqlist}
{\bf Level of Impact On Project:} High \\
{\bf Nature of Impact:} Increased code complexity could lead to
difficulty of use, problems with portability and future evolution challenges.\\
{\bf Problem Likelihood:} Medium \\
{\bf Preventitive Measures Being Taken:} A layered interface has been defined 
which highlights generic grid properties and expresses them through common abstractions.
Many groups will attempt to deploy the system in their own in-house
applications and their experiences will be fed-back into the system design.\\
{\bf Fix:} Interface redesign \\
{\bf Notes:} 
\end{reqlist}

\sreq{Inadequate performance for certain grids}
Grid operations, like halo-update and transpose, can have significant impact on parallel scaling.
Regriddings between different grids can also greatly impact scaling.
A general purpose system that can scale and perform as well as specialized, application and
grid specific forms, could be hard to devise. Deficiencies in this areana will
impact adoption.
\begin{reqlist} 
{\bf Level of Impact On Project:} High \\ 
{\bf Nature of Impact:} Low performance would impact community adoption and
would impact milestone acievement.\\
{\bf Problem Likelihood:} Medium \\
{\bf Preventitive Measures Being Taken:} Quantitative measures of grid primitives
performance are being devised and will be tracked.\\
{\bf Fix:} Develop specialized forms and increase level of resources
to developing these parts of the system.\\
{\bf Notes:} 
\end{reqlist}

\sreq{Inadequate performance for certain platforms}
Grid operations can be specialized to take advatage of platform specific
architectural features. In a general purpose system it can be hard to compete with these
specialized forms. Low performance in this area could deter community adoption of ESMF.
\begin{reqlist}
{\bf Level of Impact On Project:} Medium \\
{\bf Nature of Impact:} Low performance would impact community adoption and
would impact milestone acievement.\\
{\bf Problem Likelihood:} Medium \\
{\bf Preventitive Measures Being Taken:} Quantitative measures of grid primitives
performance are being devised and will be tracked.\\
{\bf Fix:} Develop specialized forms and increase level of resources
to developing these parts of the system. Reduce feature set for which
optimized support is developed.
\\
{\bf Notes:} 
\end{reqlist}

\req{Infrastructure - Utilities Level Issues}
Inadequate resources for developing highly portable and versatile
signaling, performance monitoring, I/O, communication, memory sharing and namespace sharing tools.
\begin{reqlist}
{\bf Level of Impact On Project:} High \\
{\bf Nature of Impact:} \\
{\bf Problem Likelihood:} Medium \\
{\bf Preventitive Measures Being Taken:} Tools developed under other projects (for example PetSC, ALICE, TAU)
will be adopted wherever possible.
\\
{\bf Fix:} Devote more resources to this area. Reduce feature set for which
highly optimized support is provided initially. \\
{\bf Notes:} 
\end{reqlist}


\req{Project Management}
ESMF involves many institutions and a geographical distributed team. 
This can create competing time pressures and difficulty in communication.

\begin{reqlist}
{\bf Level of Impact On Project:} High \\
{\bf Nature of Impact:} Communication difficulties could result in system quality and feature deficiencies. \\
{\bf Problem Likelihood:} Low \\
{\bf Preventitive Measures Being Taken:} Detailed scheduling. Frequent (bi-weekly) telephone
conferencing. Quarterly meetings. Location of core framework implementation and development
team in one place. Easy sharing of managed development code through source forge. 
Extensive use of e-mail lists. Frequent progress reviews and open dialog on project status.
\\
{\bf Fix:} Hosting of team members at NCAR where needed.
Site visits by development team members will be undertaken as needed.
\\
{\bf Notes:} 
\end{reqlist}

\sreq{Overambitious scheduling of new development}
\begin{reqlist}
{\bf Level of Impact On Project:} Medium \\
{\bf Nature of Impact:} Slippages in scheduling could delay dependent development and deployment efforts and could impact milestone payments.  \\
{\bf Problem Likelihood:} Medium \\
{\bf Preventitive Measures Being Taken:} Early prototyping efforts to explore most
complex areas of development.
\\
{\bf Fix:} Reductions in feature set if needed.
\\
{\bf Notes:} 
\end{reqlist}


\sreq{QA Coding Standards may be ignored}
THe core framework and EVA suite are required to meet very specific agreed code quality and coding
standards for milestone acceptance. Failure to develop code that meets the
standard will cause project funding delays.
\begin{reqlist}
{\bf Level of Impact On Project:} High \\
{\bf Nature of Impact:} \\
{\bf Problem Likelihood:} Fairly Low \\
{\bf Preventitive Measures Being Taken:} Developers are aware of the
coding standards and a peer review and independent QA validation process
has been established.
\\
{\bf Fix:} Closer involvement of external QA team in ongoing code 
assesment. Development of automated tools to assist in code standard validation.
\\
{\bf Notes:} 
\end{reqlist}


\sreq{Cross-project milestone coordination}
Several phases in the project rely on earlier milestones under a separate
project. Dependent phases could be seriously impacted by earlier delays.
\begin{reqlist}
{\bf Level of Impact On Project:} High \\
{\bf Nature of Impact:} \\
{\bf Problem Likelihood:} Fairly Low \\
{\bf Preventitive Measures Being Taken:} Joint technical lead team has representation
from all projects. Weekly ESMF wide telecons are used to track upcoming
milestones. Project phases are planned jointly and timelines mapped out
together at joint meetings. Technical lead team tracks milestone
work schedule and coordinates multiple milestone countdown reviews prior 
to milestones.
\\
{\bf Fix:} Reassign milestone work responsibilities as appropriate.
Increase NASA CT {\it milestone validation team} involvement in pre-milestone
countdown reviews.
\\
{\bf Notes:} 
\end{reqlist}

\section{Generic Project Risks}

This section documents how more general risk issues, that are faced by all complex projects,
are addressed with ESMF.

\sreq{Staffing issues}
\begin{reqlist}
{\bf Level of Impact On Project:} \\
{\bf Nature of Impact:} \\
{\bf Problem Likelihood:} \\
{\bf Preventitive Measures Being Taken:} \\
{\bf Fix:} \\
{\bf Notes:} 
\end{reqlist}

\end{document}
