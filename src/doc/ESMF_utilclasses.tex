\section{Infrastructure: Utilities}
\label{sec:utilclasses}

\subsection{Introduction}

The previous Infrastructure section described the data handling 
objects in the ESMF library.
This section describes the lowest level
objects in the Infrastructure layer, the Utility classes.

At first glance the Utility classes have very little in common;
some are lightweight classes which maintain no internal state; 
others have state and exist for the lifetime of the application.  
Some run within a single process; others exist per processor and 
may communicate to other instantiations of itself on different processors.
The advantage of collecting these functions 
into a base-level utility set is to help avoid circular referencing.
In most cases they occupy the lowest level of the object hierarchy.

Methods provided by Utility classes are available for use by  
all other classes in the ESMF library.  
Some of these methods will be called directly by user-supplied
code; others are intended to provide support for internal ESMF 
library code.

\subsection{Overview}

<< by their very nature these utility routines have very little 
in common.  after reading this section and the following ones,
i'd like to know if people think these short overview paragraphs 
are useful, or do they duplicate the details which follow in the 
class detail sections?   nsc. >>

The Layout object is used by any part of the library which needs
to decompose a task/problem/dataset into subsets.
The decomposition must allow for mapping certain subsets onto the
underlying machine model in a preferential way; e.g. keeping
vertical blocks of a 3D decomposition on processors which share
a memory pool.
It is required to interact with both the Processing Element lists
and the Machine Model in order to accomplish this preferential
decomposition.

The abstraction of a Processor Element provides a layer between
the underlying hardware processor ID and the application's view
of how to address individual processors.  It also maintains
grouping information to indicate collections of processors which
share some characteristic or resource.

The Machine Model encapsulates all information about the
hardware configuration, memory structure, and available
communications protocols and bandwith.  It furnishes this
information to other parts of the library which can
make decisions about partitioning or selection of algorithms
based on the amount and types of resources available.

The Basic Communications routines provide a uniform interface to
the variety of interprocess communication mechanisms available
on different hardware platforms.  
The basic functions include methods to send
and receive data, such as scatter, gather, send, receive,
reduction methods such as sum, global min/max, and methods
to synchronize such as barrier.  << does this need more
detail, or is this level ok for this point in the document??  nsc. >>

The Time Manager provides all general purpose time methods, both
for computing time instants (dates and times) and time intervals
(the difference between 2 time instants).   It also supports 
setting and handling alarm events, either one-shot or repeating.
Supported calendars include Gregorian, Julian, no-leap, 360-day, 
generic, and no-calendar.
Time intervals can range from the very brief -- any rational fraction
of a second -- to geologic time scales.
The Time Manager also allows the user to control the format of
time instants and intervals as they are returned.

Many parts of the library are required to maintain unique name
spaces for their objects.  The Name Registry supplies methods
for defining name spaces and adding and querying names in those
spaces.  In general there is will be one Registry object per
process, but if the namespace spans processes then Registry objects
will have to do global communication in order to verify names
are unique within the entire application.

The Error Handler provides both uniform handling of errors and
a way for user code to select how errors will be handled.
An integer error code can be returned from the library to the
calling code, or the library can print an error message and exit.

The Log Utility manages the complexity of serializing diagnostic
messages produced by a multiprocess application.  It provides
methods for uniformly formatting the messages to facilitate
post-processing by filters and automated tools.

The Diagnostic routines provide assistance with both performance
profiling and code debugging.  Timing routines support measurement
of both serial and parallel time intervals around certain sections
of code.  Debugging routines allow dynamic control over the level
of detail, enabling or disabling of different functional categories,
and assist with bit-level comparisons for debugging strategies which 
involve running a simulation under differing conditions and
comparing intermediate results.  These routines use methods from
the Time class for computing profile information, and from 
the Log class to collect and output results.


\subsection{Class Details}

\subsubsection{Layout (ESMF\_Layout)}
\label{sec:layout} 
\begin{description}
\item [Description] A layout is a description of a computational domain that
may describe the decomposition of an Application, a Component, a Field Group, a Field, or 
a Distributed Grid.
If no Layout is specified for an object, it can inherit its layout from an object
higher in the data hierarchy.  
\item [Function] The Layout class maintains an N-dimensional set of decomposition
elements.  It provides methods for querying the set and requesting different
types of decompositions.  It maintains the mapping of tasks to decompositions.
\end{description}

\subsubsection{Processor Element list (ESMF\_PElist)}
\label{sec:pelist} 
\begin{description}
\item [Description] A Processor Element list is a list of physical processor IDs
available to participate in the computation.  
It includes a unique identifier for each PE which may differ from the
hardware processor ID, 
information about what memory or communication groupings it belongs to (e.g. subsets 
of PEs which share the same physical memory), 
and other identifiers for virtual resources.  
\item [Function] The PElist class encapsulates information needed to schedule
jobs in parallel, and is aggregated into the Layout class (see Section~\ref{layout}).
It provides query methods to the Layout object, and some limited query methods
may be available at the application level.
\end{description}

\subsubsection{Machine Model (ESMF\_MachineModel)}
\label{sec:machinemodel} 
\begin{description}
\item [Description] A Machine Model describes the characteristics of memory and
communication characteristics which influence choices for accomplishing a parallel 
decomposition of a large problem.  For the most part the application does not
interact with this class of object, although some query routines are provided.
\item [Function] The MachineModel class encapsulates all hardware dependent information
about the memory subsystem (e.g. physical shared memory, distributed shared memory, clustered
memory), what communication transport mechanisms are available (MPI, OpenMP, etc), 
whether the processors are homogeneous or heterogeneous, and 
any other relevant hardware information (e.g. processor speed).
It is aggregated into the Layout class (see Section~\ref{layout}), and provides
methods for querying all characteristics needed by the Layout object to do 
problem decomposition and distribution of work.
The MachineModel class does expose some limited query methods to the application, 
to allow for the possibility of making run-time choices of algorithms based on hardware type.
\end{description}

\subsubsection{Machine (ESMF\_Machine)} 
\begin{description}
\item [Note] << is this the same as my MachineModel above??  nsc. >>
\item [Description] The Machine class provides a representation of 
key features of computer hardware and system software.  These
features include memory attributes and configuration, processor type and speed,
interconnect attributes, and system library availability.
\item [Function]
The main purpose of the Machine is to store hardware and system software
information needed by the framework or application programmer in a general
form, but with little abstraction.  This information can be used to perform resource 
allocation, data distribution, and dynamic load balancing.  The Machine can be queried
for platform type(s), number of processors, number of threads, and number of 
nodes.  It may optionally provide information on quantities such as bandwidth and 
latency through active tests.  
\end{description}

\subsubsection{Basic Communications (ESMF\_BasicComm)}
\label{sec:basiccomm} 
\begin{description}
\item [Description] This library is a wrapper for MPI and other vendor-supplied 
message passing libraries.
\item [Function] The Basic Communication library provides a generic interface
and efficient communications for the ESMF.  Methods include scatter, gather, send,
receive, synchronize. 
\end{description}

\subsubsection{Time Manager (ESMF\_TimeMgr)}
\label{sec:timemgr} 
\begin{description}
\item [Description] The Time Manager provides time services to other objects
in the ESMF library and to application code.  It provides either absolute
time functions called Instants, or time differences called Intervals.
\item [Function] The Time Manager provides all general purpose time methods. 
It provides methods for supporting a variety of calendar time and dates,
for computing time intervals and instants from the very brief -- any 
rational fraction of a second -- to geologic time scales, and for
setting and handling alarm events, either one-time or repeating.
\end{description}

\subsubsection{Registry (ESMF\_Registry)}
\label{sec:registry} 
\begin{description}
\item [Description] A Registry maintains a list of (name, ID) pairs per 
namespace.  It supports the many classes in the system which are
required to have unique names, either per address space or over the entire
application.  
\item [Function] The Registry class supports multiple namespaces (e.g. per
field, per component, per exchange packet pool).  It provides methods to
create, query, and list namespaces and define their scope.  
Within a namespace it provides methods to
verify a name exists, add a (name, ID) pair if the name is unique,
generate a new unique name, retrieve an ID by name, and list existing names.
\end{description}

\subsubsection{Error Handler (ESMF\_Error)}
\label{sec:error} 
\begin{description}
\item [Description] The Error Handler allows the application to control 
the behavior of the library in case of error.  The library can return
control to the calling code with an integer error code and allow the
caller to handle the error.  An Error Print routine can
supply uniform text error messsage in addition to the integer error code.
Alternatively, the Error Handler can print an error message and exit.
\item [Function] The Error Handler class provides methods for selecting
the behavior of the library in case of error, methods to return uniform
text in addition to integer error codes, and methods to print the file
name, line number, and a description of the error before exiting the process.
\end{description}

\subsubsection{Log (ESMF\_Log)}
\label{sec:log} 
\begin{description}
\item [Description] The Log utility is intended to organize diagnostic
output, which may be generated in parallel at unpredictable times in
a multiprocessor environment.  It also attempts to organize
the diagnostic output so that searches and filters may be easily constructed. 
\item [Function] The Log class will be for diagnostic output. 
The bandwidth is assumed to be moderate to small, i.e. it will not be used 
to output large streams of numerical model output data.  
It provides methods for writing to the log, for specifying uniform
formatting information, for selecting a single unified log vs.\  a log 
file per processor.
\end{description}

\subsubsection{Diagnostics (ESMF\_Diagnostics)}
\label{sec:diagnostics} 
\begin{description}
\item [Description] Diagnostics are collections of routines for
gathering information about how the simulation is behaving.  
These include both routines for finding program bugs
as well as understanding execution bottlenecks.
\item [Function] The Diagnostic class consists of two subclasses,
a Debug class for supporting methods for specifying
diagnostic information, and enabling and disabling the production of
that information.  The Performance class provides methods
for enabling and disabling timing routines in sections of
code, in both serial and parallel execution modes.
\end{description}



% \subsubsection{Foo (ESMF\_Foo)}
% \label{sec:Foo} 
% \begin{description}
% \item [Description] A Foo is ...
% \item [Function] The Foo class ...
% \end{description}




