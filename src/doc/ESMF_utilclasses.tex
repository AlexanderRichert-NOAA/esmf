\section{Utility Routines}

\subsection{Introduction}

Utilities that may be utilitized by any other class in the ESMF.  
Collecting these functions into a base-level utility set helps to 
avoid circular referencing.

These routines are callable by user level code as well.

Other stuff for in here: log, error

More on time manager.

\subsection{Overview}

<< some stuff here >>

\subsection{Class Details}

\subsubsection{Layout (ESMF\_Layout)}
\label{sec:layout} 
\begin{description}
\item [Description] A layout is a description of a computational domain that
may describe the decomposition of an Application, a Component, a Field Group, a Field, or 
a Distributed Grid.
If no Layout is specified for an object, it can inherit its layout from an object
higher in the data hierarchy.  
\item [Function] The Layout class maintains an N-dimensional set of decomposition
elements.  It provides methods for querying the set and requesting different
types of decompositions.  It maintains the mapping of tasks to decompositions.
\end{description}

\subsubsection{Processor Element list (ESMF\_PElist)}
\label{sec:pelist} 
\begin{description}
\item [Description] A Processor Element list is a list of physical processor IDs
available to participate in the computation.  
It includes a unique identifier for each PE which may differ from the
hardware processor ID, 
information about what memory or communication groupings it belongs to (e.g. subsets 
of PEs which share the same physical memory), 
and other identifiers for virtual resources.  
\item [Function] The PElist class encapsulates information needed to schedule
jobs in parallel, and is aggregated into the Layout class (see Section~\ref{layout}).
It provides query methods to the Layout object, and some limited query methods
may be available at the application level.
\end{description}

\subsubsection{Machine Model (ESMF\_MachineModel)}
\label{sec:machinemodel} 
\begin{description}
\item [Description] A Machine Model describes the characteristics of memory and
communication characteristics which influence choices for accomplishing a parallel 
decomposition of a large problem.  For the most part the application does not
interact with this class of object, although some query routines are provided.
\item [Function] The MachineModel class encapsulates all hardware dependent information
about the memory subsystem (e.g. physical shared memory, distributed shared memory, clustered
memory), what communication transport mechanisms are available (MPI, OpenMP, etc), 
whether the processors are homogeneous or heterogeneous, and 
any other relevant hardware information (e.g. processor speed).
It is aggregated into the Layout class (see Section~\ref{layout}), and provides
methods for querying all characteristics needed by the Layout object to do 
problem decomposition and distribution of work.
The MachineModel class does expose some limited query methods to the application, 
to allow for the possibility of making run-time choices of algorithms based on hardware type.
\end{description}

\subsubsection{ (ESMF\_Machine)} 
\begin{description}
\item [Note] << is this the same as my MachineModel above??  nsc. >>
\item [Description] The Machine class provides a representation of 
key features of computer hardware and system software.  These
features include memory attributes and configuration, processor type and speed,
interconnect attributes, and system library availability.
\item [Function]
The main purpose of the Machine is to store hardware and system software
information needed by the framework or application programmer in a general
form, but with little abstraction.  This information can be used to perform resource 
allocation, data distribution, and dynamic load balancing.  The Machine can be queried
for platform type(s), number of processors, number of threads, and number of 
nodes.  It may optionally provide information on quantities such as bandwidth and 
latency through active tests.  
\end{description}

\subsubsection{Basic Communications (ESMF\_BasicComm)}
\begin{description}
\item [Description] This library is a wrapper for MPI and other vendor-supplied 
message passing libraries.
\item [Function] The Basic Communication library provides a generic interface
and efficient communications for the ESMF.  Methods include scatter, gather, send,
receive, synchronize. 
\end{description}

\subsubsection{Time Manager (ESMF\_Date, ESMF\_DT)}
\begin{description}
\item [Description] The date and time interval methods in the ESMF provide date
calculations based on a number of different calendars.

\end{description}






