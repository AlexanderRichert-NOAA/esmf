% $Id$
%
% Earth System Modeling Framework
% Copyright 2002-2018, University Corporation for Atmospheric Research, 
% Massachusetts Institute of Technology, Geophysical Fluid Dynamics 
% Laboratory, University of Michigan, National Centers for Environmental 
% Prediction, Los Alamos National Laboratory, Argonne National Laboratory, 
% NASA Goddard Space Flight Center.
% Licensed under the University of Illinois-NCSA License.

\section{The ESMF Application Programming Interface}

The ESMF Application Programming Interface (API) is based on the
object-oriented programming concept of a {\bf class}.  A class is a 
software construct that is used for grouping a set of related variables 
together with the subroutines and functions that operate on them.  We 
use classes in ESMF because they help to organize the code, and often 
make it easier to maintain and understand.  A particular instance
of a class is called an {\bf object}.  For example, Field is an 
ESMF class.  An actual Field called {\tt temperature} is an object. 
That is about as far as we will go into software engineering
terminology.  

The C interface is implemented so that the variables associated
with a class are stored in a C structure.  For example, an 
{\tt ESMC\_Field} structure stores the data array, grid 
information, and metadata associated with a physical field.
The structure for each class is defined in a C header file. 
The operations associated with each class are also
defined in the header files.

The header files for ESMF are bundled together and can be accessed with a 
single {\tt include} statement, {\tt \#include "ESMC.h"}.  By convention,
the C entry points are named using ``ESMC'' as a prefix.

\subsection{Standard Methods and Interface Rules}

ESMF defines a set of standard methods and interface rules that
hold across the entire API.  These are: 

\begin{itemize}

\item
\begin{sloppypar}
{\tt ESMC\_<Class>Create()} and {\tt ESMC\_<Class>Destroy()}, for
creating and destroying objects of ESMF classes that require internal memory
management (- called ESMF deep classes). The {\tt ESMC\_<Class>Create()} method
allocates memory for the object itself and for internal variables, and
initializes variables where appropriate.  It is always written as a
function that returns a derived type instance of the class, i.e. an object.
\end{sloppypar}

\item 
\begin{sloppypar}
{\tt ESMC\_<Class>Set()} and {\tt ESMC\_<Class>Get()}, for setting and 
retrieving a particular item or flag.  In general, these methods are overloaded
for all cases where the item can be manipulated as a name/value pair.  If
identifying the item requires more than a name, or if the class is of
sufficient complexity that overloading in this way would result in an
overwhelming number of options, we define specific
{\tt ESMC\_<Class>Set<Something>()} and {\tt ESMC\_<Class>Get<Something>()}
interfaces.
\end{sloppypar}

\item {\tt ESMC\_<Class>Add()}, {\tt ESMC\_<Class>AddReplace()},
{\tt ESMC\_<Class>Remove()}, and {\tt ESMC\_<Class>Replace()}, for manipulating
objects of ESMF container classes - such as {\tt ESMC\_State} and
{\tt ESMC\_FieldBundle}. For example, the {\tt ESMC\_FieldBundleAdd()}
method adds another Field to an existing FieldBundle object.

\item {\tt ESMC\_<Class>Print()}, for printing the contents of an object to 
standard out.  This method is mainly intended for debugging.

\item {\tt ESMC\_<Class>ReadRestart()} and {\tt ESMC\_<Class>WriteRestart()}, 
for saving the contents of a class and restoring it exactly.  Read
and write restart methods have not yet been implemented for most
ESMF classes, so where necessary the user needs to write restart 
values themselves.

\item 
\begin{sloppypar}
{\tt ESMC\_<Class>Validate()}, for determining whether a class is 
internally consistent.  For example, {\tt ESMC\_FieldValidate()} validates
the internal consistency of a Field object.
\end{sloppypar}

\end{itemize}

\subsection{Deep and Shallow Classes}
\label{sec:deepshallow}

The ESMF contains two types of classes.

{\bf Deep} classes require
{\tt ESMC\_<Class>Create()} and {\tt ESMC\_<Class>Destroy()} calls.
They involve memory allocation, take significant time to set up (due to
memory management) and should not be created in a time-critical portion of code.
Deep objects persist even after the method in which they were created has
returned. Most classes in ESMF, including GridComp, CplComp, State, Fields, FieldBundles, Arrays, ArrayBundles, Grids, and Clocks, fall into this category.

{\bf Shallow} classes do not possess {\tt ESMC\_<Class>Create()}
and {\tt ESMC\_<Class>Destroy()} calls.  They are simply declared
and their values set using an {\tt ESMC\_<Class>Set()} call.  
Examples of shallow classes are Time, TimeInterval, and ArraySpec.  Shallow classes do not take long to set up and can be declared and set within
a time-critical code segment.  Shallow objects stop existing when
the method in which they were declared has returned.  

An exception to this is when a shallow object, such as a Time, 
is stored in a deep object such as a Clock.  The Clock then
carries a copy of the Time in persistent memory.  The Time is
deallocated with the {\tt ESMC\_ClockDestroy()} call.

See Section \ref{sec:overallimpl}, Overall Design and Implementation 
Notes, for a brief discussion of deep and shallow classes from 
an implementation perspective.  For an in-depth look at the design
 and inter-language issues related to deep and shallow classes,
see the \htmladdnormallink{{\it ESMF Implementation Report}}{http://www.earthsystemmodeling.org/documents/IMPL\_repdoc/}.

\subsection{Special Methods}

The following are special methods which, in one case,
are required by any application using ESMF, and in the 
other case must be called by any application that is using 
ESMF Components.

\begin{itemize}

\item {\tt ESMC\_Initialize()} and {\tt ESMC\_Finalize()} are required 
methods that must bracket the use of ESMF within an application.  
They manage the resources required to run ESMF and shut it down
gracefully.  ESMF does not support restarts in the same executable, i.e.
{\tt ESMC\_Initialize()} should not be called after {\tt ESMC\_Finalize()}.
\item {\tt ESMC\_<Type>CompInitialize()}, {\tt ESMC\_<Type>CompRun()}, and 
{\tt ESMC\_<Type>CompFinalize()} are component methods that are used at the 
highest level within ESMF.  {\tt <Type>} may be {\tt <Grid>}, for 
Gridded Components such as oceans or atmospheres, or
{\tt <Cpl>}, for Coupler Components that are used to connect 
them.  The content of these methods is not part of the ESMF.  
Instead the methods call into associated subroutines within 
user code.

\end{itemize}

\subsection{The ESMF Data Hierarchy}

The ESMF API is organized around a hierarchy of classes that
contain model data.  The operations that are performed
on model data, such as regridding, redistribution, and halo 
updates, are methods of these classes.  

The main data classes offered by the ESMF C API, in order of increasing complexity, are:
\begin{itemize}
\item {\bf Array} An ESMF Array is a distributed, multi-dimensional 
array that can carry information such as its type, kind, rank, and 
associated halo widths.  It contains a reference to a native language array.
\item {\bf Field}  A Field represents a physical scalar or vector field.
It contains a reference to an Array along with grid information and metadata.
\item {\bf State}  A State represents the collection of data that a 
Component either requires to run (an Import State) or can make 
available to other Components (an Export State).
States may contain references to Arrays, ArrayBundles, Fields,
FieldBundles, or other States. 
\item {\bf Component}  A Component is a piece of software 
with a distinct function.  ESMF currently recognizes two types 
of Components.  Components that represent a physical domain 
or process, such 
as an atmospheric model, are called Gridded Components since they are 
usually discretized on an underlying grid.  The Components 
responsible for regridding and transferring data between Gridded 
Components are called Coupler Components.  Each Component
is associated with an Import and an Export State.  Components
can be nested so that simpler Components are contained within more
complex ones.

\end{itemize}

Underlying these data classes are native language arrays.  ESMF Arrays 
and Fields can be queried for the C pointer to the actual data.  You can
perform communication operations either on the ESMF data objects or
directly on C arrays through the VM class, which serves 
as a unifying wrapper for distributed and shared memory communication 
libraries.

\subsection{ESMF Spatial Classes}
\label{sec:spatialclasses}

Like the hierarchy of model data classes, ranging from the 
simple to the complex, ESMF is organized around a hierarchy of
classes that represent different spaces associated with a computation.
Each of these spaces can be manipulated, in order to give
the user control over how a computation is executed.  For Earth system
models, this hierarchy starts with the address space associated
with the computer and extends to the physical region described by
the application.   The main spatial classes in ESMF, from
those closest to the machine to those closest to the application, are:

\begin{itemize}

\item The {\bf Virtual Machine}, or {\bf VM} The ESMF VM is an 
abstraction of a parallel computing environment that encompasses 
both shared and distributed memory, single and multi-core systems.
Its primary purpose is resource allocation and management. Each Component
runs in its own VM, using the resources it defines. The elements of a VM
are {\bf Persistent Execution Threads}, or {\bf PETs}, that are
executing in {\bf Virtual Address Spaces}, or {\bf VASs}. A simple
case is one in which every PET is associated with a single MPI process.
In this case every PET is executing in its own private VAS. If Components
are nested, the parent Component allocates a subset of its PETs to its
children. The children have some flexibility, subject to the constraints of
the computing environment, to decide how they want to use the
resources associated with the PETs they've received.

\item {\bf DELayout}  A DELayout represents a data decomposition
(we also refer to this as a distribution).  Its
basic elements are {\bf Decomposition Elements}, or {\bf DEs}.  
A DELayout associates a set of DEs with the PETs in a VM.  DEs are not
necessarily one-to-one with PETs.  For cache blocking,
or user-managed multi-threading, more DEs than PETs may be defined.
Fewer DEs than PETs may also be defined if an application requires it.

The current ESMF C API does not provide user access to the DELayout class.

\item {\bf DistGrid}  A DistGrid represents the index space
associated with a grid.  It is a useful abstraction because
often a full specification of grid coordinates is not necessary
to define data communication patterns.  The DistGrid contains
information about the sequence and connectivity of data points,
which is sufficient information for many operations.  Arrays
are defined on DistGrids.

\item {\bf Array} An Array defines how the index space described
in the DistGrid is associated with the VAS of each PET. This association
considers the type, kind and rank of the indexed data. Fields are
defined on Arrays.

\item {\bf Grid}  A Grid is an abstraction for a logically rectangular
region in physical space.  It associates a coordinate system, a set of
coordinates, and a topology to a collection of grid cells.  Grids in ESMF
are comprised of DistGrids plus additional coordinate information.

\item {\bf Mesh}  A Mesh provides an abstraction for an unstructured 
grid.  Coordinate information is set in nodes, which represent
vertices or corners.  Together the nodes establish the boundaries
of mesh elements or cells.

\item {\bf LocStream}  A LocStream is an abstraction for a set of 
unstructured data points without any topological relationship to each 
other.

\item {\bf Field}  A Field may contain more dimensions than the 
Grid that it is discretized on.  For example, for convenience 
during integration, a user may want to define a single Field object 
that holds snapshots of data at multiple times.  Fields also 
keep track of the stagger location of a Field data point within its 
associated Grid cell.

\end{itemize}

\subsection{ESMF Maps}

In order to define how the index spaces of the spatial classes relate
to each other, we require either implicit rules (in which case the
relationship between spaces is defined by default), or special Map arrays
that allow the user to specify the desired association.  The form of the 
specification is usually that the position of the array element carries
information about the first object, and the value of the array element carries
information about the second object.  ESMF includes a {\tt distGridToArrayMap},
a {\tt gridToFieldMap}, a {\tt distGridToGridMap}, and others.

\subsection{ESMF Specification Classes}

It can be useful to make small packets
of descriptive parameters.  ESMF has one of these:
\begin{itemize}
\item {\bf ArraySpec}, for storing the specifics, such as type/kind/rank,
of an array.
\end{itemize}

\subsection{ESMF Utility Classes}

There are a number of utilities in ESMF that can be used independently.
These are:
\begin{itemize}
\item {\bf Attributes}, for storing metadata about Fields,
FieldBundles, States, and other classes.
(Not currently available through the ESMF C API.)
\item {\bf TimeMgr}, for calendar, time, clock and alarm functions.
\item {\bf LogErr}, for logging and error handling.
\item {\bf Config}, for creating resource files that can replace namelists
as a consistent way of setting configuration parameters.
\end{itemize}

\section{Integrating ESMF into Applications}

Depending on the requirements of the application, the user may 
want to begin integrating ESMF in either a top-down or bottom-up 
manner.  In the top-down approach, tools at the superstructure 
level are used to help reorganize and structure the interactions
among large-scale components in the application.  It is appropriate
when interoperability is a primary concern; for example, when 
several different versions or implementations of components are going 
to be swapped in, or a particular component is going to be used 
in multiple contexts.  Another reason for deciding on a top-down 
approach is that the application contains legacy code that for 
some reason (e.g., intertwined functions, very large,
highly performance-tuned, resource limitations) there is little 
motivation to fully restructure.  The superstructure can usually be 
incorporated into such applications in a way that is non-intrusive.

In the bottom-up approach, the user selects desired utilities 
(data communications, calendar management, performance profiling,
logging and error handling, etc.) from the ESMF infrastructure 
and either writes new code using them, introduces them into 
existing code, or replaces the functionality in existing code 
with them.  This makes sense when maximizing code reuse and 
minimizing maintenance costs is a goal.  There may be a specific
need for functionality or the component writer may be starting
from scratch.  The calendar management utility is a popular
place to start.

\subsection{Using the ESMF Superstructure}

The following is a typical set of steps involved in adopting
the ESMF superstructure.  The first two tasks, which occur 
before an ESMF call is ever made, have the potential to be 
the most difficult and time-consuming.  They are the work 
of splitting an application into components and ensuring that
each component has well-defined stages of execution.  ESMF
aside, this sort of code structure helps to promote application
clarity and maintainability, and the effort put into it is likely
to be a good investment.

\begin{enumerate}

\item Decide how to organize the application as discrete Gridded 
and Coupler Components.  This might involve reorganizing code
so that individual components are cleanly separated and their 
interactions consist of a minimal number of data exchanges.

\item Divide the code for each component into initialize, run, and
finalize methods.  These methods can be multi-phase, e.g., 
{\tt init\_1, init\_2}.

\item Pack any data that will be transferred between components
into ESMF Import and Export State data structures.  This is done
by first wrapping model data in either ESMF Arrays or Fields.
Arrays are simpler to create and use than Fields, but carry less
information and have a more limited range of operations.
These Arrays and Fields are then added to Import and
Export States.  They may be packed into ArrayBundles or
FieldBundles first, for more efficient communications.
Metadata describing the model data can also be added.
At the end of this step, the data to be transferred between
components will be in a compact and largely self-describing
form.

\item Pack time information into ESMF time management data 
structures.

\item Using code templates provided in the ESMF distribution, create
ESMF Gridded and Coupler Components to represent each component
in the user code.

\item Write a set services routine that sets ESMF entry 
points for each user component's initialize, run, and finalize 
methods.

\item Run the application using an ESMF Application Driver.

\end{enumerate} 


\subsection{Constants}

Named constants are used throughout ESMF to specify the values of many 
arguments with multiple well defined values in a consistent way.  These 
constants are defined by a derived type that follows this pattern:

\begin{verbatim}
ESMF_<CONSTANT_NAME>_Flag
\end{verbatim}

The values of the constant are then specified by this pattern:

\begin{verbatim}
ESMF_<CONSTANT_NAME>_<VALUE1>
ESMF_<CONSTANT_NAME>_<VALUE2>
ESMF_<CONSTANT_NAME>_<VALUE3>
...
\end{verbatim}

A master list of all available constants can be found in section 
\ref{const:cmaster}.

\section{Overall Rules and Behavior}

\subsection{Local and Global Views and Associated Conventions}

ESMF data objects such as Fields are distributed over
DEs, with each DE getting a portion of the data.  Depending
on the task, a local or global view of the object may be
preferable.  In a local view, data indices start with the first
element on the DE and end with the last element on the same DE.
In a global view, there is an assumed or specified order to
the set of DEs over which the object is distributed.  Data
indices start with the first element on the first DE, and
continue across all the elements in the sequence of DEs.
The last data index represents the number of elements in the
entire object.  The DistGrid provides the mapping between
local and global data indices.

The convention in ESMF is that entities with a global view
have no prefix.  Entities with a DE-local (and in some cases,
PET-local) view have the prefix ``local.''

Just as data is distributed over DEs, DEs themselves can be
distributed over PETs.  This is an advanced feature for users
who would like to create multiple local chunks of data, for
algorithmic or performance reasons.
Local DEs are those DEs that are located on the local PET.
Local DE labeling always starts at 0 and goes to localDeCount-1,
where localDeCount is the number of DEs on the local PET.
Global DE numbers also start at 0 and go to deCount-1.
The DELayout class provides the mapping between local
and global DE numbers. 

\subsection{Allocation Rules}

The basic rule of allocation and deallocation for the ESMF is:
whoever allocates it is responsible for deallocating it.

ESMF methods that allocate their own space for data will
deallocate that space when the object is destroyed. 
Methods which accept a user-allocated buffer, for example
{\tt ESMC\_FieldCreate()} with the {\tt ESMF\_DATACOPY\_REFERENCE} flag,
will not deallocate that buffer at the time the object is
destroyed.  The user must deallocate the buffer
when all use of it is complete.

Classes such as Fields, FieldBundles, and States may have Arrays, 
Fields, Grids and FieldBundles created externally and associated with
them.  These associated items are not destroyed along with the rest  
of the data object since it is possible for the items to be added 
to more than one data object at a time (e.g. the same Grid could 
be part of many Fields).  It is the user's responsibility to delete 
these items when the last use of them is done.

\subsection{Assignment, Equality, Copying and Comparing Objects}

The equal sign assignment has not been overloaded in ESMF, thus resulting in
the standard C behavior. This behavior has been documented as the first
entry in the API documentation section for each ESMF class. For deep ESMF
objects the assignment results in setting an alias the the same ESMF object
in memory. For shallow ESMF objects the assignment is essentially a equivalent
to a copy of the object. For deep classes the equality operators have been
overloaded to test for the alias condition as a counter part to the assignment
behavior. This and the not equal operator are documented following the
assignment in the class API documentation sections. 

Deep object copies are implemented as a special variant of the
{\tt ESMC\_<Class>Create()} methods. It takes an existing deep object as
one of the required arguments. At this point not all deep classes have
{\tt ESMC\_<Class>Create()} methods that allow object copy.

Due to the complexity of deep classes there are many aspects when comparing two
objects of the same class. ESMF provide {\tt ESMC\_<Class>Match()} methods,
which are functions that return a class specific match flag. At this point not
all deep classes have {\tt ESMC\_<Class>Match()} methods that allow deep object
comparison.

%\subsection{Attributes}
%
%Attributes are (name, value) pairs, where
%the name is a character string and the value can be either a single
%value or list of {\tt int}, {\tt double}, or {\tt char *} values.
%Attributes can be associated with Fields, FieldBundles, and States. 
%Mixed types are not allowed in a single attribute, and all attribute
%names must be unique within a single object.    Attributes are set
%by name, and can be retrieved either directly by name or by querying
%for a count of attributes and retrieving names and values
%by index number.



\section{Overall Design and Implementation Notes}
\label{sec:overallimpl}

\begin{enumerate}

\item {\bf Deep and shallow classes.}  The deep and shallow classes 
described in Section \ref{sec:deepshallow} differ in how and where they
are allocated within a multi-language implementation environment.  We
distinguish between the implementation language, which is the language
a method is written in, and the calling language, which is the language
that the user application is written in.  Deep classes are allocated 
off the process heap by the implementation language.  Shallow classes
are allocated off the stack by the calling language.  

\item {\bf Base class.} All ESMF classes are built upon a Base class,
which holds a small set of system-wide capabilities.  

\end{enumerate}
