\section{ESMF Classes}

We divide the ESMF classes into two main categories, those associated with the coupling 
superstructure and those that are part of the utility infrastructure.  The superstructure or data 
classes are based on a hierarchical 
calling tree of increasingly abstract data structures that represent the field data associated 
with the physical systems being modeled.  Infrastructure or utility classes are independent 
of the data classes, though they too have a hierarchical structure; higher-level utilities
employ general-purpose tools such as a message log.

In the listing of classes below we provide a description of each class and its function.

\subsection{Coupling Superstructure}

\subsubsection{Component (ESMF\_Comp)} 
\begin{description}
\item [Description] A Component is a functionally related computational entity that represents 
a large system.  {\it All} Components have a Name, a Status (ESMF\_ACTIVE, ESMF\_ERROR, etc.), and a Map that 
describes the tasks and/or heaps over which the Component is distributed.  Components may be 
nested within other Components.
\item [Function] Some of the basic methods associated with a Component are
initialize, finalize, run, and stop.  A Component can be queried for its Name, Map, and Status.  If it is a Gridded Component, it can can be queried for its Distributed Grid; if it is an I/O Component it can be queried for an I/O
Specification.  When running as part of a brokered Application, a Component can 
register those Fields it can export and those needed for it to run.
\end{description}

\subsubsection{Application (ESMF\_App)}
\begin{description} 
\item [Description] We define an Application (ESMF\_App) as a special kind of Component 
that is itself composed of a set of sub-Components that interact to form a complete scientific
application.  
\item [Function] The ESMF\_App class is responsible for managing those functions that relate 
to an entire scientific application running under ESMF.  The ESMF\_App initialize method 
must be called at the start of any user application operating under the framework, and
the ESMF\_App finalize method at its end.  At initialization the Application allocates and 
configures any resources needed to run the framework.  The Application also specifies whether 
the system will be brokered using the Registry or not.  The ESMF\_App class can be queried 
for information such as an experiment name, model name, run type (ESMF\_INIT, 
ESMF\_BRANCH, etc.), and for an overall Status.  It can also be queried for
information on any Component that it includes, including its Name, Map, and
Status.
\end{description}

\subsubsection{Coupler (ESMF\_Coupler)}
\begin{description}
\item [Description] A Coupler is a specialized type of Component that encompasses all the 
functionality needed to communicate
data between two Components. It may include multiple Route calls and Transforms, and
may be separated into initialize and run stages.
\item [Function] The basic methods of a Coupler are send, which may be blocking or
non-blocking, and receive. 
\end{description}

\subsubsection{Route (ESMF\_Route)}
\label{sec:route}
\begin{description}
\item [Description] A Route is a data transfer path established between two Field Groups,
and it is created using their Maps.  A Route may be separated into initialize and run stages.
\item [Function] Routes have send methods, which may be blocking or nonblocking, and receives.
\end{description}

\subsubsection{Registry (ESMF\_Registry)} 
\label{sec:registry}
\begin{description}
\item [Description]  A Registry is an entity that stores Component Names, Maps, and metadata,
including descriptions of those Fields available for export by the Component and those needed to run.   
\item [Function] The Registry can add a Component, remove a Component, or update the information
for a given Component.  Use of the Registry is optional.
\end{description}

\subsubsection{Gridded Component (ESMF\_GComp)} 
\label{sec:gridcomp}
\begin{description}
\item [Description] A Gridded Component is a specialized type of Component that is associated
with a single Distributed Grid.  It may include one or more Field Groups that are discretized 
on the same grid and distributed in the same fashion.  A Gridded Component is the representation 
of most component models, such as ocean models.  
\item [Function] Gridded Components can provide all their associated data (get state), or can
set this data (set state).  Gridded Components can be regridded, in which case the regridding directive
propagates through all the Fields and Field Groups associated with the Component.  
\end{description}

\subsubsection{I/O Component (ESMF\_IOComp)} 
\begin{description}
\item [Description] An I/O Component represents the input and output of data from an external
device as a transfer of data to another Component.  
\item [Function]
\end{description}

\subsubsection{Transform (ESMF\_XForm)} 
\begin{description}
\item [Description] A Transform takes one or more physical quantities defined using one set 
of units or representation
and translates them as needed to a different set of units or representation, for example, 
potential temperature
to temperature.  Transform objects are specialized by the application developer.
\item [Function] The Transform class is an abstraction introduced for the
purpose of standardizing high-level coupling interfaces.  It may be overloaded
to take sets of individual Fields or Field Groups.  Methods may include a
separate initialization and run. 
\end{description}

\subsubsection{Map (ESMF\_Map)}
\label{sec:map} 
\begin{description}
\item [Description] A Map is a description of a computational domain that
may describe the decomposition of an Application, a Component, a Field Group, a Field, or 
a Distributed Grid.
If no Map is specified for an object, it can inherit its Map from an object
higher in the data hierarchy.  For example, if the map for a Field is not
specified, it will be assigned the Map associated with the Component.  A
default Map is assigned if none have been specified anywhere in an 
Application. 
\item [Function] A Map describes the decomposition of a data object.  It can be
queried for the details of that decomposition.
\end{description}

\subsubsection{Clock (ESMF\_Clock)}
\begin{description}
\item [Description] A Clock advances model time during integration and stores
information relating to integration, such as the start date, model timestep, number of 
timesteps taken, and stop date or timestep.
\item [Function] Some of the methods associated with a Clock are advance time,
get previous time, get current time, and test if last step. 
\end{description}  

\subsubsection{Distributed Grid (ESMF\_DGrid)} 
\begin{description}
\item [Description] Contains a decomposition over heaps and tasks 
as well as a grid specification.  A grid may be defined on a specific set of heaps and tasks, and 
different grids within the same executable may be mapped to different sets of heaps 
and tasks.  The way grids are mapped may change over the course of model execution.  
Global grids and subsets (masks, regional grids) must be supported.  Time dependent 
grid subsets, such as satellite swaths, are a projected capability. 
\end{description}

\subsubsection{Field Group (ESMF\_FieldGroup)} 
\begin{description}
\item [Description] A Field Group is a set of fields that is discretized on the same grid.  
Field Groups are the basic units of exchange between Components.  
\item [Function] A Field Group can be regridded or redistributed, can be queried for a 
list of the fields that it contains.  It can return its Distributed Grid. 
\end{description}

\subsubsection{Field (ESMF\_Field)}
\begin{description} 
\item [Description] A Field represents a single physical field or the components of a 
vector field.  Attributes include metadata, such as FieldName, FieldUnits, etc., that describe the
Field.
\item [Function] A Field can be regridded or redistributed, and can be queried on its
Distributed Grid or its metadata.
\end{description}

\subsection{Utility Infrastructure}

\subsubsection{Basic Utilities (ESMF\_BasicUtil)} 
\begin{description}
\item [Description] Utilities that may be utilitized by any other class in the ESMF.  
Collecting these functions into a base-level utility set helps to 
avoid circular referencing.
\item [Function] 
\end{description}

\subsubsection{Basic Communications (ESMF\_BasicComm)}
\begin{description}
\item [Description] This library is a wrapper for MPI and other vendor-supplied 
message passing libraries.
\item [Function] The Basic Communication library provides a generic interface
and efficient communications for the ESMF.  Methods include scatter, gather, send,
receive, synchronize. 
\end{description}

\subsubsection{ (ESMF\_Machine)} 
\begin{description}
\item [Description] The Machine class provides a representation of 
key features of computer hardware and system software.  These
features include memory attributes and configuration, processor type and speed,
interconnect attributes, and system library availability.
\item [Function]
The main purpose of the Machine is to store hardware and system software
information needed by the framework or application programmer in a general
form, but with little abstraction.  This information can be used to perform resource 
allocation, data distribution, and dynamic load balancing.  The Machine can be queried
for platform type(s), number of processors, number of threads, and number of 
nodes.  It may optionally provide information on quantities such as bandwidth and 
latency through active tests.  
\end{description}

\subsubsection{Time Management (ESMF\_Date, ESMF\_DT)}
\begin{description}
\item [Description] The date and time interval methods in the ESMF provide date
calculations based on a number of different calendars.

\end{description}
















