\subsection{Demonstration Application}
\label{sec:demo}

The {\tt ESMF\_COUPLED\_FLOW} demonstration illustrates use of both the
ESMF infrastructure and superstructure.  It is described in detail in 
Section~\ref{sec:demo}.

\subsubsection{Running the Coupled Wave Demonstration}

To run the demo starting from ESMF source code, type 

\begin{verbatim}
gmake ESMF_COUPLED_FLOW
\end{verbatim}

from the \$(ESMF\_DIR) directory.  This will compile the 
ESMF and the demo and run them.

To run the demo when using a precompiled shared
library packet, {\tt cd} into the {\tt CoupledFlowExe}
directory and type

\begin{verbatim}
gmake run
\end{verbatim}






























