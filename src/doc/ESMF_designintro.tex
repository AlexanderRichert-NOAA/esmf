\section{Introduction}

In this {\it Earth System Modeling Framework (ESMF) Architecture Document} we describe 
the main design features that will enable us to achieve interoperability, ease of 
use, performance portability, and reuse for climate, weather, and data assimilation
applications.  Our goal is to create a software system capable of satisfying
the extensive functional requirements laid out in the {\it ESMF 
Requirements Document}, while retaining in its design a measure of intuitiveness
and simplicity. 

We define a framework as a structured collection of software building blocks 
that can be used or customized to develop model components, assemble them into an 
application, and run the application.  Unlike a software library, in a framework some
software elements are partly defined or unimplemented.  For example, a framework
may require that a gridded model provide a {\tt GetGrid} method, and specify 
a format for describing the grid, but leave the implementation of the method up
to the framework user.  This relatively non-intrusive standardization enables 
different model components to interoperate with each other, and to use fully
complete, general methods provided by the framework (in this case, perhaps a 
{\tt Regrid} method).

The functional split between developing and combining components is the
foremost feature of the ESMF architecture.  The simplest view of the ESMF 
is that it consists of an {\it infrastructure} of utilities and data 
structures for building model components and a {\it superstructure} for coupling 
and running them.  This document describes the interplay between ESMF 
infrastructure and superstructure, and how they relate to user-provided components.

We provide in the remainder of this introductory section a summary of the 
scope of the ESMF.  The scope and motivating requirements are explained in more detail in the {\it ESMF Requirements Document}.  In Section~\ref{sec:background}, {\it Evolution of Earth System Frameworks} we place the ESMF in the context of existing frameworks in our community.  Section~\ref{sec:strategies} is 
a {\it Design Overview}.  It includes 
a description of the basic structure of an ESMF application and a discussion of 
the design strategies
used in ESMF.  Sections~\ref{sec:superclasses},~\ref{sec:fieldclasses}, and
\ref{sec:utilclasses} contain descriptions of the 
major classes in the framework, their structure, their relationships to each other, 
and the sequence of events for key interactions.  Section~\ref{sec:fwdesign}, {\it 
Framework-Wide Design Elements}, describes in detail the conventions and behavior
that we wish to apply across the ESMF software.  It
includes a description of function naming conventions and a base class whose behavior all 
framework classes will share.  Section~\ref{sec:glos} is a glossary of terms.








