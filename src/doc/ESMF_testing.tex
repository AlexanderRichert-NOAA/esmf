% $Id: ESMF_testing.tex,v 1.28 2009/01/06 16:39:26 murphysj Exp $

\subsection{Running ESMF Self-Tests}
\label{testing}

Robustness and portability are primary goals of the ESMF development
effort.  To ensure that these goals are met, the ESMF includes a
comprehesive suite of tests.  They allow testing and validation of
everything from individual functions to complete system tests.  These
test suites are used by the ESMF development team as part of their
regular development process.  ESMF users can run the testing suites to
verify that the framework software was built and installed properly,
and is running correctly on a particular platform.

Test targets will compile the ESMF library if it has not already been built.

\subsubsection{Setting up ESMF to run Test Suite Applications}
\label{ESMFRunSetting}

Unless the ESMF library was built in MPI-bypass mode (mpiuni), all applications
compiled and linked against ESMF automatically become MPI applications and must
be executed as such. The ESMF test suite and example applications are no
different in this respect.

Details of how to execute MPI applications vary widely from system to system.
ESMF uses an mpirun script mechanism to abstract away most of these differences.
All ESMF makefile targets that require the execution of applications do this by
launching the application via the executable specified in the 
{\tt ESMF\_MPIRUN} variable. ESMF assumes that an MPI applications can be 
launched across {\tt N} processes by calling

\begin{verbatim}
$(ESMF_MPIRUN) -np N application
\end{verbatim}

and that the output of the application arrives at the calling shell via 
{\tt stdout} and {\tt stderr}.

On systems that allow direct launching of MPI application via a suitable
{\tt mpirun} facility, ESMF can use it directly. This is the ESMF default for
all those configurations that come with a suitable {\tt mpirun}. In these
cases the {\tt ESMF\_MPIRUN} environment variable does not need to be set by
the user.

There are systems, however, that allow direct launching of MPI application but
provide a launch mechanism that is incompatible with ESMF's assumptions. In
these cases a simple mpirun wrapper is required. The ESMF {\tt ./scripts}
directory contains wrappers for several cases in this class, e.g. for
interactive POE access on IBM machines and {\tt aprun}, as well as
{\tt yod} on Cray machines. The ESMF configurations will access the
appropriate wrapper scripts by default if necessary.

Finally, there are those systems that utilize batch software to access the
parallel execution environment. One option is to execute the ESMF test targets
from within a batch session, either interactively or from within a script. In 
this case the batch software does not add any additional complexity for ESMF.
The same issues discussed above, of how to launch an MPI application, apply 
directly. 

However, in some cases it is more convenient to execute the ESMF test target
on the front-end machine, and have ESMF access the batch software each time it
needs to launch an application. In fact, on IBM systems this is often the only
working option, because the integrated POE system will execute each application
on the exact same number of processes specified during batch access, regardless
of how many ways parallel a specific application needs to be run.

Two modes of operation need to be considered for the ESMF batch access. First,
if interactive batch access is available, it is straight forward to write an
{\tt mpirun} script that fulfills the ESMF requirements outlined above. The
ESMF {\tt ./scripts} directory contains several scripts that access various
parallel launching facilities though interactive LSF.

Second, if interactive batch access is not available, a more complex scripting
approach is necessary. The basic requirements in this case are that ESMF must
be able to launch MPI applications across {\tt N} processes by calling

\begin{verbatim}
$(ESMF_MPIRUN) -np N application  ,
\end{verbatim}

that the output of the application will be available in a file named
{\tt application.stdout} after the script finishes, and that the
{\tt ESMF\_MPIRUN} script blocks execution until {\tt application.stdout}
has become accessible.

The ESMF {\tt ./scripts} directory contains scripts of this flavor for a wide
variety of batch systems. Most of these scripts, when called through ESMF,
will generate a customized, temporary batch script for a specific executable
"on the fly" and submit this batch script to the queuing software. The script
then waits for completion of the submitted job, after which it copies the
output, received through a system specific mechanism, into the prescribed file.

Regardless of whether the batch system access is interactive or not, it is
often necessary to specify various system specific options when calling the
batch submission tool. ESMF utilizes the {\tt ESMF\_MPIBATCHOPTIONS} environment
variable to pass user supplied values to the batch system.

The environment variable {\tt ESMF\_MPISCRIPTOPTIONS} is available to pass
user specified options to the actual script specified by {\tt ESMF\_MPIRUN}.
However, {\tt ESMF\_MPISCRIPTOPTIONS} will only be added automatically to the 
{\tt ESMF\_MPIRUN} call if the specified {\tt ESMF\_MPIRUN} can be found in the
ESMF {\tt ./scripts} directory.

Finally, the value of {\tt ESMF\_MPILAUNCHOPTIONS} is passed to the MPI launch
facility by default, i.e if {\tt ESMF\_MPIRUN} was not specified by the user.
In case the user specifies {\tt ESMF\_MPIRUN} to be anything else but scripts
out of the ESMF {\tt ./scripts} directory, it is the user's responsibility to
add {\tt ESMF\_MPISCRIPTOPTIONS} to {\tt ESMF\_MPIRUN} and/or to utilize
{\tt ESMF\_MPILAUNCHOPTIONS} within the specified script.

The possibilities covered by the generic scripts provided in the ESMF
{\tt ./scripts} directory, combined with the {\tt ESMF\_MPISCRIPTOPTIONS},
{\tt ESMF\_MPIBATCHOPTIONS}, and {\tt ESMF\_MPILAUNCHOPTIONS} environment
variables, will satisfy the majority of common situations. There are, however,
circumstances for which a customized, user-provided mpirun script is necessary.
One such situation arises with the LoadLeveler batch software. LoadLeveler
typically requires a list of options specified in the actual batch script. This
is most easily handled by a script that produces such a system and user specific
script "on the fly". Another situation is where certain modules or software
packages need to be made available inside the batch script. Again this is most
easily handled by a customized script the user writes and provides to ESMF via
the {\tt ESMF\_MPIRUN} environment variable. This will override any default
settings for the configuration and rely on the user provided script instead.

Users that face the need to write a customized mpirun script for their
parallel execution environment are encouraged to start with the closest match 
from the ESMF {\tt ./scripts} directory and customize it to their situation.
The best way to see how the existing scripts are used on the supported
platforms is to go to the 
\htmladdnormallink{http://www.esmf.ucar.edu/download/platforms/}
{http://www.esmf.ucar.edu/download/platforms/} web page and follow the link 
for the platform of interest. Each test report contains the output of
{\tt gmake info}, which lists the settings of the {\tt ESMF\_MPIxxx} 
environment variables.

Furthermore, the {\tt esmfcontrib} repository on SourceForge hosts a 
{\tt scripts} module that contains scripts that were customized to certain
user environments. These scripts are not generic enough to be included in the
ESMF distribution, but users faced with the need to customize a script to their
environment may find the script collection on {\tt esmfcontrib} a valuable
resource. Please refer to
\htmladdnormallink{http://sourceforge.net/projects/esmfcontrib}
{http://sourceforge.net/projects/esmfcontrib} on how to access 
{\tt esmfcontrib}, and the {\tt scripts} module in particular.

\subsubsection{Running ESMF Unit Tests}

\label{UnitTestDescription}
The unit tests provided with the ESMF library evaluate the following:
\begin{itemize}
\item correctness of individual functions
\item behavior of individual modules or classes
\item appropriate error handling
\end{itemize}

Unit tests can be run in either an exhaustive or a non-exhaustive (sanity check)
mode.  The exhaustive mode includes the sanity check tests.  Typically, sanity
checks for each ESMF capability include creating and destroying an object and 
testing its basic function using a valid argument set.  In the exhaustive mode,
a wide range of valid and non-valid arguments are evaluated for correct behavior.

\label{RunUnitTests}

The following commands are used to build and run the unit tests provided with 
the ESMF:
\begin{verbatim}
        gmake [ESMF_TESTEXHAUSTIVE=<ON,OFF>] tests
        gmake [ESMF_TESTEXHAUSTIVE=<ON,OFF>] tests_uni
\end{verbatim}

The {\tt tests\_uni} target runs the tests on a single processor. 
The {\tt tests} target runs the test on multiple processors.

The non-exhaustive set of unit tests should all pass.  At this point in 
development, the exhaustive tests do not all pass.  Current problems with 
unit tests are being tracked and corrected by the ESMF development team.

The results of running the unit tests can be found in the following location:
\begin{verbatim}
${ESMF_DIR}/test/test${ESMF_BOPT}/${ESMF_OS}.${ESMF_COMPILER}.${ESMF_ABI}.${ESMF_SITE}
\end{verbatim}

For example, if your esmf source files have been placed in: 
\begin{verbatim}
       /usr/local/esmf
\end{verbatim}

If your platform is a Linux uni-processor that has an installed Lahey
Fortran compiler and ESMF\_COMPILER has been set to lahey, then the build
system configuration file will be:

\begin{verbatim}
      build_config/Linux.lahey.default/build_rules.mk
\end{verbatim}

If you want to run a debug version of non-exhaustive unit tests,
then you use these commands from /usr/local/esmf:

\begin{verbatim}
       setenv ESMF_DIR /usr/local/esmf
       gmake ESMF_BOPT=g ESMF_SITE=lahey ESMF_TESTEXHAUSTIVE=OFF tests_uni
\end{verbatim}


If you are using ksh, then replace the setenv command with:
\begin{verbatim}
       export ESMF_DIR=/usr/local/esmf
\end{verbatim}

The results of the unit tests will be in:
\begin{verbatim}
       /usr/local/esmf/test/testg/Linux.lahey.32.default/
\end{verbatim}

At the end of unit test execution a script runs to analyze the results.

The script output indicates whether there are any unit test failures.
If any unit tests fail, please check if the failures are listed as known bugs in the ESMF release
page \htmladdnormallink{http://www.esmf.ucar.edu/download/releases.shtml}{http://www.esmf.ucar.edu/download/releases.shtml}
for your platform and compiler.
If the failures are not listed please contact ESMF Support at \htmladdnormallink{esmf\_support@ucar.edu}
{mailto:esmf\_support@ucar.edu}
Please indicate which unit tests are failing, and attach the output of the "gmake info" command to the email.


The script output indicates whether there are any unit test failures.
The following is a sample from the script output:

\begin{verbatim}


The unit tests in the following files all pass:

src/Infrastructure/Array/tests/ESMF_ArrayUTest.F90
src/Infrastructure/ArrayDataMap/tests/ESMF_ArrayDataMapUTest.F90
src/Infrastructure/Base/tests/ESMF_BaseUTest.F90
src/Infrastructure/FieldBundle/tests/ESMF_FieldBundleUTest.F90
src/Infrastructure/FieldBundleDataMap/tests/ESMF_FieldBundleDataMapUTest.F90
src/Infrastructure/Config/tests/ESMF_ConfigUTest.F90
src/Infrastructure/DELayout/tests/ESMF_DELayoutUTest.F90
src/Infrastructure/Field/tests/ESMF_FRoute4UTest.F90
src/Infrastructure/Field/tests/ESMF_FieldUTest.F90
src/Infrastructure/FieldComm/tests/ESMF_FieldGatherUTest.F90
src/Infrastructure/FieldDataMap/tests/ESMF_FieldDataMapUTest.F90
src/Infrastructure/Grid/tests/ESMF_GridUTest.F90
src/Infrastructure/IOSpec/tests/ESMF_IOSpecUTest.F90
src/Infrastructure/LocalArray/tests/ESMF_ArrayDataUTest.F90
src/Infrastructure/LocalArray/tests/ESMF_ArrayF90PtrUTest.F90
src/Infrastructure/LocalArray/tests/ESMF_LocalArrayUTest.F90
src/Infrastructure/LogErr/tests/ESMF_LogErrUTest.F90
src/Infrastructure/Regrid/tests/ESMF_Regrid1UTest.F90
src/Infrastructure/Regrid/tests/ESMF_RegridUTest.F90
src/Infrastructure/TimeMgr/tests/ESMF_AlarmUTest.F90
src/Infrastructure/TimeMgr/tests/ESMF_CalRangeUTest.F90
src/Infrastructure/TimeMgr/tests/ESMF_ClockUTest.F90
src/Infrastructure/TimeMgr/tests/ESMF_TimeIntervalUTest.F90
src/Infrastructure/TimeMgr/tests/ESMF_TimeUTest.F90
src/Infrastructure/VM/tests/ESMF_VMBarrierUTest.F90
src/Infrastructure/VM/tests/ESMF_VMBroadcastUTest.F90
src/Infrastructure/VM/tests/ESMF_VMGatherUTest.F90
src/Infrastructure/VM/tests/ESMF_VMScatterUTest.F90
src/Infrastructure/VM/tests/ESMF_VMSendVMRecvUTest.F90
src/Infrastructure/VM/tests/ESMF_VMUTest.F90
src/Superstructure/Component/tests/ESMF_CplCompCreateUTest.F90
src/Superstructure/Component/tests/ESMF_GridCompCreateUTest.F90
src/Superstructure/State/tests/ESMF_StateUTest.F90


The following unit test files failed to build, failed to execute or crashed during execution:

src/Infrastructure/TimeMgr/tests/ESMF_CalendarUTest.F90
src/Infrastructure/VM/tests/ESMF_VMSendRecvUTest.F90


The following unit test files had failed unit tests:

src/Infrastructure/Field/tests/ESMF_FRoute8UTest.F90
src/Infrastructure/Grid/tests/ESMF_GridCreateUTest.F90


The following individual unit tests fail:

  FAIL  DELayout Get Test, ESMF_FRoute8UTest.F90, line 139                                                                                                                                                                                                       
  FAIL  Grid Distribute Test, ESMF_GridCreateUTest.F90, line 198                                                                                                                                                                                                 


The stdout files for the unit tests can be found at:
/home/bluedawn/svasquez/script_dirs/daily_builds/esmf/test/testO/AIX.default.64.default

Found 1224 exhaustive multi processor unit tests, 1220 pass and 4 fail.


\end{verbatim}

The following is an example of the output generated when a unit test fails:
\begin{verbatim}
ESMF_FieldUTest.stdout: FAIL  Unique default Field names Test, FLD1.5.1 & 1.7.1,
                        ESMF_FieldUTest.F90, line 204  Field names not unique
\end{verbatim}

\subsubsection{Running ESMF System Tests}
\label{SystemTestDescription}

The system tests provided with the ESMF library evaluate:
\begin{itemize}
\item interface agreement between parts of the system
\item behavior of the system as a whole
\end{itemize}

The current system test suite includes tests that perform layout
reduction operations, redistribution-transpose, halo operations,
component creation and intra-grid communication.  Some of the system
tests are no longer compatible with the current API, but are included
in the release for completeness.  A complete description of each
available system test and its current compatibility status can be
found at the ESMF website,
\htmladdnormallink{http://www.esmf.ucar.edu}{http://www.esmf.ucar.edu}.  
The testing
and validation page is accessible from the {\bf Development} 
link on the navigation bar.

The following commands are used to build and run the system tests:

\begin{verbatim}
        gmake [SYSTEM_TEST=xxx] system_tests
        gmake [SYSTEM_TEST=xxx] system_tests_uni
\end{verbatim}

The {\tt system\_tests\_uni} target runs the tests on a single processor. 
The {\tt system\_tests} target runs the test on multiple processors.

If a particular SYSTEM\_TEST is not specified, then all available system tests 
are built and run.

The results of the test can be found in the following location:
\begin{verbatim}
${ESMF_DIR}/test/test${ESMF_BOPT}/${ESMF_OS}.${ESMF_COMPILER}.${ESMF_ABI}.${ESMF_SITE}
\end{verbatim}

For example, if your ESMF source files have been placed in your home directory:
\begin{verbatim}
       ~/esmf
\end{verbatim}

and your platform and compiler configuration is:
\begin{verbatim}
       Alpha multi-processor using the native compiler
\end{verbatim}

and you want to run an optimized version of system test SimpleCoupling,
then you use these commands from the directory {\tt \~/esmf}. 
\begin{verbatim}
       setenv ESMF_PROJECT <project_name>
       gmake ESMF_DIR=`pwd` SYSTEM_TEST=ESMF_SimpleCoupling system_tests
\end{verbatim}

If you are using ksh then replace the setenv command with
this:

\begin{verbatim}
       export ESMF_PROJECT=<project_name>
\end{verbatim}

The results will be in:
\begin{verbatim}
~/esmf/test/testO/OSF1.default.64.default/ESMF_SimpleCouplingSTest.stdout
\end{verbatim}

At the end of system test execution a script runs to analyze the results.


The script output indicates whether there are any system test failures.
If any system tests fail, please check if the failures are listed as known bugs in the ESMF release
page \htmladdnormallink{http://www.esmf.ucar.edu/download/releases.shtml}{http://www.esmf.ucar.edu/download/releases.shtml}
for your platform and compiler.
If the failures are not listed please contact ESMF Support at \htmladdnormallink{esmf\_support@ucar.edu}
{mailto:esmf\_support@ucar.edu}
Please indicate which system tests are failing, and attach the output of the "gmake info" command to the email.



The script output indicates whether there are any system test failures.
The following is a sample from the script output:

\begin{verbatim}


The following system tests passed:


src/system_tests/ESMF_CompCreate/ESMF_CompCreateSTest.F90
src/system_tests/ESMF_FieldExcl/ESMF_FieldExclSTest.F90
src/system_tests/ESMF_FieldHalo/ESMF_FieldHaloSTest.F90
src/system_tests/ESMF_FieldHaloPer/ESMF_FieldHaloPerSTest.F90
src/system_tests/ESMF_FieldRedist/ESMF_FieldRedistSTest.F90
src/system_tests/ESMF_FieldRegrid/ESMF_FieldRegridSTest.F90
src/system_tests/ESMF_FieldRegridMulti/ESMF_FieldRegridMultiSTest.F90
src/system_tests/ESMF_FieldRegridOrder/ESMF_FieldRegridOrderSTest.F90
src/system_tests/ESMF_FlowComp/ESMF_FlowCompSTest.F90
src/system_tests/ESMF_FlowWithCoupling/ESMF_FlowWithCouplingSTest.F90
src/system_tests/ESMF_SimpleCoupling/ESMF_SimpleCouplingSTest.F90
src/system_tests/ESMF_VectorStorage/ESMF_VectorStorageSTest.F90


The following system tests failed, did not build, or did not execute:


src/system_tests/ESMF_FieldRegridConserv/ESMF_FieldRegridConsrvSTest.F90
src/system_tests/ESMF_RowReduce/ESMF_RowReduceSTest.F90




The stdout files for the system_tests can be found at:
/home/bluedawn/svasquez/script_dirs/daily_builds/esmf/test/testO/AIX.default.64.default

Found 14 system tests, 12 passed and 2 failed.


\end{verbatim}

