% $Id: ESMF_testing.tex,v 1.4 2003/04/25 20:11:01 svasquez Exp $

\subsection{Running ESMF Self-Tests}
\label{testing}

Robustness and portability are primary goals of the ESMF development effort. To ensure that these goals are met, the ESMF includes a comprehesive suite of tests. They allow test and validation of everything from individual functions to complete system tests. These test suites are used by the ESMF development team as part of their regular development process. Model developers can also use the testing suites to verify that the software was built and installed properly. It can also assist them in the debugging process when integrating user supplied model components. 

In order to run the self-tests, the ESMF library source code must be built and installed.  See 
Section~\ref{InstallProcedures} for installation procedures and Section~\ref{BuildTestSuite} for 
information on building ESMF tests.  

\subsubsection{Running ESMF Unit Tests}

\label{UnitTestDescription}
The unit tests provided with the ESMF library evaluate the following:
\begin{itemize}
\item correctness of individual functions
\item behavior of individual modules or classes
\end{itemize}

\label{RunUnitTests}

The following commands are used to build and run the unit tests provided with the ESMF:
\begin{verbatim}
        gmake BOPT=<g,O> [ESMF_EXHAUSTIVE=<ON,OFF>] tests
        gmake BOPT=<g,O> [ESMF_EXHAUSTIVE=<ON,OFF>] tests_uni
\end{verbatim}

The non-exhaustive set of unit tests should all pass.

Using the "tests_uni" target runs the tests, on a single processor, while
the target "tests" run the test on multiple processors.


The results of the test can be found in the following location:
\begin{verbatim}
       ${ESMF_DIR/test/test${BOPT}/${ESMF_ARCH}
\end{verbatim}

For example: 

If your esmf source files have been placed in: 
\begin{verbatim}
       /usr/local/esmf
\end{verbatim}

and your platform and compiler configuration is:
\begin{verbatim}
       Linux uni-processor using the LF95 compiler
\end{verbatim}

and you want to run a debug version of exhaustive unit tests,
then you use the command:
\begin{verbatim}
       gmake BOPT=g ESMF_EXHAUSTIVE=ON tests_uni
\end{verbatim}

and will find the results in:
\begin{verbatim}
       /usr/local/esmf/test/testg/linux_lf95/
\end{verbatim}

In addition, at the end of unit tests execution a script runs to analyze the results.
An example of the script output follows:
\begin{verbatim}
The following is the analysis of the Unit Tests results:

Executable unit tests found:
-rwxr-xr-x    1 svasquez scd        322499 Apr 25 10:31 ESMF_ArrayBasicUTest
-rwxr-xr-x    1 svasquez scd        325957 Apr 25 10:31 ESMF_ArrayF90PtrUTest
-rwxr-xr-x    1 svasquez scd        338415 Apr 25 10:31 ESMF_ArrayUTest
-rwxr-xr-x    1 svasquez scd        322897 Apr 25 10:31 ESMF_BaseUTest
-rwxr-xr-x    1 svasquez scd        341901 Apr 25 10:31 ESMF_BundleUTest
-rwxr-xr-x    1 svasquez scd        330480 Apr 25 10:31 ESMF_ClockUTest
-rwxr-xr-x    1 svasquez scd        339364 Apr 25 10:31 ESMF_FRouteUTest
-rwxr-xr-x    1 svasquez scd        336536 Apr 25 10:31 ESMF_FieldUTest
-rwxr-xr-x    1 svasquez scd        330849 Apr 25 10:31 ESMF_GridUTest
-rwxr-xr-x    1 svasquez scd        354073 Apr 25 10:32 ESMF_StateUTest

All of the executable unit tests should have a corresponding stdout file.
If not, it's an indication that the unit test was not executed, or that it failed to execute.

Unit Tests stdout files found: 
-rw-r--r--    1 svasquez scd           576 Apr 25 10:31 ESMF_ArrayBasicUTest.stdout
-rw-r--r--    1 svasquez scd           960 Apr 25 10:31 ESMF_ArrayF90PtrUTest.stdout
-rw-r--r--    1 svasquez scd          7791 Apr 25 10:31 ESMF_ArrayUTest.stdout
-rw-r--r--    1 svasquez scd            99 Apr 25 10:31 ESMF_BaseUTest.stdout
-rw-r--r--    1 svasquez scd          1690 Apr 25 10:31 ESMF_BundleUTest.stdout
-rw-r--r--    1 svasquez scd         73209 Apr 25 10:31 ESMF_ClockUTest.stdout
-rw-r--r--    1 svasquez scd          2585 Apr 25 10:31 ESMF_FieldUTest.stdout
-rw-r--r--    1 svasquez scd           399 Apr 25 10:31 ESMF_GridUTest.stdout
-rw-r--r--    1 svasquez scd          6484 Apr 25 10:32 ESMF_StateUTest.stdout

Unit test stdout files of zero length indicate that the unit test
did not run because it failed to compile or it failed to execute. 

112  Unit Tests passed 

No Unit Tests Failed.

\end{verbatim}

If a unit test fails, the failure will be listed as follows:
\begin{verbatim}
ESMF_FieldUTest.stdout: FAIL  Unique default Field names Test, FLD1.5.1 & 1.7.1, ESMF_FieldUTest.F90, line 204  Field names not unique
\end{verbatim}


\subsubsection{Running ESMF System Tests}
\label{SystemTestDescription}

The System tests provided with the ESMF library evaluate the following:
\begin{itemize}
\item interface agreement between parts of the system
\item behavior of the system as a whole
\end{itemize}

The current system test suite includes tests that perform layout reduction operations, redistribution-transpose, halo operations, component creation and intra-grid communication. A complete description of each available system test can be found at the ESMF website on the developers page. 


The following commands are used to build and run the system tests:

\begin{verbatim}
        gmake BOPT=<g,O> [SYSTEM_TEST=NNNNN] system_tests
        gmake BOPT=<g,O> [SYSTEM_TEST=NNNNN] system_tests_uni
\end{verbatim}

If SYSTEM\_TEST is not specified, then all available system tests will be built and run.

The results of the test can be found in the following location:
\begin{verbatim}
       ${ESMF_DIR/test/test${BOPT}/${ESMF_ARCH}
\end{verbatim}

For example: 

If your esmf source files have been placed in your home directory:
\begin{verbatim}
       ~/esmf
\end{verbatim}

and your platform and compiler configuration is:
\begin{verbatim}
       Alpha multi-processor using the native compiler
\end{verbatim}

and you want to run an optimized version of system test number 62502,
then you use the command:
\begin{verbatim}
       gmake BOPT=O SYSTEM_TEST=62502 system_tests
\end{verbatim}

and will find the results in:
\begin{verbatim}
       ~/esmf/test/testO/alpha/62502 
\end{verbatim}

\subsubsection{Running the ESMF VAlidation (EVA) Suite}
\label{EVATestDescription}

The ESMF VAlidation(EVA) Suite is a collection of seven codes representative of those used in climate, 
weather, and data assimulation. These codes are currently used for ESMF prototyping. They will eventually 
provide the basis for ESMF tutorial examples.

The EVA Suite User's Guide (http://www.esmf.ucar.edu/esmf\_docs/EVA\_usrdoc/index.html) describes how to 
compile and run these codes, which can be downloaded from 
SourceForge (http://cvs.sourceforge.net/cgi-bin/viewcvs.cgi/esmf/eva\_src/). 




























