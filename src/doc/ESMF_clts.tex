% $Id$

ESMF comes with a set of bundled command line tools (CLT).
These applications include convenient access to general information 
about an ESMF installation, and regrid weight file generation (sometimes
referred to as "offline" regridding). This section provides assistance with
respect to building and running the bundled CLTs. If you are using a
pre-installed ESMF on your system, follow the local instructions provided by
the installer or system admin of how to access and run the ESMF CLTs.
Often access is as simple as loading a configuration module to have the
correct path to the ESMF CLT binaries added to your {\tt PATH}
environment variable.

There are two ways a user may choose to build and access the bundled ESMF 
CLTs. Users that prefer not to go through the full ESMF installation 
process have the option to build the bundled CLTs inside of the ESMF 
source tree, very similar to how the unit tests, system tests and examples are 
built. This option is outlined in section \ref{quickapps} and should only be 
considered by users that want quick access to the CLTs and are not 
interested in a sharable installation or the development of portable scripts and 
makefiles that use the CLTs. Users interested in the latter should 
consider the more standard second option outlined below.

The bundled ESMF CLTs are built automatically in the process of 
installing ESMF following the instructions given in section \ref{InstallESMF}. 
On systems that offer system-wide ESMF installations (e.g. via modules or 
similar mechanisms) the user need not worry about the build and installation 
details. Once installed, the CLTs are accessible through their precise 
location on the system. For this purpose every ESMF installation provides a file 
named {\tt esmf.mk} that contains the variable {\tt ESMF\_APPSDIR} which 
specifies the precise CLT path.

The {\tt esmf.mk} mechanism used for CLT access is the same as the one 
described in section \ref{sec:Use} for writing robust and portable user 
makefiles for building and linking user CLTs against an ESMF 
installation. One feature of the {\tt esmf.mk} mechanism is that only one single 
piece of information must be known about an ESMF installation to use it, and 
that is the location of file {\tt esmf.mk} itself. The location of this file 
should be documented by the party that installed ESMF on the system. We 
recommend that a single ESMF specific environment variable ESMFMKFILE be 
provided by the system that points to the {\tt esmf.mk} file. See section 
\ref{InstallESMF} for the related discussion aimed at the person that installs 
ESMF on a system.

Once the exact location of the bundled ESMF CLT files has been 
determined, either by inspecting the associated {\tt esmf.mk} file, or by using 
the {\tt ESMF\_APPSDIR} makefile variable directly in the user script or 
makefile, the CLTs can be executed following the system specific rules 
for execution. The details will depend on whether ESMF was built with or without 
MPI dependency. In the latter case the system specific rules for launching 
parallel CLTs must be followed. System specific execution details on 
this level are outside of ESMF's scope. However, ESMF does offer specific 
CLT use examples as part of the {\it external\_demos} module described 
online at the 
\htmladdnormallink{External Demos webpage}{http://www.earthsystemmodeling.org/users/code_examples/external_demos/external_demos.shtml}. 
For most systems, the MPI version of the ESMF bundled CLTs can be 
executed by a command equivalent to:

\begin{verbatim}

mpirun -np X $(ESMF_APPSDIR)/<clt-name>

\end{verbatim}
 
where {\tt X} specifies the total number of PETs and {\tt clt-name} is the 
name of the specific ESMF command line tool to be executed.
 
All bundled ESMF CLTs support the standard \verb+ '--help'+ command line 
option that prints out information on its proper use.  More detailed 
instructions of the individual CLTs are available in the "Command Line Tools" 
section of the {\it ESMF Reference Manual}.
