\section{Scope}

This section describes the scope of the ESMF software. Specific
requirements will be derived from the functionality described here,
and presented in more detail in other requirements documents.

The ESMF provides two primary services: to facilitate 
coupling of Earth system model (ESM) components, and to support 
lower-level tasks widely used in Earth Science modeling on 
high performance computer platforms.  We refer to the first of these 
services as the ESMF coupling {\it superstructure} and the second 
as the ESMF utility {\it infrastructure}.\\

\noindent The ESMF consists of:
\begin{itemize}
\item an interface specification for coupling ESM components; 
\item a reference implementation accompanied by extensive documentation; 
\item a suite of application examples demonstrating how the 
ESMF software is used in practice.
\end{itemize}

Initially the model components that the ESMF will support will be atmosphere, 
ocean, land and sea ice models, and data assimilation systems.  Other 
well-defined high-level functions, such as I/O, may also be represented 
as components in the ESMF superstructure.  It is at the leve of these large-scale
components that the ESMF interface specification applies.  Each compliant
component will provide a specified set of methods that will enable it to 
interoperate with other components.  For the most part, we expect components
to be able to provide the desired interface functionality through wrapping
of internal data structures and methods, not through massive reconfiguration
of model code.

\subsection{Coupling Superstructure}

All inter-component communication in the framework occurs through 
component calls to the superstructure. \\

\noindent The main functions of the ESMF coupling superstructure are:
\begin{itemize}
\item to perform any merging, interpolation, or regridding 
of data necessary for communication between ESM components;
\item to transfer data between ESM components;
\item to coordinate execution of ESM components. 
\end{itemize}

The superstructure provides a high-level interface that enables 
applications to be assembled from multiple components.  It coordinates
the transfer of gridded, distributed field data and ungridded 
observational data between components.  The superstructure will be able 
to merge data originating from multiple source grids, and will 
provide spatial and temporal averaging of data.

The ESMF will be capable of operating in multiple modes of execution.
ESM components may exist within the same executable, within multiple 
executables, and in mixed modes.  Components may execute sequentially, 
concurrently, and in mixed modes.  

Functionality for regridding, interpolation, redistribution and other 
data communication will be abstracted away from the high-level coupling
interface.  It is provided by a parallel support layer, described next.  
This support layer can be used within applications 
to perform efficient parallel grid operations.

\subsection{Parallel Support Layer}

The ESMF contains the software necessary to support the data
decomposition and communication requirements of the superstructure and
individual components.  This includes tools for describing field
data discretized on a wide variety of distributed grids.  The ESMF
supports distributed data operations in a distributed-memory environment, 
in a shared memory environment, and in a hybrid computational environment 
where the platform is a cluster of shared-memory multiprocessors.

\subsubsection{Fields}
The ESMF supports representation, regridding, and other high-level
collective operations on vector and scalar fields.  A field is represented 
by metadata, a description of its associated distributed grid, and field 
data itself.

\subsubsection{Gridded and Observational Data}

The ESMF supports regridding, interpolation, redistribution,
transfer, and merging of data discretized on the following types of grids:

\begin{itemize}
\item logically rectangular grids;
\item reduced (cut-out) and regional grids;
\item unstructured grids (e.g., land grids);
\item phase space grids (e.g., spectral, Fourier);
\item nested grids;
\item masked regions and halo regions;
\item non-gridded data sets that are changing in time;
\item cubed sphere and icosahedral grids.
\end{itemize}

\subsubsection{Grid Operations}

The ESMF provides interpolation algorithms for supported grids.
All interpolation algorithms included in ESMF are {\it linear} 
and independent of data. Adaptive procedures are not planned to be 
part the ESMF at this stage. Thus, reconstruction of gridded data from 
observations (analysis) will not be a part of framework and will be 
rather provided by data assimilation components.

The ESMF provides first-order and higher-order interpolation 
methods.  The ESMF supports conservative remapping/interpolation 
between any two grids. Non-conservative methods will also 
be supported.

Dynamic load balancing will be provided for default decompositions, 
and general dynamic load-balancing tools for specialized decompositions.

\subsection{Utility Infrastructure}

The ESMF includes general purpose utility routines for use by both 
the ESMF coupler and application codes.  These utilities include 
but are not limited to:
\begin{itemize}
\item I/O utilities;
\item performance profiling;
\item time management; and
\item error handling.
\end{itemize}

\subsubsection{I/O}

The ESMF I/O utilities will have generic interfaces to enable greater
component interoperability, flexibility, and ease of use, and will be 
high-performance.  As much as possible, the utilities will offer scalable 
parallel I/O.

The ESMF will support I/O of self-describing data the following formats:
netCDF, HDF, binary, GRIB, and BUFR.  Others such as the EOS HDF and ODF 
data formats are desired as well but may be lower priority.

\subsubsection{Performance Profiling}

A performance profiling utility will enable application developers to 
instrument code segments with timers for both CPU and wallclock 
times.  The profiling utility will also offer access to harware statistics 
by offering an interface to a package such as PAPI or PCL.  Profiling
utilities will be integrated into a general logging service that will 
provide timing, informational and debugging output in several useful formats.

\subsubsection{Time Management}

The ESMF will include a library for performing routine 
calculations with dates and time intervals.  For higher level model
time management, the ESMF will include clocks for advancing and 
reporting model time and retaining model integration information, 
and alarms for initiating both unique and periodic events.

\subsubsection{Fault Tolerance}

The ESMF will offer comprehensive and integrated error handling services.  
A mechanism to detect component failure and shut down the entire 
application will be provided.

\subsection{Future Plans}

The following are capabilities that might conceivably be included in ESMF, but will not be 
developed under CAN 00-OES-01 funding:
\begin{itemize}
\item a Graphical User Interface;
\item mathematical libraries for operations other than regridding and interpolation;
\item standard scientific modules, such as a library for calculating orbital parameters;
\item a database of intra-model components, such as convection schemes;
\item a database for storing experiments and related information;
\item a mechanism for job submission;
\item support for advanced data assimilation algorithms;
\item optimized support for additional model grids.
\end{itemize}

If additional resources are made available, the ESMF project foresees extending the 
initial framework.  Each of the features listed above is a possible addition. 




















