% $Id: ESMF_install.tex,v 1.25 2003/05/28 21:23:05 flanigan Exp $

\subsection{ESMF Download Options}

Major releases of the ESMF software can be downloaded by following
the instructions on the 
the {\bf Downloads \& Documentation} page on the ESMF 
website, \htmladdnormallink{http://www.esmf.ucar.edu}{http://www.esmf.ucar.edu}.  There are two options for using the ESMF:

\begin{itemize}
\item Download a pre-built ESMF shared object library and
test applications for a particular platform.  If you choose
this approach, you can skip ahead to Section~\ref{UsingLibrary},
Using the ESMF.  
\item Download the full ESMF source code, and compile and link
the framework to any necessary system libraries.  This will
result in a shared object file (with a *.so extension)
that can be linked in with the user's code, or with the demo
{ESMF\_COUPLED\_FLOW} executable.  In this case you will need
to follow all the instructions in subsequent sections of the 
{\it Quick Start} guide, beginning with Section~\ref{InstallProcedures},
Installation.
\end{itemize}

You may find it necessary to build the ESMF yourself
if we do not offer a shared object library for the current
version of your compiler.  The compiler versions that we offer
shared objects for are noted on the download web page.

\subsection{Installation}
\label{InstallProcedures}

% $Id$


The following compilers and utilities are required for compiling, linking and
testing the ESMF software:
\begin{itemize}
\item Fortran90 (or later) compiler;
\item C++ compiler;
\item MPI implementation compatible with the above compilers (but see below);
\item GNU's \htmladdnormallink{gcc compiler}{http://gcc.gnu.org} -
for a standard cpp preprocessor implementation;
\item \htmladdnormallink{GNU Make}{http://www.gnu.org/software/make/make.html}; 
\item \htmladdnormallink{Perl}{http://www.perl.com/download.csp} - for running
test scripts.
\end{itemize} 

Internal packages that can optionally reference external libraries:
\begin{itemize}
\item LAPACK - version 3.x or newer
\item ParallelIO (PIO) - version 2.5.7 or newer
\item yaml-cpp - tag yaml-cpp-0.6.2 or newer
\end{itemize}

Optional external packages that must be specified for certain functions:
\begin{itemize}
\item NetCDF - version 3.6.x or newer (version 4.4 or newer required by PIO)
\item parallel-NetCDF - version 1.2.0 or newer (version 1.12 or newer required by PIO)
\item Xerces - version 3.1.0 or newer
\end{itemize}  

ESMF can be built using a single-processor MPI-bypass library
that comes with ESMF by setting {\tt ESMF\_COMM=mpiuni}. This allows ESMF applications
to be linked and run in single-process mode.

In order to build html and pdf versions of the ESMF documentation, 
\htmladdnormallink{\LaTeX}{http://www.latex-project.org},
the \htmladdnormallink{latex2html}{http://www.latex2html.org}
conversion utility, and the Unix/Linux {\tt dvipdf} utility must be installed.
The csh shell is also required to complete the documentation build.


\subsubsection{ESMF Environment Variables}

The following environment variables must be set:
\begin{verbatim}
  ESMF_DIR      top-level ESMF directory
  ESMF_ARCH     platform and compiler configuration
\end{verbatim}

On Alpha machines an additional environment variable needs
to be set:

\begin{verbatim}
  ESMF_PROJECT  Load Sharing Facility (LSF) project name 
\end{verbatim}

On an Alpha machine, test and demo applications are run using 
the bsub command.  The value of ESMF\_PROJECT is used as the 
argument for bsub's -P option. The -P option assigns a job to 
a specific project.  


\subsubsection{Supported Platforms}
% $Id: 

% List of architectures supported.  This file is 
% meant to be included in a user doc.

The following platforms are currently supported:

\begin{tabular}{lll}
{\tt ESMF\_ARCH}  & {\tt alpha}      &  OSF1, native compilers. \\
                  & {\tt IRIX64}     &  IRIX, MIPSpro/mpt 64 bit. \\
                  & {\tt rs6000\_sp}  &  AIX, mpxlf90\_r, mpcc\_r, and mpCC\_r 32 bit.  \\
                  & {\tt rs6000\_64}  &  AIX, mpxlf90\_r, mpcc\_r, and mpCC\_r
64 bit.  \\
\end{tabular}


Simultaneous multiple architecture builds are supported, with
one restriction; the test cases may only be run on one platform at a time. 

\subsubsection{Building the ESMF Libraries}
\label{BuildESMF}

Build the library with the command:
\begin{verbatim}
  gmake BOPT=g  
\end{verbatim}
  for a debug version or
\begin{verbatim}
  gmake BOPT=O  
\end{verbatim}
  for an optimized version.

% GNU make requirement.  File in build/doc
% $Id: 

% Text about GNU make  This file is 
% meant to be included in a user doc.

GNU make is required to build the library.  On some
systems this will be just the command \texttt{make}.  On others 
it might be installed as \texttt{gmake} or even \texttt{gnumake}.
In any event, use the --version option with the make command
to determine if it is GNU make.


Build options that enable you to copy the library and *.mod files to
specified directories are explained in Section~\ref{BuildOptions}. 

\subsubsection{Building the ESMF Documentation}
\label{BuildDocumentation}

\noindent To build documentation:
\begin{verbatim}
  gmake dvi           ! Makes the dvi files.
  gmake pdf           ! Makes the pdf files.
  gmake html          ! Creates the html directory.
  gmake alldoc        ! Builds all the above documents.
\end{verbatim}

\noindent To build documentation for one module:

\noindent First change directory to the where the desired module's documentation resides;  for
example, to build the {\tt TimeMgr} documentation start with:

\begin{verbatim}
cd ${ESMF_DIR}/src/Infrastructure/TimeMgr/doc
\end{verbatim}

\noindent Next issue one of the following commands:
\begin{verbatim}
  gmake dvi      ! Builds local dvi files.
  gmake pdf      ! Builds local pdf files.
  gmake html     ! Builds local html files.
  gmake alldoc   ! Builds all of the local documents.
\end{verbatim}

\noindent The output from this local documentation build is in the top level {\tt doc}
directory, as with the previous commands.






