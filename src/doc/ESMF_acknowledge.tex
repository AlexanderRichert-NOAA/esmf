
\vspace{2in}
\begin{center}
{\bf Acknowledgements}
\end{center}

The ESMF software is based on the contributions of a broad community.
Below are the software packages that are included in ESMF or strongly 
influenced our design.  We'd like to express our gratitude to the 
developers of these codes for access to their software as well as their 
ideas and advice.

\begin{itemize}

\item Parallel I/O (PIO) developers at NCAR and DOE Laboratories for their 
excellent work on this package and their help in making it work with ESMF

\item The Spherical Coordinate Remapping and Interpolation Package (SCRIP) 
from Los Alamos, which informed the design of our regridding functionality

\item The Model Coupling Toolkit (MCT) from Argonne National Laboratory,
on which we based our sparse matrix multiply approach to general 
regridding

\item The Inpack configuration attributes package from NASA Goddard, 
which was adapted for use in ESMF by members of NASA Global Modeling and 
Assimilation group

\item The Flexible Modeling System (FMS) package from GFDL and the 
Goddard Earth Modeling System (GEMS) from NASA Goddard, both of which 
provided inspiration for the overall ESMF architecture 

\item The Common Component Architecture (CCA) effort within the Department
of Energy, from which we drew many ideas about how to design components

\item The Vector Signal Image Processing Library (VSIPL) and its
predecessors, which informed many aspects of our design, and the 
radar system software design group at Lincoln Laboratory

\item The Portable, Extensible Toolkit for Scientific Computation (PETSc) 
package from Argonne National Laboratories, on which we 
based our initial makefile system

\item The Community Climate System Model (CCSM) and Weather Research and
Forecasting (WRF) modeling groups at NCAR, who have provided valuable
feedback on the design and implementation of the framework

\end{itemize}