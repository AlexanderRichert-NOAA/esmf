
\section{The ESMF Application Programming Interface}

The ESMF Application Programming Interface (API) is based on the
object-oriented programming concept of a {\bf class}.  A class is a 
software construct that's used for grouping a set of related variables 
together with the subroutines and functions that operate on them.  We 
use classes in ESMF because they help to organize the code, and often 
make it easier to maintain and understand.  A particular instance
of a class is called an {\bf object}.  For example, Field is an 
ESMF class.  An actual Field called {\tt temperature} is an object. 
That is about as far as we will go into software engineering
terminology.  

The Fortran interface is implemented so that the variables associated
with a class are stored in a derived type.  For example, an 
{\tt ESMF\_Field} derived type stores the data array, grid 
information, and metadata associated with a physical field.
The derived type for each class is stored in a Fortran module, and 
the operations associated with each class are defined as module
procedures.  We use the Fortran features of generic functions and
optional arguments extensively to simplify our interfaces.

The modules for ESMF are bundled together and can be accessed with a 
single {\tt USE} statement, {\tt USE ESMF\_Mod}.

\subsection{Standard Methods and Interface Rules}

ESMF defines a set of standard methods and interface rules that
hold across the entire API.  These are: 

\begin{itemize}

\item {\tt ESMF\_<Class>Create()} and {\tt ESMF\_<Class>Destroy()}, for creating and 
destroying classes.  The {\tt ESMF\_<Class>Create()} method allocates 
memory for the class structure itself and for internal variables, and
initializes variables where appropriate.  It is always written as a 
Fortran function that returns a derived type instance of the class.

\item {\tt ESMF\_<Class>Set()} and {\tt ESMF\_<Class>Get()}, for setting 
and retrieving a particular item or flag.  In general, these methods are
overloaded for all cases where the item can be manipulated as a
name/value pair.  If identifying the item requires more than a 
name, or if the class is of sufficient complexity that overloading
in this way would result in an overwhelming number of options, we 
define specific {\tt ESMF\_<Class>Set<Something>()} and 
{\tt ESMF\_<Class>Get<Something>()} interfaces.

\item {\tt ESMF\_<Class>Add()} and 
{\tt ESMF\_<Class>Remove()} for manipulating 
items that can be appended or inserted into a list of like
items within a class.  For example, the {\tt ESMF\_StateAdd()}
method adds another Field to the list of Fields contained
in the State class.

\item {\tt ESMF\_<Class>Print()}, for printing the contents of a class to 
standard out.  This method is mainly intended for debugging.

\item {\tt ESMF\_<Class>ReadRestart()} and {\tt ESMF\_<Class>WriteRestart()}, 
for saving the contents of a class and restoring it exactly.  Read
and write restart methods have not yet been implemented for most
ESMF classes, so where necessary the user needs to write restart 
values themselves.

\item {\tt ESMF\_<Class>Validate()}, for determining whether a class is 
internally consistent.  For example, {\tt ESMF\_FieldValidate()} checks 
whether the Array and Grid associated with a Field are consistent.

\end{itemize}

{\bf EXAMPLE}

In this simple example, an ESMF Field is created with the 
name {\tt 'temp'}.  

\begin{verbatim}

USE ESMF_Mod

type ESMF_Field :: field

field = ESMF_FieldCreate('temp')

\end{verbatim}

\subsection{Deep and Shallow Classes}
\label{sec:deepshallow}

The ESMF contains two types of classes.  {\bf Deep} classes require
{\tt ESMF\_<Class>Create()} and {\tt ESMF\_<Class>Destroy()} calls.
They take significant time to set up and should not be created
in a time-critical portion of code.  Deep objects persist 
even after the method in which they were created has returned.
Most classes in ESMF, including Fields, FieldBundles, Arrays,
ArrayBundles, Grids, and Clocks, fall into this category.

{\tt Shallow} classes do not require {\tt ESMF\_<Class>Create()}
 and {\tt ESMF\_<Class>Destroy()} calls.  They can simply be declared
and their values set using an {\tt ESMF\_<Class>Set()} call.  
Examples of shallow classes are Times, TimeIntervals, and 
ArraySpecs.  Shallow classes do not take long to set up and can
be declared and set within
a time-critical code segment.  Shallow objects stop existing when
the method in which they were declared has returned.  

An exception to this is when a shallow object, such as a Time, 
is stored in a deep object such as a Clock.  The Clock then
carries a copy of the Time in persistent memory.  The Time is
deallocated with the {\tt ESMF\_ClockDestroy()} call.

See Section \ref{sec:overallimpl}, Overall Design and Implementation 
Notes, for a brief discussion of deep and shallow classes from 
an implementation perspective.  For an in-depth look at the design 
and inter-language issues related to deep and shallow classes,
see the \htmladdnormallink{{\it ESMF Implementation Report}}{http://www.esmf.ucar.edu/documents/IMPL_repdoc}.

\subsection{Special Methods}

The following are special methods which, in one case,
are required by any application using ESMF, and in the 
other case must be called by any application that is using 
ESMF Components.

\begin{itemize}

\item {\tt ESMF\_Initialize()} and {\tt ESMF\_Finalize()} are required 
methods that must bracket the use of ESMF within an application.  
They manage the resources required to run ESMF and shut it down
gracefully.
\item {\tt ESMF\_<Type>CompInitialize()}, {\tt ESMF\_<Type>CompRun()}, and 
{\tt ESMF\_<Type>CompFinalize()} are component methods that are used at the 
highest level within ESMF.  {\tt <Type>} may be {\tt <Grid>}, for 
Gridded Components such as oceans or atmospheres, or
{\tt <Cpl>}, for Coupler Components that are used to connect 
them.  The content of these methods is not part of the ESMF.  
Instead the methods call into associated Fortran subroutines within 
user code.

\end{itemize}

\subsection{The ESMF Data Hierarchy}

The ESMF API is organized around an hierarchy of classes that 
contain model data.  The operations that are performed
on model data, such as regridding, redistribution, and halo 
updates, are methods of these classes.  

The main data classes in ESMF, in order of increasing complexity, are:
\begin{itemize}
\item {\bf Array} An ESMF Array is a distributed, multi-dimensional 
array that can carry information such as its type, kind, rank, and 
associated halo widths.  It contains a reference to a native Fortran array.
\item {\bf ArrayBundle}  An ArrayBundle is a collection of Arrays, not
necessarily distributed in the same manner.  It is useful for performing
collective data operations and communications. 
\item {\bf Field}  A Field represents a physical scalar or vector field.
It contains a reference to an Array along with grid information and metadata.
\item {\bf FieldBundle}  A FieldBundle is a collection of Fields discretized 
on the same grid.  The staggering of data points may be different for 
different Fields within a FieldBundle.  Like the ArrayBundle, it is
useful for performing collective data operations and
communications.
\item {\bf State}  A State represents the collection of data that a 
Component either requires to run (an Import State) or can make 
available to other Components (an Export State).
States may contain references to Arrays, ArrayBundles, Fields,
FieldBundles, or other States. 
\item {\bf Component}  A Component is a piece of software 
with a distinct function.  ESMF currently recognizes two types 
of Components.  Components that represent a physical domain 
or process, such 
as an atmospheric model, are called Gridded Components since they are 
usually discretized on an underlying grid.  The Components 
responsible for regridding and transferring data between Gridded 
Components are called Coupler Components.  Each Component
is associated with an Import and an Export State.  Components
can be nested so that simpler Components are contained within more
complex ones.

\end{itemize}

Underlying these data classes are native language arrays.  ESMF allows 
you to reference an existing Fortran array to an ESMF Array or 
Field so that ESMF data classes can be readily 
introduced into existing code.  You can perform communication operations 
directly on Fortran arrays through the VM class, which serves 
as a unifying wrapper for distributed and shared memory communication 
libraries.

\subsection{ESMF Spatial Classes}
\label{sec:spatialclasses}

Like the hierarchy of model data classes, ranging from the 
simple to the complex, ESMF is organized around an hierarchy of 
classes that represent different spaces associated with a computation.
Each of these spaces can be manipulated, in order to give
the user control over how a computation is executed.  For Earth system
models, this hierarchy starts with the address space associated
with the computer and extends to the physical region described by
the application.   The main spatial classes in ESMF, from
those closest to the machine to those closest to the application, are:

\begin{itemize}

\item The {\bf Virtual Machine}, or {\bf VM} The ESMF VM is an 
abstraction of a parallel computing environment that encompasses 
both shared and distributed memory, single and multi-core systems.
Its primary purpose is resource allocation and management. Each Component
runs in its own VM, using the resources it defines. The elements of a VM
are {\bf Persistent Execution Threads}, or {\bf PETs}, that are
executing in {\bf Virtual Address Spaces}, or {\bf VASs}. A simple
case is one in which every PET is associated with a single MPI process.
In this case every PET is executing in its own private VAS. If Components
are nested, the parent component allocates a subset of its PETs to its
children. The children have some flexibility, subject to the constraints of
the computing environment, to decide how they want to use the
resources associated with the PETs they've received.

\item {\bf DELayout}  A DELayout represents a data decomposition
(we also refer to this as a distribution).  Its
basic elements are {\bf Decomposition Elements}, or {\bf DEs}.  
A DELayout associates a set of DEs with the PETs in a VM.  DEs are not
necessarily one-to-one with PETs.  For cache blocking,
or user-managed multi-threading, more DEs than PETs may be defined.
Fewer DEs than PETs may also be defined if an application requires it.

\item {\bf DistGrid}  A DistGrid represents the index space
associated with a grid.  It is a useful abstraction because
often a full specification of grid coordinates is not necessary
to define data communication patterns.  The DistGrid contains
information about the sequence and connectivity of data points,
which is sufficient information for many operations.  Arrays
are defined on DistGrids.

\item {\bf Array} An Array defines how the index space described
in the DistGrid is associated with the VAS of each PET. This association
considers the type, kind and rank of the indexed data. Fields are
defined on Arrays.

\item {\bf Grid}  A Grid is an abstraction of a physical space.  
It associates a coordinate system, a set of coordinates, and 
a topology to a collection of grid cells.  Grids in ESMF are
comprised of DistGrids plus additional coordinate information. 

\item {\bf Field}  A Field may contain more dimensions than the 
Grid that it is discretized on.  For example, for convenience 
during integration, a user may want to define a single Field object 
that holds snapshots of data at multiple times.  Fields also 
keep track of the stagger location of a Field data point within its 
associated Grid cell.

\end{itemize}

\subsection{ESMF Maps}

In order to define how the index spaces of the spatial classes relate
to each other, we require either implicit rules (in which case the
relationship between spaces is defined by default), or special Map arrays
that allow the user to specify the desired association.  The form of the 
specification is usually that the position of the array element carries
information about the first object, and the value of the array element carries
information about the second object.  ESMF includes a {\tt distGridToArrayMap},
a {\tt gridToFieldMap}, a {\tt distGridToGridMap}, and others.

\subsection{ESMF Specification Classes}

It can be useful to make small packets
of descriptive parameters.  ESMF has one of these:
\begin{itemize}
\item {\bf ArraySpec}, for storing the specifics, such as type/kind/rank,
of an array.
\end{itemize}

\subsection{ESMF Utility Classes}

There are a number of utilities in ESMF that can be used independently.
These are:
\begin{itemize}
\item {\bf Attributes}, for storing metadata about Fields,
FieldBundles, States, and other classes.
\item {\bf TimeMgr}, for calendar, time, clock and alarm functions.
\item {\bf LogErr}, for logging and error handling.
\item {\bf Config}, for creating resource files that can replace namelists
as a consistent way of setting configuration parameters.
\end{itemize}

\section{Overall Rules and Behavior}

\subsection{Allocation Rules}

The basic rule of allocation and deallocation for the ESMF is:
whoever allocates it is responsible for deallocating it.

ESMF methods that allocate their own space for data will
deallocate that space when the object is destroyed. 
Methods which accept a user-allocated buffer, for example
{\tt ESMF\_FieldCreate()} with the {\tt ESMF\_DATA\_REF} flag,
will not deallocate that buffer at the time the object is
destroyed.  The user must deallocate the buffer
when all use of it is complete.

Classes such as Fields, FieldBundles, and States may have Arrays, 
Fields, Grids and FieldBundles created externally and associated with
them.  These associated items are not destroyed along with the rest  
of the data object since it is possible for the items to be added 
to more than one data object at a time (e.g. the same Grid could 
be part of many Fields).  It is the user's responsibility to delete 
these items when the last use of them is done.

\subsection{Equality and Copying Objects}

The equal sign operator in ESMF does not generate any special 
behavior on the part of the framework.  If the user decides to set
one object equal to another, the internal contents will simply be
copied.  That means that if there is a pointer within the object
being copied, the pointer will be replicated and the data pointed to 
will be referenced by the object copy.  As a matter of style and 
safety, users should try to avoid exploiting such implicit behavior.
A preferable approach is to use a class creation or duplication 
method.  Unfortunately, not all classes have duplication methods yet.

\subsection{Attributes}

Attributes are (name, value) pairs, where
the name is a character string and the value can be either a single
value or list of {\tt int}/{\tt I*4}, {\tt double}/{\tt R*8},
logical ({\tt ESMF\_Logical}), or {\tt char *}/{\tt character} values.
Attributes can be associated with Fields, FieldBundles, and States. 
Mixed types are not allowed in a single attribute, and all attribute
names must be unique within a single object.    Attributes are set
by name, and can be retrieved either directly by name or by querying
for a count of attributes and retrieving names and values
by index number.

\section{Integrating ESMF into Applications}

Depending on the requirements of the application, the user may 
want to begin integrating ESMF in either a top-down or bottom-up 
manner.  In the top-down approach, tools at the superstructure 
level are used to help reorganize and structure the interactions
among large-scale components in the application.  It is appropriate
when interoperability is a primary concern; for example, when 
several different versions or implementations of components are going 
to be swapped in, or a particular component is going to be used 
in multiple contexts.  Another reason for deciding on a top-down 
approach is that the application contains legacy code that for 
some reason (e.g., intertwined functions, very large,
highly performance-tuned, resource limitations) there is little 
motivation to fully restructure.  The superstructure can usually be 
incorporated into such applications in a way that is non-intrusive.

In the bottom-up approach, the user selects desired utilities 
(data communications, calendar management, performance profiling,
logging and error handling, etc.) from the ESMF infrastructure 
and either writes new code using them, introduces them into 
existing code, or replaces the functionality in existing code 
with them.  This makes sense when maximizing code reuse and 
minimizing maintenance costs is a goal.  There may be a specific
need for functionality or the component writer may be starting
from scratch.  The calendar management utility is a popular
place to start.

\subsection{Using the ESMF Superstructure}

The following is a typical set of steps involved in adopting
the ESMF superstructure.  The first two tasks, which occur 
before an ESMF call is ever made, have the potential to be 
the most difficult and time-consuming.  They are the work 
of splitting an application into components and ensuring that
each component has well-defined stages of execution.  ESMF
aside, this sort of code structure helps to promote application
clarity and maintainability, and the effort put into it is likely
to be a good investment.

\begin{enumerate}

\item Decide how to organize the application as discrete Gridded 
and Coupler Components.  This might involve reorganizing code
so that individual components are cleanly separated and their 
interactions consist of a minimal number of data exchanges.

\item Divide the code for each component into initialize, run, and
finalize methods.  These methods can be multi-phase, e.g., 
{\tt init\_1, init\_2}.

\item Pack any data that will be transferred between components
into ESMF Import and Export State data structures.  This is done
by first wrapping model data in either ESMF Arrays or Fields.
Arrays are simpler to create and use than Fields, but carry less
information and have a more limited range of operations.
These Arrays and Fields are then added to Import and
Export States.  They may be packed into ArrayBundles or
FieldBundles first, for more efficient communications.
Metadata describing the model data can also be added.
At the end of this step, the data to be transferred between
components will be in a compact and largely self-describing
form.

\item Pack time information into ESMF time management data 
structures.

\item Using code templates provided in the ESMF distribution, create
ESMF Gridded and Coupler Components to represent each component
in the user code.

\item Write a set services routine that sets ESMF entry 
points for each user component's initialize, run, and finalize 
methods.

\item Run the application using an ESMF Application Driver.

\end{enumerate} 











