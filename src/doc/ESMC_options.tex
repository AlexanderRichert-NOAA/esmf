% $Id$

\section{Appendix A: Master List of Constants}
\label{const:cmaster}

\subsection{ESMC\_CALKIND}
This flag is documented in section \ref{const:calkindflag_c}.

\subsection{ESMC\_COORDSYS}
This flag is documented in section \ref{const:ccoordsys}.

\subsection{ESMC\_DECOMP}
\label{const:cdecompflag}
{\sf DESCRIPTION:\\}
Indicates how DistGrid elements are decomposed over DEs.

The type of this flag is:

{\tt type(ESMC\_Decomp\_Flag)}

The valid values are:
\begin{description}
\item [ESMC\_DECOMP\_BALANCED]
      Decompose elements as balanced as possible across DEs. The maximum 
      difference in number of elements per DE is 1, with the extra elements on
      the lower DEs.
\item [ESMC\_DECOMP\_CYCLIC]
      Decompose elements cyclically across DEs.
\item [ESMC\_DECOMP\_RESTFIRST]
      Divide elements over DEs. Assign the rest of this division to the first
      DE.
\item [ESMC\_DECOMP\_RESTLAST]
      Divide elements over DEs. Assign the rest of this division to the last DE.
\end{description}

\subsection{ESMC\_FILEFORMAT}
This flag is documented in section \ref{const:cfileformat}.

\subsection{ESMC\_GRIDITEM}
This flag is documented in section \ref{const:cgriditem}.

\subsection{ESMC\_GRIDSTATUS}
This flag is documented in section \ref{const:cgridstatus}.

\subsection{ESMC\_INDEX}
{\sf DESCRIPTION:\\}
Indicates whether index is local (per DE) or global (per object).

The type of this flag is:

{\tt type(ESMC\_IndexFlag)}
\label{const:indexflag}
The valid values are:
\begin{description}
\item [ESMC\_INDEX\_DELOCAL]
      Indicates that DE-local index space starts at lower bound 1 for each DE.
\item [ESMC\_INDEX\_GLOBAL]
      Indicates that global indices are used. This means that DE-local index
      space starts at the global lower bound for each DE.
\item [ESMC\_INDEX\_USER]
      Indicates that the DE-local index bounds are explicitly set by the user.
\end{description}

\subsection{ESMC\_LINETYPE}
\label{opt:lineType}

{\sf DESCRIPTION:\\}  This argument allows the user to select the path of the line which connects two points on the surface of a sphere.
This in turn controls the path along which distances are calculated and the shape of the edges that make up a cell.

The type of this flag is:

{\tt type(ESMC\_LineType\_Flag)}

The valid values are:
\begin{description}
\item [ESMC\_LINETYPE\_CART]
   Cartesian line. When this option is specified distances are calculated in a straight line through the 3D Cartesian space
   in which the sphere is embedded. Cells are approximated by 3D planes bounded by 3D Cartesian lines between their corner vertices.
   When calculating regrid weights, this line type is currently the default for the following regrid methods: ESMC\_REGRIDMETHOD\_BILINEAR,
   ESMC\_REGRIDMETHOD\_PATCH, ESMC\_REGRIDMETHOD\_NEAREST\_STOD, and  ESMC\_REGRIDMETHOD\_NEAREST\_DTOS.
\item [ESMC\_LINETYPE\_GREAT\_CIRCLE]
   Great circle line. When this option is specified distances are calculated along a great circle path (the shortest distance
   between two points on a sphere surface). Cells are bounded by great circle paths between their corner vertices. When calculating regrid
   weights, this line type is currently the default for the following regrid method: ESMC\_REGRIDMETHOD\_CONSERVE.
\end{description}


\subsection{ESMC\_LOGKIND}
This flag is documented in section \ref{const:clogkindflag}.

\subsection{ESMC\_LOGMSG}
This flag is documented in section \ref{const:clogmsgflag}.

\subsection{ESMC\_MESHELEMTYPE}
This flag is documented in section \ref{const:cmeshelemtype}.

\subsection{ESMF\_METHOD}
\label{const:cmethod}

{\sf DESCRIPTION:\\}  
Specify standard ESMF Component method.

The type of this flag is:

{\tt type(ESMF\_Method\_Flag)}

The valid values are:
\begin{description}
\item [ESMF\_METHOD\_FINALIZE]
      Finalize method.
\item [ESMF\_METHOD\_INITIALIZE]
      Initialize method.
\item [ESMF\_METHOD\_READRESTART]
      ReadRestart method.
\item [ESMF\_METHOD\_RUN]
      Run method.
\item [ESMF\_METHOD\_WRITERESTART]
      WriteRestart method.
\end{description}

\subsection{ESMC\_POLEKIND}
This flag is documented in section \ref{const:cpolekind}.

\subsection{ESMC\_REGION}
\label{const:cregion}
{\sf DESCRIPTION:\\}
Specifies various regions in the data layout of an Array or Field object.

The type of this flag is:

{\tt type(ESMC\_Region\_Flag)}

The valid values are:
\begin{description}
\item [ESMC\_REGION\_TOTAL]
      Total allocated memory.
\item [ESMC\_REGION\_SELECT]
      Region of operation-specific elements.
\item [ESMC\_REGION\_EMPTY]
      The empty region contains no elements.
\end{description}

\subsection{ESMC\_REGRIDMETHOD}
This flag is documented in section \ref{opt:cregridmethod}.

\subsection{ESMC\_STAGGERLOC}
This flag is documented in section \ref{const:cstaggerloc}.

\subsection{ESMC\_TYPEKIND}
\label{const:ctypekind}

{\sf DESCRIPTION:\\}
Named constants used to indicate type and kind combinations supported by the
overloaded ESMC interfaces. The corresponding Fortran kind-parameter constants 
are described in the ESMF\_TYPEKIND section of Appendices of the ESMF Fortran 
reference manual.

The type of these named constants is:

{\tt type(ESMC\_TypeKind\_Flag)}

The named constants are:
\begin{description}
\item [ESMC\_TYPEKIND\_I1]
      Indicates 1 byte integer.
\item [ESMC\_TYPEKIND\_I2]
      Indicates 2 byte integer.
\item [ESMC\_TYPEKIND\_I4]
      Indicates 4 byte integer.
\item [ESMC\_TYPEKIND\_I8]
      Indicates 8 byte integer.
\item [ESMC\_TYPEKIND\_R4]
      Indicates 4 byte real.
\item [ESMC\_TYPEKIND\_R8]
      Indicates 8 byte real.
\end{description}

\subsection{ESMC\_UNMAPPEDACTION}
\label{const:unmappedaction}
{\sf DESCRIPTION:\\}
Indicates what action to take with respect to unmapped destination points
and the entries of the sparse matrix that correspond to these points.

The type of this flag is:

{\tt type(ESMC\_UnmappedAction\_Flag)}

The valid values are:
\begin{description}
	\item[ESMC\_UNMAPPEDACTION\_ERROR]
	An error is issued when there exist destination points in a regridding
	operation that are not mapped by corresponding source points.
	\item[ESMC\_UNMAPPEDACTION\_IGNORE]
	Destination points which do not have corresponding source points are 
	ignored and zeros are used for the entries of the sparse matrix
	that is generated.
\end{description}

