\section{Introduction}
\label{sec:intro}

In this {\it Earth System Modeling Framework (ESMF) Architecture Document} we describe 
the main design features that will enable us to achieve interoperability, ease of 
use, performance portability, and reuse for climate, weather, and data assimilation
applications.  Our goal is to create a software system capable of satisfying
the extensive functional requirements laid out in the \htmladdnormallink{{\it ESMF 
Requirements Document}}{http://www.esmf.ucar.edu/esmf\_docs/ESMF\_reqdoc/} 
\cite{bib:ESMFreqdoc}, while retaining in its design a measure of intuitiveness 
and simplicity. 

This document is intended for those who require a detailed view of the ESMF
from a software engineering perspective; for example, a software developer who
wishes to extend the ESMF to include significantly new functionality.  It assumes 
a basic knowledge of object-oriented design concepts and terminology, familiarity 
with the C, C++, and FORTRAN programming languages, and an understanding of 
diagramming using the Unified Modeling Language (UML).  Most users will 
not need or want this level of technical detail, and are advised to turn to 
the {\it ESMF User's Guide}{\http://www.esmf.ucar.edu/esmf\_docs/ESMF\_usrdoc/} 
and the {\it ESMF User's Reference}{\http://www.esmf.ucar.edu/esmf\_docs/ESMF\_refdoc/} 
as their primary documentation.  These documents contain overviews of the system 
design tailored to the user's perspective.

The ESMF is a structured collection of software building blocks 
that can be used or customized to develop model components, assemble them into an 
application, and run the application.  Unlike a software library, in a framework some
software elements are partly defined or unimplemented.  For example, a framework
may require that a gridded model provide a {\tt GetGrid} method, and specify 
a format for describing the grid, but leave the implementation of the method up
to the framework user.  This relatively non-intrusive sort of standardization enables 
different model components to interact with each other, and also
use general methods provided by the framework (for example a framework supplied
{\tt Regrid} method).

The functional split between developing and combining components is the
foremost feature of the ESMF architecture.  The simplest view of the ESMF 
is that it consists of a layered {\it infrastructure} of utilities and data 
structures for building model components and a {\it superstructure} for coupling 
and running them.  User-provided components fit between these 
layers like the filling in a sandwich.  Each successively higher layer builds
on elements of the one below it.  Utilities such as communication
primitives, logging and profiling tools are used in methods that
manipulate field and grid data structures.  These data structures may be
tightly integrated into user-supplied components, or they may simply 
help to wrap a component's native data structures in order to provide 
a standard interface for coupling.  The superstructure provides a systematic 
approach to sequencing and synchronizing user-supplied components that 
interact to form an application.  The main layers in the ESMF architecture are 
shown in Figure~\ref{fig:sandwich}.  

This document is organized as follows.  Section~\ref{sec:shortscope} is a summary of the 
scope and motivating requirements of the ESMF.  These are explained in more detail in 
the {\it ESMF Requirements Document}.
Section~\ref{sec:strategies} describes how ESMF relates 
to a number of software design paradigms.  Section~\ref{sec:conventions} lists the
conventions used to describe the ESMF design.
Sections~\ref{sec:superclasses},~\ref{sec:fieldclasses}, and
\ref{sec:utilclasses} contain descriptions of the 
major classes in the framework, their structure, their relationship to each other, 
and the sequence of events for key interactions.  Section~\ref{sec:implications} 
contains a preliminary definition of ESMF compliance, and outlines the sorts
of modifications that will be required to adopt the framework.  Section~\ref{sec:glos} is a 
glossary of terms. Prior to these detailed technical discussion, we examine existing frameworks within our 
community and how they have motivated and influenced ESMF design



\subsection{Evolution of Earth Science Frameworks}
\label{sec:archbackground}

The ESMF represents a next step in the evolution of Earth science
frameworks.  The ideas and the experience that inform its design
are drawn from widely used Earth science modeling systems 
developed by ESMF collaborators such as FMS, GEMS, the modular NCAR 
Flux Coupler, and the WRF Advanced Software Framework (e.g., \cite{fms, gems,
wrf, ncarcpl}).  Other ESMF collaborators 
have contributed to the development of modeling and simulation tools in related 
fields \cite{deluca, petsc}.  Each of these efforts is a modular,
portable, high-performance system for creating multi-component applications
on parallel computing platforms.  Each also represents a productive
collaboration between scientists, computational scientists and software
engineers that has continued into ESMF.  

Object-oriented design concepts are used extensively in existing climate and
weather frameworks (see Section~\ref{sec:oop} for more on object-oriented design). 
For example, in GEMS encapsulation 
via derived types and modules increases portability, hides details of the 
computing platform from the user, and results in a very simple and natural
interface for creating coupling applications.  
In FMS and other codes, polymorphism and operator overloading are 
employed throughout to simplify interfaces.  Coupling systems such 
as the NCAR Flux Coupler use the concept of generic components to
increase interoperability.  The WRF system is one of the few 
examples anywhere of an operational code with a programming model 
that effectively and simply allows users to create data structures 
for a cluster-type architecture that mixes shared and distributed memory, 
and that has been carefully designed for high performance on both vector 
supercomputers and microprocessor-based platforms \cite{wrfperfport}.

In ESMF, we intend to retain the scientist-friendly 
aspect of these frameworks as well as the strongest features of their design.  
At the interface, they present coupling and
communication in terms that scientists can easily understand, and they 
offer a FORTRAN language interface that scientists are already familiar 
with.  

Many of these frameworks share features and design strategies in common, yet 
they are as a whole not
interoperable with each other, nor can any currently address the requirements
of the broad Earth system community.  In ESMF, the creators of 
these successful smaller frameworks have joined to collectively create an
entirely new system that extends the features of their frameworks and provides 
enhanced cross-domain interoperability and reuse.








