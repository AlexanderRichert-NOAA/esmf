% $Id: ESMF_use.tex,v 1.21 2005/05/06 20:10:39 nscollins Exp $

\subsection{Using the ESMF}
\label{UsingLibrary}

To use ESMF from Fortran, add the directory that contains
the ESMF {\tt *.mod} file(s),

\begin{verbatim}
$ESMF_DIR/mod/mod$ESMF_BOPT/$ESMF_ARCH.$ESMF_COMPILER.$ESMF_PREC.$ESMF_SITE
\end{verbatim} 

to your search path for {\tt *.mod} files.  For most compilers this path 
is identified either with a {\tt -I} or a {\tt -M}.  You must also link 
with the ESMF library.  For most compilers, adding the {\tt -L} directory 
search flag with the following directory:

\begin{verbatim}
$ESMF_DIR/lib/lib$ESMF_BOPT/$ESMF_ARCH.$ESMF_COMPILER.$ESMF_PREC.$ESMF_SITE
\end{verbatim} 

followed by the {\tt -lesmf} flag, will link in the ESMF library.

More details of how to link on specific platforms are included in the 
next section.

There is a single ESMF module, called {\tt ESMF\_Mod}, that should be 
included in applications with the Fortran {\tt USE} statement.  It 
is not necessary to include any header files in Fortran.

To use ESMF from C/C++, link with the ESMF library 
and include the {\tt ESMC.h} file. 

\subsubsection{Shared Object Libraries}

On some platforms, a shared object library is created in addition to the
standard {\tt .a} library.
Shared object libraries are libraries that are loaded by the first program 
that uses them. All programs that start afterwards automatically use the 
existing shared library. The library is kept in memory as long as any 
active program is still using it. 

Since shared object libraries are pre-linked to system libraries, using them
simplifies life for the user when a variety of system libraries are
required or when system libraries vary a great deal on a 
platform-to-platform basis.  ESMF requires linking to both Fortran90 and
C++ libraries on a set of very non-standardized platforms, and using
shared objects helps to hide some of this complexity.

The order in which shared libraries are presented to 
the linker is important. Library routines must be called before they are 
defined. So, if a library {\bf A} uses functionality provided by library 
{\bf B}, then library {\bf A} must appear before library {\bf B} on the 
link line. 

\subsubsection{Linking}

When building the ESMF libraries on platforms that support both
32 and 64 bit addressing, verify that the ESMF\_PREC environment 
variable is set to match the compile option that was specified 
to build your application.

To link a Fortran application to the ESMF libraries please 
refer to the {\tt link\_rules} files found in the following directories:

\begin{verbatim}
	$ESMF_DIR/build_config/AIX.default.default
	$ESMF_DIR/build_config/IRIX64.default.default
	$ESMF_DIR/build_config/Linux.intel.default
	$ESMF_DIR/build_config/Linux.lahey.default
	$ESMF_DIR/build_config/Linux.pgi.default
	$ESMF_DIR/build_config/OSF1.default.default
\end{verbatim}


\subsubsection{Customized SITE Files}

In an effort to provide platform specific information for building ESMF 
and linking the libraries with your application, a SourceForge 
site, {\tt esmfcontrib}, has been created.
To locate the platform makefiles for a specific institution, check out 
the {\tt build\_config\_files} using the appropriate CVSROOT.
The URL for the {\tt esmfcontrib} SourceForge site is:

\begin{verbatim}
        http://sourceforge.net/projects/esmfcontrib/
\end{verbatim}

Additionally, you may check out all the platform makefile fragments 
for a particular institution from the {\tt esmfcontrib} site. For example, 
to check out the available makefile fragments for platforms at the
National Center for Atmospheric Research, {\tt ncar}, change directories to

\begin{verbatim}
 	$ESMF_DIR/build_config
\end{verbatim}

and use the following CVS command:

\begin{verbatim}
	cvs -z3 -d:ext:$username@cvs.sourceforge.net:/cvsroot/esmfcontrib checkout ncar
\end{verbatim}

The following directories will be checked out:

\begin{verbatim}
	AIX.default.bluesky
	Linux.lahey.longs
\end{verbatim}

To build using these makefiles you must set the environment 
variable {\tt ESMF\_SITE} to {\tt bluesky}, or {\tt longs}.

At the present time, we have files for the following institutions:

\begin{verbatim}
anl  - Argonne National Laboratory
cola - Center for Ocean-Land-Atmosphere Studies
gsfc - Goddard Space Flight Center
mit  - Massachusetts Institute of Technology
ncar - National Center for Atmospheric Research
\end{verbatim}


Users are encouraged to contribute pertinent information to the 
{\tt esmfcontrib} respository.




