% $Id$

Building user applications against an ESMF installation requires that the 
compiler and linker be able to find the appropriate ESMF header, module and 
library files. If this procedure has been documented by the installer of the 
ESMF library on your system then follow the directions provided.

In the absence of installation specific instructions there are two standard
methods supported by ESMF to build user code against the installation. The
first method is based on a GNU makefile fragment that can be included by the
user code build infrastructure. This method requires that the user application
also uses the
\htmladdnormallink{GNU Make}{http://www.gnu.org/software/make/make.html}
system. The second method is based on 
\htmladdnormallink{CMake}{https://cmake.org/}.


\subsection{esmf.mk method}
\label{sec:EsmfMkMethod}

Every ESMF installation provides a file named {\tt esmf.mk} that contains the 
information needed to build a user application gainst the installation. The
location of the {\tt esmf.mk} file should be documented by the party that
installed ESMF on the system. We recommend that a single ESMF specific 
environment variable, {\tt ESMFMKFILE}, be provided by the system that points to 
the {\tt esmf.mk} file. See section \ref{InstallESMF} for the related discussion 
aimed at the person that installs ESMF on a system.

The information in {\tt esmf.mk} is defined in form of variables. In fact, 
syntactically {\tt esmf.mk} is a makefile fragment and can be imported by an 
application specific makefile via the {\tt include} command. All the variables 
in {\tt esmf.mk} start with the "{\tt ESMF\_}" prefix to prevent conflicts. The 
information in {\tt esmf.mk} is fully specified and is not affected by any 
variables set in the user's environment.

The information defined in {\tt esmf.mk} includes Fortran compiler and linker, 
as well as C++ compiler and linker. It further includes the recommended Fortran 
and C++ specific compiler and linker flags for building ESMF applications. One 
way of using the {\tt esmf.mk} is to glean the necessary information from it. 
This information can then be used either directly on the command line when 
compiling a user application, or to hardwire the settings into the application 
specific build system. However, the recommended use of {\tt esmf.mk} is to 
include this file in the application specific makefile directly via the 
{\tt include} command.

The {\tt Makefile} template below demonstrates how a user build system can be 
constructed to leverage the {\tt esmf.mk} file. In practice, most user build 
systems will be more complex. However, this template does show that the added 
complexity introduced by using {\tt esmf.mk} is minimal. Examples of how to use 
this build system in realistic user scenarios can be found in the 
\htmladdnormallink{external demos}{http://www.earthsystemmodeling.org/users/code_examples/external_demos/external_demos.shtml}.

The advantages of using {\tt esmf.mk}, over hard coding suitable compiler and 
linker flags into the user build system directly, are robustness and portability. 
Robustness is a consequence of the fact that everything defined in {\tt esmf.mk} 
corresponds to the exact settings used during the ESMF library build 
(consistency) and during the ESMF test suite build. Using {\tt esmf.mk} thus 
guarantees that the user application is build in the exact same manner as the 
ESMF test suite applications that undergo strict regression testing before every 
ESMF release. Portability means that a user build system, which uses 
{\tt esmf.mk} in the way the template {\tt Makefile} demonstrates, will function 
as expected on any system where ESMF was successfully installed and tested, 
without the need of modifying anything. Every {\tt esmf.mk} is generated during 
a specific ESMF installation using the ESMF tested settings for the host 
platform.

\begin{verbatim}

################################################################################
### Makefile template for user ESMF application, leveraging esmf.mk mechanism ##
################################################################################

################################################################################
### Finding and including esmf.mk ##############################################

# Note: This fully portable Makefile template depends on finding environment
#       variable "ESMFMKFILE" set to point to the appropriate "esmf.mk" file,
#       as is discussed in the User's Guide.
#       However, you can still use this Makefile template even if the person
#       that installed ESMF on your system did not provide for a mechanism to
#       automatically set the environment variable "ESMFMKFILE". In this case
#       either manually set "ESMFMKFILE" in your environment or hard code the
#       location of "esmf.mk" into the include statement below.
#       Notice that the latter approach has negative impact on portability.

ifneq ($(origin ESMFMKFILE), environment)
$(error Environment variable ESMFMKFILE was not set.)
endif

include $(ESMFMKFILE)

################################################################################
### Compiler and linker rules using ESMF_ variables supplied by esmf.mk ########

.SUFFIXES: .f90 .F90 .c .C

.f90:
	$(ESMF_F90COMPILER) -c $(ESMF_F90COMPILEOPTS) $(ESMF_F90COMPILEPATHS) \
          $(ESMF_F90COMPILEFREENOCPP) $<
	$(ESMF_F90LINKER) $(ESMF_F90LINKOPTS) $(ESMF_F90LINKPATHS) \
          $(ESMF_F90LINKRPATHS) -o $@ $*.o $(ESMF_F90ESMFLINKLIBS)        

.F90:
	$(ESMF_F90COMPILER) -c $(ESMF_F90COMPILEOPTS) $(ESMF_F90COMPILEPATHS) \
          $(ESMF_F90COMPILEFREECPP) $(ESMF_F90COMPILECPPFLAGS) $<
	$(ESMF_F90LINKER) $(ESMF_F90LINKOPTS) $(ESMF_F90LINKPATHS) \
          $(ESMF_F90LINKRPATHS) -o $@ $*.o $(ESMF_F90ESMFLINKLIBS)        
        
.c:
	$(ESMF_CXXCOMPILER) -c $(ESMF_CXXCOMPILEOPTS) \
          $(ESMF_CXXCOMPILEPATHSLOCAL) $(ESMF_CXXCOMPILEPATHS) \
          $(ESMF_CXXCOMPILECPPFLAGS) $<
	$(ESMF_CXXLINKER) $(ESMF_CXXLINKOPTS) $(ESMF_CXXLINKPATHS) \
          $(ESMF_CXXLINKRPATHS) -o $@ $*.o $(ESMF_CXXESMFLINKLIBS)

.C:
	$(ESMF_CXXCOMPILER) -c $(ESMF_CXXCOMPILEOPTS) \
          $(ESMF_CXXCOMPILEPATHSLOCAL) $(ESMF_CXXCOMPILEPATHS) \
          $(ESMF_CXXCOMPILECPPFLAGS) $<
	$(ESMF_CXXLINKER) $(ESMF_CXXLINKOPTS) $(ESMF_CXXLINKPATHS) \
          $(ESMF_CXXLINKRPATHS) -o $@ $*.o $(ESMF_CXXESMFLINKLIBS)

################################################################################
### Sample targets for user ESMF applications ##################################

all: esmf_UserApplication esmc_UserApplication

esmf_UserApplication:

esmc_UserApplication:

################################################################################

\end{verbatim}

Notice that the {\tt ESMF\_F90LINKPATHS}, {\tt ESMF\_F90LINKRPATHS}, 
{\tt ESMF\_CXXLINKPATHS}, and {\tt ESMF\_CXXLINKRPATHS} variables used in the
linking targets might contain paths to the specific compiler version, MPI
implementation, and 3rd party libraries (see section \ref{sec:ThirdParty})
used when building ESMF. The paths are explicitly included in order to 
simplify the process of writing an application build system that is consistent
with the ESMF library that is used.

There are, however, situations where it is
desirable to let the application decide what compiler version, MPI version, 
and/or 3rd party library verson (e.g. NetCDF) to use. To this end, {\tt esmf.mk}
defines an alternative set of variables: {\tt ESMF\_F90ESMFLINKPATHS}, 
{\tt ESMF\_F90ESMFLINKRPATHS}, {\tt ESMF\_CXXESMFLINKPATHS}, and 
{\tt ESMF\_CXXESMFLINKRPATHS}. These variables only encode the precise path to
the ESMF library, and do not specify where to find the compiler, MPI, and/or 
3rd party libraries. When using this alternative set of variables, it becomes
the responsibility of the application build system to ensure the required
libraries can be found by the linker, and are compatible with the ESMF
installation.


\subsection{CMake method}

An ESMF CMake find file is located at: \textit{<ESMF source directory>/cmake/FindESMF.cmake}

The find file will parse the \textit{esmf.mk} file created following a successful
ESMF build. (See \ref{sec:EsmfMkMethod} for a description of the \textit{esmf.mk} file.)
The CMake module sets \textit{esmf.mk} variables as global CMake variables. When
using the find file, {\tt ESMFMKFILE} must be set to the filepath of \textit{esmf.mk}.
If this is NOT set, then {\tt ESMF\_FOUND} will always be {\tt FALSE}. If {\tt ESMFMKFILE}
exists, then {\tt ESMF\_FOUND=TRUE} and all ESMF makefile variables will be set
in the global scope. Optionally, set {\tt ESMF\_MKGLOBALS} to a string list to
filter makefile variables. For example, to globally scope only {\tt ESMF\_LIBSDIR}
and {\tt ESMF\_APPSDIR} variables, use this CMake command in \textit{CMakeLists.txt}: \verb|set(ESMF_MKGLOBALS "LIBSDIR" "APPSDIR")|
