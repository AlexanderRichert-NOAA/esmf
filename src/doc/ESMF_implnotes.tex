% $Id: ESMF_implnotes.tex,v 1.3 2004/03/24 14:15:36 cdeluca Exp $

\section{Overall Design and Implementation Notes}
\label{sec:overallimpl}

\begin{enumerate}

\item {\bf Deep and Shallow classes.}  The deep and shallow classes 
described in Section {sec:deepshallow} differ in how and where they
are allocated within a multi-language implementation environment.  We
distinguish between the implementation language, which is the language
a method is written in, and the calling language, which is the language
that the user application is written in.  Deep classes are allocated 
off the process heap by the implementation language.  Shallow classes
are allocated off the stack by the calling language.  

\item {Base class.} All ESMF classes are built upon a Base class.  The Base 
is used to hold system-wide capabilities, such as Attributes.  Attributes 
are implemented in the Base class so they can be attached to
any object in the system which is built on the Base object.  (This is true
for all deep objects in the system.)  Attributes are created by making a
private copy of the information provided during the Set call.  Lists of
values are supported, but they are not intended for large
data arrays.   Attribute data is copied during a Get operation.

\end{enumerate}
