% $Id: ESMF_RegridWeightGen.tex,v 1.20 2010/12/08 21:46:24 svasquez Exp $

\section{ESMF\_RegridWeightGen}
\label{sec:ESMF_RegridWeightGen}

\subsection{Description}

 In addition to the online regridding functionality, the ESMF distribution also 
contains an exectuable for generating regridding weights. This tool reads in
two grid files and outputs weights for interpolation
between the two grids. The input and output files are all in netcdf format. The grid files are either in 
the same format as is used as an input to SCRIP~\cite{ref:SCRIP}, or in the ESMF unstructured grid format~\ref{sec:example:UnstructFromFile}. The weight file is the same format as is output by SCRIP. The interpolation weights can be generated with
the bilinear, patch, or first order conservative methods decribed below. This application assumes that the
source and destination grids are spherical and that the coordinates given in the files are latitude and longitude
values. This file based regrid weight generation application is fully parallel. This application is used in the 
\htmladdnormallink{ESMF\_RegridWeightGenCheck external demo}{http://www.earthsystemmodeling.org/users/code_examples/external_demos/external_demos.shtml}, so that can serve as an example of its use.

Internally this application uses the ESMF public API to generate the interpolation weights.
If a the source or destination grid is logically rectangular, then {\tt ESMF\_GridCreate()}~\ref{sec:example:2DLogRecFromScrip} is used to create an ESMF\_Grid object. The cell center 
coordinates of the input grid are put into the center stagger location ({\tt ESMF\_STAGGERLOC\_CENTER}). In addition, for conservative regridding, the corner coordinates are also put into the corner stagger location 

({\tt ESMF\_STAGGERLOC\_CORNER}). The 2D coordinates are mapped
into 3D Cartesian coordiantes by setting the {\tt regridScheme} flag to {\tt ESMF\_REGRID\_SCHEME\_FULL3D} while calling 
{\tt ESMF\_FieldRegridStore()}.   The method
{\tt ESMF\_MeshCreate()}~\ref{sec:example:UnstructFromFile} is used to create an ESMF\_Mesh object, if the 
source or destination grid is a cubed sphere grid or an unstructured grid. When making this call, 
the flag {\tt convert3D} is set to {\tt TRUE} to convert the 2D coordinates into 3D Cartesian coordinates.
{\tt ESMF\_FieldRegridStore()} is used to generate the weight table and indicies table representing the interpolation matrix. 

The regridding occurs in 3D to avoid
problems with periodicity and with the pole singularity. This application
supports four options for handling the pole region (i.e. the empty area above the top row of the source grid or below
the bottom row of the source grid).  The first option is to leave the pole region empty (``-p none''), in this 
case if a destination point lies above or below the 
top row of the source grid, it will fail to map, yielding an error. 
With the next two options, the pole region is handled by constructing 
an artificial pole in the center of the top and bottom row of grid points and then filling
in the region from this pole to the edges of the source grid with triangles. 
The pole is located at the average of the position of the points surrounding
it, but moved outward to be at the same radius as the rest of the points
in the grid. The difference between these two artificial pole options is what value is used at the pole. 
The default pole option (``-p all'') sets the value at the pole to be the average of the values
of all of the grid points surrounding the pole. For the other option (``-p N''), the user chooses
a number N from 1 to the number of source grid points around the pole. For
each destination point, the value at the pole is then the average of the N source points
surrounding that destination point. For the last pole option (``-p teeth'') no artificial pole is constructed, instead the
pole region is covered by connecting points across the top and bottom row of the source Grid into triangles. As 
this makes the top and bottom of the source sphere flat, for a big enough difference between the size of
the source and destination pole regions, this can still result in unmapped destination points.  
Only pole option ``none'' is currently supported with the conservative interpolation method (i.e. ``-m conserve''). 

 This regridding application can be used to generate bilinear, patch, or first-order conservative interpolation weights. The default interpolation method
is bilinear. The algorithm used by this application to generate the bilinear weights is the standard one found in
many textbooks.  Each destination point is mapped to a location in the source Mesh, the position of the destination point relative 
to the source points surrounding it is used to calculate the interpolation weights. 

 This application can also be used to generate patch interpolation weights. Patch
interpolation is the ESMF version of a techique called ``patch recovery'' commonly
used in finite element modeling~\cite{PatchInterp1}~\cite{PatchInterp2}. It typically results in better approximations to values and derivatives when compared to bilinear interpolation.  
Patch interpolation works by constructing multiple polynomial patches to represent
the data in a source element. For 2D grids, these polynomials 
are currently 2nd degree 2D polynomials. The interpolated value at the destination point 
is the weighted average of the values of the patches at that point. 

The patch interpolation process works as follows. 
For each source element containing a destination point
we construct a patch for each corner node that makes up the element (e.g. 4 patches for 
quadrilateral elements, 3 for triangular elements). To construct a polynomial patch for
 a corner node we gather all the elements around that node. 
(Note that this means that the patch interpolation weights depends on the source 
element's nodes, and the nodes of all elements neighboring the source element.)  
We then use a least squares fitting algorithm to choose the set of coefficients 
for the polynomial that produces the best fit for the data in the elements. 
This polynomial will give a value at the destination point that fits the source data 
in the elements surrounding the corner node. We then repeat this process for each 
corner node of the source element generating a new polynomial for each set of elements.  
To calculate the value at the destination point we do a weighted average of the values 
of each of the corner polynomials evaluated at that point. The weight for a corner's 
polynomial is the bilinear weight of the destination point with regard to that corner.  
The patch method has a larger stencil than the bilinear, for this reason the patch weight matrix can be correspondingly larger
than the bilinear matrix (e.g. for a quadrilateral grid the patch matrix is around 4x the size of
the bilinear matrix). This can be an issue when performing a regrid weight generation operation close to the memory
limit on a machine. 



First-order conservative interpolation~\cite{ConservativeOrder1} is also available as a regridding method. This method will typically have  
a larger interpolation error than the previous two methods, but will do a much better job of preserving the value of the 
integral of data between the source and destination grid. In this method the value across each source cell
is treated as a constant. The weights for a particular destination cell, are the area of intersection of each 
source cell with the destination cell divided by the area of the destination cell.
Areas in this case are the great circle 
areas of the polygons which make up the cells (the cells around each center are defined by the corner coordinates 
in the grid file). 

\subsection{Usage}

The command line arguments are all keyword based.  Both the long keyward prefixed with \verb+ '--' + or the 
one character short keyword prefixed with {\tt '-'} are supported.  The format to run the application is 
as follows:

\begin{verbatim}
ESMF_RegridWeightGen  [--help]
                      [--source|-s] src_grid_filename 
                      [--destination|-d] dst_grid_filename 
                      [--weight|-w] out_weight_file 
                      [--method|-m] [bilinear|patch|conserve] 
                      [--pole|-p] [none|all|teeth|1|2|..] 
                      --src_type [SCRIP|ESMF] 
                      --dst_type [SCRIP|ESMF]
                      -t [SCRIP|ESMF]

where
  --help              - print the usage message 
  --source or -s      - a required argument specifying the source grid file name

  --destination or -d - a required argument specifying the destination grid file name

  --weight or -w      - a required argument specifying the output regridding 
                        weight file name

  --method or -m      - an optional argument specifying which interpolation method
                        is used.   The value can be one of the following:

                        bilinear     - for bilinear interpolation, also the default
                                       method if not specified.
                        patch        - for patch recovery interpolation
                        conserve     - for first-order conservative interpolation

  --pole or -p        - an optional argument indicating what to do with the pole.  
                        The value can be one of the following:

                        none  - No pole, the source grid ends at the top (and bottom) 
                                row of nodes specified in <source grid>.
                        all   - Construct an artificial pole placed in the center of the 
                                top (or bottom) row of nodes, but projected onto the 
                                sphere formed by the rest of the grid. The value at 
                                this pole is the average of all the pole values. This
                                is the default option.

                        teeth - No new pole point is constructed, instead the holes at 
                                the poles are filled by constructing triangles across 
                                the top and bottom row of the source Grid. This can be
                                useful because no averaging occurs, however, because 
                                the top and bottom of the sphere are now flat, for a
                                big enough mismatch between the size of the destination
                                and source pole regions, some destination points may
                                still not be able to be mapped to the source Grid. 

                        <N>   - Construct an artificial pole placed in the center of the 
                                top (or bottom) row of nodes, but projected onto the
                                sphere formed by the rest of the grid. The value at
                                this pole is the average of the N source nodes next to
                                the pole and surrounding the destination point (i.e.
                                the value may differ for each destination point. Here 
                                N ranges from 1 to the number of nodes around the pole. 

    --src_type        - an optional argument specifying the source grid file type.  The 
                        value could be either SCRIP or ESMF.  Currently, the ESMF 
                        file type is only available for the unstructured grid. The 
                        default option is SCRIP.

    --dst_type        - an optional argument specifying the destination grid file type.  
                        The value could be either SCRIP or ESMF.  Currently, the ESMF 
                        file type is only available for the unstructured grid. The 
                        default option is SCRIP.

    -t                - an optional argument specifying the file types for both the 
                        source and the destination grid files.  The default option 
                        is SCRIP.  If both -t and --src_type or --dst_type are given
                        at the same time and they disagree with each other, an error
                        message will be generated.
\end{verbatim}

