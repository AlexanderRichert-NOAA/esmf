% $Id: ESMF_RegridWeightGen.tex,v 1.11 2010/09/28 21:11:46 peggyli Exp $

\section{ESMF\_RegridWeightGen}
\label{sec:ESMF_RegridWeightGen}

\subsection{Description}

The ESMF\_RegridWeightGen application is used to generate regridding weights in an offline mode.
It takes a source grid file and a destination grid file
in either the SCRIP format or the ESMF Unstructured grid format, and generate a regridding weight 
file in the SCRIP format.  It supports bilinear, patch and conservative regridding methods.  
The source and the destination grids can be either a regularly rectangular grid or an unstructured grid
including cubed sphere grid.  This application is a parallel application that can be run 
on any number of PETs.  This application is used in the \htmladdnormallink{ESMF\_RegridWeightGenCheck external use case}{http://www.earthsystemmodeling.org/developers/test/euc/ESMF_RegridWeightGenCheck_euc.shtml}.

In this application, we use {\tt ESMF\_GridCreate}~\ref{sec:example:2DLogRecFromScrip} to create a 
ESMF\_Grid object if the source or destination grid is a logically rectangular grid.  We put the cell center 
coordinates of the input grid at the center stagger location ({\tt ESMF\_STAGGERLOC\_CENTER}) and we map the 2D coordinates
into 3D Cartesian coordiantes by setting the {\tt regridScheme} flag to {\tt ESMF\_REGRID\_SCHEME\_FULL3D} while calling 
{\tt ESMF\_FieldRegridStore()}.   We use 
{\tt ESMF\_MeshCreate}~\ref{sec:example:UnstructFromFile} to create a ESMF\_Mesh object if the 
source or destination grid is a cubed sphere grid or an unstructured grid. We set {\tt cartisian3D} flag to {\tt TRUE} to
convert the 2D coordinates into 3D Cartesian coordinates while calling {\tt ESMF\_MeshCreate()}. We then use
{\tt ESMF\_FieldRegridStore()} to generate the regridding weight table and indicies table.   

The regridding occurs in 3D to avoid
problems with periodicity and with the pole singularity. 
Unless the pole option is turned off, the polar region is handled by constructing 
an artificial point in the center of the top and bottom row of grid points. 
The pole is located at the average of the position of the points surrounding
it, but moved in the z-direction to be at the same radius as the rest of the points
in the grid. There are a couple of options for what value is used at the pole. 
The default is for the value at the pole to be the average of the values
of all of the grid points surrounding the pole. For another option, the user may also choose
a number N from 1 to the number of source grid points around the pole. For
each destination point, the value at the pole is then the average of the N source points
surrounding that destination point.

 This regridding application can be used to generate either bilinear or patch interpolation weights. The default interpolation method
is bilinear. The algorithm used by this application to generate the bilinear weights is the standard one found in
many textbooks.  Each destination point is mapped to a location in the source Mesh, the position of the destination point relative 
to the source points surrounding it is used to calculate the interpolation weights. 

 This application can also be used to generate patch interpolation weights. Patch
interpolation is the ESMF version of a techique called ``patch recovery'' commonly
used in finite element modeling~\cite{PatchInterp1}~\cite{PatchInterp2}. It typically results in better approximations to values and derivatives when compared to bilinear interpolation.  
Patch interpolation works by constructing multiple polynomial patches to represent
the data in a source element. For 2D grids, these polynomials 
are currently 2nd degree 2D polynomials. The interpolated value at the destination point 
is the weighted average of the values of the patches at that point. 

The patch interpolation process works as follows. 
For each source element containing a destination point
we construct a patch for each corner node that makes up the element (e.g. 4 patches for 
quadrilateral elements, 3 for triangular elements). To construct a polynomial patch for
 a corner node we gather all the elements around that node. 
(Note that this means that the patch interpolation weights depends on the source 
element's nodes, and the nodes of all elements neighboring the source element.)  
We then use a least squares fitting algorithm to choose the set of coefficients 
for the polynomial that produces the best fit for the data in the elements. 
This polynomial will give a value at the destination point that fits the source data 
in the elements surrounding the corner node. We then repeat this process for each 
corner node of the source element generating a new polynomial for each set of elements.  
To calculate the value at the destination point we do a weighted average of the values 
of each of the corner polynomials evaluated at that point. The weight for a corner's 
polynomial is the bilinear weight of the destination point with regard to that corner.  

Global first-order conservative interpolation weights are also available with the 
file based regrid application. When this option is selected a conservative modification
is applied to the interpolation weights using the L2 method.  The L2 method in ESMF is based
on a finite element method to constrain the interpolation for global conservation of 
mass.  The conservative option can be used with either the patch or bilinear interpolation
and any of the pole options.  If this option is selected, integration weights will be
written to the output file containing the interpolation weights. 
Note that conservation is still in beta and should only be used with extreme caution.
It can have interpolation problems with certain combinations of source and destination grid. 
Particularly those where a high resolution area in the destination aligns with a lower resolution 
region in the source. 

The command line arguments are all keyword based.  Both the long keyward prefixed with \verb+ '--' + or the 
one character short keyword prefixed with {\tt '-'} are supported.  The format to run the application is 
as follows:

\begin{verbatim}
ESMF_RegridWeightGen  [--help]
                      [--source|-s] src_grid_filename 
                      [--destination|-d] dst_grid_filename 
                      [--weight|-w] out_weight_file 
                      [--method|-m] [bilinear|patch|conserve] 
                      [--pole|-p] [all|none|1|2|..] 
                      --src_type [SCRIP|ESMF] 
                      --dst_type [SCRIP|ESMF]
                      -t [SCRIP|ESMF]

where
  --help         - print the usage message 
  --source or -s - a required argument specifying the source grid file name

  --destination or -d - a required argument specifying the destination grid file name

  --weight or -w - a required argument specifying the output regridding weight file name

  --method or -m - an optional argument specifying which interpolation method is used.  The value
                   can be one of the following:

                   bilinear     - for bilinear interpolation, also the default method if not specified.
                   patch        - for patch recovery interpolation
                   conservative - for first order conservative interpolation

   --pole or -p - an optional argument indicating what to do with the pole.  The value can be one of
                  the following:

                  none - No pole, the source grid ends at the top (and bottom) row of 
                        nodes specified in <source grid>.
                  all  - Construct an artificial pole placed in the center of the 
                        top (or bottom) row of nodes, but projected onto the sphere 
                        formed by the rest of the grid. The value at this pole is the 
                        average of all the pole values. This is the default option.
                  <N>  - Construct an artificial pole placed in the center of the 
                        top (or bottom) row of nodes, but projected onto the sphere 
                        formed by the rest of the grid. The value at this pole is the 
                        average of the N source nodes next to the pole and surrounding
                        the destination point (i.e. the value may differ for each
                        destination point. Here N ranges from 1 to the number of nodes 
                        around the pole. 

    --src_type - an optional argument specifying the source grid file type.  The value could be either
                 SCRIP or ESMF.  Currently, the ESMF file type is only available for the unstructured 
                 grid. The default option is SCRIP.

    --dst_type - an optional argument specifying the destination grid file type.  The value could be 
                 either SCRIP or ESMF.  Currently, the ESMF file type is only available for the
                 unstructured grid. The default option is SCRIP.

    -t         - an optional argument specifying the file types for both the source and the destination
                 grid files.  The default option is SCRIP.  If both -t and --src_type or --dst_type
                 are given at the same time and they disagree with each other, an error message will
                 be generated.
\end{verbatim}

