%===============================================================================
% CVS $Id: qa.tex,v 1.1 2005/05/05 22:55:21 nscollins Exp $
% CVS $Source: /mnt/twixshare/Storage/Archive-SF-Repos/ESMF_CVS_Repo/esmf/src/doc/dev_guide/Attic/qa.tex,v $
% CVS $Name:  $
%===============================================================================

\section{Quality Assurance}
\label{sec:qa}

Collective reviews and the intelligence and attentiveness of project
staff are the primary mechanisms for software quality assurance.  
Requirements documents, design and architecture documents, code including
test scripts, code documentation, the {\it Build and Test Plan}
(see Sections~\ref{sec:build} and~\ref{sec:testing}), system test
results, and this {\it Developer's Guide} and the procedures specified
therein shall be reviewed by appropriate members of the Joint Specification 
Team.  The purpose of the reviews is to look for inconsistencies,
errors, and inefficiencies, and to increase coordination and awareness
within the project.

The primary author of the item to be reviewed leads the review and is
responsible for modifying the item as per reviewer comments within a 
reasonable time frame.  The ESMF technical leads set up a review schedule
and monitor the review process overall.

In addition to the above we will designate one person - the integrator/
gatekeeper on the core development team - the quality assurance lead.
The quality assurance lead:
\begin{itemize}
\item tracks consistency between documentation and source code before
      releases;
\item ensures that tests are executed and problem reports written;
\item verifies that problem reports are resolved and the tracker updated;
\item monitors major violations of coding standards;
\item attends reviews.
\end{itemize}

As part of this person's integration and testing function, he/she also
evaluates the ESMF software via system tests before it is released.

Tasks may be added to the QA list above as the ESMF project proceeds.
Tasks may also be removed if it is apparent that one or more other people 
on the development team can more appropriately or more effectively perform 
a quality assurance function.







