%===============================================================================
% CVS $Id: roles.tex,v 1.3 2006/08/07 18:49:28 cdeluca Exp $
% CVS $Source: /mnt/twixshare/Storage/Archive-SF-Repos/ESMF_CVS_Repo/esmf/src/doc/dev_guide/roles.tex,v $
% CVS $Name:  $
%===============================================================================

\section{Groups and Roles in ESMF Development}

For more detail on the groups below, including their Terms of
Reference, see the {\it ESMF Project Plan}\cite{bib:ESMFprojectplan}.
This document is available via the {\bf Management} link on 
the ESMF website navigation bar.

\subsection{Core Team}
ESMF software implementation is led by a {\bf Core Team} based in the 
Scientific Computing Division (SCD) of the National Center for
Atmospheric Research (NCAR).  The Core Team relies on close interaction
with customers, and the work of many contributors.  Core Team
activities are open to all active ESMF developers.

\subsubsection{Core Team Roles}
All Core Team members are responsible for helping to 
develop effective project processes and for following
processes that are in place.

Most of the staff on the Core Team are developers.  Developers
interact with customers throughout the whole development 
life cycle: to understand customer requirements and preferences,
and to design, implement, document, and test ESMF software.
Developers in ESMF are to large
extent self-managed, and are responsible for setting
technical strategies that align with project objectives
and customer desires.

The Core Team has one or more testers.  The lead tester, called
the {\bf Integrator}, is responsible for running regression
tests, managing project computing accounts, preparing for releases,
generating project metrics, and overseeing request and bug tracking.

The {\bf Core Team Manager} is responsible for overseeing development
overall, and coordinating activities of multiple developers
so that project schedules and priorities are achieved.
Responsibilities include assigning tasks, deciding on
short-term priorities based on longer-term objectives, 
and monitoring conformance to processes and conventions.
The Core Team Manager also has responsibilities not directly
related to development, such as project administration,
hiring, the acquistion of funding, and the representation of 
the Core Team to Executive Management, the Change Review 
Board, and sponsors.

\subsection{Joint Specification Team (JST)}
The Core Team, along with contributors, customers, technical
managers, and other stakeholders in development, are collectively
referred to as the {\bf ESMF Joint Specification Team}, or {\bf JST}.  
JST membership is open to members of the science, computing, and
related communities.

\subsection{Change Review Board (CRB)}
ESMF development priorities and schedules are set by a 
{\bf Change Review Board}. or {\bf CRB} that consists of individuals
associated with major ESMF initiatives and applications.  
Members of the CRB are chosen by the ESMF Executive Management.

\subsection{Quality Assurance Responsibilities}

Collective public reviews and the intelligence and attentiveness of
project staff are the two primary mechanisms for software quality assurance.  

Requirements, design, code, and project documents such as plans
are subject to public review.  The purpose of the reviews is
to look for models of good documentation, as well as inconsistencies,
errors, inefficiencies, and areas for improvement.  Reviews also
increase coordination and awareness within the ESMF project.

The Integrator ensures that support requests
and bugs are reported, tracked, and resolved, and that automated
test and backup scripts are operating correctly.

The Core Team Manager works with team to ensure that documentation
is sufficient, tracks consistency between documentation and source
code before releases, and monitors conformance to coding and
documentation standards.  The Core Team Manager is also responsible
for ensuring the quality and accuracy of the ESMF source code
distribution overall, by leading the development,
implementation and documentation of development, test, and release
procedures.










