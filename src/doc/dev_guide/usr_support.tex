\subsection{User Support}
\label{sec:usr_support}


\subsubsection{Roles}
The {\bf Advocate} is the staff person assigned to a particular code e.g. GEOS-5. See section ~\ref{core} for a full definition and list of responsibilities. 
the {\bf Handler} is the staff person who actually solves the request. The advocate and the Handler may be the same person or they may be different. See section ~\ref{core} for complete definitiona and list of responsibilities.

\subsubsection{Support Categories}
{\bf New} is a request that has not been replied to.
{\bf Closed} is a request that has been fixed to the user's satisfaction.
{\bf Pending} is a request that has been fixed to the Handler's satisfaction but has not yet been approved by the user.


\subsubsection{Work Flow}
\begin{enumerate}
\item Message received
\item Silverio or in his absence Sylvia, generates a SourceForge Bug, Feature, or Support request ticket
\item If the request contains more than one topic, then Silverio will open multiple tickets, one per topic. This can been done
initially if obvious, or later if more research indicates it is necessary. 

\begin{itemize}
\item The top line of the entry should be WHO: <Requestor Name>.
\item Indicate the institution and model if known
\item Keep title of initial email and the title of the SF ticket the
same or close enough to be able to determine they are one and the same
\item Assign the ticket to the Advocate of that component. 
\end{itemize}

\item Initial contact is made by:

\begin{itemize}
\item The Advocate or
\item A Handler assigned by the Advocate
\item Sylvia if the Advocate is unavailable
\end{itemize}

\item The Handler, which may be the Advocate or another developer assigned by the Advocate, works on the issue keeping the requester informed. Once the work has been completed to the Handlers satisfaction, the ticket can be coded on SourceForge as pending by the Handler.  The mail folder on 
on the IMAP server can also be moved to pending. This can be done by either the Advocate or the Handler, whichever the Advocate prefers.

\item The Handler contacts the customer to gage their satisfaction in the solution. If the customer is satisfied, the ticket may be closed by the Advocate, and the mail folder on the IMAP server moved from open to 
closed by the Advocate.  If the customer does not respond, an attempt at contact will be made every two-weeks for one month.  After one month with no contact, a pending ticket may be closed. The Advocate can choose to hand off this part of the process to Sylvia if desired. 
\end{enumerate}

\subsubsection{General Guidelines for Handling Tickets}
\begin{itemize}
\item Include title and ticket number on all correspondence.
\item Make initial contact within 48 hours even if just to say message received
\item Copy esmf\_support on all replies.
\item Bugs that are fixed should be marked Closed, and Fixed. They should never be deleted. 
\item Bugs that are duplicates should be marked Deleted, and Duplicate. 
\item If the main issue in a bug, feature request, or support request has not been implemented it should stay Open.
\item The Handler is the person responsible for marking a ticket Pending and moving the mail folder from Open to Pending.
\item The Advocate is the person responsible for marking a ticket Closed and moving the mail folder to Closed
\end{itemize}

\subsubsection{IMAP mail archives}
\begin{enumerate}
\item Folders are labeled by component and consolidated as necessary.
\item Silverio will create a folder labeled with the request number. 
\item Silverio will then copy each related email message to the appropriate 
numbered folder. 
\item The number folders will be filed under Open, Closed, or Pending folders 
corresponding to status of the request on the SourceForge system.
\item When a ticket has been closed or marked Pending, the Advocate will move the numbered folder from the Open folder to the Closed or Pending folder.
\item There will be only one New folder to which all unresponded requests will be placed. Once a request has been responded to, Silverio will move the folder to the appropriate institution. Advocates may also move mailfolders around.
\end{enumerate}

\subsubsection{INFO:Code (subject) mail messages}
Advocates need to share the information they have received from their codes with the rest of the Core team. This will be done by sending an email to esmf\_support with a subject line labeled INFO:Code e.g. INFO:CCSM, INFO:GEOS-5, follwed by a subject line e.g. status of support requests or new code contact.  These messages will be filed on the IMAP server (see above section) under the code referenced. All information about a code that is general and not relatedto a specific support request will be archived in this manner. 

\subsubsection{freeCRM}
A client relationship managment tool (freeCRM http://www.freecrm.com) is being used to archive codes, their affiliated contacts and applicable funding information. The following is a list of roles and responsibilities associated with this sofware:
\begin{itemize}
\item Advocates are responsible for the accuracy and completeness of all information associated with codes to which they are assigned.  This information includes a pull down menu that specifies the state of the code's ESMF'ization.
\item The Tester is responsible for creating a back up of all freeCRM data on a monthly basis.
\item The Core Team Manager is responsible for the accuracy and completness of all funding related information.
\item The Support Manager is responsible for creating model 'companies' and informing the tester of any additions so that the back up scripts can be modified. They are additionally responsible for conducting monthly quality control checks of all information in the system.  
\end{itemize} 

\subsection{Battle Rhythm}
There are numerious tasks that need to occur on a cyclical basis. The following lists these tasks, their periodicity, who should complete the task, and the expected results.

\subsubsection{Support}
\begin{enumerate}
\item Monthly
  \begin{itemize}
  \item The support manager will conduct a ticket review meeting in which new tickets will be reviewed, placed on the appropriate tracker, given appropriate key words e.g. LONG:KEY. During this meeting, ticket Handler's will be expected to discussed tickets they have closed or changed to pending. The Core Team Mangager will review and accept or reject the justification for the closures.
   \item The Tester will create a back up file of all freeCRM data and place the back up on the UCAR web server.
  \end{itemize}

\item Weekly
  \begin{itemize}
  \item The Support Manager will purge the Maia Mailguard spam filter
  \item The Support Manager will update the esmf\_community, esmf\_info, esmf\_support, and esmf\_jst mailing lists with the email addresses of those individuals who have downloaded the software.
  \item The Support Manager will review the IMAP folders moving items that have been responded to into the appropriate category and spot check for errors such as missfilings.
  \end{itemize}

\end{enumerate}  
 
