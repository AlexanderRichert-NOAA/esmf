     % $Id: tracking.tex,v 1.11 2006/11/07 22:12:35 murphysj Exp $

\section{Tracking and Metrics}


\subsection{Release Schedule}
\label{sec:build}
The ESMF Release Schedule is generated by the Content Review Board (CRB) (see section ~\ref{crb}) during its quarterly meetings.  The release schedule is posted on the ESMF website home page under {\bf Quick Links}.

\subsection{Tasks Tracking}
\label{tracking}

\subsubsection{Trackers}
SourceForge provides the development team with a series of trackers (see section ~\ref{tracking_tools} for more detail). An item entered into a tracker is
called a {\bf ticket}. Tickets are assigned a unique number and can move from tracker to tracker. 

\subsubsection{Summary}
There are multiple steps to properly tracking tasks. The following sections cover these steps, which are summarized here:
\begin{enumerate}
\item Enter main task in Task Tracker
\item Use key words to link sub tasks in other trackers with the main task in the Task Tracker
\item Label all tickets with {\bf LONG:} that will take longer than two calender weeks
\item Put the completion time estimate after the title (DESIGN/COMPLETION)
\item Label tickets need for the next release with a priority nine.
\end {enumerate}

\subsubsection{Task List (Tracker)}

The task list is used to archive past, current, and future development tasks.  The CRB uses uses this list to determine future release schedules.  It's updated by the Core Team Manager quarterly, prior to CRB meetings.  It is browsable by anyone and is located at
\htmladdnormallink{{http://www.sourceforge.net/projects/esmf (click on Tasks)}}{http://www.sourceforge.net/projects/esmf}.

\subsubsection{Tracking Multiple Sub Tasks Related to the Same Task}

While the task list contains the master list of all items necessary to the development of a particular release, other trackers may contain sub tasks 
necessary to the completion of the main task. In order to identify these sub tasks, developers should label both the master task and the sub tasks with a 
unique key. The key should describe the task, be unique, and be follwed by a colon. Careful placement of keys will allow the CRB to quickly visualize
through the search mechanism all tasks necessary for a new release. The key is highlighted in the example below:

\vspace{5mm}
In Task Tracker: {\bf STDIZE\_INIT}: Standardize Initialization Behavior

\vspace{5mm}
In Bug Tracker:
\begin{itemize}
\item LONG:{\bf STDIZE\_INIT}:Std hand. of Verify rc for declared obj (4)
\item LONG:{\bf STDIZE\_INIT}:Field count from uninitialized Bundle(4)
\item {\bf STDIZE\_INIT}:Fld ct on uninit Bndle is not 0 on nag (4)
\item LONG:{\bf STDIZE\_INIT}:BndleGetFildNmes crashes if Bndle uninit(4)
\end{itemize}

\subsubsection{Labeling Tasks > 2 Weeks}

The CRB only manages tasks that will take longer than two calender weeks. All tickets that are estimated to be two calender weeks or long should
be labeled with {\bf LONG:} prefixed to the title of the ticket. 

\vspace{5mm}
Examples:
\begin{itemize}
\item  {\bf LONG:} wants one-sided communications options (2/3) 
\item  {\bf LONG:} typekind vs type and kind (4)
\end{itemize}

\subsubsection{Estimating Task Completion Time}

The following guidelines should be used when determining task completion time:
\begin{itemize}

\item Estimates are in calendar weeks.
\item Estimates do not include the 25% time spent on support.  Thus items are allocated at 3 weeks per month.
\item Estimates do not include integration and release preparation.
\item Estimates do include any research time that's part of the developer's position. Calendar weeks are not corrected for this.
\item Design estimates do include looking at other packages, design, time for iterating on telecons, design reviews, feedback, and prototyping.
\item Implementation estimates do include implementation, code reivews, testing, and documentation. Documentation time and testing time should both
equal coding time. Alos include any peripheral capabilities that will be needed to support the main item.
\item Use the past as a guide and estimate long.
\end {itemize}

\subsubsection{Labeling Completion Time on Tasks}

An estimated time to complete should be placed at the end of any ticket title labeled with the {\bf LONG:} key. If the time includes a design phase,
this should be listed first e.g. (3/4) equals 3 weeks design plus 4 additional weeks for completion. Examine the two examples in the preceeding section. 
The first item will take two weeks to design and 3 additional weeks to complete. The second item has no design phase and will take 4 weeks to complete. 

\subsubsection{Prioritization of Task Tickets}

The SourceForge trackers allow the user to set the priority of a ticket. Priority nine is used to label tickets necessary for the next release. These are the only tickets that should be labeled at this level. Priority levels one through four can be used by individual developers to prioritize their assigned 
tickets that are unrelated to the next release. All other tickets should be labeled with a priority five.

\subsection{Trackers}
\label{tracking_tools}

All trackers categorize requests similiarly. Pull down menus can assign various criteria including Assignee, Status (Open, Closed, etc.), Category (Array, Regrid, etc.), and Group (Bug, Feature Request, etc.)  Tickets can be sorted and viewed based on these criteria and can be moved from one tracker to another. For example a ticket may start out as support request and be moved to the bug tracker for resolution.  To move a ticket, simply change the type using the type pull down menu. 

\subsubsection{Support Requests}

Support requests are monitored on the Support Request tracker on SourceForge and the esmf\_support mail archive on the IMAP server. (See \htmladdnormallink{Support Tracker}{http://sourceforge.net/tracker/?group_id=38089\&atid=421186}). The complete instructions for handling support requests is located in Section ~\ref{sec:usr_support}.

\subsubsection{Bugs}

Bugs are tracked using the \htmladdnormallink{{SourceForge Bug Tracker}}{http://sourceforge.net/tracker/?group_id=38089\&atid=421185}. Bugs are issues with capabilities that already exist e.g. optimization, documentation, and tests. Bugs can be of short or long duration.  Long duration bugs (> 2 weeks to implement) are marked with a LONG: prefix in the title.  There is no prefix for short duration bugs.

Bugs may be identified independently, or they may originate as a support request. If the bug originated independently, Silverio will create the initial entry. If it originated as a support request, then the {\bf Handler or Advocate} (see section ~\ref{core} for definitions) will transfer the support request to the bug tracker. 

When a bug is opened, provide as much detail as possible to help the Handler reproduce the problem. Indicate where the bug was found e.g. trunk, branch, or both. A bug that exists on both the trunk and a branch will generally be fixed only on the trunk. If the bug is fixed on both the trunk and branch, the ticket must  state this explicitly. When the bug is fixed, the Handler adds a note to the ticket and changes the status to Pending. Once the customer who originated the bug or initial support request has confirmed the fix, the ticket can be closed. 

\subsubsection{Features}

Features are various requests that involve creating new functionality. The prefix LONG is to be used to identify tasks with a duration greater than 2 weeks. Features are tracked using the 
\htmladdnormallink{{Feature Tracker}}{http://sourceforge.net/tracker/?group_id=38089\&atid=421188}

\subsubsection{Operations}
Current infrastructure related items are tracked using the \htmladdnormallink{{Operations Tracker}}{http://sourceforge.net/tracker/?group_id=38089\&atid=421187}

\subsubsection{Code}
The Advocates (see section ~\ref{core} for definition) and their assigned codes are listed
in the \htmladdnormallink{{Code Tracker}}{http://sourceforge.net/tracker/?group_id=38089\&atid=421189}

\subsubsection{Task}
The \htmladdnormallink{{Task Tracker}}{http://sourceforge.net/tracker/?group_id=38089} lists various tasks need before each release. Only the CRB can change this tasker. 

\subsection{Metrics}

The Integrator is responsible for tracking various aspects of the
ESMF project to measure the implementation and testing progress.
\subsubsection{Unit and System Tests Coverage}
Throughout the ESMF software development process, the software is constantly
being unit and system tested. It is essential to know the percentage of the
code that has been fully or partially tested. A script has been written that 
lists all the public interface subprograms (functions and subroutines) and 
determines which of these have been unit and/or system tested. From this script
output, the tested percentage can be calculated.
\subsubsection{ESMF Requirements Coverage}
An Excel spreadsheet is maintained listing the requirements as described in the
ESMF Requirements Document. A hardcopy of the spreadsheet has been posted in a
convenient place for the core team. Using color coding the core team indicates
which requirements have been implemented and of these which have been tested.
\subsubsection{Source Lines of Code, SLOC}
The ESMF SLOC information is provided in an internal ESMF webpage. The
SLOC data is presented in graph form with the following columns:
\begin{enumerate}
\item[Fortran] 
\item[C++] 
\item[c] 
\item[Makefiles] 
\item[SLOC Total] 
\item[Lines of text] 
\end{enumerate}

The graphs are a monthly breakdown from January 1, 2002 to the present.
























