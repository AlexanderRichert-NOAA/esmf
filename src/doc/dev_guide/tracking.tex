% $Id: tracking.tex,v 1.6 2006/10/20 17:16:29 murphysj Exp $

\section{Tracking and Metrics}
\label{sec:tracking}

\subsection{Tasks and Releases}
\label{sec:build}

\subsubsection{Release Schedule}
The ESMF Release Schedule is generated by the Content Review Board (CRB) (see \S~\ref{during its
quarterly meetings.  The release schedule is posted on the ESMF website home page under 
{\bf Quick Links}.

\subsubsection{Task List}

The task List is used to archive past, current, and future development tasks.  The CRB uses uses this list
to determine future release schedules.  It's updated by the Core Team Manager quarterly, prior to CRB meetings.  It is browsable by anyone and is located at
\href{http://www.sourceforge.net/projects/esmf}{http://www.sourceforge.net/projects/esmf (click on tasks)} .

\subsection{Trackers}
\label{sec:tracking_tools}

All trackers categorize requests similiarly. Pull down menus can assign various criteria including Assignee, Status (Open, Closed, etc.), Category (Array, Regrid, etc.), and Group (Bug, Feature Request, etc.)  Tickets can be sorted and viewed based on these criteria and can be moved from one tracker to another. For example a ticket may start out as support request and be moved to the bug tracker for resolution.  To move a ticket, simply change the type using the type pull down menu. 

\subsubsection{Support Requests}

Support requests are monitored on the Support Request tracker on SourceForge and the esmf\_support mail archive on the IMAP server. (See \href{http://sourceforge.net/tracker/?group_id=38089\&atid=421186}{Support Request Tracker}). The complete instructions for handling support requests is located in  \S~\ref{sec:usr_support}

\subsubsection{Bugs}

Bugs are tracked using the \href{http://sourceforge.net/tracker/?group_id=38089\&atid=421185}{Source Forge Bug Tracker}). Bugs are issues with capabilities that already exist e.g. optimization, documentation, and tests. Bugs can be of short or long duration.  Long duration bugs (> 2 weeks to implement) are marked with a LONG: prefix in the title.  There is no prefix for short duration bugs.

Bugs may be identified independently, or they may originate as a support request. If the bug originated independently, Silverio will create the initial entry. If it originated as a support request, then the {\bf Handler or Advocate} (see \S~\ref{Advocate} for definitions) will transfer the support request to the bug tracker. 

When a bug is opened, provide as much detail as possible to help the Handler reproduce the problem. Indicate where the bug was found e.g. trunk, branch, or both. A bug that exists on both the trunk and a branch will generally be fixed only on the trunk. If the bug is fixed on both the trunk and branch, the ticket must  state this explicitly. When the bug is fixed, the Handler adds a note to the ticket and changes the status to Pending. Once the customer who originated the bug or initial support request has confirmed the fix, the ticket can be closed. 

\subsubsection{Features}

Features are various requests that involve creating new functionality. The prefix LONG is to be used to identify tasks with a duration greater than 2 weeks. Features are tracked using the 
(\href{http://sourceforge.net/tracker/?group_id=38089\&atid=421188}{Feature Tracker})

\subsubsection{Operations}
Current infrastructure related items are tracked using the(\href{http://sourceforge.net/tracker/?group_id=38089\&atid=421187}{Operations Tracker})

\subsubsection{Code}
The Advocates (see \S~\ref{Advocate} for definition) and their assigned codes are listed
in the (\href{http://sourceforge.net/tracker/?group_id=38089\&atid=421189}{Code Tracker})

\subsubsection{Task}
The (\href{http://sourceforge.net/tracker/?group_id=38089}{Task Tracker}) lists various tasks need before each release. Only the CRB can change this tasker. 


\subsection{Metrics}

The Integrator is responsible for tracking various aspects of the
ESMF project to measure the implementation and testing progress.
\subsubsection{Unit and System Tests Coverage}
Throughout the ESMF software development process, the software is constantly
being unit and system tested. It is essential to know the percentage of the
code that has been fully or partially tested. A script has been written that 
lists all the public interface subprograms (functions and subroutines) and 
determines which of these have been unit and/or system tested. From this script
output, the tested percentage can be calculated.
\subsubsection{ESMF Requirements Coverage}
An Excel spreadsheet is maintained listing the requirements as described in the
ESMF Requirements Document. A hardcopy of the spreadsheet has been posted in a
convenient place for the core team. Using color coding the core team indicates
which requirements have been implemented and of these which have been tested.
\subsubsection{Source Lines of Code, SLOC}
The ESMF SLOC information is provided in an internal ESMF webpage. The
SLOC data is presented in graph form with the following columns:
\begin{enumerate}
\item[Fortran] 
\item[C++] 
\item[c] 
\item[Makefiles] 
\item[SLOC Total] 
\item[Lines of text] 
\end{enumerate}

The graphs are a monthly breakdown from January 1, 2002 to the present.
























