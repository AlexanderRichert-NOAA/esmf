% $Id: tracking.tex,v 1.4 2006/08/07 18:49:28 cdeluca Exp $

\section{Tracking and Metrics}
\label{sec:tracking}

\subsection{Tasks and Releases}
\label{sec:build}

\subsubsection{Release Schedule}
The ESMF Release Schedule is generated by the CRB during its
quarterly meetings.  It's posted on the ESMF website home
page under {\bf Quick Links}.

\subsubsection{Core Team Task List}

The Core Team maintains a Task List to archive past, current,
and future development tasks.  This tool is
what the CRB uses during their meetings to build up a new
release schedule.  It's updated by the Core Team Manager 
quarterly, prior to CRB meetings.  Like other SourceForge
tools it is browsable by anyone.  The Task List is at:  
\htmladdnormallink{http://www.sourceforge.net/projects/esmf} 
{http://www.sourceforge.net/projects/esmf}, under the 
{\it Tasks} link. 

\subsection{Support Requests}
\label{sec:tracking_tools}

Developers use the Support Request tracker on the ESMF 
SourceForge site to monitor and prioritize user requests.
Customers write the esmf\_support@ucar.edu list and their
requests are entered into the tracker by a Core Team
member.  (See \htmladdnormallink{http://www.sourceforge.net/projects/esmf}
{http://www.sourceforge.net/projects/esmf}, under {\bf Support Requests}
on the menu bar.

The Support Request tracker categorizes requests by Assignee, 
Status (Open, Closed, etc.), Category (Array, Regrid, etc.), and
Group (Bug, Feature Request, etc.)  Requests can be sorted and
viewed based on these keys.

\subsection{Bugs}

The Bugs tracker on the main ESMF SourceForge site is used
to record bugs reports.  It works in much the same way as
the Support Request tracker.
(See \htmladdnormallink{http://www.sourceforge.net/projects/esmf}
{http://www.sourceforge.net/projects/esmf}, under {\bf Bugs}
on the menu bar.

When a bug is identified, the tester opens a bug report on SourceForge, giving as much detail
about the bug as possible to help the developer reproduce the problem. When the developer
fixes the bug, he/she logs into SourceForge, adds a note to the bug report, and changes the status
to Pending. This triggers SourceForge to send an email to the originator of the bug report. It is
the responsibility of the originator to verify that the bug has been fixed and close the bug report.

When a bug is found, it may exist on the trunk, a branch or both. The originator of the bug report
should state where the bug was found in the bug report.  A bug that exists on both the trunk and a 
branch will generally be fixed only on the trunk. If the bug is fixed on both the trunk and branch, the 
developer must state this explicitly in the bug report.  Customer queries about known problems for a 
release will be directed to the developers.

\subsection{Metrics}

The Integrator is responsible for tracking various aspects of the
ESMF project to measure the implementation and testing progress.
\subsubsection{Unit and System Tests Coverage}
Throughout the ESMF software development process, the software is constantly
being unit and system tested. It is essential to know the percentage of the
code that has been fully or partially tested. A script has been written that 
lists all the public interface subprograms (functions and subroutines) and 
determines which of these have been unit and/or system tested. From this script
output, the tested percentage can be calculated.
\subsubsection{ESMF Requirements Coverage}
An Excel spreadsheet is maintained listing the requirements as described in the
ESMF Requirements Document. A hardcopy of the spreadsheet has been posted in a
convenient place for the core team. Using color coding the core team indicates
which requirements have been implemented and of these which have been tested.
\subsubsection{Source Lines of Code, SLOC}
The ESMF SLOC information is provided in an internal ESMF webpage. The
SLOC data is presented in graph form with the following columns:
\begin{enumerate}
\item[Fortran] 
\item[C++] 
\item[c] 
\item[Makefiles] 
\item[SLOC Total] 
\item[Lines of text] 
\end{enumerate}

The graphs are a monthly breakdown from January 1, 2002 to the present.
























