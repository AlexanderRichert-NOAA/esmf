% CVS $Id$

\subsection{Preprocessor Usage}

\subsubsection{Using the Preprocessor For Generic Fortran Code}

Fortran allows the creation of generic subprogram names via the
{\tt INTERFACE} facility.  This lets the caller use a simplified
single name for a call.  Then, based on the callers actual argument list,
the compiler will choose an appropriate specific routine to perform
the task.

When several specific variants of a generic call exist, and only
differ in minor ways, for example by ESMF object type, it can be
desirable to have the variants generated from a single definition.
This allows improvements in the reliability and maintainability of
the code.  Feature code can be added in a single place, and all
of the specific routines will automatically incorporate the new code.

By convention, generic ESMF Fortran code uses the GNU C preprocessor
macro facility to generate specific routines.  These routines generally
differ only in the type, kind, and rank of the dummy arguments.  The
internal code is otherwise identical.

When writing a cpp macro, the definition must be 'one line' long.
Since coding an entire subprogram requires many lines to implement,
a backslash character is used at the end of each line to concatenate
multiple lines from the input file into a single, very long, preprocessor
line.  After macro expansion, each preprocessor line needs to be
converted into a sequence of Fortran statements.  So again by
convention, at-sign characters, @, are used as the second-to-last
character in each line.  After cpp has been run, the long lines from
the result are split at the @ characters by converting them to newline
characters.  The {\tt tr} command is used for this translation.

ESMF Fortran files which need to be preprocessed in this manner use
the file name suffix {\tt .cppF90}.  This triggers the makefile system
to use the following sequence when processing the file:

\begin{itemize}

\item The gcc version of cpp is run.  This causes a first pass of
preprocessing to occur.  All macro expansion for generic routines
should be performed in this pass.

\item The {\tt tr} command is run to perform several transformations:

\begin{enumerate}

\item Convert @ characters to newline characters.  As described above, this
converts the one very long line of any cpp macro expansion into multiple
source lines for the compiler.

\item Convert caret, \^\, characters to \# characters
in order to allow selected preprocessing directives to pass through to the
second level of preprocessing performed by the actual compiler.

\item Convert vertical bar, |, characters to apostrophe (single
quote) characters.  Vertical bars should be used where apostrophe characters
need to appear.  This is needed because some versions of cpp look for matching
apostrophes in order to properly issue code.  Note that only lines within a
macro definition, i.e., those ending in @\, need to use this transformation.

\end{enumerate}

\item The Fortran compiler is run.  The built-in preprocessor is used for
the second preprocessing pass.  This pass is used for other preprocessing
tasks - such as including header files needed for compilation, or for
conditional compilation of system-dependent code.

\end{itemize}

A simple example of generic code written in this style is:

\begin{verbatim}

  module ESMF_Example_mod
    use ESMF
    implicit none

! The following header needs to be preprocessed by the compiler,
! not the first cpp pass.  So use a caret as the first character,
! rather than a pound-sign.

^include "my_header.inc"

! Define a generic name with three specific routines

    interface ESMF_generic_name
      module procedure ESMF_ArraySpecifc
      module procedure ESMF_FieldSpecfic
      module procedure ESMF_MeshSpecific
    end interface

  contains

! Define a macro for expansion.  Each line is terminated with a
! @\ sequence.  Also note the use of ## for concatenation
! and # for stringization.

#define MySpecificMacro(mclass, mthis)  \
  subroutine mclass##Specific (mthis, rc) @\
    type(mclass), intent(inout) :: mthis @\
    integer, optional :: rc @\
 @\
    print *, |class = |, #mclass @\
 @\
    if (present (rc)) rc = ESMF_SUCCESSFUL @\
 @\
  end subroutine mclass##Specfic @\

! Expand macro for a few classes

MySpecificMacro(ESMF_Array, array)
MySpefificMacro(ESMF_Field, field)
MySpecificMacro(ESMF_Mesh, mesh)

end module ESMF_Example_mod

\end{verbatim}

Character string concatenation via the Fortran concatenation operator {\tt //}
can be problematic due to the first pass preprocessing treating it as a C++ inline
comment.  An alternative is to use the {\tt ESMF\_StringConcat} function.

\subsubsection{System Dependent Strategy Using Preprocessor} Since the code
must compile across different platforms, a strategy must be adopted to
handle the system differences. Examples of system differences are: the
subroutines (bcopy vs memcopy) or include filenames (strings.h vs str.h,
etc).

Rather than putting architecture names in all the source files, there will
be an ESMF-wide file which contains sections like this: 
\begin{verbatim}

 #ifdef sgi 
 #define HAS_VFORK 0 
 #define BCOPY_FASTER 1 
 #define FOPEN_3RD_ARG 1 
 #endif

 #ifdef sun 
 #define HAS_VFORK 1 
 #define BCOPY_FASTER 0 
 #define FOPEN_3RD_ARG 1 
 #endif

\end{verbatim} This allows system-dependent code to be bracketed with
meaningful names: 

\begin{verbatim} 

 #if HAS_VFORK
    vfork();
 #else
    fork(); 
    exec();
 #endif

\end{verbatim} 
and not an endless string of architecture names.

