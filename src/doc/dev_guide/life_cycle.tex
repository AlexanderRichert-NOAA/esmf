%============================================================================
% CVS $Id: life_cycle.tex,v 1.1 2005/05/05 22:55:21 nscollins Exp $
% CVS $Source: /mnt/twixshare/Storage/Archive-SF-Repos/ESMF_CVS_Repo/esmf/src/doc/dev_guide/Attic/life_cycle.tex,v $
% CVS $Name:  $
%===============================================================================

\section{Software Development Cycle}
\label{sec:life_cycle}

The development environment for the ESMF project has 
several defining characteristics.  First, the ESMF development team, the Joint
Specification Team, is distributed.  This makes incorporating simple, 
efficient communication mechanisms into the development process essential.  
Second, the development team is large, diverse, and mostly self-managed.  
Procedures and conventions must be clear and well documented in order for
the whole team to adopt them.  Third, the initial set of target 
applications for the ESMF, referred to as the Joint Milestone 
Codeset (JMC), is known.  Many of those who developed the JMC are 
active participants in the ESMF project.  We therefore have the 
opportunity to guarantee the ESMF's viability by allowing future users 
and real codes to guide framework development.  For example, we 
prioritize requirements based on the number of JMC applications 
that need a capability.  

\subsection{Software Process Model}

The ESMF software development cycle is based on the staged 
delivery model \cite{mcconnell96}.  The steps in this software development
model are:

\begin{enumerate}
\item {\bf Software Concept}  Collect and itemize the high-level requirements of the system and identify the basic functions that the system must perform.

\item {\bf Requirements Analysis} Write and review a requirements document - a detailed statement of the scientific and computational requirements for the software.

\item {\bf Architectural Design} Define a high-level software architecture that outlines the functions, relationships, and interfaces for major components. Write and review an architecture document.

\item {\bf Stage 1, 2, ..., n} Repeat the following steps creating a potentially releasable product at the end of each stage. Each stage produces a more robust, complete version of the software.

\begin{itemize}
\item {\it Detailed Design} Create a detailed design document and API specification. Incorporate the interface specification 
into a detailed design document and review the design.

\item {\it Code Construction and Unit Testing} Implement the interface, 
debug and unit test.

\item {\it System Testing}  Assemble the complete system, verify that the 
code satisfies all requirements.

\item {\it Release} Create a potentially releasable product, including User's Guide and User's Reference Manual. Frequently code produced at intermediate stages software will be used internally.
\end{itemize}

\item {\bf Code Distribution and Maintenance}  Official public release 
of the software, beginning of maintenance phase. 

\end{enumerate}

We have customized and extended this standard model to suit the ESMF project.  
Subsequent sections describe how the ESMF team shall implement each development step.
It is essential for process definition to be an ongoing:  developers must 
be able to abandon processes that are not working, and to refine processes
as lessons are learned during development.  We therefore anticipate a 
period of process refinement prior to each major stage of development.
We shall expedite development overall by advancing the progress of a small 
subset of the framework functionality before the rest.  This shall allow 
us to experiment with document templates and procedures for a particular
stage, such as detailed design, before undertaking that stage wholesale.

\subsection{Software Concept}

The first step in the standard staged development cycle consists of 
collecting and itemizing the high-level requirements of the software, and 
identifying the basic functions that it must perform.  Participants in
the ESMF project completed this exercise in the process of developing
a unified set of proposals.  A summary of the high-level requirements
of ESMF -- a statement of project scope and vision -- is included 
in the {\it General Requirements} part of the {\it ESMF Requirements Document}\cite{bib:ESMFreqdoc}.

\subsection {Requirements Analysis}

The first section of the {\it ESMF Requirements Document} outlines 
the major ESMF capabilities necessary to meet project milestones and achieve 
project goals.  The second part of the document is a detailed 
requirements specification for each functionality class included in 
the framework.  

\subsubsection{Community Input}
We encourage input from the community on requirements.
An open community meeting was held on May 30, 2002 in order to obtain
feedback on the requirements assembled by the ESMF team and to elicit
additional requirements.  Hardcopy and web-based requirements submission
forms were made available to community members to facilitate the collection
process.  Additional requirements received are catalogued and viewable 
on the ESMF website, via the {\bf Development} link on the navigation bar.

To submit a requirement or comment on an existing requirement on-line,
visit the ESMF website and click on the {\bf Development} link.
Instructions for submitting forms are available on this page.

We place highest priority on the requirements imposed by our contractual
obligation to NASA as described in our milestones, and by the needs of
the codes in the JMC codeset.  We will consider adding additional software
features desired or required by the community on a case-by-case basis.
Factors we would consider in making such a decision would include value
to the community overall and resources required to implement the feature.

\subsubsection{Requirements Review Teams and Procedures}

At the beginning of the requirements process for each functionality 
class, the ESMF Deployment Team shall be polled to find out which 
JMC codes plan to 
use the functionality.  These listings are included in the  
{\it ESMF Requirements Document}.  The review teams for requirements
{\it must include at minimum a representative for each JMC code that 
is planning to use the class or utility}.  

Generally the review process will include at least two reviews 
conducted via email on the esmf\_tech@ucar.edu list, and at least
one teleconference or live review.

Reviews continue until they are done, or until common sense dictates that
the process should stop.  We are to a large extent constrained by the 
timetable specified in our milestones.

Typically the first documents of a given type take longer to prepare than
subsequent documents, since we use the initial documents to refine format
and process.  The ESMF technical leads prepare review schedules based on
factors such as the maturity of the document format, imminent milestones,
and concurrent activities.  A signoff review is held when a document is
deemed complete or when additional work on it is not likely to be as 
productive as proceeding with another task.  For example,
we may temporarily halt work on a design document in order to 
engage in an associated prototyping activity.  
The appropriate time for signoff is decided by consensus.

\subsubsection{Requirements Document Organization and Conventions}

Requirements documents shall be organized so that specific requirements are
listed under a titled topic.  For example, under the title {\bf Platforms}, 
the ability to run on an SGI Origin, IBM SP, Compaq ES, etc. are 
listed as individual and separately numbered requirements.   

Templates for requirements documents are available in the ESMF document
template set (see Section~\ref{sec:doc_templates}).

\subsubsection{Attributes of Requirements}

Each specific requirement possesses the following attributes:  priority, 
source, verification, status, and notes, the last of which is optional.  
These are typical for requirements
analysis ~\cite{wiegers}.  We'll now look at them in more detail.

\begin{description}
\item [Priority] The purpose of the priority attribute is to associate
each requirement with the milestones and longer term project goals that 
it satisfies.  Each requirement is assigned a number from 1-3, with
values defined as follows:
\begin{enumerate}

\item This capability is directly required for a milestone OR
half or more of the JMC applications that could use the utility or class
in which this capability is embedded would require this
capability in order to maintain their existing functionality;
 
\item Less than half of the JMC applications that could use
the utility or class in which this capability is embedded 
would require this capability in order to maintain their existing
functionality.

\item This capability is desired in order to extend the existing 
functionality of one or more JMC codes.

\end{enumerate}

If some capability merits additional explanation to describe 
its priority, those preparing requirements are encouraged 
to elaborate.

Changes to requirements' priorities during development will be
reviewed by the technical leads and other members of the Joint
Specification Team on an as-needed basis.  General requirements 
will be reviewed overall before being updated and submitted for 
Milestones F, G and K.  This occurs on roughly a yearly basis.
 
\item [Source] The source attribute traces each capability
to the applications to which it applies.  In addition to applications
particular people or organizations may be noted.   This attribute 
helps to identify those that can provide further 
information and who may also be potential testers and users.  It
prevents the inclusion of features that have little likelihood of
being used.

\item [Verification] The verification attribute specifies an objective
and quantitive strategy for assessing whether a requirement is
satisfied.  Typical values include {\it code inspection}, 
{\it unit test} and {\it system test}.
Some capabilities may require the preparation of special data sets.

\item [Status] Throughout the course of this project it will be 
useful for us to track what has been accomplished and to archive 
ideas for extensions and improvements.  The status attribute identifies
each capability as:
\begin{itemize}
\item {\bf proposed}; this indicates an item that has been accepted as useful, but
that is not scheduled for implementation.
\item {\bf approved-1}; this indicates an item approved for implementation as part
of the 1st code release at Milestone F
\item {\bf approved-2}; this indicates an item approved for implementation as part 
of the 2nd code release at Milestone G
\item {\bf implemented}
\item {\bf verified}
\item {\bf rejected} (with reason cited); this indicates an item that has been actively rejected by
the review team 
\end{itemize}
Whether the cabability exists in other packages or models
is also helpful to note.

\item [Notes] This is a catch-all for additional information such
as background, references, related design and implementation issues, 
risk factors, and so on.

\end{description}

\subsection{Architectural Design}

The architecture of the framework is presented in an {\it ESMF 
Architecture Document} in terms of an object model \cite{bib:ESMFarchdoc}.  This document describes 
how the functionality specified in the {\it ESMF Requirements Document} is 
apportioned into classes, and how those classes interact.  It also
describes the overall layering strategy for the framework, the machine 
model and the programming model.  The glossary of ESMF terms started in the 
{\it ESMF Requirements Document} was refined and extended in this 
document.

The architecture document was reviewed by all members of the Joint
Specification Team, and by an ESMF Peer Review Board.

The comments received back on the {\it Architecture Document}, informally and
from the Peer Review Board, indicated that the presentation of the document
was ineffective at conveying how the ESMF worked. Although the 
document was full of detailed and complex diagrams, the terminology and 
diagrams were oriented to software engineers and were not especially
scientist-friendly.  The detailed diagrams also made the document difficult 
to maintain.  The lessons learned during this stage of development will
influence the strategy for presentation of subsequent documents.  

\subsection{Detailed Design}

Users interact with the ESMF through its Application Programming
Interface (API).  The clarity, consistency, and functionality
embodied by the API is a primary factor in determining
whether users will be willing and able to adopt the ESMF.  
As such, it is exceedingly important that the design - and in
particular, the API - be thoroughly reviewed by the ESMF application 
team before the framework is widely deployed.  These reviews require
iteration.  The ESMF API will change rapidly over the first year of 
development, as we integrate and standardize classes
developed by different developers, and as implementation reveals 
what parts of the design do and do not work well.   

\subsubsection{Community Input}

As with requirements, the ESMF team encourages design input from the community.
Comments on the emerging ESMF API were solicited at the community meeting 
held on May 15, 2003.  Submit comments on design by sending mail 
to \htmladdnormallink{\bf esmf\_support@ucar.edu}
{mailto://esmf\_support@ucar.edu}.

\subsubsection{Initial Design Reviews}

Initial design reviews were held in the early prototype
stage of new classes, typically a few months after design began and 
when some prototype code was available.  The design documents prepared
included descriptions and, as appropriate, diagrams of classes, 
showing class relationships and interactions.  The documents 
also included API specifications for all class methods, and 
descriptions of data structures.  Where feasible at this stage
of development, discussion of design options was included.  To facilitate
document development and standardization, a design template was prepared 
for use by all developers, and included in the set of ESMF document 
templates (see Section~\ref{sec:doc_templates}).  

As with the {\it ESMF Requirements Document}, members of the Joint 
Specification Team prepared and reviewed the design documents. 

The initial set of design reviews established general outlines for 
code construction.  In most cases, designs changed dramatically as
implementation got underway, and as ideas were prototyped.  Object 
models were found to be ineffective compared to API inspection and 
code examples in conveying how a particular piece of code would work.

\subsubsection{Final Interface Reviews}

It is clear that another round of intensive class API reviews
shall be necessary.  These shall occur when the implementation 
of a class is complete, or close to it.  Each ESMF application group 
shall review 
and sign off on the final interfaces for each class.  The intent 
of these reviews is to ensure that the interfaces are adequate 
and easy to use, and to bring the interfaces into line so 
that they follow standard ESMF conventions for design policies, 
method names, and documentation procedures.  

The review procedure for each class shall be as follows:

\begin{enumerate}
\item A telecon review shall be held in which the full class API 
and examples of use are provided.  

\item The reviewed material and a summary of requested changes will
be posted to the esmf\_tech@ucar.edu list.  Comments will be
collected for a week, and should be sent to the same list.
The commentary period may be extended as necessary.

\item Code and documentation will be modified to reflect the review
results.  

\item Modified examples and documentation for the class will be
posted to a webpage maintained by the Core Team.  The material
posted will be the section of the Reference Manual that 
describes the class.  Each application group will have two weeks 
to do a final review and sign off on the interfaces.
 
\end{enumerate}

\subsection {Code Construction and Unit Testing}

Code shall be written in accordance with the interface specifications
presented in design documents and the coding conventions described in 
Section ~\ref{sec:code_conv}.  Code reviews shall be conducted by members 
of the Core Team.

Each class shall be accompanied by a suite of unit tests.  The user
shall have the option to build either ``sanity check'' type tests or an 
exhaustive suite.  The exhaustive tests will include tests of all
functionalities and a variety of valid and invalid input values. The
sanity tests will be a minumum set of tests to indicate whether, for
example, the software has been installed correctly.  The NASA Computational 
Technologies team will review and supplement the unit tests to ensure 
that software requirements are met.

\subsection {System Testing }

System testing and integration shall be performed incrementally on the 
evolving body of ESMF software, in accordance with the {\it Build and Test 
Plan}.

The ESMF VAlidation (EVA) suite shall be used to test the use of the 
ESMF system as portions of it become available.  For additional discussion
of test and validation procedures, see Section~\ref{sec:testing}.

\subsection {Release and Iteration }

The ESMF project has two major public software releases, one scheduled for 
April 2003 and one scheduled for April 2004.  There are internal software 
releases monthly.  The tagging convention for public and internal releases 
is described in Section~\ref{sec:tagging}.

Major releases shall be accompanied by thorough documentation, including an 
{\it ESMF User's Guide} and {\it ESMF Reference Manual}.  User documentation
shall include code excerpts, examples to demonstrate ideas, existing bugs,
and restrictions. Detailed test documentation will be made available to
the NASA evaluation team, to allow team members to understand the extent of
testing, plan deviations, and links to requirements.

In addition, Milestone K (January 2005) includes the provision of
a tutorial application that is part of the ESMF VAlidation (EVA) Suite.  The 
entire EVA Suite
itself shall be available for download via SourceForge and the ESMF 
website to users who wish to experiment with other application examples.  
Releases shall be announced via
mailing lists, on the ESMF website, and at community meetings.  Source
code shall be available for download from the ESMF website and from the 
SourceForge site.

At the close of both internal and public release cycles, we plan to 
iterate on requirements, architecture, design, and implementation in 
order to refine and extend the ESMF software and documentation.  The
goal will be to maintain a close-to-deliverable system throughout 
development, with functionality and optimization increasing
over time.

\subsection {Maintenance and Extension} 

Routine maintenance of the framework will include ports to new 
platforms and modifications for new compilers.  Extensions to the 
framework may occur at a wide range of different scales, from adding
a new performance optimization to creating an extensive library
of tools for advanced support of data assimilation systems.  
The version of the {\it ESMF User's Guide} delivered in 
January 2005 will provide instructions for routine extensions of 
the framework -- for example, adding support for an additional grid.
Other extensions to the framework will be facilitated by the system
documentation provided in the {\it User's Guide} and {\it Reference Manual}.  

We will consider open source contributions after our first mature 
software release at Milestone G (April 2004).  We anticipate that if we 
accept code contributions from others not in our collaboration the
contributors will interact with a Change Review Board.  We feel that
it is premature to detail those interactions at this time.  We will
upate this document with the specifics such interactions in the
Milestone G timeframe.

NCAR has committed to providing ongoing support for maintenance and 
development of the framework, but the magnitude of that support, and
whether it will be supplemented by contributions from other institutions
or agencies, is unknown.  











