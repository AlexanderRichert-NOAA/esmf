\section{Definition of ESMF Adoption}
\label{sec:implications}

The definition of adoption below applies to JMC codes during the
ESMF project's initial three-year period of NASA funding.  Adoption
may be considered synonymous with the term compliance for the 
purpose of NASA contracts.

\subsubsection{Partial Adoption}
In order to achieve partial adoption, a JMC code component must 
implement, or adopt default implementations, of the complete set of 
standard ESMF component interface methods including the following
capabilities:

\begin{itemize}
\item It must be able to be instantiated in parallel configurations.

\item It must provide implementations of methods for creation, deletion, 
configuration, initialization, finalization, run, read and write 
restart, and others as necessary for control by an ESMF application 
framework.

\item It must provide method implementations to allow it to be queried 
for its distribution, state (i.e. fields available for export, fields 
required for import, etc.), run status and other pertinent 
information.

\item Communication with other JMC code components must be mediated by an 
ESMF coupler component using framework communication services, such 
that neither JMC component needs to maintain information about the 
specific component that it is being coupled to. 

\item Data and information to be exchanged with other JMC code components
must be provided through ESMF constructs and utilities (i.e. ESMF state,
bundles, fields, time, grid, decomposition, etc.) These must include
pertinent metadata information and provide a standard format for
exchanging information. JMC code components must use the public
interface methods provided by the ESMF utilities and constructs and not
directly manipulate their internal data.

\item The JMC components must be able to accept ESMF time management 
information.

\item Data and information to be exchanged with other JMC code components
must be provided through ESMF constructs and utilities (i.e. ESMF state,
bundles, fields, grid, etc.) These must include
pertinent metadata information and provide a standard format for
exchanging information.  JMC code components must use the public 
interface methods provided by the ESMF utilities and not directly 
manipulate the internal data of those utilities.  

\end{itemize}

\subsubsection{Full Adoption}

For full adoption JMC code component must satisfy all requirements 
described for partial adoption.  In addition, for full adoption a
component must:

\begin{itemize}
\item Extensively use internally three or more utilities from the following
set:
   I/O, parameter specification, log/error, performance profiling, time
   management, grid communication services.

\item Adopt the standard ESMF grid communication services and constructs
   internally to the extent necessary to allow interoperability with
   other weather, climate, and data assimilation components that have 
   achieved full adoption.

\item Adopt the standard ESMF time management utilities for calendars,
time instants, time spans, alarms, etc. internally to the extent
necessary to allow interoperability with other weather, climate, and 
data assimilation components that have achieved full adoption.

\item Adopt design features that eliminate or minimize as much as possible
the potential for name space conflicts of variables, methods, etc.
between components.

\item Adopt design features that eliminate or minimize as much as possible
the potential for I/O conflicts between components during reads/writes
of configuration, state, errors, logs, performance analysis, etc.
\end{itemize}


