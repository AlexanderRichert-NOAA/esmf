%===============================================================================
% CVS $Id: intro.tex,v 1.4 2006/08/01 22:02:59 cdeluca Exp $
% CVS $Source: /mnt/twixshare/Storage/Archive-SF-Repos/ESMF_CVS_Repo/esmf/src/doc/dev_guide/intro.tex,v $
% CVS $Name:  $
%===============================================================================

\section{Introduction}
\label{sec:intro}

\subsection{About this Document}
The Earth System Modeling Framework (ESMF) {\it Software Developer's Guide},
or simply the {\it Guide}, is the reference handbook that describes
the practices, standards, and conventions recommended for ESMF core software
development.  It is updated as needed.

Suggestions on how to improve the {\it Guide} should be sent to 
\htmladdnormallink{esmf\_support@ucar.edu}{mailto:esmf\_support@ucar.edu}.

\subsection{Supplementary Information}
An important source of supplementary information is the ESMF website:
\begin{center}
\htmladdnormallink{{\bf http://www.esmf.ucar.edu}}{http://www.esmf.ucar.edu}
\end{center}
This website is referred to throughout the {\it Guide}.  
It includes a link to the ESMF CVS repository, the current release
schedule, task lists, and many other types of project information.

The {\it Guide} doesn't contain instructions on how to adapt user
codes for ESMF.  This information is included in the {\it ESMF User's
Guide}\cite{bib:ESMFusrdoc}, which is available via the {\bf Downloads \& Documentation} link on the ESMF website navigation bar.

\subsection{Acknowledgements}
Sections of this document were derived from the \htmladdnormallink{{\it GFDL 
Flexible Modeling System Developers' Manual}}{http://www.gfdl.gov/\~{}vb/FMSManual}
\cite{FMSdevguide} and the \htmladdnormallink{{\it Community Climate System Model Software
  Developer's Guide}}{http://www.ccsm.ucar.edu/csm/working\_groups/Software/dev\_guide/dev\_guide}
\cite{CCSMdevguide}.  These have been adapted with the permission of the respective authors.

\section{Groups Involved in ESMF Development}

For more detail on the groups below, including their Terms of
Reference, see the {\it ESMF Project Plan}.

\subsection{Core Team}
ESMF software implementation is led by a {\bf Core Team} based in the 
Scientific Computing Division (SCD) of the National Center for
Atmospheric Research (NCAR).  The Core Team relies on close interaction
with customers, and the work of many contributors.  Core Team
activities are open to active ESMF developers.

\subsection{Joint Specification Team (JST)}
The Core Team, along with contributors, customers, technical
managers, and other stakeholders in development, are collectively
referred to as the {\bf ESMF Joint Specification Team}, or {\bf JST}.  
JST membership is open to members of the science, computing, and
related communities.

ESMF development priorities and schedules are set by a 
{\bf Change Review Board}. or {\bf CRB} that consists of individuals
associated with major ESMF initiatives and applications.  
Members of the CRB are chosen by the ESMF Executive Management.
























