\section{Implementation Language and Language Interoperability Strategy}

It is possible to represent the fundamental features of object-oriented 
software -- polymorphism, inheritance and encapsulation -- in a variety of languages, 
including the usual choices for high-performance systems: C, C++ and F90. \footnote{Albeit 
with differing levels of difficulty and effectiveness.}  Perhaps the best evidence for 
this claim is that widely used object-oriented libraries and frameworks have been 
written in each of these languages.  Given the above, we can assume that the architecture 
and design of the ESMF will be largely independent of its implementation language.  We
anticipate being able to interpret the architecture described in this document, expressed
in the Unified Modeling Language, in whatever implementation language is chosen.  The rationale for our decision is presented in the ESMF Implementation Report.





