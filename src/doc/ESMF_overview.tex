
{\it 
In this section, we describe: 
\begin{itemize}
\item the goals and unique role of ESMF in Earth science;
\item the relevant features of the problem domain that ESMF encompasses; and
\item the key design features that are derived from the requirements of the ESMF domain and 
enable the ESMF to satisfy its goals. 
\end{itemize}
}

The {\bf Earth System Modeling Framework (ESMF)} is a software toolkit that increases 
software reuse, component interoperability, performance portability, and ease of use in 
Earth science applications.  The focus on Earth systems - a diverse but bounded domain - is 
critical to the ability of the ESMF to achieve its goals.  This is where the ESMF's value and 
capability come from - enabling interoperability but able to retain some efficiency and structure 
because the domain is bounded.  The scope is also unique - there are larger projects and smaller 
projects, but only ESMF spans institutions that are all interested in collaboration. The fifteen 
initial testbeds for the ESMF represent a wide range of research and operational applications in 
climate, weather, and data assimilation \cite{ref:models}.  Thus the ESMF must efficiently 
support a variety of grids, programming paradigms, and computing platforms.  However, the ESMF 
domain is restricted enough to allow the creation of specific data and control constructs.  A 
challenge of the project is to support high-performance operations in the face of the generality 
that's required.  Commonality in applications can be found at both the {\bf infrastructure} 
level, in model development tools for communication, I/O, time management, and other basic 
functions; and at the {\b superstructure} level, in an architecture and standards for 
combining geophysical components into applications.

The ESMF design is rooted in the requirements and characteristics of the applications that 
the framework must support.  In the sections below, these characteristics are described.  
Then we describe the major architectural implications that spring from the nature of the 
domain.

The key characteristics of the Earth science code that the ESMF must address are as follows.
\begin{enumerate}

\item Applications contain sets of complex software components that represent different 
physical domains, such as land, ocean, atmosphere and sea ice.  Other components may perform 
data assimilation or analysis.  These components are "coupled" so that the output fields of 
some of the components are used 
as the input fields for other components.  Coupling may include data transfers, unit 
conversions, flux 
calculations, regridding, data transfers, and other transformations.  The software that 
couples components is often viewed as a component itself.  

\item The same models or components may be run in a variety of configurations.  Scientists 
may swap in
alternative implementations of a domain or function in order to create new applications 
(e.g., different 
ocean models, different dynamical cores); may use different subsets and combinations 
of a particular set of 
components (e.g., an atmospheric model may be run with an ocean model for a 
hurricane application, and 
with a full complement of land, ocean, and sea ice models for a climate simulation); may 
run ensembles of the same component; and as a matter of course attempt endless variations 
and combinations of these research strategies.

\item Applications run on a variety of parallel computing systems and require high performances. 

\item Applications have many diverse users, from graduate students to software engineers 
to .  Applications are under continuous scientific development, and scientists and developers 
frequently wish to modify and update the code.

\end{enumerate}

The characteristics above inform a software architecture.  The key architectural features 
of ESMF are described below.

\begin{description}
\item[Component-based architecture]  Applications running under ESMF are organized as sets of 
interacting {\bf Components}.  A Component is
a software representation of some domain or function; for example, a land model or data 
assimilation 
system.  All Components share common behavior in ESMF.  This architectural model lends itself 
naturally to systems whose functionality is segregable.

\item[Types of Components]  ESMF Components fall into a number of categories.  The two primary 
types are Gridded 
Components ({\tt ESMF\_GridComp}) and ESMF Coupler Components ({\tt ESMF\_CplComp}).  A 
Gridded Component represents 
a physical domain that contains fields discretized on a grid.  A Coupler Component performs the operations required for two or more components to interact with each other.

\item[Local communication]

One of the important features of the ESMF architecture is that all communication between 
Gridded Components is 
local.  Another way of stating this is that all inter-component communication is mediated 
by a Coupler Component.    
\end{description}





















