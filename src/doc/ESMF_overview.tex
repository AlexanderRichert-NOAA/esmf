\section{Overview}

In this {\it Architecture Document} we describe the computing environment of the Earth 
System Modeling Framework (ESMF), the layering strategy we will employ, the major classes 
in the framework, and how these classes will interact.  We also provide detailed definitions of 
terms; these are summarized in an extensive glossary. 

NOTES: \newline
Some aspects of this Architecture Document will not be resolved until
another ESMF document, the Implementation Report, is nearer completion.  This Report
will examine language issues and programming model as they relate to the ESMF.

ESMF architecture will be largely independent of implementation language
and language interoperability strategy, but the presence or absence of a language feature 
in our chosen implementation may influence some framework interfaces and structural decisions.

The programming model is integral to the design of many framework interfaces.  
By programming model we mean the strategy employed for utilizing computing hardware, 
OS, and standard library or vendor-supplied software (e.g., MPI or other message-passing 
software, Posix threads, OMP).  The programming model describes the level at which 
features of the computing environment will be exposed, and the abstractions used to 
simplify interfaces.

In this document we assume a minimum amount of abstraction, and allow both distributed 
memory and threading constructs to be visible to the application developer.  If we find 
that additional abstraction can be incorporated without a significant performance penalty 
and without degrading the flexibility of the framework we may simplify interfaces further.




