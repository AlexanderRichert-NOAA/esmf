\section{What is the Earth System Modeling Framework?}

The Earth System Modeling Framework (ESMF) is a structured collection of 
software building blocks that can be used or customized to develop 
Earth system model components, and assemble them into applications.  
The simplest view of the ESMF is that it consists of an
{\it infrastructure} of utilities and data structures for creating 
model components, and a {\it superstructure} for coupling them.  
User code sits between these two layers, making calls to the infrastructure
libraries beneath it and being scheduled and synchronized by the 
superstructure above it.  The configuration resembles a sandwich, as
shown in Figure 1.

The ESMF architecture is scalable, flexible paradigm for building highly 
complex climate, weather, and related applications from components such
as atmospheric models, land models, and data assimilation systems.  The 
ESMF is not a single master application into which all components must fit; 
rather it is a way of developing components so that they can be used 
in many different user-written applications.  Model components that adopt 
ESMF are usable in different contexts without code modification, and may be
incorporated into other ESMF-based modeling systems within the Earth 
science community.  In addition to the high-level organization ESMF
offers, it provides a set of robust, portable, performance 
optimized libraries for regridding, data transfers, I/O, time management, 
and other common functions.  ESMF users may choose to extensively rewrite 
their codes to take advantage of the ESMF infrastructure, or they may 
decide to simply wrap components in ESMF interfaces in order to
adopt the ESMF architecture and coupling services.

\section{The ESMF User Guide for Release 1.0}

This {\it ESMF User Guide} will eventually serve as an introduction for the 
new ESMF user and as a reference for the experienced user.  Since ESMF 
Release 1.0 is the first public ESMF release, and is a prototype rather than 
a production-ready package, we assume you are a potential user interested in 
learning more about the ESMF software.  This edition of the {\it User Guide} 
is designed to guide you through that process.  

The purpose of ESMF Release 1.0 is to provide a first look at the ESMF
Application Programming Interface (API), and to demonstrate the viability 
of the ESMF architecture and implementation.  Section \ref{sec:QuickStart} contains 
a {\it Quick Start} guide that explains how to install the ESMF software and 
run a demonstration program, {\tt ESMF\_COUPLED\_WAVE}, that illustrates both ESMF 
utilities and coupling services.  Section \ref{sec:ArchOver} is an 
architectural overview that describes the framework's basic goals and features.  
More detail on ESMF structure and operation, such as a description of the 
directory structure and how to run the ESMF self-tests, is provided in Section 
\ref{sec:TechOver}.  Section \ref{sec:Adoption} looks ahead to the steps 
required to adapt a component for use within ESMF.  Finally, to help you become 
familiar with ESMF terminology, the last section in the {\it User Guide} is 
a glossary.  

While we are delighted to have potential users experiment with ESMF, the
ESMF team can offer only limited support to those who are trying to incorporate 
the framework into applications at this early stage.  The ESMF is still
too young to be considered viable infrastructure for active research and 
operational codes.  For ESMF Release 1.0, we have focused on implementing 
major architectural features and basic functionality.  The current performance 
characteristics and memory requirements of the software are unlikely to resemble 
those in later releases.

Our next major release will occur in April 2004.  At that time, the {\it ESMF 
User Guide} will be expanded to include a comprehensive section on how to adapt 
application codes for the framework, and support staff will be available to 
assist users with ESMF adoption.  We do not anticipate preparing other public
releases or patches for Release 1.0 between now and the next release - our focus
will be on developing the ESMF software and making Release 2.0 a robust, portable,
high -performance product.  We are relying on Earth system modelers to provide 
us with the design feedback essential for creating a shared, easy to use, high 
performance software framework.  

Those curious about specific interfaces should refer to the {\it ESMF Reference 
Manual for Fortran90}, which contains a detailed listing and description of 
the ESMF API.

















