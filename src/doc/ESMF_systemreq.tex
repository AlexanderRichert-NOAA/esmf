% $Id$


The following compilers and utilities are required for compiling, linking and
testing the ESMF software:
\begin{itemize}
\item Fortran90 (or later) compiler;
\item C++ compiler;
\item MPI implementation compatible with the above compilers (but see below);
\item GNU's \htmladdnormallink{gcc compiler}{http://gcc.gnu.org} -
for a standard cpp preprocessor implementation;
\item \htmladdnormallink{GNU Make}{http://www.gnu.org/software/make/make.html}; 
\item \htmladdnormallink{Perl}{http://www.perl.com/download.csp} - for running
test scripts.
\end{itemize} 

Internal packages that can optionally reference external libraries:
\begin{itemize}
\item LAPACK - version 3.x or newer
\item ParallelIO (PIO) - version 2.5.7 or newer
\item yaml-cpp - tag yaml-cpp-0.6.2 or newer
\end{itemize}

Optional external packages that must be specified for certain functions:
\begin{itemize}
\item NetCDF - version 3.6.x or newer (version 4.4 or newer required by PIO)
\item parallel-NetCDF - version 1.2.0 or newer (version 1.12 or newer required by PIO)
\item Xerces - version 3.1.0 or newer
\end{itemize}  

ESMF can be built using a single-processor MPI-bypass library
that comes with ESMF by setting {\tt ESMF\_COMM=mpiuni}. This allows ESMF applications
to be linked and run in single-process mode.

In order to build html and pdf versions of the ESMF documentation, 
\htmladdnormallink{\LaTeX}{http://www.latex-project.org},
the \htmladdnormallink{latex2html}{http://www.latex2html.org}
conversion utility, and the Unix/Linux {\tt dvipdf} utility must be installed.
The csh shell is also required to complete the documentation build.
