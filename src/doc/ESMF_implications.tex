\section{ESMF Compliance and Framework Adoption}
\label{sec:implications}

\subsection{ESMF Compliance}

We define ESMF compliance for software components, including
both {\tt Coupler} and {\tt Gridded Component}s.  In order to 
achieve compliance, 
components must provide a minimal set of standard ESMF-defined 
interfaces.  The implementation of these interfaces must enable the  
components to be instantiated in parallel configurations
and destroyed; to be initialized, finalized, and run; to be queried for 
distribution, state, run status and other pertinent information;
and, for {\tt Gridded Component}s, to interoperate with other components such
that they do not need to maintain information about the coupling context.

\subsection{Implications of Architecture for Framework Adoption}

In this section we outline the types of changes that will be necessary for
application groups to adapt their software components for ESMF.  {\it Since the
functional interface to the ESMF is not yet fully defined, nor have we 
explored all the implications of the architectural definition, this list is 
necessarily tentative, incomplete, and non-specific.}

\begin{itemize}
\item Since {\tt Gridded Component}s will run on the same 
{\tt DE}s as the {\tt Application Component}, often in conjunction 
with a {\tt Coupler}, 
it is necessary to subroutinize
them and provide standard methods such as {\tt ESMF\_CompInit}, 
{\tt ESMF\_CompRun}, and {\tt ESMF\_CompFinalize}.  For {\tt Gridded 
Component}s
with a time iteration loop, we expect the {\tt ESMF\_CompRun} command to 
take a time interval or number of timesteps as one of its arguments.
\item {\tt Gridded Component}s must define ESMF import and export {\tt State} 
structures.
\item {\tt Coupler Component}s must define and return in a specified 
format any transforms that {\tt Gridded Component}s involved 
in a coupling interaction will execute.
\item Components must implement standard query methods to return
their {\tt Layout}, {\tt Grid(s)}, {\tt States}, run status, and other 
information in a standard format.
\end{itemize}

Components {\it will not} need to recast their internal data structures.
They {\it will not} be required to insert return points at each timestep.
Thus the amount of internal reorganization of codes required should be 
kept to a minimum.  We hope, however, to provide high-performance infrastructure,
both data structures and utilities, that can be leveraged by groups who see
the ESMF effort as an opportunity to restructure their codes.



