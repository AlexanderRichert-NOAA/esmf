\section{ESMF Compliance and Framework Adoption}
\label{sec:implications}

\subsection{ESMF Compliance}

The definition of compliance below applies to JMC codes during the
ESMF project's initial three-year period of NASA funding.

\subsubsection{Partial Compliance}
In order to achieve partial compliance, a JMC code component must 
implement, or adopt default implementations, of the complete set of 
standard ESMF component interface methods including the following
capabilities:

\begin{itemize}
\item It must be able to be instantiated in parallel configurations.

\item It must provide implementations of methods for creation, deletion, 
configuration, initialization, finalization, run, read and write 
restart, and others as necessary for control by an ESMF application 
framework.

\item It must provide method implementations to allow it to be queried 
for its distribution, state (i.e. fields available for export, fields 
required for import, etc.), run status and other pertinent 
information.

\item Communication with other JMC code components must be mediated by an 
ESMF coupler component using framework communication services, such 
that neither JMC component needs to maintain information about the 
specific component that it is being coupled to. 

\item Data and information to be exchanged with other JMC code components
must be provided through ESMF constructs and utilities (i.e. ESMF state,
bundles, fields, time, grid, decomposition, etc.) These must include
pertinent metadata information and provide a standard format for
exchanging information. JMC code components must use the public
interface methods provided by the ESMF utilities and constructs and not
directly manipulate their internal data.

\item The JMC components must be able to accept ESMF time management 
information.

\item Data and information to be exchanged with other JMC code components
must be provided through ESMF constructs and utilities (i.e. ESMF state,
bundles, fields, grid, etc.) These must include
pertinent metadata information and provide a standard format for
exchanging information.  JMC code components must use the public 
interface methods provided by the ESMF utilities and not directly 
manipulate the internal data of those utilities.  

\end{itemize}

\subsubsection{Full Compliance}

A fully compliant JMC code component must satisfy all requirements 
described for partial compliance.  In addition, a fully compliant 
component must:

\begin{itemize}
\item Extensively use internally three or more utilities from the following
set:
   I/O, parameter specification, log/error, performance profiling, time
   management, grid communication services.

\item Adopt the standard ESMF grid communication services and constructs
   internally to the extent necessary to allow interoperability with
   other compliant weather, climate, and data assimilation
   components.

\item Adopt the standard ESMF time management utilities for calendars,
time instants, time spans, alarms, etc. internally to the extent
necessary to allow interoperability with other compliant weather,
climate, and data assimilation components.

\item Adopt design features that eliminate or minimize as much as possible
the potential for name space conflicts of variables, methods, etc.
between components.

\item Adopt design features that eliminate or minimize as much as possible
the potential for I/O conflicts between components during reads/writes
of configuration, state, errors, logs, performance analysis, etc.
\end{itemize}

JMC codes are required to be compliant by April 2004.

\subsection{Implications of Architecture for Framework Adoption}

In this section we outline the types of changes that will be necessary for
application groups to adapt their software components for ESMF.  {\it Since the
functional interface to the ESMF is not yet fully defined, nor have we 
explored all the implications of the architectural definition, this list is 
necessarily tentative, incomplete, and non-specific.}

\begin{itemize}
\item Since {\tt Gridded Component}s will run on the same 
{\tt DE}s as the {\tt Application Component}, often in conjunction 
with a {\tt Coupler}, 
it is necessary to subroutinize
them and provide standard methods such as {\tt ESMF\_CompInit}, 
{\tt ESMF\_CompRun}, and {\tt ESMF\_CompFinalize}.  For {\tt Gridded 
Component}s
with a time iteration loop, we expect the {\tt ESMF\_CompRun} command to 
take a time interval or number of timesteps as one of its arguments.
\item {\tt Gridded Component}s must define ESMF import and export {\tt State} 
structures.
\item {\tt Coupler Component}s must define and return in a specified 
format any transforms that {\tt Gridded Component}s involved 
in a coupling interaction will execute.
\item Components must implement standard query methods to return
their {\tt DELayout}, {\tt Grid(s)}, {\tt States}, run status, and other 
information in a standard format.
\end{itemize}

Components {\it will not} need to recast their internal data structures.
They {\it will not} be required to insert return points at each timestep.
Thus the amount of internal reorganization of codes required should be 
kept to a minimum.  We hope, however, to provide high-performance infrastructure,
both data structures and utilities, that can be leveraged by groups who see
the ESMF effort as an opportunity to restructure their codes.



