% $Id: ESMF_infradataoverview.tex,v 1.29 2012/10/15 16:52:44 oehmke Exp $

\section{Overview}

The ESMF infrastructure data classes are part of the framework's 
hierarchy of structures for handling Earth system model data and 
metadata on parallel platforms.  The hierarchy is in complexity; the 
simplest data class in the infrastructure represents a distributed data
array and the most complex data class represents a bundle of physical 
fields that are discretized on the same grid.  Data class methods 
are called both from user-written code and from other classes 
internal to the framework. 

Data classes are distributed over {\bf DE}s, or {\bf Decomposition Elements}.  
A DE represents a piece of a decomposition.  A DELayout is a collection
of DEs with some associated connectivity that describes a specific 
distribution.  For example, the distribution of a grid divided 
into four segments in the x-dimension would be expressed in ESMF as
a DELayout with four DEs lying along an x-axis. This abstract concept 
enables a data decomposition to be defined in 
terms of threads, MPI processes, virtual decomposition elements, or
combinations of these without changes to user code.  This is a
primary strategy for ensuring optimal performance and portability
for codes using the ESMF for communications.

ESMF data classes are useful because they provide a standard, 
convenient way for developers to collect together information 
related to model or observational data.  The information assembled 
in a data class includes a data pointer, a set of attributes 
(e.g. units, although attributes can also be user-defined), and a 
description of an associated grid.  The same set of information within 
an ESMF data object can be used by the framework to arrange 
intercomponent data transfers, to perform I/O, for communications
such as gathers and scatters, for simplification of interfaces 
within user code, for debugging, and for other functions.  
This unifies and organizes codes overall so that the user need not
define different representations of metadata for the same field 
for I/O and for component coupling.  

Since it is critical that users be able to introduce ESMF into their
codes easily and incrementally, ESMF data classes can be created based 
on native Fortran pointers.  Likewise, there are methods for retrieving 
native Fortran pointers from within ESMF data objects.  This allows
the user to perform allocations using ESMF, and to retrieve Fortran
arrays later for optimized model calculations.  The ESMF data classes 
do not have associated differential operators or other mathematical 
methods.

For flexibility, it is not necessary to build an ESMF data object 
all at once.  For example, it's possible to create a 
field but to defer allocation of the associated field data until 
a later time.


\begin{center}  
\begin{tabular}{|p{6in}|}
\hline
\vspace{.01in}
{\bf Key Features} \\[.01in]
Hierarchy of data structures designed specifically for the Earth 
system domain and high performance, parallel computing. \\
Multi-use ESMF structures simplify user code overall. \\
Data objects support incremental construction and deferred allocation. \\ 
Native Fortran arrays can be associated with or retrieved from ESMF data
objects, for ease of adoption, convenience, and performance. \\[.03in] \hline
\end{tabular}
\end{center}

\subsection{Infrastructure Data Classes}

The main classes that are used for model and observational data manipulation
are as follows:

\begin{itemize}

\item {\bf Array}  An ESMF Array contains a data pointer, 
information about its associated datatype, precision, and 
dimension.  

Data elements in Arrays are partitioned into categories 
defined by the role the data element plays in distributed halo 
operations.  Haloing - sometimes called ghosting - is the 
practice of copying portions of array data to multiple memory 
locations to ensure that data dependencies can be satisfied 
quickly when performing a calculation.  ESMF Arrays contain 
an {\bf exclusive} domain, which contains data elements
updated exclusively and definitively by a given DE; a 
{\bf computational} domain, which contains all data elements
with values that are updated by the DE in computations; and 
a {\bf total} domain, which includes both the computational 
domain and data elements from other DEs which may be read 
but are not updated in computations.

\item {\bf ArrayBundle} ArrayBundles are collections of
Arrays that are stored in a single object.  Unlike FieldBundles,
they don't need to be distributed the same way across PETs.  The
motivation for ArrayBundles is both convenience and performance.

\item {\bf Field}  A Field holds model and/or observational 
data together with its underlying grid or set of spatial 
locations.  It provides methods for configuration, 
initialization, setting and retrieving data values, 
data I/O, data regridding, and manipulation of attributes.

\item {\bf FieldBundle} Groups of Fields on the same underlying 
physical grid can be collected into a single object called a FieldBundle.  
A FieldBundle provides two major functions: it allows groups of 
Fields to be manipulated using a single identifier, for example 
during export or import of data between Components; and 
it allows data from multiple Fields to be packed together 
in memory for higher locality of reference and ease in 
subsetting operations.  Packing a set of Fields into a single
FieldBundle before performing a data communication allows the set 
to be transferred at once rather than as a Field at a time.
This can improve performance on high-latency platforms.

FieldBundle objects contain methods for setting and retrieving constituent 
fields, regridding, data I/O, and reordering of data in memory.

\end{itemize}

\subsection{Regrid}\label{sec:regrid}

 This section describes the regridding methods provided by ESMF. Regridding, also called remapping or interpolation, is 
 the process of changing the grid that underlies data values while preserving qualities of the original data. Different 
 kinds of transformations are appropriate for different problems. Regridding may be needed when communicating data between
 Earth system model components such as land and atmosphere, or between different data sets to support operations such as visualization.

 Regridding can be broken into two stages. The first stage is generation of an interpolation weight matrix that describes how points in 
 the source grid contribute to points in the destination grid. The second stage is the multiplication of values on the source grid by the
 interpolation weight matrix to produce values on the destination grid. This occurs through a parallel sparse matrix multiply.

 There are two options for accessing ESMF regridding functionality: {\bf offline} and {\bf integrated}. Offline regridding is a process whereby interpolation 
 weights are generated by a separate ESMF application, not within the user code. The ESMF offline regridding application also only generates the interpolation 
 matrix, the user is responsible for reading in this matrix and doing the actual interpolation (multiplication by the sparse matrix) in their code.
 Please see Section~\ref{sec:ESMF_RegridWeightGen} for a description of the offline regridding application and the options it supports. 
 In constrast to offline regridding, integrated regridding is a process whereby interpolation weights are generated via subroutine calls during the
 execution of the user's code. The integrated regridding can also perform the parallel sparse 
 matrix multiply. In other words, ESMF integrated regridding allows a user to perform the whole process of interpolation within their code.
 The rest of this section further describes ESMF integrated regridding. Figure~\ref{Regriddingcapabilities} shows a comparison of the capabilities 
 of offline and integrated regridding. 

\begin{table}[ht]
\centering
\vspace{0.2cm}
\begin{tabular}{| l | l | c c |}
\hline
& & Online & Offline \\ [0.5ex]
\hline
2D Polygons & Triangles & $\surd$ & $\surd$ \\
& Quadrilaterals & $\surd$ & $\surd$ \\
\hline
3D Polygons & Hexahedrons & $\surd$ & $\surd$ \\
\hline
Regridding & Bilinear & $\surd$ & $\surd$ \\
& Patch & $\surd$ & $\surd$ \\
& Conservative (1st order) & $\surd$ & $\surd$ \\
\hline
Masking & Destination & $\surd$ & $\surd$ \\
& Source & $\surd$ &  $\surd$ \\
& Unmapped points & $\surd$ & $\surd$ \\
\hline
Pole Options & Full circle average & $\surd$ & $\surd$ \\
& N-point average & $\surd$ & $\surd$ \\
& Teeth pole & $\surd$ & $\surd$ \\[1ex]
\hline
\end{tabular}
\label{Regriddingcapabilities}
\caption{Comparison of the offline vs. online regridding capabilities of ESMF}
\end{table}


\subsubsection{Basic usage}
 The basic flow of using ESMF integrated regridding is as follows. First a source and destination grid object are created, both can be either a Grid or Mesh. 
 Coordinates are set during Mesh creation, but for the Grid they must be set separately using the {\tt ESMF\_GridAddCoord()} and {\tt ESMF\_GridGetCoord()} methods. 
 Next Fields are built on the source and destination grid objects. These Fields are then passed into {\tt ESMF\_FieldRegridStore()}. The user can either get a 
 sparse matrix from this call and/or a {\tt routeHandle}. If the user gets the sparse matrix then they are responsible for deallocating it, but other than that
 can use it as they wish. The {\tt routeHandle} can be used in the {\tt ESMF\_FieldRegrid()} call to perform the actual interpolation of data from the source 
 to the destination field. This interpolation can be repeated for the same set of Fields as long as the coordinates at the staggerloc involved in the
 regridding in the associated grid object don't change. The same {\tt routeHandle} can also be used between any pair of Fields which is weakly congruent 
to the pair used to create the {\tt routeHandle}.  Congruent Fields possess matching DistGrids and the shape of the 
local array tiles matches between the Fields for every DE. For weakly congruent Fields the sizes 
of the undistributed dimensions, that vary faster with memory than the first distributed dimension,
 of similar Fields that differ in the number of elements in the left most undistributed dimensions.             
 You can apply the routehandle between any set of Fields weakly congruent to the original Fields used to create the routehandle without 
 incurring an error. However, if you want the routehandle to be the same interpolation between the grid objects upon which the Fields are built as was calculated with the original {\tt ESMF\_FieldRegridStore()} call, then there
 are additional constraints on the grid objects. To be the same interpolation, the grid objects upon which the 
 Fields are build must contain the same coordinates at the stagger locations involved in the regridding as
 the original source and destination Fields used in the {\tt ESMF\_FieldRegridStore()} call.                                               The routehandle represents the interpolation between the grid objects as they were during the {\tt ESMF\_FieldRegridStore()} call.        So if the coordinates at the stagger location in the grid objects change, a new call to {\tt ESMF\_FieldRegridStore()}                    is necessary to compute the interpolation between that new set of coordinates. When finished with the {\tt routeHandle} 
 {\tt ESMF\_FieldRegridRelease()} should be used to free the associated memory. 


\subsubsection{Spherical grids and poles}
In the case that the Grid is on a sphere ({\tt coordSys=ESMF\_COORDSYS\_SPH\_DEG or ESMF\_COORDSYS\_SPH\_RAD})
then the coordinates given in the Grid are interpreted as latitude and longitude values. The coordinates can either be in degrees or radians as indicated by the 
{\tt coordSys} flag set during Grid creation. As is true with many global models, this application currently assumes the latitude and longitude refer to positions on a 
perfect sphere, as opposed to a more complex and accurate representation of the earth's true shape such as would be used in a GIS system. (ESMF's current user base doesn't 
require this level of detail in representing the earth's shape, but it could be added in the future if necessary.)

For Grids on a sphere, the regridding occurs in 3D Cartesian to avoid
problems with periodicity and with the pole singularity. This library
 supports four options for handling the pole region (i.e. the empty area above the top row of the source grid or below
 the bottom row of the source grid).  Note that all of these pole options currently only work for the Fields build on the Grid class and not for those built on 
 the Mesh class. The first option is to leave the pole region empty ({\tt polemethod=ESMF\_POLEMETHOD\_NONE}), in this 
 case if a destination point lies above or below the 
 top row of the source grid, it will fail to map, yielding an error (unless {\tt unmappedaction=ESMF\_UNMAPPEDACTION\_IGNORE} is specified).  
 With the next two options, the pole region is handled by constructing 
 an artificial pole in the center of the top and bottom row of grid points and then filling
 in the region from this pole to the edges of the source grid with triangles. 
 The pole is located at the average of the position of the points surrounding
 it, but moved outward to be at the same radius as the rest of the points
 in the grid. The difference between these two artificial pole options is what value is used at the pole. 
 The default pole option ({\tt polemethod=ESMF\_POLEMETHOD\_ALLAVG}) sets the value at the pole to be the average of the values
 of all of the grid points surrounding the pole. For the other option ({\tt polemethod=ESMF\_POLEMETHOD\_NPNTAVG}), the user chooses
 a number N from 1 to the number of source grid points around the pole. The value N is set via the argument {\tt regridPoleNPnts}. For
 each destination point, the value at the pole is then the average of the N source points
 surrounding that destination point. For the last pole option ({\tt polemethod=ESMF\_POLEMETHOD\_TEETH}) no artificial pole is constructed, instead the
 pole region is covered by connecting points across the top and bottom row of the source Grid into triangles. As 
 this makes the top and bottom of the source sphere flat, for a big enough difference between the size of
 the source and destination pole regions, this can still result in unmapped destination points.  
 Only pole option {\tt ESMF\_POLEMETHOD\_NONE} is currently supported with the conservative interpolation method 
(i.e. {\tt regridmethod=ESMF\_REGRIDMETHOD\_CONSERVE}). 


\subsubsection{Masking}
 Masking is the process whereby parts of a grid can be marked to be ignored during an operation, such as regridding. Masking can be used on a source grid to 
 indicate that certain portions of the grid should not be used to generate regridded data. This is useful, for example, if a portion of source grid contains 
 unusable values. Masking can also be used on a destination grid to indicate that the portion of the field built on that part of the Grid should not receive 
 regridded data. This is useful, for example, when part of the grid isn't being used (e.g. the land portion of an ocean grid).

 ESMF currently supports masking for Fields built on structured Grids and element masking for Fields built on unstructured Meshes. The user may mask out points 
 in the source Field or destination Field or both. To do masking the user sets mask information in the Grid (see~\ref{sec:usage:items}) or 
 Mesh (see~\ref{sec:mesh:mask}) upon which the Fields passed into the 
 {\tt ESMF\_FieldRegridStore()} call are built. The `srcMaskValues' and `dstMaskValues' arguments to that call can then be used to specify which values in that mask 
 information indicate that a location should be masked out. For example, if `dstMaskValues' is set to (/1,2/), then any location that has a value of 1 or 2 in 
 the mask information of the Grid or Mesh upon which the destination Field is built will be masked out.

 Masking behavior differs slightly between regridding methods. For non-conservative regridding methods (e.g. bilinear or high-order patch), masking is done on
 points. For these methods, masking a destination point means that the point won't participate in regridding (e.g. won't be interpolated to). For these methods, 
 masking a source point means that the entire source cell using that point is masked out. In other words, if any corner point making up a source cell is masked 
 then the cell is masked. For conservative regridding methods (e.g. first-order conservative) masking is done on cells. Masking a destination cell means that the 
 cell won't participate in regridding (e.g. won't be interpolated to). Similarly, masking a source cell means that the cell won't participate in regridding 
 (e.g. won't be interpolated from). For any type of interpolation method (conservative or non-conservative) the masking is set on the location upon which the 
 Fields passed into the regridding call are built. For example, if Fields built on {\tt ESMF\_STAGGERLOC\_CENTER} are passed into the {\tt ESMF\_FieldRegridStore()} 
 call then the masking should also be set in {\tt ESMF\_STAGGERLOC\_CENTER}.


\subsubsection{Unmapped destination points}
 If a destination point can't be mapped to a location in the source grid, the user has two options. The user may ignore those destination points
 that can't be mapped by setting the {\tt unmappedaction} argument to {\tt ESMF\_UNMAPPEDACTION\_IGNORE}. (Ignored points won't be included in
 the sparse matrix or routeHandle output from the {\tt ESMF\_FieldRegridStore()} call.)   The user also has the option to return
 an error if unmapped destination points exist. This is the default behavior, so the user can either not set the {\tt unmappedaction} argument
 or the user can set it to {\tt ESMF\_UNMAPPEDACTION\_ERROR}. At this point ESMF does not support extrapolation to destination points outside 
 the unmasked source Field. 


\subsubsection{Interpolation methods: bilinear}\label{sec:interpolation:bilinear}
 Bilinear interpolation calculates the value for the 
 destination point as a combination of multiple linear interpolations, one for each dimension of the Grid. Note that for ease of 
 use, the term bilinear interpolation is used for 3D interpolation in ESMF as well, although it should more properly be referred 
 to as trilinear interpolation.

\smallskip

 In 2D, ESMF supports bilinear regridding between any combination of the following:
 \begin{itemize}
 \item Structured Grids composed of a single logically rectangular patch
 \item Unstructured Meshes composed of any combination of triangles and quadrilaterals (e.g. rectangles)
 \end{itemize}

\smallskip

 In 3D, ESMF supports bilinear regridding between any combination of the following:
 \begin{itemize}
 \item Structured Grids composed of a single logically rectangular patch
 \item Unstructured Meshes composed of hexahedrons (e.g. cubes)
 \end{itemize}

\smallskip

 To use the bilinear method the user may created their Fields on any stagger location for Grids or the node location ({\tt ESMF\_MESHLOC\_NODE}) for Meshes.
 For Grids, the stagger location upon which the Field was built must contain coordinates. 

\subsubsection{Interpolation methods: higher-order patch}\label{sec:interpolation:patch}

 Patch (or higher-order) interpolation is the ESMF version of a technique called ``patch recovery'' commonly
 used in finite element modeling~\cite{PatchInterp1}~\cite{PatchInterp2}. It typically results in better approximations to 
 values and derivatives when compared to bilinear interpolation.
 Patch interpolation works by constructing multiple polynomial patches to represent
 the data in a source cell. For 2D grids, these polynomials
 are currently 2nd degree 2D polynomials. One patch is constructed for each corner of the source cell, and the patch is constructed 
 by doing a least squared fit through the data in the cells surrounding the corner. The interpolated value at the destination point is 
 then a weighted average of the values of the patches at that point. The patch method has a larger
 stencil than the bilinear, for this reason the patch weight matrix can be correspondingly larger
 than the bilinear matrix (e.g. for a quadrilateral grid the patch matrix is around 4x the size of
 the bilinear matrix). This can be an issue when performing a regrid operation close to the memory
 limit on a machine.  

\smallskip

 In 2D, ESMF supports patch regridding between any combination of the following:
 \begin{itemize}
 \item Structured Grids composed of a single logically rectangular patch
 \item Unstructured Meshes composed of any combination of triangles and quadrilaterals (e.g. rectangles)
 \end{itemize}

\smallskip

 Patch regridding is currently not supported in 3D.

\smallskip

 To use the patch method the user may created their Fields on any stagger location for Grids or the node location ({\tt ESMF\_MESHLOC\_NODE}) for Meshes.
 For Grids, the stagger location upon which the Field was built must contain coordinates. 


\subsubsection{Interpolation methods: first-order conservative}\label{sec:interpolation:conserve}
 First-order conservative interpolation~\cite{ConservativeOrder1} is also available as a regridding method. This method 
 will typically have  
 a larger local interpolation error than the previous two methods, but will do a much better job of preserving the value
 of the  integral of data between the source and destination grid. In this method the value across each source cell
 is treated as a constant. The weights for a particular destination cell are the area of intersection of each 
 source cell with the destination cell divided by the area of the destination cell. For cartesian grids, the area of a grid cell is the typical cartesian area. 
 For grids on a sphere, cell areas are calculated by connecting the corner coordinates of each grid cell with great circles. If the user doesn't specify
 cell areas in the involved Grids or Meshes, then the conservation will hold for the areas as calculated by 
 ESMF. This means the following equation will hold:  sum-over-all-source-cells(Vsi*Asi) = sum-over-all-destination-cells(Vdj*A'dj), where
 V is the variable being regridded and A' is the area of a cell as calculated by ESMF.  The subscripts s and d refer to source and destination values, and the i and j are the source 
 and destination grid cell indices (flattening the arrays to 1 dimension). If the user does specify the area's in the Grid or Mesh, then the conservation will be adjusted to work for the areas 
 provided by the user. This means the following equation will hold:  sum-over-all-source-cells(Vsi*Asi) = sum-over-all-destination-cells(Vdj*Adj),
 where A is the area of a cell as provided by the user. 

 The user should be aware that because of the conservation relationship between the source and destination fields, the more the total source area
 differs from the total destination area the more the values of the source field will differ from the corresponding values of the destination field, likely giving a higher 
 interpolation error. It is best to have the total source and destination areas the same (this will automatically be true if no user areas are specified). For source and destination grids 
 which only partially overlap the areas which should be the same are the areas of the overlapping regions of the source and destination. 

 Note that for grids on a sphere the conservative interpolation assumes great circle edges to cells. This means that the
 edges of a cell won't necessarily be
 the same as a straight line in latitude longitude. For small edges, this difference will be small, but for long edges it
 could be significant. This means if
 the user expects cell edges as straight lines in latitude longitude space, they should avoid using one large cell with 
 long edges to compute an average over a region (e.g. over an ocean basin). The 
 user should also avoid using cells which contain one edge that runs half way or more around the earth, because the 
 regrid weight calculation assumes the 
 edge follows the shorter great circle path. Also, there isn't a unique great circle edge defined between points on the 
 exact opposite side of the earth from one another (antipodal points). 
 However, the user can work around both of these problem by breaking the long edge into two smaller edges by inserting 
 an extra node, or by breaking the large target grid cells 
 into two or more smaller grid cells. This allows the application to resolve the ambiguity in edge direction. 

 It is important to note that the current implementation of conservative regridding doesn't normalize the interpolation 
 weights by the destination fraction. This means that for a destination
 grid which only partially overlaps the source grid the destination field which is output from the regrid operation 
 should be divided by the corresponding destination fraction to yield the 
 true interpolated values for cells which are only partially covered by the source grid. The fraction also needs to be 
 included when computing the total source and destination integrals. 

 The following pseudo-code shows how to compute the total source integral ({\tt src\_total}) given the source field values
 ({\tt src\_field}), the source area ({\tt src\_area}) from the {\tt ESMF\_FieldRegridGetArea()} call, and
 the source fraction ({\tt src\_frac}) from the {\tt ESMF\_FieldRegridStore()} call:

\begin{verbatim}
 src_total=0.0
 for each source element i
    src_total=src_total+src_field(i)*src_area(i)*src_frac(i)
 end for
\end{verbatim}

 The following pseudo-code shows how to compute the total destination integral ({\tt dst\_total}) given the
 destination field values ({\tt dst\_field}) resulting
 from the {\tt ESMF\_FieldRegrid()} call, the destination area ({\tt dst\_area}) from the {\tt ESMF\_FieldRegridGetArea()}
 call,
 and the destination fraction ({\tt dst\_frac}) from the {\tt ESMF\_FieldRegridStore()} call. It also 
 shows how to adjust the destination field ({\tt dst\_field}) resulting from the {\tt ESMF\_FieldRegrid()} call by the
 fraction ({\tt dst\_frac}) from the {\tt ESMF\_FieldRegridStore()} call: 

\begin{verbatim}

 dst_total=0.0
 for each destination element i
    if (dst_frac(i) not equal to 0.0) then
       dst_total=dst_total+dst_field(i)*dst_area(i) 
       dst_field(i)=dst_field(i)/dst_frac(i)
        If mass computed here after dst_field adjust, would need to be:
        dst_total=dst_total+dst_field(i)*dst_area(i)*dst_frac(i) 
    end if
 end for
\end{verbatim}

\smallskip

 In 2D, ESMF supports conservative regridding between any combination of the following:
 \begin{itemize}
 \item Structured Grids composed of a single logically rectangular patch
 \item Unstructured Meshes composed of any combination of triangles and quadrilaterals (e.g. rectangles)
 \end{itemize}

\smallskip

 In 3D, ESMF supports conservative regridding between any combination of the following:
 \begin{itemize}
 \item Structured Grids composed of a single logically rectangular patch
 \item Unstructured Meshes composed of hexahedrons (e.g. cubes) and tetrahedras.
 \end{itemize}

\smallskip

 To use the conservative method the user must have created their Fields on the center 
 stagger location ({\tt ESMF\_STAGGERLOC\_CENTER} in 2D or {\tt ESMF\_STAGGERLOC\_CENTER\_VCENTER} in 3D) for Grids  or the element location ({\tt ESMF\_MESHLOC\_ELEMENT}) for Meshes.
 For Grids, the corner stagger location ({\tt ESMF\_STAGGERLOC\_CORNER} in 2D or {\tt ESMF\_STAGGERLOC\_CORNER\_VFACE} in 3D) must contain coordinates describing the outer perimeter of the Grid cells. 

\subsubsection{Troubleshooting guide}

 The below is a list of problems users commonly encounter with regridding and potential solutions. 
 This is by no means an exhaustive list, so if none of these problems fit your case, or if the solutions
 don't fix your problem, please feel free to email esmf support (esmf\_support@list.woc.noaa.gov).

 \bigskip
 
 {\bf Problem:} Regridding is too slow.

 \medskip

 {\bf Possible Cause:} The {\tt ESMF\_FieldRegridStore()} method is called more than is necessary. \newline
 The {\tt ESMF\_FieldRegridStore()} operation is a complex one and can be 
 relatively slow for some cases (large Grids, 3D grids, etc.) 
 
 \smallskip

 {\bf Solution:} Reduce the number of {\tt ESMF\_FieldRegridStore()} calls to the minimum necessary. The
 routeHandle generated by the {\tt ESMF\_FieldRegridStore()} call depends on only four factors: the 
 stagger locations that the input Fields are created on, the coordinates in the Grids the input Fields
 are built on at those stagger locations, the padding of the input Fields 
 (specified by the {\tt totalWidth} arguments in {\tt FieldCreate}) and the size of the tensor
 dimensions in the input Fields (specified by the {\tt ungridded} arguments in {\tt FieldCreate}). 
 For any pair of Fields which share these attributes with the Fields used in the
 {\tt ESMF\_FieldRegridStore} call  the same routeHandle can be used. Note, that the data in the 
 Fields does NOT matter, the same routeHandle can be used no matter how the data in the Fields changes.

 \smallskip

 In particular:
 \begin{itemize}

 \item If Grid coordinates do not change during a run, then the {\tt ESMF\_FieldRegridStore()} call can be
 done once between a pair of Fields at the beginning and the resulting routeHandle used for each 
 timestep during the run. 

 \item If a pair of Fields was created with exactly the same arguments to {\tt ESMF\_FieldCreate()} as the 
 pair of Fields used during an {\tt ESMF\_FieldRegridStore()} call, then the resulting routeHandle can 
 also be used between that pair of Fields. 
 \end{itemize}

 \bigskip
 
 {\bf Problem:} Distortions in destination Field at periodic boundary.

 \medskip

 {\bf Possible Cause:} The Grid overlaps itself. With a periodic Grid, the regrid system expects
  the first point to not be a repeat of the last point. In other words,
  regrid constructs its own connection and overlap between the first and last points of the
  periodic dimension and so the Grid doesn't need to contain these. If the Grid does, then this
  can cause problems. 

 \smallskip

 {\bf Solution:} Define the Grid so that it doesn't contain the overlap point. This typically means simply making
 the Grid one point smaller in the periodic dimension.  If a Field 
 constructuted on the Grid needs to contain these overlap points then the user can use the
 {\tt totalWidth} arguments to include this extra padding in the Field. Note, however, 
 that the regrid won't update these extra points, so the user will have to do a copy to fill the points
 in the overlap region in the Field.  


\subsection{File-based Regrid API}
\input{../Superstructure/PreESMFMod/doc/ESMF_RegridWeightGen_fapi}
