%                **** IMPORTANT NOTICE *****
% This LaTeX file has been automatically produced by ProTeX v. 1.1
% Any changes made to this file will likely be lost next time
% this file is regenerated from its source. Send questions 
% to Arlindo da Silva, dasilva@gsfc.nasa.gov
 
\setlength{\parskip}{0pt}
\setlength{\parindent}{0pt}
\setlength{\baselineskip}{11pt}
 
%--------------------- SHORT-HAND MACROS ----------------------
\def\bv{\begin{verbatim}}
\def\ev{\end{verbatim}}
\def\be{\begin{equation}}
\def\ee{\end{equation}}
\def\bea{\begin{eqnarray}}
\def\eea{\end{eqnarray}}
\def\bi{\begin{itemize}}
\def\ei{\end{itemize}}
\def\bn{\begin{enumerate}}
\def\en{\end{enumerate}}
\def\bd{\begin{description}}
\def\ed{\end{description}}
\def\({\left (}
\def\){\right )}
\def\[{\left [}
\def\]{\right ]}
\def\<{\left  \langle}
\def\>{\right \rangle}
\def\cI{{\cal I}}
\def\diag{\mathop{\rm diag}}
\def\tr{\mathop{\rm tr}}
%-------------------------------------------------------------

\markboth{Left}{Source File: NUOPC\_Driver.F90,  Date: Mon Mar 28 16:35:43 PDT 2011
}

\bigskip{\sf MODULE:}
\begin{verbatim}  module NUOPC_Driver
\end{verbatim}

\bigskip{\sf DESCRIPTION:\\}
Component that drives Model, Mediator, and Connector components. For every Driver time step the same run sequence, i.e. sequence of Model, Mediator, and Connector {\tt Run} methods is called. The run sequence is fully customizable. The default run sequence implements explicit time stepping.

\bigskip{\sf SUPER:}
\begin{verbatim}  ESMF_GridComp
\end{verbatim}

\bigskip{\sf USE DEPENDENCIES:}
\begin{verbatim}  use ESMF
\end{verbatim}

\bigskip{\sf SETSERVICES:}
\begin{verbatim}  subroutine SetServices(driver, rc)
    type(ESMF_GridComp)   :: driver
    integer, intent(out)  :: rc
\end{verbatim}

\bigskip{\sf INITIALIZE:}
\begin{itemize}
\item phase 0: ({\sc Required, NUOPC Provided})
  \begin{itemize}
  \item Initialize the {\tt InitializePhaseMap} Attribute according to the NUOPC Initialize Phase Definition (IPD) version 00 (see section \ref{IPD} for a precise definition). The default implementation sets the following mapping:
    \begin{itemize}
    \item IPDv00p1 = 1: ({\sc Required, NUOPC Provided})
    \end{itemize}  
  \end{itemize}  
\item IPDv00p1 ({\sc NUOPC Provided})
  \begin{itemize}
  \item Allocate and initialize internal data structures.
  \item If the internal clock is not yet set, set the default internal clock to be a copy of the incoming clock, but only if the incoming clock is valid.
  \item {\it Required specialization} to set component services: {\tt label\_SetModelServices}.
  \begin{itemize}
    \item Call {\tt NUOPC\_DriverAddComp()} for all Model, Mediator, and Connector components to be added.
    \item Optionally replace the default clock. 
  \end{itemize}
  \item {\it Optional specialization} to set run sequence: {\tt label\_SetRunSequence}.

  \item Execute Initialize phase=0 for all Model, Mediator, and Connector components. This is the method where each component is required to initialize its {\tt InitializePhaseMap} Attribute.
  \item {\it Optional specialization} to analyze and modify the {\tt InitializePhaseMap} Attribute of the child components before the Driver uses it: {\tt label\_ModifyInitializePhaseMap}.
  \item Drive the initialize sequence for the child components, compatible with up to IPDv04, as documented in section \ref{IPD}.
  \end{itemize}  
\end{itemize}

\bigskip{\sf RUN:}
\begin{itemize}
\item phase 1: ({\sc Required, NUOPC Provided})
  \begin{itemize}
  \item If the incoming clock is valid, set the internal stop time to one time step interval on the incoming clock.
  \item Drive the time stepping loop, from current time to stop time, incrementing by time step.
  \begin{itemize}
    \item For each time step iteration the Model and Connector components Run() methods are being called according to the run sequence.
  \end{itemize}  
  \end{itemize}    
\end{itemize}

\bigskip{\sf FINALIZE:}
\begin{itemize}
\item phase 1: ({\sc Required, NUOPC Provided})
  \begin{itemize}
  \item {\it Optional specialization} to finalize driver component: {\tt label\_Finalize}.
  \item Execute the Finalize() methods of all Connector components in order.
  \item Execute the Finalize() methods of all Model components in order.
  \item Destroy all Model components and their import and export states.
  \item Destroy all Connector components.
  \item Internal clean-up.
  \end{itemize}      
\end{itemize}

\mbox{}\hrulefill\ 

%...............................................................

