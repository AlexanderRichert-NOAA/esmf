%                **** IMPORTANT NOTICE *****
% This LaTeX file has been automatically produced by ProTeX v. 1.1
% Any changes made to this file will likely be lost next time
% this file is regenerated from its source. Send questions 
% to Arlindo da Silva, dasilva@gsfc.nasa.gov
 
\setlength{\parskip}{0pt}
\setlength{\parindent}{0pt}
\setlength{\baselineskip}{11pt}
 
%--------------------- SHORT-HAND MACROS ----------------------
\def\bv{\begin{verbatim}}
\def\ev{\end{verbatim}}
\def\be{\begin{equation}}
\def\ee{\end{equation}}
\def\bea{\begin{eqnarray}}
\def\eea{\end{eqnarray}}
\def\bi{\begin{itemize}}
\def\ei{\end{itemize}}
\def\bn{\begin{enumerate}}
\def\en{\end{enumerate}}
\def\bd{\begin{description}}
\def\ed{\end{description}}
\def\({\left (}
\def\){\right )}
\def\[{\left [}
\def\]{\right ]}
\def\<{\left  \langle}
\def\>{\right \rangle}
\def\cI{{\cal I}}
\def\diag{\mathop{\rm diag}}
\def\tr{\mathop{\rm tr}}
%-------------------------------------------------------------

\markboth{Left}{Source File: NUOPC\_Driver.F90,  Date: Mon Mar 28 16:35:43 PDT 2011
}

\bigskip{\sf MODULE:}
\begin{verbatim}  module NUOPC_Driver
\end{verbatim}

\bigskip{\sf DESCRIPTION:\\}
Driver component that drives model and connector components. The default is to use explicit time stepping. Each driver time step, the same sequence of model and connector components' {\tt Run} methods are called. The run sequence is fully customizable.

\bigskip{\sf SUPER:}
\begin{verbatim}  ESMF_GridComp
\end{verbatim}

\bigskip{\sf USE DEPENDENCIES:}
\begin{verbatim}  use ESMF
\end{verbatim}

\bigskip{\sf SETSERVICES:}
\begin{verbatim}  subroutine routine_SetServices(gcomp, rc)
    type(ESMF_GridComp)   :: gcomp
    integer, intent(out)  :: rc
\end{verbatim}

\bigskip{\sf INITIALIZE:}
\begin{itemize}
\item phase 1: {\sc Provided:}
  \begin{itemize}
  \item Initialize the internal state.
  \item Set default internal clock to be a copy of the incoming clock.
  \item {\it Required specialization} to set number of model components, {\tt modelCount}, in the internal state: {\tt label\_SetModelCount}.
  \item Allocate internal storage according to {\tt modelCount}.
  \item {\it Optional specialization} to provide Model and Connector {\tt petList} members in the internal state: {\tt label\_SetModelPetList}.
  \item Create Model components with their import and export States.
  \item Attach standard NUOPC Model Component metadata.
  \item Create Connector components.
  \item Attach standard NUOPC Connector Component metadata.
  \item Initialize the default run sequence.
  \item {\it Required specialization} to set component services: {\tt label\_SetModelServices}. 
  \begin{itemize}
    \item Call into SetServices() for all Model and Connector components.
    \item Optionally replace the default clock. 
    \item Optionally replace the default run sequence.
  \end{itemize}
  \item Execute Model components' Initialize() phase 0, in order.
  \item Execute Connector components' Initialize() phase 0, in order.
  \item Execute Model components' Initialize() phase 1, in order.
  \item Execute Connector components' Initialize() phase 1, in order.
  \item Execute Model components' Initialize() phase 2, in order.
  \item Execute Connector components' Initialize() phase 2, in order.
  \item Execute Model components' Initialize() phase 3, in order.
  \item Execute Model components' Initialize() phase 4, in order.
  \end{itemize}  
\end{itemize}

\bigskip{\sf RUN:}
\begin{itemize}
\item phase 1: {\sc Provided:}
  \begin{itemize}
  \item Time stepping loop, from current time to stop time, incrementing by time step.
  \item For each time step iteration, the Model and Connector components Run() methods are being called according to the run sequence.
  \end{itemize}    
\end{itemize}

\bigskip{\sf FINALIZE:}
\begin{itemize}
\item phase 1: {\sc Provided:}
  \begin{itemize}
  \item {\it Optional specialization} to finalize driver component: {\tt label\_Finalize}.
  \item Execute all Connector components' Finalize() methods in order.
  \item Execute all Model components' Finalize() methods in order.
  \item Destroy all Model components and their import and export states.
  \item Destroy all Connector components.
  \item Deallocate the run sequence.
  \item Deallocate the internal state.
  \end{itemize}      
\end{itemize}

\bigskip{\sf INTERNALSTATE:}
\begin{verbatim}  label_InternalState

  type type_InternalState
    type(type_InternalStateStruct), pointer :: wrap
  end type

  type type_InternalStateStruct
    integer                           :: modelCount
    type(type_PetList),  pointer      :: modelPetLists(:)
    type(type_PetList),  pointer      :: connectorPetLists(:,:)
    !--- private members ----------------------------------------
    type(ESMF_GridComp), pointer      :: modelComp(:)
    type(ESMF_State),    pointer      :: modelIS(:), modelES(:)
    type(ESMF_CplComp),  pointer      :: connectorComp(:,:)
    type(NUOPC_RunSequence), pointer  :: runSeq(:)  ! size may increase dynamic.
    integer                           :: runPhaseToRunSeqMap(10)
  end type

  type type_PetList
    integer, pointer :: petList(:)  ! lists that are set here transfer ownership
  end type
  
\end{verbatim}
%...............................................................

