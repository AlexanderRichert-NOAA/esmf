%                **** IMPORTANT NOTICE *****
% This LaTeX file has been automatically produced by ProTeX v. 1.1
% Any changes made to this file will likely be lost next time
% this file is regenerated from its source. Send questions 
% to Arlindo da Silva, dasilva@gsfc.nasa.gov
 
\setlength{\parskip}{0pt}
\setlength{\parindent}{0pt}
\setlength{\baselineskip}{11pt}
 
%--------------------- SHORT-HAND MACROS ----------------------
\def\bv{\begin{verbatim}}
\def\ev{\end{verbatim}}
\def\be{\begin{equation}}
\def\ee{\end{equation}}
\def\bea{\begin{eqnarray}}
\def\eea{\end{eqnarray}}
\def\bi{\begin{itemize}}
\def\ei{\end{itemize}}
\def\bn{\begin{enumerate}}
\def\en{\end{enumerate}}
\def\bd{\begin{description}}
\def\ed{\end{description}}
\def\({\left (}
\def\){\right )}
\def\[{\left [}
\def\]{\right ]}
\def\<{\left  \langle}
\def\>{\right \rangle}
\def\cI{{\cal I}}
\def\diag{\mathop{\rm diag}}
\def\tr{\mathop{\rm tr}}
%-------------------------------------------------------------

\markboth{Left}{Source File: NUOPC\_Driver.F90,  Date: Mon Mar 28 16:35:43 PDT 2011
}

\bigskip{\sf MODULE:}
\begin{verbatim}  module NUOPC_Driver
\end{verbatim}

\bigskip{\sf DESCRIPTION:\\}
Component that drives and coordinates initialization of its child components: Model, Mediator, and Connector components. For every Driver time step the same run sequence, i.e. sequence of Model, Mediator, and Connector {\tt Run} methods is called. The run sequence is fully customizable. The default run sequence implements explicit time stepping.

\bigskip{\sf SUPER:}
\begin{verbatim}  ESMF_GridComp
\end{verbatim}

\bigskip{\sf USE DEPENDENCIES:}
\begin{verbatim}  use ESMF
\end{verbatim}

\bigskip{\sf SETSERVICES:}
\begin{verbatim}  subroutine SetServices(driver, rc)
    type(ESMF_GridComp)   :: driver
    integer, intent(out)  :: rc
\end{verbatim}

\bigskip{\sf SEMANTIC SPECIALIZATION LABELS:}
\begin{itemize}
  \item Initialize:
  \begin{itemize}
    \item {\bf label\_SetModelServices}
    \begin{itemize}
      \item Optional. By default driver has no child components.
      \item Use {\tt NUOPC\_DriverAddComp()} repeatedly to add child components to the driver.
      \item Use {\tt NUOPC\_CompAttributeSet()} or {\tt NUOPC\_CompAttributeIngest()} to set attributes on child components.
      \item Create and set driver clock with startTime, stopTime, and timeStep, if not done by the driver's parent.
    \end{itemize}
    \item {\bf label\_SetRunSequence}
    \begin{itemize}
      \item Optional. By default drive child components in the sequence they were added.
      \item Define and set a RunSequence either by calling {\tt NUOPC\_DriverIngestRunSequence()}, or by using the {\tt NUOPC\_DriverNewRunSequence()} and {\tt NUOPC\_DriverAddRunElement()} API.
    \end{itemize}
    \item {\bf label\_ModifyInitializePhaseMap}
    \begin{itemize}
      \item Optional. By default InitializePhaseMap attributes are not modified.
      \item Modify the InitializePhaseMap attribute on the child components as desired. This is very rarely needed.
    \end{itemize}
    \item {\bf label\_ModifyCplLists}
    \begin{itemize}
      \item Optional. By default CplList attributes are  not modified.
      \item Modify the CplList attribute on the child components as desired. This can be useful to set custom Connection Options for specific Field pairs.
    \end{itemize}
  \end{itemize}
  \item Run:
  \begin{itemize}
    \item {\bf label\_SetRunClock}
      \begin{itemize}
        \item Optional. By default driver clock is not modified each time its Run is entered.
        \item Modify the driver clock before executing RunSequence. This is very rarely needed.
      \end{itemize}
    \item {\bf label\_ExecuteRunSequence}
    \begin{itemize}
      \item Optional. By default use NUOPC generic RunSequence execution.
      \item Implement a custom RunSequence execution. This is very rarely needed.
    \end{itemize}
  \end{itemize}
  \item Finalize:
  \begin{itemize}
    \item {\bf label\_Finalize}
    \begin{itemize}
      \item Optional. By default do nothing.
      \item Destroy any objects created during Initalize.
    \end{itemize}
  \end{itemize}
\end{itemize}


\mbox{}\hrulefill\ 

%...............................................................

