% $Id$
%
The NUOPC Layer uses standard metadata on Fields to guide the decision making that is implemented in generic code. The generic {\tt NUOPC\_Connector} component, for instance, uses the {\tt StandardName} Attribute to construct a list of matching Fields between the import and export States. The NUOPC Field Dictionary provides a software implementation of a controlled vocabulary for the {\tt StandardName} Attribute. It also associates each registered {\tt StandardName} with canonical {\tt Units}, a default {\tt LongName}, and a default {\tt ShortName}.

The NUOPC Layer provides a number of default entries in the Field Dictionary, shown in the table below. The {\tt StandardName} Attribute of all default entries complies with the Climate and Forecast (CF) conventions as documented at \htmladdnormallink{http://cf-pcmdi.llnl.gov/}{http://cf-pcmdi.llnl.gov/}. 

Currently it is typically that a user of the NUOPC Layer extends the Field Dictionary by calling the {\tt  NUOPC\_FieldDictionaryAddEntry()} interface to add additional entries. It is our intention to grow the number of default entries over time, and to more strongly leverage the NUOPC Field Dictionary to ensure meta data interoperability between codes that use the NUOPC Layer.

Besides the {\tt StandardName} Attribute, the NUOPC Layer currently only uses the {\tt Units} entry to verify that Fields are given in their canonical units. The plan is to extend this to support unit conversion in the future. The default {\tt LongName} and default {\tt ShortName} associations are provided as a convenience to the implementor of NUOPC compliant components; the NUOPC Layer itself does not base any decisions on these two Attributes.
