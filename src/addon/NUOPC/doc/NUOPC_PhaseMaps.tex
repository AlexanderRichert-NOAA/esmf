% $Id$
%

\label{PhaseMaps}

ESMF introduces the concept of standard component methods: {\tt Initialize}, {\tt Run}, and {\tt Finalize}. ESMF further recognizes the need for being able to split each of the standard methods into multiple phases. On the ESMF level, phases are implemented by a simple integer {\tt phase} index. The NUOPC layer adds an abstraction layer that allows phases to be referenced by label.

For complex scenarios, e.g. multiple versions of multi-stage initialize sequences, the use of an integer based phase index quickly becomes confusing. The NUOPC Layer addresses this issue by introducing three component level attributes: {\tt InitializePhaseMap}, {\tt RunPhaseMap}, and {\tt FinalizePhaseMap}. These attributes map logical NUOPC phase labels to integer ESMF phase indices. A NUOPC compliant component fully documents its available phases through the phase maps.

Currently the NUOPC Layer leverages the {\tt InitializePhaseMap} during the initialization loop that is implemented by the generic {\tt NUOPC\_Driver}. It looks for phase map entries according to the initialize phase definition outlined in section \ref{IPD}. The {\tt RunPhaseMap} is used when setting up run sequences in the Driver. The {\tt NUOPC\_DriverAddRunElement()} takes the {\tt phaseLabel} argument, and uses the {\tt RunPhaseMap} attribute internally to translates the label into the corresponding ESMF phase index. The {\tt FinalizePhaseMap} is currently not used within the NUOPC Layer.

Within NUOPC, components under a driver are also referenced by label. Every component is associated with a label when it is added to a driver through the {\tt NUOPC\_DriverAddComp()} call. Multiple instances of the same component can be added to a driver, provided each instance is given a unique label. Connectors between components are identified by providing the label of the source component and destination component.
