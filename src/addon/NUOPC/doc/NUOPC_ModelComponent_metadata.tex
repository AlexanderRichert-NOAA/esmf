\label{CompMeta}
The Model and Mediator Component metadata is implemented as an ESMF Attribute Package:

\begin{itemize}
    \item Convention: NUOPC
    \item Purpose: General
    \item Includes:
    \begin{itemize}
      \item CIM Model Component Simulation Description (see for example the \htmladdnormallink{Component Attribute packages}{http://www.earthsystemmodeling.org/esmf\_releases/public/ESMF\_5\_2\_0rp2/ESMF\_refdoc/node6.html\#SECTION06022100000000000000} section in the ESMF v5.2.0rp2 documentation)
    \end{itemize} 
    \item Description: Model/Mediator component description and nesting metadata. 
\end{itemize}

\begin{tabular}{|p{5cm}|p{5cm}|p{35mm}|}
     \hline\hline
     {\bf Attribute name} & {\bf Definition} & {\bf Controlled vocabulary}\\
     \hline\hline
     {\tt Verbosity} & String value controlling the verbosity of INFO messages.& high, low\\ \hline
     {\tt InitializePhaseMap} & List of string values, mapping the logical NUOPC initialize phases, of a specific Initialize Phase Definition (IPD) version, to the actual ESMF initialize phase number under which the entry point is registered.& IPDvXXpY=Z, where XX = two-digit revision number, e.g. 01, Y = logical NUOPC phase number, Z = actual ESMF phase number, with Y, Z > 0 and Y, Z < 10 \\ \hline
     {\tt RunPhaseMap} & List of string values, mapping the logical NUOPC run phases to the actual ESMF run phase number under which the entry point is registered.& {\em label-string}=Z, where {\em label-string} can be chosen freely, and Z = actual ESMF phase number. \\ \hline
     {\tt FinalizePhaseMap} & List of string values, mapping the logical NUOPC finalize phases to the actual ESMF finalize phase number under which the entry point is registered.& {\em label-string}=Z, where {\em label-string} can be chosen freely, and Z = actual ESMF phase number. \\ \hline
     {\tt NestingGeneration} & Integer value enumerating nesting level.& 0, 1, 2, ...\\ \hline
     {\tt Nestling} & Integer value enumerating siblings within the same generation.& 0, 1, 2, ...\\ \hline
     {\tt InitializeDataComplete} & String value indicating whether all initialize data dependencies have been satisfied.& false, true\\ \hline
     {\tt InitializeDataProgress} & String value indicating whether progress is being made resolving initialize data dependencies.& false, true\\ \hline
     \hline
\end{tabular}
