%                **** IMPORTANT NOTICE *****
% This LaTeX file has been automatically produced by ProTeX v. 1.1
% Any changes made to this file will likely be lost next time
% this file is regenerated from its source. Send questions 
% to Arlindo da Silva, dasilva@gsfc.nasa.gov
 
\setlength{\parskip}{0pt}
\setlength{\parindent}{0pt}
\setlength{\baselineskip}{11pt}
 
%--------------------- SHORT-HAND MACROS ----------------------
\def\bv{\begin{verbatim}}
\def\ev{\end{verbatim}}
\def\be{\begin{equation}}
\def\ee{\end{equation}}
\def\bea{\begin{eqnarray}}
\def\eea{\end{eqnarray}}
\def\bi{\begin{itemize}}
\def\ei{\end{itemize}}
\def\bn{\begin{enumerate}}
\def\en{\end{enumerate}}
\def\bd{\begin{description}}
\def\ed{\end{description}}
\def\({\left (}
\def\){\right )}
\def\[{\left [}
\def\]{\right ]}
\def\<{\left  \langle}
\def\>{\right \rangle}
\def\cI{{\cal I}}
\def\diag{\mathop{\rm diag}}
\def\tr{\mathop{\rm tr}}
%-------------------------------------------------------------

\markboth{Left}{Source File: NUOPC\_Model.F90,  Date: Mon Mar 28 16:35:43 PDT 2011
}

\bigskip{\sf MODULE:}
\begin{verbatim}  module NUOPC_Model
\end{verbatim}

\bigskip{\sf DESCRIPTION:\\}
Model component with a default {\em explicit} time dependency. Each time the {\tt Run} method is called the model integrates one timeStep forward on the passed in parent clock. The internal clock is advanced at the end of each {\tt Run} call. The component flags incompatibility during {\tt Run} if the current time of the incoming clock does not match the current time of the internal clock.

\bigskip{\sf SUPER:}
\begin{verbatim}  NUOPC_ModelBase
\end{verbatim}

\bigskip{\sf USE DEPENDENCIES:}
\begin{verbatim}  use ESMF
\end{verbatim}

\bigskip{\sf SETSERVICES:}
\begin{verbatim}  subroutine SetServices(model, rc)
    type(ESMF_GridComp)   :: model
    integer, intent(out)  :: rc
\end{verbatim}

\bigskip{\sf INITIALIZE:}
\begin{itemize}
\item phase 0: Set Initialize Phase Definition Version ({\sc Required, NUOPC Provided})
  \begin{itemize}
  \item Initialize the {\tt InitializePhaseMap} Attribute according to the NUOPC Initialize Phase Definition (IPD) version 00 (see section \ref{IPD} for a precise definition). The default implementation sets the following mapping:
    \begin{itemize}
    \item IPDv00p1 = 1: ({\sc Required, Implementor Provided})
      \begin{itemize}
      \item Advertise Fields in import and export States.
      \end{itemize}
    \item IPDv00p2 = 2: ({\sc Required, Implementor Provided})
      \begin{itemize}
      \item Realize the advertised Fields in import and export States.
      \end{itemize}  
    \item IPDv00p3 = 3: ({\sc Required, NUOPC Provided})
      \begin{itemize}
      \item Check compatibility of the Fields' Connected status.
      \end{itemize}
    \item IPDv00p4 = 4: ({\sc Required, NUOPC Provided})
      \begin{itemize}
      \item Handle Field data initialization. Time stamp the export Fields.
      \end{itemize}
    \end{itemize}  
  \end{itemize}  
\item IPDv00p1, IPDv01p1, IPDv02p1, IPDv03p1, IPDv04p1, IPDv05p1: Advertise fields in import and export States ({\sc Required, Implementor Provided})
  \begin{itemize}
  \item Advertise fields in import/export states using one of the two {\tt NUOPC\_Advertise} methods (\ref{NUOPC_AdvertiseField}, \ref{NUOPC_AdvertiseFields}). The methods require Standard Names for each field, and the Standard Names must appear in the NUOPC Field Dictionary or a runtime error is generated.  {\tt NUOPC\_Advertise} accepts a {\tt TransferOfferGeomObject} argument which may be one of:
    \begin{itemize}
    \item ``will provide'' (default) - The field will provide its own geometric object (i.e., Grid, Mesh, or LocStream)
    \item ``can provide''- The field can provide its own geometric object, but only if the connected field in the other component will not provide it
    \item ``cannot provide'' - The field cannot provide its own geometric object. It must accept a geometric object from a connected field.
    \end{itemize}
  See section \ref{TransferGeom} for more details about transferring geometric objects between NUOPC components.
  Memory is not allocated for advertised fields, but attributes are set on the field which can be used in later phases, especially for determining if another component can provide and/or consume the advertised field.
  \end{itemize}
\item IPDv00p2, IPDv01p3, IPDv02p3, IPDv03p3, IPDv04p3, IPDv05p4: Realize field {\em providing} a geometric object ({\sc Required*, Implementor Provided})
  \begin{itemize}
  \item Realize connected import and export fields that have their TransferActionGeomObject attribute set to ``provide'', i.e., that will provide their own geometric object (i.e., Grid, Mesh, or LocStream). ``provide'' is the default value of TransferActionGeomObject. Realize means an ESMF\_Field object is created on a geometric object and memory for the field is allocated or referenced.

The {\tt NUOPC\_Realize} methods (\ref{NUOPC_RealizeComplete}, \ref{NUOPC_RealizeField}) are used to realize fields. Only previously advertised fields can be realized and the field's name is used to search the state for the previously advertised field.

*Note: This phase is not required if all fields are {\em accepting} a geometric object.
  \end{itemize}
\item IPDv03p4, IPDv04p4, IPDv05p5: Modify decomposition of accepted geometric object ({\sc Optional, Implementor Provided})
  \begin{itemize}
  \item Optionally modify the decomposition information of any accepted geometric object by replacing the DistGrid. In the case of the Grid geometric object, this can be accomplished by retrieving the Grid (and its DistGrid) from the Field, creating a new DistGrid with modified decomposition, creating a new Grid on the new (modified) DistGrid, and then using {\tt ESMF\_FieldEmptySet} to replace the existing Grid with the new one.

This phase is useful when accepting a Grid from another component, but when the PET counts differ between components. In this case, a new decomposition needs to be set based on the current processor count.
  \end{itemize}
\item IPDv03p5, IPDv04p5, IPDv05p6: Realize fields {\em accepting} a geometric object ({\sc Required*, Implementor Provided})
  \begin{itemize}
  \item Realize connected import and export fields that have their TransferActionGeomObject attribute set to ``accept'', i.e., that will accept a geometric object from a connected field in another component.  If the generic {\tt NUOPC\_Connector} is used, at this point the full geometric object has already been set in the field and only a call to {\tt ESMF\_FieldEmptyComplete} is required to allocate memory for the field.

The {\tt NUOPC\_Realize} methods (\ref{NUOPC_RealizeComplete}, \ref{NUOPC_RealizeField}) are used to realize fields. Only previously advertised fields can be realized and the field's name is used to search the state for the previously advertised field.

  *Note: This phase is not required if all fields are {\em providing} a geometric object.
  \end{itemize}
\item IPDv00p3, IPDv01p4, IPDv02p4, IPDv03p6, IPDv04p6, IPDv05p7: Verify import fields connected and set clock ({\sc NUOPC Provided})
  \begin{itemize}
  \item If the model internal clock is found to be not set, then set the model internal clock as a copy of the incoming clock. 
  \item {\it Optional specialization} to set the internal clock and/or alarms: {\tt label\_SetClock}.
  \item Check compatibility, ensuring all advertised import Fields are connected.
  \end{itemize}   
\item IPDv00p4, IPDv01p5:  Initialize export fields ({\sc NUOPC Provided})
  \begin{itemize}
  \item {\it Optional specialization} to initialize export Fields: {\tt label\_DataInitialize}
  \item Time stamp Fields in export State for compatibility checking.
  \end{itemize}    
\item IPDv02p5, IPDv03p7, IPDv04p7, IPDv05p8: Initialize export fields ({\sc NUOPC Provided})
  \begin{itemize}
  \item {\it Optional specialization} to initialize export Fields: {\tt label\_DataInitialize}
  \item Timestamp Fields in export State for compatibility checking.
  \item Set Component metadata used to resolve initialize data dependencies.
  \end{itemize}    
\end{itemize}

\bigskip{\sf RUN:}
\begin{itemize}
\item phase 1: ({\sc Required, NUOPC Provided})
  \begin{itemize}
  \item {\sc Specialization Required/Provided}: {\tt label\_SetRunClock} to check and set the internal Clock against the incoming Clock.
  \begin{itemize}
  \item {\tt IF} (Phase specific specialization present): Execute the phase specific specialization.
  \item {\tt ELSE}: Execute the phase independent specialization. {\sc Provided}: By default check that internal Clock and incoming Clock agree on current time and that the time step of the incoming Clock is a multiple of the internal Clock time step. Under these conditions set the internal stop time to one time step interval of the incoming Clock. Otherwise exit with error, flagging an incompatibility.
  \end{itemize}
  \item {\sc Specialization Required/Provided}: {\tt label\_CheckImport} to check Fields in the import State.
  \begin{itemize}
  \item {\tt IF} (Phase specific specialization is present): Execute the phase specific specialization.
  \item {\tt ELSE}: Execute the phase independent specialization. {\sc Provided}: By default check that all import Fields are at the current time of the internal Clock.
  \end{itemize}
  \item Time stepping loop: starting at current time, running to stop time of the internal Clock.
  \begin{itemize}
  \item Timestamp the Fields in the export State according to the current time of the internal Clock.
  \item {\sc Specialization Required}: {\tt label\_Advance} to execute model code.
  \item Advance the current time of the internal Clock according to the time step of the internal Clock.
  \end{itemize}
  \item Timestamp all Fields in the export State according to the current time of the internal Clock, which now is identical to the stop time of the internal Clock.
  \end{itemize}    
\end{itemize}

\bigskip{\sf FINALIZE:}
\begin{itemize}
\item phase 1: ({\sc Required, NUOPC Provided})
  \begin{itemize}
  \item {\it Optional specialization} to finalize custom parts of the component: {\tt label\_Finalize}.
  \end{itemize}      
\end{itemize}

\subsubsection{Initialize Phase Specialization - label\_SetClock}
{\sc Optional, Implementor Provided}\\
{\em Called from: IPDv00p3, IPDv01p4, IPDv02p4, IPDv03p6, IPDv04p6}\\

The specialization method can change aspects of the internal clock, which defaults to a copy of the incoming parent clock. For example, the timeStep size may be changed and/or Alarms may be set on the clock.\\

The method {\tt NUOPC\_CompSetClock(comp, externalClock, stabilityTimeStep)} (\ref{NUOPC_GridCompSetClock}) can be used to set the internal clock as a copy of externalClock, but with a timeStep that is less than or equal to the stabilityTimeStep. At the same time it ensures that the timeStep of the external clock is a multiple of the timeStep of the internal clock. If the stabilityTimeStep argument is not provided then the internal clock will simply be set as a copy of the external clock.

\subsubsection{Initialize Phase Specialization - label\_DataInitialize}
{\sc Optional, Implementor Provided}\\
{\em Called from: IPDv00p4, IPDv01p5, IPDv02p5, IPDv03p7, IPDv04p7}\\

The specialization method should initialize field data in the export state. Fields in the export state will be timestamped automatically by the calling phase, so there is no need to do it here.

\subsubsection{Run Phase Specialization - label\_SetRunClock}
{\sc Required, NUOPC Provided}\\
{\em Called from: default run phase}\\

A specialization method to check and set the internal clock against the incoming clock. This method is called by the default run phase.\\

If not overridden, the default method will check that the internal clock and incoming clock agree on the current time and that the time step of the incoming clock is a multiple of the internal clock time step. Under these conditions set the internal stop time to one time step interval of the incoming clock. Otherwise exit with error, flagging an incompatibility.

\subsubsection{Run Phase Specialization - label\_CheckImport}
{\sc Required, NUOPC Provided}\\
{\em Called from: default run phase}\\

A specialization method to verify import fields before advancing in time. If not overridden, the default method verifies that all import fields are at the current time of the internal clock.

\subsubsection{Run Phase Specialization - label\_Advance}
{\sc Required, Implementor Provided}\\
{\em Called from: default run phase}\\

A specialization method that advances the model forward in time by one timestep of the internal clock. This method will be called iteratively by the default run phase until reaching the stop time on the internal clock.

\subsubsection{Run Phase Specialization - label\_TimestampExport}
{\sc Required, NUOPC Provided}\\
{\em Called from: default run phase}\\

A specialization method to set the timestamp on export fields after the model has advanced. If not overridden, the default method sets the timestamp on all export fields to the stop time on the internal clock (which is also now the current model time).

\subsubsection{Finalize Phase Specialization - label\_Finalize}
{\sc Optional, Implementor Provided}\\
{\em Called from: default finalize phase}\\

An optional specialization method for custom finalization code and deallocations of user data structures.

\mbox{}\hrulefill\ 

%...............................................................
