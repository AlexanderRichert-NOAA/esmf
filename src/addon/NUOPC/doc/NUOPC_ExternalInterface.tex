% $Id$
%

\label{ExternalInterface}

Complete applications can easily be built by assembling NUOPC compliant components. Many such NUOPC applications are in productive use across several institutions. The top level of such applications is typically implemented via a very thin application layer holding the main program that calls into the top level driver component that derives from {\tt NUOPC\_Driver}. Model components sit under the top level driver, interacting with one another and the driver through the NUOPC protocols. Complex systems have one or more component hierarchy levels under the top level driver as discussed in the previous section.

There are situation, however, where a NUOPC application needs to be controlled by an outside component. Such an outside component does not derive from any of the generic NUOPC components, and cannot be expected to implement the complete NUOPC protocol. Typically such an external component implements its own control structure outside of NUOPC and ESMF. One example of such a situation are data assimilation systems that want to drive a NUOPC forecast application. 

In order to facilitate the external access into a NUOPC application, the {\tt NUOPC\_Driver} provides an {\em external interface}. This interface is implemented through the standard ESMF component methods: Initialize, Run, and Finalize. This interface with the top level NUOPC driver allows an external component to control and interact with the entire NUOPC application.

The standard ESMF component interfaces hold {\tt importState}, {\tt exportState}, and a {\tt clock} argument. These arguments are used to pass data in and out of the NUOPC application, and control the time stepping of the NUOPC model, respectively. The top level driver of a NUOPC application has access to any field that is advertised by any of the components and therefore serves as a single point of access for the entire application.

The external NUOPC interface is currently defined by the Initialize, Run, and Finalize phases documented in the following table.

\vspace*{2ex}
\begin{longtable}[h]{|p{.30\textwidth}|p{.20\textwidth}|p{.50\textwidth}|}
     \hline\hline
     {\bf methodFlag} & {\bf phaseLabel} & {\bf Meaning}\\
     \hline\hline
     {\tt ESMF\_METHOD\_INITIALIZE} & label\_ExternalAdvertise & Called after the external component has 
                                                                 set up the import- and exportStates with 
                                                                 fields that it plans to interact with.
                                                                 On the NUOPC application side this call will
                                                                 got through the complete advertise cylce.\\ \hline
     {\tt ESMF\_METHOD\_INITIALIZE} & label\_ExternalRealize   & Called after the external component has
                                                                 been informed about the connected status of
                                                                 the fields in the import- and exportState.
                                                                 On the NUOPC application side this call will
                                                                 finish setting up RouteHandles between all
                                                                 components involved.\\ \hline
     {\tt ESMF\_METHOD\_INITIALIZE} & label\_ExternalDataInit  & Trigger a complete data initialize throughout
                                                                 the NUOPC application. The expectation is that
                                                                 all components reset their data consistent with
                                                                 the {\tt clock} argument.\\ \hline
     {\tt ESMF\_METHOD\_RUN}        &                          & The default Run() method steps the NUOPC
                                                                 application forward in time according to
                                                                 the {\tt clock} argument.\\ \hline
     {\tt ESMF\_METHOD\_FINALIZE}   & label\_ExternalReset     & Inform the NUOPC application about a {\tt clock}
                                                                 reset.\\ \hline
     {\tt ESMF\_METHOD\_FINALIZE}   &                          & Completely finalize and shut down the NUOPC application.\\ \hline\hline
\end{longtable}

Here {\tt methodFlag} and {\tt phaseLabel} corrsepond to the respective arguments of method {\tt NUOPC\_CompSearchPhaseMap()}. This method is used to determine the actual ESMF phase index needed when calling into {\tt ESMF\_GridCompInitialize()}, {\tt ESMF\_GridCompRun()}, or {\tt ESMF\_GridCompFinalize()}. In cases where no {\tt phaseLabel} is indicated, the default phase is used for the implementation, accessible by not specifying the argument.

The following code example demonstrates typical use of the external interface.

\begin{verbatim}
  ! Call "ExternalAdvertise" Initialize for the earth system Component
  call NUOPC_CompSearchPhaseMap(nuopcApp, methodflag=ESMF_METHOD_INITIALIZE, &
    phaseLabel=label_ExternalAdvertise, phaseIndex=phase, rc=rc)
    ! check rc
  call ESMF_GridCompInitialize(nuopcApp, phase=phase, clock=clock, &
    importState=esmImportState, exportState=esmExportState, userRc=urc, rc=rc)
    ! check rc and urc

  ! Call "ExternalRealize" Initialize for the earth system Component
  call NUOPC_CompSearchPhaseMap(nuopcApp, methodflag=ESMF_METHOD_INITIALIZE, &
    phaseLabel=label_ExternalRealize, phaseIndex=phase, rc=rc)
    ! check rc
  call ESMF_GridCompInitialize(nuopcApp, phase=phase, clock=clock, &
    importState=esmImportState, exportState=esmExportState, userRc=urc, rc=rc)
    ! check rc and urc

  ! Call "ExternalDataInit" Initialize for the earth system Component
  call NUOPC_CompSearchPhaseMap(nuopcApp, methodflag=ESMF_METHOD_INITIALIZE, &
    phaseLabel=label_ExternalDataInit, phaseIndex=phase, rc=rc)
    ! check rc
  call ESMF_GridCompInitialize(nuopcApp, phase=phase, clock=clock, &
    importState=esmImportState, exportState=esmExportState, userRc=urc, rc=rc)
    ! check rc and urc

  ! Run the earth system Component
  call ESMF_GridCompRun(nuopcApp, clock=clock, &
    importState=esmImportState, exportState=esmExportState, userRc=urc, rc=rc)
    ! check rc and urc

  ! Finalize the earth system Component
  call ESMF_GridCompFinalize(nuopcApp, clock=clock, &
    importState=esmImportState, exportState=esmExportState, userRc=urc, rc=rc)
    ! check rc and urc
\end{verbatim}




