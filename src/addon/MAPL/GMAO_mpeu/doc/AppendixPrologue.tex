% +-======-+ 
%  Copyright (c) 2003-2007 United States Government as represented by 
%  the Admistrator of the National Aeronautics and Space Administration.  
%  All Rights Reserved.
%  
%  THIS OPEN  SOURCE  AGREEMENT  ("AGREEMENT") DEFINES  THE  RIGHTS  OF USE,
%  REPRODUCTION,  DISTRIBUTION,  MODIFICATION AND REDISTRIBUTION OF CERTAIN 
%  COMPUTER SOFTWARE ORIGINALLY RELEASED BY THE UNITED STATES GOVERNMENT AS 
%  REPRESENTED BY THE GOVERNMENT AGENCY LISTED BELOW ("GOVERNMENT AGENCY").  
%  THE UNITED STATES GOVERNMENT, AS REPRESENTED BY GOVERNMENT AGENCY, IS AN 
%  INTENDED  THIRD-PARTY  BENEFICIARY  OF  ALL  SUBSEQUENT DISTRIBUTIONS OR 
%  REDISTRIBUTIONS  OF THE  SUBJECT  SOFTWARE.  ANYONE WHO USES, REPRODUCES, 
%  DISTRIBUTES, MODIFIES  OR REDISTRIBUTES THE SUBJECT SOFTWARE, AS DEFINED 
%  HEREIN, OR ANY PART THEREOF,  IS,  BY THAT ACTION, ACCEPTING IN FULL THE 
%  RESPONSIBILITIES AND OBLIGATIONS CONTAINED IN THIS AGREEMENT.
%  
%  Government Agency: National Aeronautics and Space Administration
%  Government Agency Original Software Designation: GSC-15354-1
%  Government Agency Original Software Title:  GEOS-5 GCM Modeling Software
%  User Registration Requested.  Please Visit http://opensource.gsfc.nasa.gov
%  Government Agency Point of Contact for Original Software:  
%  			Dale Hithon, SRA Assistant, (301) 286-2691
%  
% +-======-+ 
\appendix
\addcontentsline{toc}{part}{APPENDIX}
 
  
%/////////////////////////////////////////////////////////////
\newpage
%
%\markboth{Left}
%   {Source File: m\_Filename.F90, Date: Thu Nov 15 22:24:22 GMT 2001}
% 
%/////////////////////////////////////////////////////////////
\section{Programming References} \label{app:ProLogues}
%
  \subsection{Module m\_Filename - Filename manipulation routines }
  
\bigskip{\sf INTERFACE:}
\begin{verbatim} 
     module m_Filename
       public :: Filename_base		! basename()
       public :: Filename_dir		! dirname()
 \end{verbatim}
%
%{\sf REVISION HISTORY:}
%\begin{verbatim}   	14Feb00	- Jing Guo <guo@dao.gsfc.nasa.gov>
%  		- initial prototype/prolog/code\end
%{verbatim}
 
%/////////////////////////////////////////////////////////////
 
\mbox{}\hrulefill\ 

  \subsubsection{Filename\_base - basename}
  
\bigskip{\sf INTERFACE:}
\begin{verbatim} 
     function Filename_base(cstr,sfx)
       character(len=*)         ,intent(in) :: cstr
       character(len=*),optional,intent(in) :: sfx
       character(len=len(cstr))             :: Filename_base
 \end{verbatim}
%
%{\sf REVISION HISTORY:}
%\begin{verbatim}   	14Feb00	- Jing Guo <guo@dao.gsfc.nasa.gov>
%  		- initial prototype/prolog/code\end{verbatim}
 
%/////////////////////////////////////////////////////////////
 
\mbox{}\hrulefill\ 
 

  \subsubsection{Filename\_dir - dirname}

\bigskip{\sf INTERFACE:}
\begin{verbatim} 
     function Filename_dir(cstr)
       character(len=*),intent(in) :: cstr
       character(len=len(cstr))    :: Filename_dir
 \end{verbatim}
%
%{\sf REVISION HISTORY:}
%\begin{verbatim}   	14Feb00	- Jing Guo <guo@dao.gsfc.nasa.gov>
%  		- initial prototype/prolog/code\end{verbatim}

%\markboth{Left}{Source File: m\_IndexBin\_char.F90,  Date: Thu Nov 15 22:24:22 GMT 2001
%}

 
%/////////////////////////////////////////////////////////////
 
\mbox{}\hrulefill\ 
 
\subsection{Module m\_IndexBin\_char - Template of indexed bin-sorting module }

\bigskip{\sf INTERFACE:}
\begin{verbatim} 
     module m_IndexBin_char
       public :: IndexBin
 \end{verbatim}
%
%{\sf REVISION HISTORY:}
%\begin{verbatim}   	17Feb99 - Jing Guo <guo@thunder> - initial prototype/prolog/code\end{verbatim}
 
%/////////////////////////////////////////////////////////////
 
\mbox{}\hrulefill\ 
 
\subsubsection{IndexBin - Indexed sorting for a single value}
 
\bigskip{\sf INTERFACE:}
\begin{verbatim} 
     subroutine IndexBin(n,indx,keys,key0,ln0)
       integer, intent(in)                        :: n
       integer, dimension(n), intent(inout)       :: indx
       character(len=*), dimension(n), intent(in) :: keys
       character(len=*), intent(in)               :: key0 ! value
       integer,optional,intent(out)               :: ln0
 \end{verbatim}
%
%{\sf REVISION HISTORY:}
%\begin{verbatim}   	16Feb99 - Jing Guo <guo@thunder> - initial prototype/prolog/code
%  	27Sep99 - Jing Guo <guo@thunder> - Fixed a bug pointed out by
%  					   Chris Redder\end{verbatim}
 
%/////////////////////////////////////////////////////////////
 
\mbox{}\hrulefill\ 
 
  \subsubsection{IndexBin - Indexed sorting into a set of given bins}

\bigskip{\sf INTERFACE:}
\begin{verbatim} 
     subroutine IndexBin(n,indx,keys,bins,lcs,lns)
       integer, intent(in)                      :: n
       integer, dimension(n),intent(inout)      :: indx
       character(len=*),dimension(n),intent(in) :: keys
       character(len=*),dimension(:),intent(in) :: bins ! values
       integer, dimension(:),intent(out)        :: lcs ! locs. of the bins
       integer, dimension(:),intent(out)        :: lns ! sizes of the bins
 \end{verbatim}
%
%{\sf REVISION HISTORY:}
%\begin{verbatim}   	16Feb99 - Jing Guo <guo@thunder> - initial prototype/prolog/code\end{verbatim}
 
%/////////////////////////////////////////////////////////////
 
\mbox{}\hrulefill\ 
 

  \subsubsection{IndexBin - IndexBin wrapped without working arrays}
  
\bigskip{\sf INTERFACE:}
\begin{verbatim} 
     subroutine IndexBin(n,indx,keys,bins)
       integer,             intent(in)           :: n
       integer,dimension(n),intent(inout)        :: indx
       character(len=*),dimension(n),intent(in)  :: keys
       character(len=*),dimension(:),intent(in)  :: bins ! values
 \end{verbatim}
%
%{\sf REVISION HISTORY:}
%\begin{verbatim}   	17Feb99 - Jing Guo <guo@thunder> - initial prototype/prolog/code\end{verbatim}

%\markboth{Left}{Source File: m\_IndexBin\_integer.F90,  Date: Thu Nov 15 22:24:22 GMT 2001
%}

 
%/////////////////////////////////////////////////////////////
 
\mbox{}\hrulefill\ 
 

  \subsection{Module m\_IndexBin\_integer - Template of indexed bin-sorting module }

\bigskip{\sf INTERFACE:}
\begin{verbatim} 
     module m_IndexBin_integer
       public :: IndexBin
       interface IndexBin
          module procedure IndexBin0_, IndexBin1_, IndexBin1w_
       end interface
 \end{verbatim}
%
%{\sf REVISION HISTORY:}
%\begin{verbatim}   	17Feb99 - Jing Guo <guo@thunder> - initial prototype/prolog/code\end{verbatim}
 
%/////////////////////////////////////////////////////////////
 
\mbox{}\hrulefill\ 
 
  \subsubsection{IndexBin0\_ - Indexed sorting for a single value}

\bigskip{\sf INTERFACE:}
\begin{verbatim} 
     subroutine IndexBin0_(n,indx,keys,key0,ln0)
       integer, intent(in) :: n
       integer, dimension(n), intent(inout) :: indx
       integer, dimension(n), intent(in) :: keys
       integer, intent(in) :: key0 ! The key value to be moved to front
       integer,optional,intent(out) :: ln0
 \end{verbatim} %{\sf REVISION HISTORY:}
%\begin{verbatim}   	16Feb99 - Jing Guo <guo@thunder> - initial prototype/prolog/code
%  	27Sep99 - Jing Guo <guo@thunder> - Fixed a bug pointed out by
%  					   Chris Redder\end{verbatim}
 
%/////////////////////////////////////////////////////////////
 
\mbox{}\hrulefill\ 
 
  \subsubsection{IndexBin1\_ - Indexed sorting into a set of given bins}

\bigskip{\sf INTERFACE:}
\begin{verbatim} 
     subroutine IndexBin1_(n,indx,keys,bins,lcs,lns)
       integer, intent(in) :: n
       integer, dimension(n),intent(inout) :: indx
       integer, dimension(n),intent(in)    :: keys
       integer, dimension(:),intent(in)    :: bins! values of the bins
       integer, dimension(:),intent(out)   :: lcs ! locs. of the bins
       integer, dimension(:),intent(out)   :: lns ! sizes of the bins
 \end{verbatim} %{\sf REVISION HISTORY:}
%\begin{verbatim}   	16Feb99 - Jing Guo <guo@thunder> - initial prototype/prolog/code\end{verbatim}
 
%/////////////////////////////////////////////////////////////
 
\mbox{}\hrulefill\ 
 
  \subsubsection{IndexBin1w\_ - IndexBin1\_ wrapped without working arrays}

\bigskip{\sf INTERFACE:}
\begin{verbatim} 
     subroutine IndexBin1w_(n,indx,keys,bins)
       integer,             intent(in)    :: n
       integer,dimension(n),intent(inout) :: indx
       integer,dimension(n),intent(in)    :: keys
       integer,dimension(:),intent(in)    :: bins ! values of the bins
 \end{verbatim} %{\sf REVISION HISTORY:}
%\begin{verbatim}   	17Feb99 - Jing Guo <guo@thunder> - initial prototype/prolog/code\end{verbatim}

%\markboth{Left}{Source File: m\_IndexBin\_logical.F90,  Date: Thu Nov 15 22:24:22 GMT 2001
%}

 
%/////////////////////////////////////////////////////////////
 
\mbox{}\hrulefill\ 
 
  \subsection{Module m\_IndexBin\_logical - Template of indexed bin-sorting module }

\bigskip{\sf INTERFACE:}
\begin{verbatim} 
     module m_IndexBin_logical
       public :: IndexBin
       interface IndexBin
         module procedure IndexBin0_
       end interface
 \end{verbatim} %{\sf REVISION HISTORY:}
%\begin{verbatim}   	17Feb99 - Jing Guo <guo@thunder> - initial prototype/prolog/code\end{verbatim}
 
%/////////////////////////////////////////////////////////////
 
\mbox{}\hrulefill\ 
 

  \subsubsection{IndexBin0\_ - Indexed sorting for a single value}

\bigskip{\sf INTERFACE:}
\begin{verbatim} 
     subroutine IndexBin0_(n,indx,keys,key0,ln0)
       integer, intent(in) :: n
       integer, dimension(n), intent(inout) :: indx
       logical, dimension(n), intent(in) :: keys
       logical, intent(in) :: key0 ! The key value to be moved to front
       integer,optional,intent(out) :: ln0
 \end{verbatim} %{\sf REVISION HISTORY:}
%\begin{verbatim}   	16Feb99 - Jing Guo <guo@thunder> - initial prototype/prolog/code
%  	27Sep99 - Jing Guo <guo@thunder> - Fixed a bug pointed out by
%  					   Chris Redder\end{verbatim}

%\markboth{Left}{Source File: m\_List.F90,  Date: Thu Nov 15 22:24:22 GMT 2001
%}

 
%/////////////////////////////////////////////////////////////
 
\mbox{}\hrulefill\ 
 
  \subsection{Module m\_List - a list manager }

\bigskip{\sf INTERFACE:}
\begin{verbatim} 
     module m_List 
       public :: List		! The class data structure
       public :: init
       public :: clean
       public :: index
       public :: nitem
       public :: get
       public :: assignment(=)
 
     type List
       character(len=1),dimension(:),pointer :: bf
       integer,       dimension(:,:),pointer :: lc
     end type List
 \end{verbatim} %{\sf REVISION HISTORY:}
%\begin{verbatim}   	22Apr98 - Jing Guo <guo@thunder> - initial prototype/prolog/code\end{verbatim}
 
%/////////////////////////////////////////////////////////////
 
\mbox{}\hrulefill\ 
 
  \subsubsection{init\_ - initialized a List from a character string}

  	A list is a string in the form of ``\verb"cat:tiger:lion"'',
     or ``\verb"lat:lon:lev"''.  Through the initialization call, the
     items delimited by ``\verb":"'' are stored as an array of sub-
     strings of a long string, accessible through an array of substring
     indices.  The only constraints now on the valid list entries are,
     (1) the value of an entry does not contain ``\verb":"'', and (2)
     The leading and the trailing blanks are insignificant, although
     any imbeded blanks are.
  
\bigskip{\sf INTERFACE:}
\begin{verbatim} 
     subroutine init_(aList,Values)
       type(List),intent(out)	  :: aList  ! an indexed string values
       character(len=*),intent(in) :: Values ! ":" delimited names
 \end{verbatim} %{\sf REVISION HISTORY:}
%\begin{verbatim}   	22Apr98 - Jing Guo <guo@thunder> - initial prototype/prolog/code\end{verbatim}
 
%/////////////////////////////////////////////////////////////
 
\mbox{}\hrulefill\ 
 
  \subsubsection{initStr\_ initialize with a String type}

\bigskip{\sf INTERFACE:}
\begin{verbatim} 
     subroutine initStr_(aList,pstr)
       type(List),intent(out)     :: aList  ! an indexed string values
       type(String),intent(in)	  :: pstr
 \end{verbatim} %{\sf REVISION HISTORY:}
%\begin{verbatim}   	23Apr98 - Jing Guo <guo@thunder> - initial prototype/prolog/code\end{verbatim}
 
%/////////////////////////////////////////////////////////////
 
\mbox{}\hrulefill\ 
 
  \subsubsection{initStr1\_ initialize with an array of Strings}

\bigskip{\sf INTERFACE:}
\begin{verbatim} 
     subroutine initStr1_(aList,strs)
       type(List),intent(out)               :: aList  ! an indexed string values
       type(String),dimension(:),intent(in) :: strs
 \end{verbatim} %{\sf REVISION HISTORY:}
%\begin{verbatim}   	23Apr98 - Jing Guo <guo@thunder> - initial prototype/prolog/code\end{verbatim}
 
%/////////////////////////////////////////////////////////////
 
\mbox{}\hrulefill\ 
 

  \subsubsection{clean\_ - clean a List variable}

\bigskip{\sf INTERFACE:}
\begin{verbatim} 
     subroutine clean_(aList)
       type(List),intent(inout) :: aList
 \end{verbatim} %{\sf REVISION HISTORY:}
%\begin{verbatim}   	22Apr98 - Jing Guo <guo@thunder> - initial prototype/prolog/code\end{verbatim}
 
%/////////////////////////////////////////////////////////////
 
\mbox{}\hrulefill\ 
 
\subsubsection{nitem\_ - number of items in the list}

\bigskip{\sf INTERFACE:}
\begin{verbatim} 
     function nitem_(aList)
       type(List),intent(in) :: aList
       integer               :: nitem_
 \end{verbatim} %{\sf REVISION HISTORY:}
%\begin{verbatim}   	22Apr98 - Jing Guo <guo@thunder> - initial prototype/prolog/code\end{verbatim}
 
%/////////////////////////////////////////////////////////////
 
\mbox{}\hrulefill\ 
 
 \subsubsection{index\_ - lookup a list for a given item name}

\bigskip{\sf INTERFACE:}
\begin{verbatim} 
     function index_(aList,item)
       type(List),      intent(in) :: aList	! a List of names
       character(len=*),intent(in) :: item	! a given item name
       integer                     :: index_
 \end{verbatim} %{\sf REVISION HISTORY:}
%\begin{verbatim}   	22Apr98 - Jing Guo <guo@thunder> - initial prototype/prolog/code\end{verbatim}
 
%/////////////////////////////////////////////////////////////
 
\mbox{}\hrulefill\ 
 

  \subsubsection{indexStr\_ - lookup a list for a given item name}

\bigskip{\sf INTERFACE:}
\begin{verbatim} 
     function indexStr_(aList,itemStr)
       type(List),      intent(in) :: aList	! a List of names
       type(String),    intent(in) :: itemStr
       integer                     :: indexStr_
 \end{verbatim} %{\sf REVISION HISTORY:}
%\begin{verbatim}   	22Apr98 - Jing Guo <guo@thunder> - initial prototype/prolog/code\end{verbatim}
 
%/////////////////////////////////////////////////////////////
 
\mbox{}\hrulefill\ 
 
  \subsubsection{copy\_ - Copy a List. Pointers are copied as allocatables}

\bigskip{\sf INTERFACE:}
\begin{verbatim} 
     subroutine copy_(yL,xL)	! yL=xL
       type(List),intent(out) :: yL
       type(List),intent(in)  :: xL
 \end{verbatim} %{\sf REVISION HISTORY:}
%\begin{verbatim}   	22Apr98 - Jing Guo <guo@thunder> - initial prototype/prolog/code\end{verbatim}
 
%/////////////////////////////////////////////////////////////
 
\mbox{}\hrulefill\ 
 

  \subsubsection{get\_ - return a numbered item from the List}

\bigskip{\sf INTERFACE:}
\begin{verbatim} 
     subroutine get_(itemStr,ith,aList)
       type(String),intent(out) :: itemStr
       integer,     intent(in)  :: ith
       type(List),  intent(in)  :: aList
 \end{verbatim} %{\sf REVISION HISTORY:}
%\begin{verbatim}   	23Apr98 - Jing Guo <guo@thunder> - initial prototype/prolog/code\end{verbatim}
 
%/////////////////////////////////////////////////////////////
 
\mbox{}\hrulefill\ 
 

  \subsubsection{getall\_ - return all items from the List}

\bigskip{\sf INTERFACE:}
\begin{verbatim} 
     subroutine getall_(itemStr,aList)
       type(String),intent(out) :: itemStr
       type(List),  intent(in)  :: aList
 \end{verbatim} %{\sf REVISION HISTORY:}
%\begin{verbatim}   	23Apr98 - Jing Guo <guo@thunder> - initial prototype/prolog/code\end{verbatim}
 
%/////////////////////////////////////////////////////////////
 
\mbox{}\hrulefill\ 
 
  \subsubsection{getrange\_ - return a range of items from the List}

\bigskip{\sf INTERFACE:}
\begin{verbatim} 
     subroutine getrange_(itemStr,i1,i2,aList)
       type(String),intent(out) :: itemStr
       integer,     intent(in)  :: i1
       integer,     intent(in)  :: i2
       type(List),  intent(in)  :: aList
 \end{verbatim} %{\sf REVISION HISTORY:}
%\begin{verbatim}   	23Apr98 - Jing Guo <guo@thunder> - initial prototype/prolog/code\end{verbatim}
 
%/////////////////////////////////////////////////////////////
 
\mbox{}\hrulefill\ 
 
  \subsubsection{set\_indices\_ - set the indices of given items}

\bigskip{\sf INTERFACE:}
\begin{verbatim} 
     subroutine set_indices_(indices,aList,values)
       integer,dimension(:),intent(out) :: indices
       type(List),intent(in)            :: aList  ! an indexed string values
       character(len=*),intent(in)      :: Values ! ":" delimited names
 \end{verbatim} %{\sf REVISION HISTORY:}
%\begin{verbatim}   	31May98 - Jing Guo <guo@thunder> - initial prototype/prolog/code\end{verbatim}

%\markboth{Left}{Source File: m\_MergeSorts.F90,  Date: Thu Nov 15 22:24:22 GMT 2001
%}

 
%/////////////////////////////////////////////////////////////
 
\mbox{}\hrulefill\ 
 

  \subsection{Module m\_MergeSorts - Tools for incremental indexed-sorting }

     This tool module contains basic sorting procedures, that in
     addition to a couple of standard Fortran 90 statements in the
     array syntex, allow a full range sort or unsort operations.
     The main characteristics of the sorting algorithm used in this
     module are, a) stable, and b) index sorting.
  
\bigskip{\sf INTERFACE:}
\begin{verbatim} 
     module m_MergeSorts
       public :: IndexSet
       public :: IndexSort
 
       interface IndexSet
         module procedure setn_
         module procedure set_
       end interface
       interface IndexSort
         module procedure iSortn_
         module procedure rSortn_
         module procedure dSortn_
         module procedure cSortn_
         module procedure iSort_
         module procedure rSort_
         module procedure dSort_
         module procedure cSort_
         module procedure iSort1_
         module procedure rSort1_
         module procedure dSort1_
         module procedure cSort1_
       end interface
\end{verbatim}
 
%   !EXAMPLES:
%  	...
%  	integer, intent(in) :: No
%  	type(Observations), dimension(No), intent(inout) :: obs
%  	integer, dimension(No) :: indx	! automatic array
%  	call IndexSet(No,indx)
%  	call IndexSort(No,indx,obs(1:No)%lev,descend=.false.)
%  	call IndexSort(No,indx,obs(1:No)%lon,descend=.false.)
%  	call IndexSort(No,indx,obs(1:No)%lat,descend=.false.)
%  	call IndexSort(No,indx,obs(1:No)%kt,descend=.false.)
%  	call IndexSort(No,indx,obs(1:No)%ks,descend=.false.)
%  	call IndexSort(No,indx,obs(1:No)%kx,descend=.false.)
%  	call IndexSort(No,indx,obs(1:No)%kr,descend=.false.)
%  		! Sorting
%  	obs(1:No) = obs( (/ (indx(i),i=1,No) /) )
%       	...
%  		! Unsorting
%  	obs( (/ (indx(i),i=1,No) /) ) = obs(1:No)
%       \end{verbatim}
%{\sf REVISION HISTORY:}
%\begin{verbatim}  	15Mar00	- Jing Guo
%  		. Added interfaces without the explicit size
%  		. Added interfaces for two dimensional arrays
%  	02Feb99 - Jing Guo <guo@thunder> - Added if(present(stat)) ...
%   	04Jan99 - Jing Guo <guo@thunder> - revised
%   	09Sep97 - Jing Guo <guo@thunder> - initial prototype/prolog/code\end{verbatim}
 
%/////////////////////////////////////////////////////////////
 
\mbox{}\hrulefill\ 
 
  \subsubsection{setn\_ - Initialize an array of data location indices}

\bigskip{\sf INTERFACE:}
\begin{verbatim} 
     subroutine setn_(n,indx)
       integer, intent(in)                :: n		! size of indx(:)
       integer, dimension(n), intent(out) :: indx	! indices
 \end{verbatim} %{\sf REVISION HISTORY:}
%\begin{verbatim}  	15Mar00	- Jing Guo
%  		. initial prototype/prolog/code
%  		. redefined for the original interface\end{verbatim}
 
%/////////////////////////////////////////////////////////////
 
\mbox{}\hrulefill\ 
 
  \subsubsection{set\_ - Initialize an array of data location indices}

\bigskip{\sf INTERFACE:}
\begin{verbatim} 
     subroutine set_(indx)
       integer, dimension(:), intent(out) :: indx	! indices
 \end{verbatim} %{\sf REVISION HISTORY:}
%\begin{verbatim}  	15Mar00	- Jing Guo
%  		. Modified the interface, by removing the explicit size
%   	09Sep97 - Jing Guo <guo@thunder> - initial prototype/prolog/code
%   	04Jan99 - Jing Guo <guo@thunder> - revised prolog format\end{verbatim}
 
%/////////////////////////////////////////////////////////////
 
\mbox{}\hrulefill\ 
 

  \subsubsection{iSortn\_ - A stable merge index sorting of INTs.}

\bigskip{\sf INTERFACE:}
\begin{verbatim} 
     subroutine iSortn_(n,indx,keys,descend,stat)
       integer,intent(in) :: n
       integer, dimension(n), intent(inout) :: indx
       integer, dimension(n), intent(in)    :: keys
       logical, optional, intent(in)        :: descend
       integer, optional, intent(out)       :: stat
 \end{verbatim} %{\sf REVISION HISTORY:}
%\begin{verbatim}  	15Mar00	- Jing Guo
%  		. initial prototype/prolog/code
%  		. redefined for the original interface\end{verbatim}
 
%/////////////////////////////////////////////////////////////
 
\mbox{}\hrulefill\ 
 

  \subsubsection{rSortn\_ - A stable merge index sorting REALs.}

\bigskip{\sf INTERFACE:}
\begin{verbatim} 
     subroutine rSortn_(n,indx,keys,descend,stat)
       integer,intent(in)                   :: n
       integer, dimension(n), intent(inout) :: indx
       real(SP),dimension(n), intent(in)    :: keys
       logical, optional, intent(in)        :: descend
       integer, optional, intent(out)       :: stat
 \end{verbatim}%{\sf REVISION HISTORY:}
%\begin{verbatim}  	15Mar00	- Jing Guo
%  		. initial prototype/prolog/code
%  		. redefined for the original interface\end{verbatim}
 
%/////////////////////////////////////////////////////////////
 
\mbox{}\hrulefill\ 
 

  \subsubsection{dSortn\_ - A stable merge index sorting DOUBLEs.}

\bigskip{\sf INTERFACE:}
\begin{verbatim} 
     subroutine dSortn_(n,indx,keys,descend,stat)
       integer,intent(in)                   :: n
       integer, dimension(n), intent(inout) :: indx
       real(DP), dimension(n), intent(in)   :: keys
       logical, optional, intent(in)        :: descend
       integer, optional, intent(out)       :: stat
 \end{verbatim}%{\sf REVISION HISTORY:}
%\begin{verbatim}  	15Mar00	- Jing Guo
%  		. initial prototype/prolog/code
%  		. redefined for the original interface\end{verbatim}
 
%/////////////////////////////////////////////////////////////
 
\mbox{}\hrulefill\ 
 

  \subsubsection{cSortn\_ - A stable merge index sorting of CHAR(*)s.}

\bigskip{\sf INTERFACE:}
\begin{verbatim} 
     subroutine cSortn_(n,indx,keys,descend,stat)
       integer,intent(in)                         :: n
       integer, dimension(n), intent(inout)       :: indx
       character(len=*), dimension(n), intent(in) :: keys
       logical, optional, intent(in)              :: descend
       integer, optional, intent(out)             :: stat
 \end{verbatim}%{\sf REVISION HISTORY:}
%\begin{verbatim}  	15Mar00	- Jing Guo
%  		. initial prototype/prolog/code
%  		. redefined for the original interface\end{verbatim}
 
%/////////////////////////////////////////////////////////////
 
\mbox{}\hrulefill\ 
 

  \subsubsection{iSort\_ - A stable merge index sorting of INTs.}

\bigskip{\sf INTERFACE:}
\begin{verbatim} 
     subroutine iSort_(indx,keys,descend,stat)
       integer, dimension(:), intent(inout) :: indx
       integer, dimension(:), intent(in)    :: keys
       logical, optional, intent(in)        :: descend
       integer, optional, intent(out)       :: stat
 \end{verbatim}%{\sf REVISION HISTORY:}
%\begin{verbatim}  	15Mar00	- Jing Guo
%  		. Modified the interface, by removing the explicit size
%  	02Feb99 - Jing Guo <guo@thunder> - Added if(present(stat)) ...
%   	04Jan99 - Jing Guo <guo@thunder> - revised the prolog
%   	09Sep97 - Jing Guo <guo@thunder> - initial prototype/prolog/code\end{verbatim}
 
%/////////////////////////////////////////////////////////////
 
\mbox{}\hrulefill\ 
 

  \subsubsection{rSort\_ - A stable merge index sorting REALs.}

\bigskip{\sf INTERFACE:}
\begin{verbatim} 
     subroutine rSort_(indx,keys,descend,stat)
       integer, dimension(:), intent(inout) :: indx
       real(SP),dimension(:), intent(in)    :: keys
       logical, optional, intent(in)        :: descend
       integer, optional, intent(out)       :: stat
 \end{verbatim}%{\sf REVISION HISTORY:}
%\begin{verbatim}  	15Mar00	- Jing Guo
%  		. Modified the interface, by removing the explicit size
%  	02Feb99 - Jing Guo <guo@thunder> - Added if(present(stat)) ...
%   	04Jan99 - Jing Guo <guo@thunder> - revised the prolog
%   	09Sep97 - Jing Guo <guo@thunder> - initial prototype/prolog/code\end{verbatim}
 
%/////////////////////////////////////////////////////////////
 
\mbox{}\hrulefill\ 
 
  \subsubsection{dSort\_ - A stable merge index sorting DOUBLEs.}

\bigskip{\sf INTERFACE:}
\begin{verbatim} 
     subroutine dSort_(indx,keys,descend,stat)
       integer, dimension(:), intent(inout) :: indx
       real(DP), dimension(:), intent(in)   :: keys
       logical, optional, intent(in)        :: descend
       integer, optional, intent(out)       :: stat
 \end{verbatim}%{\sf REVISION HISTORY:}
%\begin{verbatim}  	15Mar00	- Jing Guo
%  		. Modified the interface, by removing the explicit size
%  	02Feb99 - Jing Guo <guo@thunder> - Added if(present(stat)) ...
%   	04Jan99 - Jing Guo <guo@thunder> - revised the prolog
%   	09Sep97 - Jing Guo <guo@thunder> - initial prototype/prolog/code\end{verbatim}
 
%/////////////////////////////////////////////////////////////
 
\mbox{}\hrulefill\ 
 

  \subsubsection{cSort\_ - A stable merge index sorting of CHAR(*)s.}

\bigskip{\sf INTERFACE:}
\begin{verbatim} 
     subroutine cSort_(indx,keys,descend,stat)
       integer, dimension(:), intent(inout)       :: indx
       character(len=*), dimension(:), intent(in) :: keys
       logical, optional, intent(in)              :: descend
       integer, optional, intent(out)             :: stat
 \end{verbatim}%{\sf REVISION HISTORY:}
%\begin{verbatim}  	15Mar00	- Jing Guo
%  		. Modified the interface, by removing the explicit size
%  	02Feb99 - Jing Guo <guo@thunder> - Added if(present(stat)) ...
%   	04Jan99 - Jing Guo <guo@thunder> - revised the prolog
%   	09Sep97 - Jing Guo <guo@thunder> - initial prototype/prolog/code\end{verbatim}
 
%/////////////////////////////////////////////////////////////
 
\mbox{}\hrulefill\ 
 
  \subsubsection{iSort1\_ - A stable merge index sorting of INTs.}

\bigskip{\sf INTERFACE:}
\begin{verbatim} 
     subroutine iSort1_(indx,keys,ikey,descend,stat)
       integer, dimension(:), intent(inout) :: indx
       integer, dimension(:,:), intent(in)  :: keys
       integer,intent(in)                   :: ikey
       logical, optional, intent(in)        :: descend
       integer, optional, intent(out)       :: stat
 \end{verbatim}%{\sf REVISION HISTORY:}
%\begin{verbatim}  	15Mar00	- Jing Guo
%  		. initial prototype/prolog/code
%  		. Copied code from iSort_
%  		. Extended the interface and the algorithm to handle
%  		  2-d arrays with an index.\end{verbatim}
% 
%/////////////////////////////////////////////////////////////
 
\mbox{}\hrulefill\ 
 
  \subsubsection{rSort1\_ - A stable merge index sorting REALs.}

\bigskip{\sf INTERFACE:}
\begin{verbatim} 
     subroutine rSort1_(indx,keys,ikey,descend,stat)
       integer, dimension(:), intent(inout) :: indx
       real(SP),dimension(:,:), intent(in)  :: keys
       integer,intent(in)                   :: ikey
       logical, optional, intent(in)        :: descend
       integer, optional, intent(out)       :: stat
 \end{verbatim}%{\sf REVISION HISTORY:}
%\begin{verbatim}  	15Mar00	- Jing Guo
%  		. initial prototype/prolog/code
%  		. Copied code from rSort_
%  		. Extended the interface and the algorithm to handle
%  		  2-d arrays with an index.\end{verbatim}
 
%/////////////////////////////////////////////////////////////
 
\mbox{}\hrulefill\ 
 
  \subsubsection{dSort1\_ - A stable merge index sorting DOUBLEs.}

\bigskip{\sf INTERFACE:}
\begin{verbatim} 
     subroutine dSort1_(indx,keys,ikey,descend,stat)
       integer, dimension(:), intent(inout) :: indx
       real(DP), dimension(:,:), intent(in) :: keys
       integer,intent(in)                   :: ikey
       logical, optional, intent(in)        :: descend
       integer, optional, intent(out)       :: stat
 \end{verbatim}%{\sf REVISION HISTORY:}
%\begin{verbatim}  	15Mar00	- Jing Guo
%  		. initial prototype/prolog/code
%  		. Copied code from dSort_
%  		. Extended the interface and the algorithm to handle
%  		  2-d arrays with an index.\end{verbatim}
 
%/////////////////////////////////////////////////////////////
 
\mbox{}\hrulefill\ 
 
  \subsubsection{cSort1\_ - A stable merge index sorting of CHAR(*)s.}

\bigskip{\sf INTERFACE:}
\begin{verbatim} 
     subroutine cSort1_(indx,keys,ikey,descend,stat)
       integer, dimension(:), intent(inout)         :: indx
       character(len=*), dimension(:,:), intent(in) :: keys
       integer,intent(in)                           :: ikey
       logical, optional, intent(in)                :: descend
       integer, optional, intent(out)               :: stat
 \end{verbatim}%{\sf REVISION HISTORY:}
%\begin{verbatim}  	15Mar00	- Jing Guo
%  		. initial prototype/prolog/code
%  		. Copied code from cSort_
%  		. Extended the interface and the algorithm to handle
%  		  2-d arrays with an index.\end{verbatim}

%\markboth{Left}{Source File: m\_Permuter.F90,  Date: Thu Nov 15 22:24:23 GMT 2001
%}

 
%/////////////////////////////////////////////////////////////
 
\mbox{}\hrulefill\ 
 
  \subsection{Module m\_Permuter - permute/unpermute }

\bigskip{\sf INTERFACE:}
\begin{verbatim} 
     module m_Permuter
       public :: permute	
       public :: unpermute
 
     interface permute
       module procedure	
         permutei_,	&	! integer in place
         permuteio_,	&	! integer with an output
         permutei1_,	&	! integer in place
         permuteio1_,	&	! integer with an output
         permuter_,	&	! real in place
         permutero_,	&	! real with an output
         permuter1_,	&	! real in place
         permutero1_,	&	! real with an output
         permuted_,	&	! dble in place
         permutedo_,	&	! dble with an output
         permuted1_,	&	! dble in place
         permutedo1_,	&	! dble with an output
         permutel_,	&	! logical in place
         permutelo_,	&	! logical with an output
         permutel1_,	&	! logical in place
         permutelo1_		! logical with an output
     end interface
 
     interface unpermute
       module procedure	&
         unpermutei_,	&	! integer in place
         unpermuteio_,	&	! integer with an output
         unpermutei1_,	&	! integer in place
         unpermuteio1_,	&	! integer with an output
         unpermuter_,	&	! real in place
         unpermutero_,	&	! real with an output
         unpermuter1_,	&	! real in place
         unpermutero1_,	&	! real with an output
         unpermuted_,	&	! dble in place
         unpermutedo_,	&	! dble with an output
         unpermuted1_,	&	! dble in place
         unpermutedo1_,	&	! dble with an output
         unpermutel_,	&	! logical in place
         unpermutelo_,	&	! logical with an output
         unpermutel1_,	&	! logical in place
         unpermutelo1_		! logical with an output
     end interface
 \end{verbatim}%{\sf REVISION HISTORY:}
%\begin{verbatim}   	25Aug99	- Jing Guo <guo@dao.gsfc.nasa.gov>
%  		- initial prototype/prolog/code\end{verbatim}
 
%/////////////////////////////////////////////////////////////
 
\mbox{}\hrulefill\ 
 

  \subsubsection{permutei\_ - permute an integer array according to indx[]}

\bigskip{\sf INTERFACE:}
\begin{verbatim} 
     subroutine permutei_(ary,indx,n)
       integer,dimension(:),intent(inout) :: ary
       integer,dimension(:),intent(in)    :: indx
       integer,             intent(in)    :: n
 \end{verbatim}%{\sf REVISION HISTORY:}
%\begin{verbatim}   	25Aug99	- Jing Guo <guo@dao.gsfc.nasa.gov>
%  		- initial prototype/prolog/code\end{verbatim}
% 
%/////////////////////////////////////////////////////////////
 
\mbox{}\hrulefill\ 
 

  \subsubsection{permuteio\_ - permute an integer array according to indx[]}

\bigskip{\sf INTERFACE:}
\begin{verbatim} 
     subroutine permuteio_(aout,ary,indx,n)
       integer,dimension(:),intent(inout) :: aout
       integer,dimension(:),intent(in )   :: ary
       integer,dimension(:),intent(in)    :: indx
       integer,             intent(in)    :: n
 \end{verbatim}%{\sf REVISION HISTORY:}
%\begin{verbatim}   	25Aug99	- Jing Guo <guo@dao.gsfc.nasa.gov>
%  		- initial prototype/prolog/code\end{verbatim}
 
%/////////////////////////////////////////////////////////////
 
\mbox{}\hrulefill\ 
 

  \subsubsection{unpermutei\_ - unpermute a \_permuted\_ integer array}

\bigskip{\sf INTERFACE:}
\begin{verbatim} 
     subroutine unpermutei_(ary,indx,n)
       integer,dimension(:),intent(inout) :: ary
       integer,dimension(:),intent(in)    :: indx
       integer,             intent(in)    :: n
 \end{verbatim}%{\sf REVISION HISTORY:}
%\begin{verbatim}   	25Aug99	- Jing Guo <guo@dao.gsfc.nasa.gov>
%  		- initial prototype/prolog/code\end{verbatim}
 
%/////////////////////////////////////////////////////////////
 
\mbox{}\hrulefill\ 
 

  \subsubsection{unpermuteio\_ - unpermute a \_permuted\_ integer array}

\bigskip{\sf INTERFACE:}
\begin{verbatim} 
     subroutine unpermuteio_(aout,ary,indx,n)
       integer,dimension(:),intent(inout) :: aout
       integer,dimension(:),intent(in)    :: ary
       integer,dimension(:),intent(in)    :: indx
       integer,             intent(in)    :: n
 \end{verbatim}%{\sf REVISION HISTORY:}
%\begin{verbatim}   	25Aug99	- Jing Guo <guo@dao.gsfc.nasa.gov>
%  		- initial prototype/prolog/code\end{verbatim}
 
%/////////////////////////////////////////////////////////////
 
\mbox{}\hrulefill\ 
 

  \subsubsection{permuter\_ - permute a real array according to indx[]}

\bigskip{\sf INTERFACE:}
\begin{verbatim} 
     subroutine permuter_(ary,indx,n)
       real(SP),dimension(:),intent(inout) :: ary
       integer ,dimension(:),intent(in)    :: indx
       integer ,             intent(in)    :: n
 \end{verbatim}%{\sf REVISION HISTORY:}
%\begin{verbatim}   	25Aug99	- Jing Guo <guo@dao.gsfc.nasa.gov>
%  		- initial prototype/prolog/code\end{verbatim}
 
%/////////////////////////////////////////////////////////////
 
\mbox{}\hrulefill\ 
 
  \subsubsection{permutero\_ - permute a real array according to indx[]}

\bigskip{\sf INTERFACE:}
\begin{verbatim} 
     subroutine permutero_(aout,ary,indx,n)
       real(SP),dimension(:),intent(inout) :: aout
       real(SP),dimension(:),intent(in)    :: ary
       integer ,dimension(:),intent(in)    :: indx
       integer ,             intent(in)    :: n
 \end{verbatim}%{\sf REVISION HISTORY:}
%\begin{verbatim}   	25Aug99	- Jing Guo <guo@dao.gsfc.nasa.gov>
%  		- initial prototype/prolog/code\end{verbatim}
 
%/////////////////////////////////////////////////////////////
 
\mbox{}\hrulefill\ 

  \subsubsection{unpermuter\_ - unpermute a \_permuted\_ real array}

\bigskip{\sf INTERFACE:}
\begin{verbatim} 
     subroutine unpermuter_(ary,indx,n)
       real(SP),dimension(:),intent(inout) :: ary
       integer ,dimension(:),intent(in)    :: indx
       integer ,             intent(in)    :: n
 \end{verbatim}%{\sf REVISION HISTORY:}
%\begin{verbatim}   	25Aug99	- Jing Guo <guo@dao.gsfc.nasa.gov>
%  		- initial prototype/prolog/code\end{verbatim}
 
%/////////////////////////////////////////////////////////////
 
\mbox{}\hrulefill\ 
 

  \subsubsection{unpermutero\_ - unpermute a \_permuted\_ real array}

\bigskip{\sf INTERFACE:}
\begin{verbatim} 
     subroutine unpermutero_(aout,ary,indx,n)
       real(SP),dimension(:),intent(inout) :: aout
       real(SP),dimension(:),intent(in)    :: ary
       integer ,dimension(:),intent(in)    :: indx
       integer ,             intent(in)    :: n
 \end{verbatim}%{\sf REVISION HISTORY:}
%\begin{verbatim}   	25Aug99	- Jing Guo <guo@dao.gsfc.nasa.gov>
%  		- initial prototype/prolog/code\end{verbatim}
 
%/////////////////////////////////////////////////////////////
 
\mbox{}\hrulefill\ 
 
  \subsubsection{permuted\_ - permute a double precision array}

\bigskip{\sf INTERFACE:}
\begin{verbatim} 
     subroutine permuted_(ary,indx,n)
       real(DP),dimension(:),intent(inout) :: ary
       integer ,dimension(:),intent(in)    :: indx
       integer ,             intent(in)    :: n
 \end{verbatim}%{\sf REVISION HISTORY:}
%\begin{verbatim}   	25Aug99	- Jing Guo <guo@dao.gsfc.nasa.gov>
%  		- initial prototype/prolog/code\end{verbatim}
 
%/////////////////////////////////////////////////////////////
 
\mbox{}\hrulefill\ 
 
  \subsubsection{permutedo\_ - permute a double precision array}

\bigskip{\sf INTERFACE:}
\begin{verbatim} 
     subroutine permutedo_(aout,ary,indx,n)
       real(DP),dimension(:),intent(inout) :: aout
       real(DP),dimension(:),intent(in)    :: ary
       integer ,dimension(:),intent(in)    :: indx
       integer ,             intent(in)    :: n
 \end{verbatim}%{\sf REVISION HISTORY:}
%\begin{verbatim}   	25Aug99	- Jing Guo <guo@dao.gsfc.nasa.gov>
%  		- initial prototype/prolog/code\end{verbatim}
 
%/////////////////////////////////////////////////////////////
 
\mbox{}\hrulefill\ 
 
  \subsubsection{unpermuted\_ - unpermute a double precision array}

\bigskip{\sf INTERFACE:}
\begin{verbatim} 
     subroutine unpermuted_(ary,indx,n)
       real(DP),dimension(:),intent(inout) :: ary
       integer ,dimension(:),intent(in)    :: indx
       integer ,             intent(in)    :: n
 \end{verbatim}%{\sf REVISION HISTORY:}
%\begin{verbatim}   	25Aug99	- Jing Guo <guo@dao.gsfc.nasa.gov>
%  		- initial prototype/prolog/code\end{verbatim}
 
%/////////////////////////////////////////////////////////////
 
\mbox{}\hrulefill\ 

  \subsubsection{unpermutedo\_ - unpermute a double precision array}

\bigskip{\sf INTERFACE:}
\begin{verbatim} 
     subroutine unpermutedo_(aout,ary,indx,n)
       real(DP),dimension(:),intent(inout) :: aout
       real(DP),dimension(:),intent(in)    :: ary
       integer ,dimension(:),intent(in)    :: indx
       integer ,             intent(in)    :: n
 \end{verbatim}%{\sf REVISION HISTORY:}
%\begin{verbatim}   	25Aug99	- Jing Guo <guo@dao.gsfc.nasa.gov>
%  		- initial prototype/prolog/code\end{verbatim}
 
%/////////////////////////////////////////////////////////////
 
\mbox{}\hrulefill\ 
 

  \subsubsection{permutel\_ - permute a real array according to indx[]}

\bigskip{\sf INTERFACE:}
\begin{verbatim} 
     subroutine permutel_(ary,indx,n)
       logical,dimension(:),intent(inout) :: ary
       integer,dimension(:),intent(in)    :: indx
       integer,             intent(in)    :: n
 \end{verbatim}%{\sf REVISION HISTORY:}
%\begin{verbatim}   	25Aug99	- Jing Guo <guo@dao.gsfc.nasa.gov>
%  		- initial prototype/prolog/code\end{verbatim}
 
%/////////////////////////////////////////////////////////////
 
\mbox{}\hrulefill\ 

  \subsubsection{permutelo\_ - permute a real array according to indx[]}

\bigskip{\sf INTERFACE:}
\begin{verbatim} 
     subroutine permutelo_(aout,ary,indx,n)
       logical,dimension(:),intent(inout) :: aout
       logical,dimension(:),intent(in)    :: ary
       integer,dimension(:),intent(in)    :: indx
       integer,             intent(in)    :: n
 \end{verbatim}%{\sf REVISION HISTORY:}
%\begin{verbatim}   	25Aug99	- Jing Guo <guo@dao.gsfc.nasa.gov>
%  		- initial prototype/prolog/code\end{verbatim}
 
%/////////////////////////////////////////////////////////////
 
\mbox{}\hrulefill\ 
 
  \subsubsection{unpermutel\_ - unpermute a \_permuted\_ logical array}

\bigskip{\sf INTERFACE:}
\begin{verbatim} 
     subroutine unpermutel_(ary,indx,n)
       logical,dimension(:),intent(inout) :: ary
       integer,dimension(:),intent(in)    :: indx
       integer,             intent(in)    :: n
 \end{verbatim}%{\sf REVISION HISTORY:}
%\begin{verbatim}   	25Aug99	- Jing Guo <guo@dao.gsfc.nasa.gov>
%  		- initial prototype/prolog/code\end{verbatim}
 
%/////////////////////////////////////////////////////////////
 
\mbox{}\hrulefill\ 
 
  \subsubsection{unpermutelo\_ - unpermute a \_permuted\_ logical array}

\bigskip{\sf INTERFACE:}
\begin{verbatim} 
     subroutine unpermutelo_(aout,ary,indx,n)
       logical,dimension(:),intent(inout) :: aout
       logical,dimension(:),intent(in)    :: ary
       integer,dimension(:),intent(in)    :: indx
       integer,             intent(in)    :: n
 \end{verbatim}%{\sf REVISION HISTORY:}
%\begin{verbatim}   	25Aug99	- Jing Guo <guo@dao.gsfc.nasa.gov>
%  		- initial prototype/prolog/code\end{verbatim}
 
%/////////////////////////////////////////////////////////////
 
\mbox{}\hrulefill\ 
 
  \subsubsection{permutei1\_ - permute an integer array according to indx[]}

\bigskip{\sf INTERFACE:}
\begin{verbatim} 
     subroutine permutei1_(ary,indx,n)
       integer,dimension(:,:),intent(inout) :: ary
       integer,dimension(:),intent(in)      :: indx
       integer,             intent(in)      :: n
 \end{verbatim}%{\sf REVISION HISTORY:}
%\begin{verbatim}   	25Aug99	- Jing Guo <guo@dao.gsfc.nasa.gov>
%  		- initial prototype/prolog/code\end{verbatim}
 
%/////////////////////////////////////////////////////////////
 
\mbox{}\hrulefill\ 
 
  \subsubsection{permuteio1\_ - permute an integer array according to indx[]}

\bigskip{\sf INTERFACE:}
\begin{verbatim} 
     subroutine permuteio1_(aout,ary,indx,n)
       integer,dimension(:,:),intent(inout) :: aout
       integer,dimension(:,:),intent(in )   :: ary
       integer,dimension(:),intent(in)      :: indx
       integer,             intent(in)      :: n
 \end{verbatim}%{\sf REVISION HISTORY:}
%\begin{verbatim}   	25Aug99	- Jing Guo <guo@dao.gsfc.nasa.gov>
%  		- initial prototype/prolog/code\end{verbatim}
 
%/////////////////////////////////////////////////////////////
 
\mbox{}\hrulefill\ 

  \subsubsection{unpermutei1\_ - unpermute a \_permuted\_ integer array}

\bigskip{\sf INTERFACE:}
\begin{verbatim} 
     subroutine unpermutei1_(ary,indx,n)
       integer,dimension(:,:),intent(inout) :: ary
       integer,dimension(:),intent(in)      :: indx
       integer,             intent(in)      :: n
 \end{verbatim}%{\sf REVISION HISTORY:}
%\begin{verbatim}   	25Aug99	- Jing Guo <guo@dao.gsfc.nasa.gov>
%  		- initial prototype/prolog/code\end{verbatim}
 
%/////////////////////////////////////////////////////////////
 
\mbox{}\hrulefill\ 

  \subsubsection{unpermuteio1\_ - unpermute a \_permuted\_ integer array}

\bigskip{\sf INTERFACE:}
\begin{verbatim} 
     subroutine unpermuteio1_(aout,ary,indx,n)
       integer,dimension(:,:),intent(inout) :: aout
       integer,dimension(:,:),intent(in)    :: ary
       integer,dimension(:),intent(in)      :: indx
       integer,             intent(in)      :: n
 \end{verbatim}%{\sf REVISION HISTORY:}
%\begin{verbatim}   	25Aug99	- Jing Guo <guo@dao.gsfc.nasa.gov>
%  		- initial prototype/prolog/code\end{verbatim}
 
%/////////////////////////////////////////////////////////////
 
\mbox{}\hrulefill\ 
 
  \subsubsection{permuter1\_ - permute a real array according to indx[]}

\bigskip{\sf INTERFACE:}
\begin{verbatim} 
     subroutine permuter1_(ary,indx,n)
       real(SP),dimension(:,:),intent(inout) :: ary
       integer ,dimension(:),intent(in)      :: indx
       integer ,             intent(in)      :: n
 \end{verbatim}%{\sf REVISION HISTORY:}
%\begin{verbatim}   	25Aug99	- Jing Guo <guo@dao.gsfc.nasa.gov>
%  		- initial prototype/prolog/code\end{verbatim}
 
%/////////////////////////////////////////////////////////////
 
\mbox{}\hrulefill\ 
 
  \subsubsection{permutero1\_ - permute a real array according to indx[]}

\bigskip{\sf INTERFACE:}
\begin{verbatim} 
     subroutine permutero1_(aout,ary,indx,n)
       real(SP),dimension(:,:),intent(inout) :: aout
       real(SP),dimension(:,:),intent(in)    :: ary
       integer ,dimension(:),intent(in)      :: indx
       integer ,             intent(in)      :: n
 \end{verbatim}%{\sf REVISION HISTORY:}
%\begin{verbatim}   	25Aug99	- Jing Guo <guo@dao.gsfc.nasa.gov>
%  		- initial prototype/prolog/code\end{verbatim}
 
%/////////////////////////////////////////////////////////////
 
\mbox{}\hrulefill\ 

  \subsubsection{unpermuter1\_ - unpermute a \_permuted\_ real array}

\bigskip{\sf INTERFACE:}
\begin{verbatim} 
     subroutine unpermuter1_(ary,indx,n)
       real(SP),dimension(:,:),intent(inout) :: ary
       integer ,dimension(:),intent(in)      :: indx
       integer ,             intent(in)      :: n
 \end{verbatim}%{\sf REVISION HISTORY:}
%\begin{verbatim}   	25Aug99	- Jing Guo <guo@dao.gsfc.nasa.gov>
%  		- initial prototype/prolog/code\end{verbatim}
 
%/////////////////////////////////////////////////////////////
 
\mbox{}\hrulefill\ 
 

  \subsubsection{unpermutero1\_ - unpermute a \_permuted\_ real array}

\bigskip{\sf INTERFACE:}
\begin{verbatim} 
     subroutine unpermutero1_(aout,ary,indx,n)
       real(SP),dimension(:,:),intent(inout) :: aout
       real(SP),dimension(:,:),intent(in)    :: ary
       integer ,dimension(:),intent(in)      :: indx
       integer ,             intent(in)      :: n
 \end{verbatim}%{\sf REVISION HISTORY:}
%\begin{verbatim}   	25Aug99	- Jing Guo <guo@dao.gsfc.nasa.gov>
%  		- initial prototype/prolog/code\end{verbatim}
 
%/////////////////////////////////////////////////////////////
 
\mbox{}\hrulefill\ 
 

  \subsubsection{permuted1\_ - permute a double precision array}

\bigskip{\sf INTERFACE:}
\begin{verbatim} 
     subroutine permuted1_(ary,indx,n)
       real(DP),dimension(:,:),intent(inout) :: ary
       integer ,dimension(:),intent(in)      :: indx
       integer ,             intent(in)      :: n
 \end{verbatim}%{\sf REVISION HISTORY:}
%\begin{verbatim}   	25Aug99	- Jing Guo <guo@dao.gsfc.nasa.gov>
%  		- initial prototype/prolog/code\end{verbatim}
 
%/////////////////////////////////////////////////////////////
 
\mbox{}\hrulefill\ 
 

  \subsubsection{permutedo1\_ - permute a double precision array}

\bigskip{\sf INTERFACE:}
\begin{verbatim} 
     subroutine permutedo1_(aout,ary,indx,n)
       real(DP),dimension(:,:),intent(inout) :: aout
       real(DP),dimension(:,:),intent(in)    :: ary
       integer ,dimension(:),intent(in)      :: indx
       integer ,             intent(in)      :: n
 \end{verbatim}%{\sf REVISION HISTORY:}
%\begin{verbatim}   	25Aug99	- Jing Guo <guo@dao.gsfc.nasa.gov>
%  		- initial prototype/prolog/code\end{verbatim}
 
%/////////////////////////////////////////////////////////////
 
\mbox{}\hrulefill\ 
 

  \subsubsection{unpermuted1\_ - unpermute a double precision array}

\bigskip{\sf INTERFACE:}
\begin{verbatim} 
     subroutine unpermuted1_(ary,indx,n)
       real(DP),dimension(:,:),intent(inout) :: ary
       integer ,dimension(:),intent(in)      :: indx
       integer ,             intent(in)      :: n
 \end{verbatim}%{\sf REVISION HISTORY:}
%\begin{verbatim}   	25Aug99	- Jing Guo <guo@dao.gsfc.nasa.gov>
%  		- initial prototype/prolog/code\end{verbatim}
 
%/////////////////////////////////////////////////////////////
 
\mbox{}\hrulefill\ 
 

  \subsubsection{unpermutedo1\_ - unpermute a double precision array}

\bigskip{\sf INTERFACE:}
\begin{verbatim} 
     subroutine unpermutedo1_(aout,ary,indx,n)
       real(DP),dimension(:,:),intent(inout) :: aout
       real(DP),dimension(:,:),intent(in)    :: ary
       integer ,dimension(:),intent(in)      :: indx
       integer ,             intent(in)      :: n
 \end{verbatim}%{\sf REVISION HISTORY:}
%\begin{verbatim}   	25Aug99	- Jing Guo <guo@dao.gsfc.nasa.gov>
%  		- initial prototype/prolog/code\end{verbatim}
 
%/////////////////////////////////////////////////////////////
 
\mbox{}\hrulefill\ 
 

  \subsubsection{permutel1\_ - permute a real array according to indx[]}

\bigskip{\sf INTERFACE:}
\begin{verbatim} 
     subroutine permutel1_(ary,indx,n)
       logical,dimension(:,:),intent(inout) :: ary
       integer,dimension(:),intent(in)      :: indx
       integer,             intent(in)      :: n
 \end{verbatim}%{\sf REVISION HISTORY:}
%\begin{verbatim}   	25Aug99	- Jing Guo <guo@dao.gsfc.nasa.gov>
%  		- initial prototype/prolog/code\end{verbatim}
 
%/////////////////////////////////////////////////////////////
 
\mbox{}\hrulefill\ 
 

  \subsubsection{permutelo1\_ - permute a real array according to indx[]}

\bigskip{\sf INTERFACE:}
\begin{verbatim} 
     subroutine permutelo1_(aout,ary,indx,n)
       logical,dimension(:,:),intent(inout) :: aout
       logical,dimension(:,:),intent(in)    :: ary
       integer,dimension(:),intent(in)      :: indx
       integer,             intent(in)      :: n
 \end{verbatim}%{\sf REVISION HISTORY:}
%\begin{verbatim}   	25Aug99	- Jing Guo <guo@dao.gsfc.nasa.gov>
%  		- initial prototype/prolog/code\end{verbatim}
 
%/////////////////////////////////////////////////////////////
 
\mbox{}\hrulefill\ 
 

  \subsubsection{unpermutel1\_ - unpermute a \_permuted\_ logical array}

\bigskip{\sf INTERFACE:}
\begin{verbatim} 
     subroutine unpermutel1_(ary,indx,n)
       logical,dimension(:,:),intent(inout) :: ary
       integer,dimension(:),intent(in)      :: indx
       integer,             intent(in)      :: n
 \end{verbatim}%{\sf REVISION HISTORY:}
%\begin{verbatim}   	25Aug99	- Jing Guo <guo@dao.gsfc.nasa.gov>
%  		- initial prototype/prolog/code\end{verbatim}
 
%/////////////////////////////////////////////////////////////
 
\mbox{}\hrulefill\ 
 

  \subsubsection{unpermutelo1\_ - unpermute a \_permuted\_ logical array}

\bigskip{\sf INTERFACE:}
\begin{verbatim} 
     subroutine unpermutelo1_(aout,ary,indx,n)
       logical,dimension(:,:),intent(inout) :: aout
       logical,dimension(:,:),intent(in)    :: ary
       integer,dimension(:),intent(in)      :: indx
       integer,             intent(in)      :: n
 \end{verbatim}%{\sf REVISION HISTORY:}
%\begin{verbatim}   	25Aug99	- Jing Guo <guo@dao.gsfc.nasa.gov>
%  		- initial prototype/prolog/code\end{verbatim}

%\markboth{Left}{Source File: m\_SortingTools.F90,  Date: Thu Nov 15 22:24:23 GMT 2001
%}

%/////////////////////////////////////////////////////////////
 
\mbox{}\hrulefill\ 

  \subsection{Module m\_SortingTools - A collection of different sorting tools }
%
%  	This module contains a collection of sorting utilities.  The
%     utilities are accessed through three generic interfaces, IndexSet(),
%     IndexSort(), and IndexBin().
%  
%  	Note that, a version of IndexBin() for real arguments is not
%     implemented due to the difficulty of comparing two real values as
%     being equal.  For example, a bin for real values may be specified
%     as a single number, a range of two numbers, a number with an
%     absolute error-bar, or a number with a relative error-bar.
%  
%  	In general, one may have to map both keys(:) and bins(:) to
%     integer indices by the a given rule, then use the integer version
%     of IndexBin() with the two integer index arrays to do the sorting.
%     This mapping rule, however, is application dependent.
%  
%  	Also note that, in principle, it is possible to use both 
%     IndexSort() and IndexBin() in the same sorting task.
  
\bigskip{\sf INTERFACE:}
\begin{verbatim} 
     module m_SortingTools
       public :: IndexSet	! define an initial list of indices
       public :: IndexSort	! index for a new rank out of the old
       public :: IndexBin	! index for sorting bins
       public :: RankSet		! define an initial list of ranks
       public :: RankMerge	! merge two arrays by re-ranking
       public :: IndexedRankMerge ! index-merge two array segments
\end{verbatim}
%
%   !EXAMPLES:
%  	- An example of using IndexSet()/IndexSort() in combination with
%     the convenience of the Fortran 90 array syntex can be found in the
%     prolog of m_MergeSorts.
%  	- An example of using IndexSet()/IndexBin(): Copying all "good"
%     data to another array.
%  	integer :: indx(n)
%  	call IndexSet(n,indx)
%  	call IndexBin(n,indx,allObs(:)%qcflag,GOOD,ln0=ln_GOOD)
%  		! Copy all "good" data to another array
%  	goodObs(1:ln_GOOD)=allObs( indx(1:ln_GOOD) )
%  		! Refill all "good" data back to their original places
%  	allObs( indx(1:ln_GOOD) ) = goodObs(1:ln_GOOD)
%  	- Similarily, multiple keys may be used in an IndexBin() call
%     to selectively sort the data.  The following code will move data
%     with kt = kt_Us,kt_U,kt_Vs,kt_V up to the front:
%  	call IndexBin(n,indx,allObs(:)%kt,(/kt_Us,kt_U,kt_Vs,kt_V/))
%  	allObs(1:n) = allObs( indx(1:n) )
%  	- Additional applications can also be implemented with other
%     argument combinations.\end{verbatim} 
%{\sf REVISION HISTORY:}
%\begin{verbatim}  	15Mar00	- Jing Guo
%  		. Added m_rankMerge module interface
%  	20Apr99 - Jing Guo
%  		- Commented "only" in use m_IndexBin_xxx to avoid an
%  		  apperent compiler bug on DEC/OSF1
%   	17Feb99 - Jing Guo <guo@thunder> - initial prototype/prolog/code\end{verbatim}

%\markboth{Left}{Source File: m\_StrTemplate.F90,  Date: Thu Nov 15 22:24:24 GMT 2001
%}

 
%/////////////////////////////////////////////////////////////
 
\mbox{}\hrulefill\ 
 

  \subsection{Module m\_StrTemplate - A template formatting a string with variables }

\bigskip{\sf INTERFACE:}
\begin{verbatim} 
     module m_StrTemplate
       public :: StrTemplate	! Substitute variables in a template
       interface StrTemplate
         module procedure strTemplate_
       end interface
 \end{verbatim}%{\sf REVISION HISTORY:}
%\begin{verbatim}   	01Jun99	- Jing Guo <guo@dao.gsfc.nasa.gov>
%  		- initial prototype/prolog/code\end{verbatim}
 
%/////////////////////////////////////////////////////////////
 
\mbox{}\hrulefill\ 

  \subsubsection{strTemplate\_ - expanding a format template to a string}

\bigskip{\sf INTERFACE:}
\begin{verbatim} 
     subroutine strTemplate_(str,tmpl,class,xid,nymd,nhms,stat)
       character(len=*),intent(out)          :: str   ! the output
       character(len=*),intent(in )          :: tmpl  ! a "format"
       character(len=*),intent(in ),optional :: class
                ! choose a UNIX or a GrADS(default) type format
       character(len=*),intent(in ),optional :: xid
                ! a string substituting a  "%s". Trailing spaces will be 
                ! ignored
       integer,intent(in ),optional          :: nymd
                ! yyyymmdd, substituting "%y4", "%y2", "%m1", "%m2", "%mc", 
                ! "%Mc', and "%MC"
       integer,intent(in ),optional          :: nhms
                ! hhmmss, substituting "%h1", "%h2", "%h3", and "%n2"
       integer,intent(out),optional          :: stat
                ! error code
 \end{verbatim}%{\sf REVISION HISTORY:}
%\begin{verbatim}   	03Jun99	- Jing Guo <guo@dao.gsfc.nasa.gov>
%  		- initial prototype/prolog/code
%         08Jan01 - da Silva: moved uppercase() to outside select() to
%                   avoid coredump on Linux/PGI.\end{verbatim}
 
%/////////////////////////////////////////////////////////////
 
\mbox{}\hrulefill\ 

  \subsubsection{GX\_ - evaluate a GrADS style string template}

\bigskip{\sf INTERFACE:}
\begin{verbatim} 
     subroutine GX_(str,tmpl,xid,nymd,nhms,stat)
       character(len=*),intent(out)         :: str
       character(len=*),intent(in )         :: tmpl
       character(len=*),optional,intent(in) :: xid
       integer,optional,intent(in)          :: nymd
       integer,optional,intent(in)          :: nhms
       integer,optional,intent(out)         :: stat
 \end{verbatim}%{\sf REVISION HISTORY:}
%\begin{verbatim}   	01Jun99	- Jing Guo <guo@dao.gsfc.nasa.gov>
%  		- initial prototype/prolog/code\end{verbatim}

%\markboth{Left}{Source File: m\_String.F90,  Date: Thu Nov 15 22:24:24 GMT 2001
%}
 
%/////////////////////////////////////////////////////////////
 
\mbox{}\hrulefill\ 
 
  \subsection{Module m\_String - string pointers }

\bigskip{\sf INTERFACE:}
\begin{verbatim} 
     module m_String
       public :: String		! The class data structure
       public :: toChar		! convert to a CHARACTER(*)
       public :: char		! convert to a CHARACTER(*)
       public :: String_init
       public :: init		! set a CHARACTER(*) type to a String
       public :: String_clean
       public :: clean		! clean a String
       public :: String_len
       public :: len		! length of a String 
       public :: String_bcast
       public :: bcast		! Broadcast a String 
       public :: String_mci
       public :: String_mco
       public :: ptr_chars
     type String
       character(len=1),dimension(:),pointer :: c
     end type String
 \end{verbatim}%{\sf REVISION HISTORY:}
%\begin{verbatim}   	22Apr98 - Jing Guo <guo@thunder> - initial prototype/prolog/code\end{verbatim}
 
%/////////////////////////////////////////////////////////////
 
\mbox{}\hrulefill\ 
 
  \subsubsection{str2ch0\_ - convert a String to a variable of characters}

\bigskip{\sf INTERFACE:}
\begin{verbatim} 
     function str2ch0_(str)
       type(String),intent(in)    :: str
       character(len=size(str%c)) :: str2ch0_
 \end{verbatim}%{\sf REVISION HISTORY:}
%\begin{verbatim}   	23Apr98 - Jing Guo <guo@thunder> - initial prototype/prolog/code\end{verbatim}
 
%/////////////////////////////////////////////////////////////
 
\mbox{}\hrulefill\ 
 
  \subsubsection{ch12ch0\_ - convert a rank-1 char-array to a rank-0 char(*)}

\bigskip{\sf INTERFACE:}
\begin{verbatim} 
     function ch12ch0_(ch1)
       character(len=1),dimension(:),intent(in) :: ch1
       character(len=size(ch1)) :: ch12ch0_
 \end{verbatim}%{\sf REVISION HISTORY:}
%\begin{verbatim}   	22Apr98 - Jing Guo <guo@thunder> - initial prototype/prolog/code\end{verbatim}
 
%/////////////////////////////////////////////////////////////
 
\mbox{}\hrulefill\ 

  \subsubsection{initc\_ - convert a character object to a String object}

\bigskip{\sf INTERFACE:}
\begin{verbatim} 
     subroutine initc_(str,chr)
       type(String),intent(out)    :: str
       character(len=*),intent(in) :: chr
 \end{verbatim}%{\sf REVISION HISTORY:}
%\begin{verbatim}   	23Apr98 - Jing Guo <guo@thunder> - initial prototype/prolog/code\end{verbatim}
 
%/////////////////////////////////////////////////////////////
 
\mbox{}\hrulefill\ 
 

  \subsubsection{inits\_ - convert a String object to a String}

\bigskip{\sf INTERFACE:}
\begin{verbatim} 
     subroutine inits_(oStr,iStr)
       type(String),intent(out) :: oStr
       type(String),intent(in ) :: iStr
 \end{verbatim}%{\sf REVISION HISTORY:}
%\begin{verbatim}   	07Feb00	- Jing Guo <guo@dao.gsfc.nasa.gov>
%  		- initial prototype/prolog/code\end{verbatim}
 
%/////////////////////////////////////////////////////////////
 
\mbox{}\hrulefill\ 
 
  \subsubsection{clean\_ - clean a defined String object}

\bigskip{\sf INTERFACE:}
\begin{verbatim} 
     subroutine clean_(str)
       type(String),intent(inout) :: str
 \end{verbatim}%{\sf REVISION HISTORY:}
%\begin{verbatim}   	23Apr98 - Jing Guo <guo@thunder> - initial prototype/prolog/code\end{verbatim}
 
%/////////////////////////////////////////////////////////////
 
\mbox{}\hrulefill\ 
 
  \subsubsection{bcast\_ - broadcast a rank-0 String}

\bigskip{\sf INTERFACE:}
\begin{verbatim} 
     subroutine bcast_(Str,root,comm,stat)
       type(String)                 :: Str   ! (IN) on the root, (OUT) elsewhere
       integer,intent(in)           :: root
       integer,intent(in)           :: comm
       integer,optional,intent(out) :: stat
 \end{verbatim}%{\sf REVISION HISTORY:}
%\begin{verbatim}   	27Apr98 - Jing Guo <guo@thunder> - initial prototype/prolog/code\end{verbatim}
 
%/////////////////////////////////////////////////////////////
 
\mbox{}\hrulefill\ 
 
  \subsubsection{mci0\_ - checking in a String scalar}

\bigskip{\sf INTERFACE:}
\begin{verbatim} 
     subroutine mci0_(marg,thread)
       type(String),    intent(in) :: marg
       character(len=*),intent(in) :: thread
 \end{verbatim}%{\sf REVISION HISTORY:}
%\begin{verbatim}   	07Feb00	- Jing Guo <guo@dao.gsfc.nasa.gov>
%  		- initial prototype/prolog/code\end{verbatim}
 
%/////////////////////////////////////////////////////////////
 
\mbox{}\hrulefill\ 

  \subsubsection{mco0\_ - checking out a String scalar}

\bigskip{\sf INTERFACE:}
\begin{verbatim} 
     subroutine mco0_(marg,thread)
       type(String),    intent(in) :: marg
       character(len=*),intent(in) :: thread
 \end{verbatim}%{\sf REVISION HISTORY:}
%\begin{verbatim}   	07Feb00	- Jing Guo <guo@dao.gsfc.nasa.gov>
%  		- initial prototype/prolog/code\end{verbatim}
 
%/////////////////////////////////////////////////////////////
 
\mbox{}\hrulefill\ 
 
  \subsubsection{mci1\_ - checking in a String scalar}

\bigskip{\sf INTERFACE:}
\begin{verbatim} 
     subroutine mci1_(marg,thread)
       type(String),dimension(:),intent(in) :: marg
       character(len=*)         ,intent(in) :: thread
 \end{verbatim}%{\sf REVISION HISTORY:}
%\begin{verbatim}   	07Feb00	- Jing Guo <guo@dao.gsfc.nasa.gov>
%  		- initial prototype/prolog/code\end{verbatim}
 
%/////////////////////////////////////////////////////////////
 
\mbox{}\hrulefill\ 

  \subsubsection{mco1\_ - checking out a String scalar}

\bigskip{\sf INTERFACE:}
\begin{verbatim} 
     subroutine mco1_(marg,thread)
       type(String),dimension(:),intent(in) :: marg
       character(len=*)         ,intent(in) :: thread
 \end{verbatim}%{\sf REVISION HISTORY:}
%\begin{verbatim}   	07Feb00	- Jing Guo <guo@dao.gsfc.nasa.gov>
%  		- initial prototype/prolog/code\end{verbatim}
 
%/////////////////////////////////////////////////////////////
 
\mbox{}\hrulefill\ 
 
  \subsubsection{mci2\_ - checking in a String scalar}

\bigskip{\sf INTERFACE:}
\begin{verbatim} 
     subroutine mci2_(marg,thread)
       type(String),dimension(:,:),intent(in) :: marg
       character(len=*)           ,intent(in) :: thread
 \end{verbatim}%{\sf REVISION HISTORY:}
%\begin{verbatim}   	07Feb00	- Jing Guo <guo@dao.gsfc.nasa.gov>
%  		- initial prototype/prolog/code\end{verbatim}
 
%/////////////////////////////////////////////////////////////
 
\mbox{}\hrulefill\ 
 
  \subsubsection{mco2\_ - checking out a String scalar}

\bigskip{\sf INTERFACE:}
\begin{verbatim} 
     subroutine mco2_(marg,thread)
       type(String),dimension(:,:),intent(in) :: marg
       character(len=*)           ,intent(in) :: thread
 \end{verbatim}%{\sf REVISION HISTORY:}
%\begin{verbatim}   	07Feb00	- Jing Guo <guo@dao.gsfc.nasa.gov>
%  		- initial prototype/prolog/code\end{verbatim}
 
%/////////////////////////////////////////////////////////////
 
\mbox{}\hrulefill\ 

  \subsubsection{mci3\_ - checking in a String scalar}

\bigskip{\sf INTERFACE:}
\begin{verbatim} 
     subroutine mci3_(marg,thread)
       type(String),dimension(:,:,:),intent(in) :: marg
       character(len=*)             ,intent(in) :: thread
 \end{verbatim}%{\sf REVISION HISTORY:}
%\begin{verbatim}   	07Feb00	- Jing Guo <guo@dao.gsfc.nasa.gov>
%  		- initial prototype/prolog/code\end{verbatim}
 
%/////////////////////////////////////////////////////////////
 
\mbox{}\hrulefill\ 
 
  \subsubsection{mco3\_ - checking out a String scalar}

\bigskip{\sf INTERFACE:}
\begin{verbatim} 
     subroutine mco3_(marg,thread)
       type(String),dimension(:,:,:),intent(in) :: marg
       character(len=*)             ,intent(in) :: thread
 \end{verbatim}%{\sf REVISION HISTORY:}
%\begin{verbatim}   	07Feb00	- Jing Guo <guo@dao.gsfc.nasa.gov>
%  		- initial prototype/prolog/code\end{verbatim}
 
%/////////////////////////////////////////////////////////////
 
\mbox{}\hrulefill\ 
 
  \subsubsection{len\_ = len of a String}

\bigskip{\sf INTERFACE:}
\begin{verbatim} 
     function len_(str)
       type(String),intent(in) :: str
       integer :: len_
 \end{verbatim}%{\sf REVISION HISTORY:}
%\begin{verbatim}   	10Apr00	- Jing Guo <guo@dao.gsfc.nasa.gov>
%  		- initial prototype/prolog/code\end{verbatim}
 
%/////////////////////////////////////////////////////////////
 
\mbox{}\hrulefill\ 
 
  \subsubsection{ptr\_chars\_ - direct}

\bigskip{\sf INTERFACE:}
\begin{verbatim} 
     function ptr_chars_(str)
       type(String),intent(in)               :: str
       character(len=1),pointer,dimension(:) :: ptr_chars_
 \end{verbatim}%{\sf REVISION HISTORY:}
%\begin{verbatim}   	10Apr00	- Jing Guo <guo@dao.gsfc.nasa.gov>
%  		- initial prototype/prolog/code\end{verbatim}

%\markboth{Left}{Source File: m\_StringLinkedList.F90,  Date: Thu Nov 15 22:24:24 GMT 2001
%}

 
%/////////////////////////////////////////////////////////////
 
\mbox{}\hrulefill\ 
 
  \subsection{Module m\_StringLinkedList - A linked-list of String }

\bigskip{\sf INTERFACE:}
\begin{verbatim} 
     module m_StringLinkedList
       public :: StringLinkedList	! The class data structure
 
 		! o An object of a StringLinkedList should be defined
 		!   as a pointer of a StringLinkedList.  It is often
 		!   represented by a pointer to the head-node of the
 		!   linked-list.
 		! o A node in a StringLinkedList is specificed by a
 		!   reference pointer.  A reference pointer is a
 		!   logical reference of a node in the list.  However,
 		!   it does not physically point to that node.  In
 		!   fact, a reference pointer normally references to
 		!   the node physically pointed by the pointer in the
 		!   node physically pointed by the reference pointer,
 		!	[this] -> [..|next] -> [..|next]
 		!   where the last node is the logically referenced
 		!   node.
 
       public :: StringLinkedList_init	! constructor
       public :: StringLinkedList_clean  ! destructor
 
 		! A _clean() action will reset a StringLinkedList to its
 		! pre-_init() status.
 
       public :: StringLinkedList_insert ! grower, insert a node
       public :: StringLinkedList_delete ! ungrower, delete a node
 
 		! Both procedures processing the node through a given
 		! reference pointer.  The reference pointer will not
 		! be modified directly through either _insert() or
 		! _delete().  It is the pointer in the node physically
 		! pointed by a reference pointer got modified.  Also,
 		! the node logically referenced by the reference
 		! pointer is either the new node for an _insert(), and
 		! the removed node for a _delete().
 
       public :: StringLinkedList_eol	! inquirer, is an end-node?
 
 		! An end-of-list situation occurs when the reference
 		! pointer is logically referencing to the end-node or
 		! beyond.  Note that an end-node links to itself.
 
       public :: StringLinkedList_next	! iterator, go to the next node.
 
       public :: StringLinkedList_count	! counter
       
 		! Count the number of nodes from this reference pointer,
 		! starting from and including the logical node but
 		! excluding the end-node.
 
       public :: StringLinkedList_get	! fetcher
 
 		! Get the value logically referenced by a reference
 		! pointer.  Return EOL if the referenced node is an
 		! EOL().  The reference pointer will be iterated to
 		! the next node if the referenced node is not an EOL.
 
     type StringLinkedList
       type(String)                   :: str
       type(StringLinkedList),pointer :: next
     end type StringLinkedList
 
     interface StringLinkedList_init  ; module procedure	&
        init_
     end interface
 
     interface StringLinkedList_clean ; module procedure	&
        clean_
     end interface
 
     interface StringLinkedList_insert; module procedure	&
        insertc_,      &        ! insert a CHARACTER(len=*) argument
        inserts_                ! insert a String argument
     end interface
 
     interface StringLinkedList_delete; module procedure	&
        delete_
     end interface
 
     interface StringLinkedList_eol   ; module procedure	&
        eol_
     end interface
 
     interface StringLinkedList_next  ; module procedure	&
        next_
     end interface
 
     interface StringLinkedList_count ; module procedure	&
        count_
     end interface
 
     interface StringLinkedList_get   ; module procedure	&
        getc_,          &       ! get as a CHARACTER(len=*)
        gets_                   ! get as a String
     end interface
 \end{verbatim}%{\sf REVISION HISTORY:}
%\begin{verbatim}   	16Feb00	- Jing Guo <guo@dao.gsfc.nasa.gov>
%  		- initial prototype/prolog/code\end{verbatim}
 
%/////////////////////////////////////////////////////////////
 
\mbox{}\hrulefill\ 
 

  \subsubsection{init\_ - initialize a StringLinkedList from a pointer}

\bigskip{\sf INTERFACE:}
\begin{verbatim} 
     subroutine init_(head)
       type(StringLinkedList),pointer :: head	! (out) a list
 \end{verbatim}%{\sf REVISION HISTORY:}
%\begin{verbatim}   	22Feb00	- Jing Guo <guo@dao.gsfc.nasa.gov>
%  		- initial prototype/prolog/code\end{verbatim}
 
%/////////////////////////////////////////////////////////////
 
\mbox{}\hrulefill\ 
 
  \subsubsection{insertc\_ - insert before the logically referenced node}

\bigskip{\sf INTERFACE:}
\begin{verbatim} 
     subroutine insertc_(cstr,this)
       character(len=*),intent(in)    :: cstr ! a new entry
       type(StringLinkedList),pointer :: this ! (in) a node
 \end{verbatim}%{\sf REVISION HISTORY:}
%\begin{verbatim}   	16Feb00	- Jing Guo <guo@dao.gsfc.nasa.gov>
%  		- initial prototype/prolog/code\end{verbatim}
 
%/////////////////////////////////////////////////////////////
 
\mbox{}\hrulefill\ 
 
  \subsubsection{inserts\_ - insert before the logically referenced node}

\bigskip{\sf INTERFACE:}
\begin{verbatim} 
     subroutine inserts_(str,this)
       type(String),intent(in)        :: str  ! a new entry
       type(StringLinkedList),pointer :: this ! (in) a node
 \end{verbatim}%{\sf REVISION HISTORY:}
%\begin{verbatim}   	16Feb00	- Jing Guo <guo@dao.gsfc.nasa.gov>
%  		- initial prototype/prolog/code\end{verbatim}
 
%/////////////////////////////////////////////////////////////
 
\mbox{}\hrulefill\ 

  \subsubsection{delete\_ - delete the logically referenced node}

\bigskip{\sf INTERFACE:}
\begin{verbatim} 
     subroutine delete_(this)
       type(StringLinkedList),pointer :: this ! (in) a node
 \end{verbatim}%{\sf REVISION HISTORY:}
%\begin{verbatim}   	17Feb00	- Jing Guo <guo@dao.gsfc.nasa.gov>
%  		- initial prototype/prolog/code\end{verbatim}
 
%/////////////////////////////////////////////////////////////
 
\mbox{}\hrulefill\ 
 
  \subsubsection{eol\_ - if the logically referenced node is an end-node}

\bigskip{\sf INTERFACE:}
\begin{verbatim} 
     function eol_(this)
       type(StringLinkedList),pointer :: this ! (in) a node
       logical                        :: eol_ ! returned value
 \end{verbatim}%{\sf REVISION HISTORY:}
%\begin{verbatim}   	23Feb00	- Jing Guo <guo@dao.gsfc.nasa.gov>
%  		- initial prototype/prolog/code\end{verbatim}
 
%/////////////////////////////////////////////////////////////
 
\mbox{}\hrulefill\ 
 
  \subsubsection{next\_ - point a reference pointer to the next node}

\bigskip{\sf INTERFACE:}
\begin{verbatim} 
     subroutine next_(this)
       type(StringLinkedList),pointer :: this ! (inout) a node
 \end{verbatim}%{\sf REVISION HISTORY:}
%\begin{verbatim}   	23Feb00	- Jing Guo <guo@dao.gsfc.nasa.gov>
%  		- initial prototype/prolog/code\end{verbatim}
 
%/////////////////////////////////////////////////////////////
 
\mbox{}\hrulefill\ 
 

  \subsubsection{count\_ - count the number of nodes}

\bigskip{\sf INTERFACE:}
\begin{verbatim} 
     function count_(this)
       type(StringLinkedList),pointer :: this ! (in) a node
       integer :: count_		! returned value
 \end{verbatim}%{\sf REVISION HISTORY:}
%\begin{verbatim}   	24Feb00	- Jing Guo <guo@dao.gsfc.nasa.gov>
%  		- initial prototype/prolog/code\end{verbatim}
 
%/////////////////////////////////////////////////////////////
 
\mbox{}\hrulefill\ 
 

  \subsubsection{getc\_ - get the logically referenced value as CHARACTERs}

\bigskip{\sf INTERFACE:}
\begin{verbatim} 
     subroutine getc_(this,cstr,eol)
       type(StringLinkedList),pointer :: this ! (inout) a node
       character(len=*),intent(out)   :: cstr ! the referenced value
       logical         ,intent(out)   :: eol  ! if the node is an end-node
 \end{verbatim}%{\sf REVISION HISTORY:}
%\begin{verbatim}   	17Feb00	- Jing Guo <guo@dao.gsfc.nasa.gov>
%  		- initial prototype/prolog/code\end{verbatim}
 
%/////////////////////////////////////////////////////////////
 
\mbox{}\hrulefill\ 
 

  \subsubsection{gets\_ - get the logically referenced value as a String}

\bigskip{\sf INTERFACE:}
\begin{verbatim} 
     subroutine gets_(this,str,eol)
       type(StringLinkedList),pointer :: this ! (inout) a node
       type(String),intent(out)       :: str  ! the referenced value
       logical     ,intent(out)       :: eol  ! if the node is an end-node
 \end{verbatim}%{\sf REVISION HISTORY:}
%\begin{verbatim}   	17Feb00	- Jing Guo <guo@dao.gsfc.nasa.gov>
%  		- initial prototype/prolog/code\end{verbatim}
 
%/////////////////////////////////////////////////////////////
 
\mbox{}\hrulefill\ 
 

  \subsubsection{clean\_ - clean the whole object from this point}

\bigskip{\sf INTERFACE:}
\begin{verbatim} 
     subroutine clean_(head,stat)
       type(StringLinkedList),pointer :: head ! (inout) a head-node
       integer,optional,intent(out)   :: stat ! return status
 \end{verbatim}%{\sf REVISION HISTORY:}
%\begin{verbatim}   	17Feb00	- Jing Guo <guo@dao.gsfc.nasa.gov>
%  		- initial prototype/prolog/code\end{verbatim}

%\markboth{Left}{Source File: m\_chars.F90,  Date: Thu Nov 15 22:24:24 GMT 2001
%}

 
%/////////////////////////////////////////////////////////////
 
\mbox{}\hrulefill\ 
 

  \subsection{Module m\_chars - a module for character class object operations }

\bigskip{\sf INTERFACE:}
\begin{verbatim} 
      module m_chars
        public	:: operator (.upper.)	! convert a string to uppercase
        public	:: uppercase
 
        public	:: operator (.lower.)	! convert a string to lowercase
        public	:: lowercase
 
        interface operator (.upper.)
          module procedure upper_case
        end interface
        interface uppercase
          module procedure upper_case
        end interface
 
        interface operator (.lower.)
          module procedure lower_case
        end interface
        interface lowercase
          module procedure lower_case
        end interface
 \end{verbatim}%{\sf REVISION HISTORY:}
%\begin{verbatim}   	16Jul96 - J. Guo	- (to do)\end{verbatim}
 
%/////////////////////////////////////////////////////////////
 
\mbox{}\hrulefill\ 
 

  \subsubsection{upper\_case - convert lowercase letters to uppercase.}

\bigskip{\sf INTERFACE:}
\begin{verbatim} 
   function upper_case(str) result(ustr)
     character(len=*), intent(in) :: str
     character(len=len(str))      :: ustr
 \end{verbatim}%{\sf REVISION HISTORY:}
%\begin{verbatim}   	13Aug96 - J. Guo	- (to do)\end{verbatim}
 
%/////////////////////////////////////////////////////////////
 
\mbox{}\hrulefill\ 
 
  \subsubsection{lower\_case - convert uppercase letters to lowercase.}

\bigskip{\sf INTERFACE:}
\begin{verbatim} 
   function lower_case(str) result(lstr)
     character(len=*), intent(in) :: str
     character(len=len(str))      :: lstr
 \end{verbatim}%{\sf REVISION HISTORY:}
%\begin{verbatim}   	13Aug96 - J. Guo	- (to do)\end{verbatim}
%
%\markboth{Left}{Source File: m\_die.F90,  Date: Thu Nov 15 22:24:24 GMT 2001
%}

%/////////////////////////////////////////////////////////////
 
\mbox{}\hrulefill\ 
 
  \subsection{Module m\_die - die with mpout flushed }

\bigskip{\sf INTERFACE:}
\begin{verbatim} 
     module m_die
       public :: die		! signal an exception
       public :: diex		! a special die() supporting macros
       public :: perr,warn	! message(s) to stderr
       public :: perr_die	! to be phased out
       public :: MP_die		! a special die() for MPI errors
       public :: MP_perr		! perr for MPI errors, from m_mpif90
       public :: MP_perr_die	! a special die() for MPI errors
       public :: assert_		! used by ASSERT() macro of assert.H
 
       interface die
         module procedure
           die0_,	& ! die(where)
           die1_,	& ! die(where,message)
           die2_,	& ! die(where,proc,ier)
           die4_	  ! die(where,mesg1,ival1,mesg2,ival2)
       end interface
 
       interface diex
         module procedure
           diex_	  ! diex(where,filename,lineno)
       end interface
 
       interface perr
         module procedure
           perr1_,	 ! perr(where,message)
           perr2_,	 ! perr(where,proc,ier)
           perr4_	 ! perr(where,mesg1,ival1,mesg2,ival2)
       end interface
       interface warn
         module procedure	
           perr1_,	 ! perr(where,message)
           perr2_,	 ! perr(where,proc,ier)
           perr4_	 ! perr(where,mesg1,ival1,mesg2,ival2)
       end interface
 
       interface perr_die
         module procedure
           die2_	  ! perr_die(where,proc,ier)
       end interface
 
       interface MP_die
         module procedure
           MPdie2_	  ! MP_die(where,proc,ier)
       end interface
       interface MP_perr_die
         module procedure
           MPdie2_	  ! MP_die(where,proc,ier)
       end interface
 
       interface assert_
         module procedure
           assertmsg_,	
           assertnomsg_
       end interface
 \end{verbatim}%{\sf REVISION HISTORY:}
%\begin{verbatim}   	26Feb98 - Jing Guo <guo@thunder> - initial prototype/prolog/code\end{verbatim}
 
%/////////////////////////////////////////////////////////////
 
\mbox{}\hrulefill\ 

  \subsubsection{die0\_ - flush(mpout) before die()}

\bigskip{\sf INTERFACE:}
\begin{verbatim} 
     subroutine die0_(where)
       character(len=*),intent(in) :: where
 \end{verbatim}%{\sf REVISION HISTORY:}
%\begin{verbatim}   	26Feb98 - Jing Guo <guo@thunder> - initial prototype/prolog/code\end{verbatim}
 
%/////////////////////////////////////////////////////////////
 
\mbox{}\hrulefill\ 

  \subsubsection{die1\_ - flush(mpout) before die()}

\bigskip{\sf INTERFACE:}
\begin{verbatim} 
     subroutine die1_(where,message)
       character(len=*),intent(in) :: where
       character(len=*),intent(in) :: message
 \end{verbatim}%{\sf REVISION HISTORY:}
%\begin{verbatim}   	26Feb98 - Jing Guo <guo@thunder> - initial prototype/prolog/code\end{verbatim}
 
%/////////////////////////////////////////////////////////////
 
\mbox{}\hrulefill\  

  \subsubsection{die2\_ - flush(mpout) before die()}

\bigskip{\sf INTERFACE:}
\begin{verbatim} 
     subroutine die2_(where,proc,ier)
       character(len=*),intent(in) :: where
       character(len=*),intent(in) :: proc
       integer,intent(in)          :: ier
 \end{verbatim}%{\sf REVISION HISTORY:}
%\begin{verbatim}   	26Feb98 - Jing Guo <guo@thunder> - initial prototype/prolog/code\end{verbatim}
 
%/////////////////////////////////////////////////////////////
 
\mbox{}\hrulefill\ 

  \subsubsection{die4\_ - flush(mpout) before die()}

\bigskip{\sf INTERFACE:}
\begin{verbatim} 
     subroutine die4_(where,mesg1,ival1,mesg2,ival2)
       character(len=*),intent(in) :: where
       character(len=*),intent(in) :: mesg1
       integer,intent(in)          :: ival1
       character(len=*),intent(in) :: mesg2
       integer,intent(in)          :: ival2
 \end{verbatim}%{\sf REVISION HISTORY:}
%\begin{verbatim}   	26Feb98 - Jing Guo <guo@thunder> - initial prototype/prolog/code\end{verbatim}
 
%/////////////////////////////////////////////////////////////
 
\mbox{}\hrulefill\ 

  \subsubsection{diex\_ - flush(mpout) before die()}

\bigskip{\sf INTERFACE:}
\begin{verbatim} 
     subroutine diex_(where,filename,line)
       character(len=*),intent(in) :: where
       character(len=*),intent(in) :: filename
       integer,intent(in)          :: line
 \end{verbatim}%{\sf REVISION HISTORY:}
%\begin{verbatim}   	26Feb98 - Jing Guo <guo@thunder> - initial prototype/prolog/code\end{verbatim}
 
%/////////////////////////////////////////////////////////////
 
\mbox{}\hrulefill\ 
 
  \subsubsection{perr1\_ - send a simple error message to \_stderr\_}

\bigskip{\sf INTERFACE:}
\begin{verbatim} 
     subroutine perr1_(where,message)
       character(len=*),intent(in) :: where
       character(len=*),intent(in) :: message
 \end{verbatim}%{\sf REVISION HISTORY:}
%\begin{verbatim}   	27Apr98 - Jing Guo <guo@thunder> - initial prototype/prolog/code\end{verbatim}
 
%/////////////////////////////////////////////////////////////
 
\mbox{}\hrulefill\ 

  \subsubsection{perr2\_ - send a simple error message to \_stderr\_}

\bigskip{\sf INTERFACE:}
\begin{verbatim} 
     subroutine perr2_(where,proc,ier)
       character(len=*),intent(in) :: where
       character(len=*),intent(in) :: proc
       integer,intent(in)          :: ier
 \end{verbatim}%{\sf REVISION HISTORY:}
%\begin{verbatim}   	27Apr98 - Jing Guo <guo@thunder> - initial prototype/prolog/code\end{verbatim}
 
%/////////////////////////////////////////////////////////////
 
\mbox{}\hrulefill\ 
 
  \subsubsection{perr4\_ - send a simple error message to \_stderr\_}

\bigskip{\sf INTERFACE:}
\begin{verbatim} 
     subroutine perr4_(where,mesg1,ival1,mesg2,ival2)
       character(len=*),intent(in) :: where
       character(len=*),intent(in) :: mesg1
       integer,intent(in)          :: ival1
       character(len=*),intent(in) :: mesg2
       integer,intent(in)          :: ival2
 \end{verbatim}%{\sf REVISION HISTORY:}
%\begin{verbatim}   	27Apr98 - Jing Guo <guo@thunder> - initial prototype/prolog/code\end{verbatim}
 
%/////////////////////////////////////////////////////////////
 
\mbox{}\hrulefill\ 

  \subsubsection{MPdie2\_ - invoke MP\_perr before die\_}

\bigskip{\sf INTERFACE:}
\begin{verbatim} 
     subroutine MPdie2_(where,proc,ier)
       character(len=*),intent(in) :: where
       character(len=*),intent(in) :: proc
       integer,intent(in)          :: ier
 \end{verbatim}%{\sf REVISION HISTORY:}
%\begin{verbatim}   	27Apr98 - Jing Guo <guo@thunder> - initial prototype/prolog/code\end{verbatim}
 
%/////////////////////////////////////////////////////////////
 
\mbox{}\hrulefill\ 
 
  \subsubsection{assertmsg\_ - an utility called by ASSERT() macro only}

\bigskip{\sf INTERFACE:}
\begin{verbatim} 
     subroutine assertmsg_(str, file, line)
       Character(Len=*), Intent(In) :: str	! a message
       Character(Len=*), Intent(In) :: file	! a filename
       Integer, Intent(In)          :: line	! a line number
 \end{verbatim}%{\sf REVISION HISTORY:}
%\begin{verbatim}   	25Aug00	- Jing Guo <guo@dao.gsfc.nasa.gov>
%  		- modified
%  		- included into m_die for easier module management
%  	before	- Tom Clune
%  		- Created for MPI PSAS implementation as a separate
%  		  module\end{verbatim}
 
%/////////////////////////////////////////////////////////////
 
\mbox{}\hrulefill\ 
 
  \subsubsection{assertnomsg\_ - an utility called by ASSERT() macro only}

\bigskip{\sf INTERFACE:}
\begin{verbatim} 
     subroutine assertnomsg_(file, line)
       Character(Len=*), Intent(In) :: file	! a filename
       Integer, Intent(In)          :: line	! a line number
 \end{verbatim}%{\sf REVISION HISTORY:}
%\begin{verbatim}   	25Aug00	- Jing Guo <guo@dao.gsfc.nasa.gov>
%  		- modified
%  		- included into m_die for easier module management
%  	before	- Tom Clune
%  		- Created for MPI PSAS implementation as a separate
%  		  module\end{verbatim}

%\markboth{Left}{Source File: m\_dropdead.F90,  Date: Thu Nov 15 22:24:25 GMT 2001
%}
 
%/////////////////////////////////////////////////////////////
 
\mbox{}\hrulefill\ 

  \subsection{Module m\_dropdead - An abort() with a style }

\bigskip{\sf INTERFACE:}
\begin{verbatim} 
     module m_dropdead
       public	:: die	! terminate a program with a condition
 
       interface die
         module procedure
           die_,	
           diex_
       end interface
 \end{verbatim}%{\sf REVISION HISTORY:}
%\begin{verbatim}   	20Feb97 - Jing Guo <guo@eramus> - defined template\end{verbatim}
 
%/////////////////////////////////////////////////////////////
 
\mbox{}\hrulefill\ 
 
\subsubsection{die\_ - Clean up and raise an exception to the OS}

     A call to die() exits the program with minimum information for
     both the user and the operating system.
  
\bigskip{\sf INTERFACE:}
\begin{verbatim} 
     subroutine die_(where)
       character(len=*),intent(in) :: where	! where it is called
 \end{verbatim}%{\sf REVISION HISTORY:}
%\begin{verbatim}   	20Feb97 - Jing Guo <guo@eramus> - defined template\end{verbatim}
 
%/////////////////////////////////////////////////////////////
 
\mbox{}\hrulefill\ 
 
  \subsubsection{diex\_ - Clean up and raise an exception to the OS}

     A call to die() exits the program with minimum information for
     both the user and the operating system.  This implementation,
     however, may be used in conjunction with with a source preprocessor
     to produce more detailed location information.
  
\bigskip{\sf INTERFACE:}
\begin{verbatim} 
     subroutine diex_(where,fnam,line)
       character(len=*),intent(in) :: where	! where it is called
       character(len=*),intent(in) :: fnam
       integer,intent(in)          :: line
 \end{verbatim}%{\sf REVISION HISTORY:}
%\begin{verbatim}   	20Feb97 - Jing Guo <guo@eramus> - defined template\end{verbatim}

%\markboth{Left}{Source File: m\_flow.F90,  Date: Thu Nov 15 22:24:25 GMT 2001
%}
 
%/////////////////////////////////////////////////////////////
 
\mbox{}\hrulefill\ 
 
  \subsection{Module m\_flow - tracing the program calling tree }

\bigskip{\sf INTERFACE:}
\begin{verbatim} 
     module m_flow
       public :: flow_ci
       public :: flow_co
       public :: flow_flush
       public :: flow_reset
 
       interface flow_ci;    module procedure ci_;    end interface
       interface flow_co;    module procedure co_;    end interface
       interface flow_flush; module procedure flush_; end interface
       interface flow_reset; module procedure reset_; end interface
 \end{verbatim}%{\sf REVISION HISTORY:}
%\begin{verbatim}   	26Feb98 - Jing Guo <guo@thunder> - initial prototype/prolog/code\end{verbatim}
 
%/////////////////////////////////////////////////////////////
 
\mbox{}\hrulefill\ 
 
  \subsubsection{ci\_ - checking in a level}

\bigskip{\sf INTERFACE:}
\begin{verbatim} 
     subroutine ci_(name)
       character(len=*),intent(in) :: name
 \end{verbatim}%{\sf REVISION HISTORY:}
%\begin{verbatim}   	26Feb98 - Jing Guo <guo@thunder> - initial prototype/prolog/code\end{verbatim}
 
%/////////////////////////////////////////////////////////////
 
\mbox{}\hrulefill\ 

  \subsubsection{co\_ - checking out a level}

\bigskip{\sf INTERFACE:}
\begin{verbatim} 
     subroutine co_(name)
       character(len=*),intent(in) :: name
 \end{verbatim}%{\sf REVISION HISTORY:}
%\begin{verbatim}   	26Feb98 - Jing Guo <guo@thunder> - initial prototype/prolog/code\end{verbatim}
 
%/////////////////////////////////////////////////////////////
 
\mbox{}\hrulefill\ 
 
  \subsubsection{flush\_ - print all remaining entries in the list}

\bigskip{\sf INTERFACE:}
\begin{verbatim} 
     subroutine flush_(lu)
       integer,intent(in) :: lu
 \end{verbatim}%{\sf REVISION HISTORY:}
%\begin{verbatim}   	26Feb98 - Jing Guo <guo@thunder> - initial prototype/prolog/code\end{verbatim}
 
%/////////////////////////////////////////////////////////////
 
\mbox{}\hrulefill\ 
 
\subsubsection{reset\_ - set the stack to empty}

\bigskip{\sf INTERFACE:}
\begin{verbatim} 
     subroutine reset_()
 \end{verbatim}%{\sf REVISION HISTORY:}
%\begin{verbatim}   	26Feb98 - Jing Guo <guo@thunder> - initial prototype/prolog/code\end{verbatim}

%\markboth{Left}{Source File: m\_inpak90.F90,  Date: Thu Nov 15 22:24:25 GMT 2001
%}  

%/////////////////////////////////////////////////////////////

\mbox{}\hrulefill\

\subsection{Module m\_inpack90}
 
  \subsubsection{I90\_allLoadF - populate a rooted database to all PEs}

\bigskip{\sf INTERFACE:}
\begin{verbatim} 
     subroutine I90_allLoadF(fname,root,comm,istat)
       character(len=*),intent(in) :: fname
       integer,intent(in)          :: root
       integer,intent(in)          :: comm
       integer,intent(out)         :: istat
 \end{verbatim}%{\sf REVISION HISTORY:}
%\begin{verbatim}   	28Jul98 - Jing Guo <guo@thunder> - initial prototype/prolog/code\end{verbatim}
 
%/////////////////////////////////////////////////////////////
 
\mbox{}\hrulefill\ 
 
  \subsubsection{push\_ - push on a new layer of the internal file \_i90\_now\_}

\bigskip{\sf INTERFACE:}
\begin{verbatim} 
     subroutine push_(ier)
       integer,intent(out) :: ier
 \end{verbatim}%{\sf REVISION HISTORY:}
%\begin{verbatim}   	05Aug98 - Jing Guo <guo@thunder> - initial prototype/prolog/code\end{verbatim}
 
%/////////////////////////////////////////////////////////////
 
\mbox{}\hrulefill\ 
 
  \subsubsection{pop\_ - pop off a layer of the internal file \_i90\_now\_}

\bigskip{\sf INTERFACE:}
\begin{verbatim} 
     subroutine pop_(ier)
       integer,intent(out) :: ier
 \end{verbatim}%{\sf REVISION HISTORY:}
%\begin{verbatim}   	05Aug98 - Jing Guo <guo@thunder> - initial prototype/prolog/code\end{verbatim}
 
%/////////////////////////////////////////////////////////////
 
\mbox{}\hrulefill\ 

  \subsubsection{i90\_fullRelease - releases the whole stack led by \_i90\_now\_}

\bigskip{\sf INTERFACE:}
\begin{verbatim} 
     subroutine i90_fullRelease(ier)
       integer,intent(out) :: ier
 \end{verbatim}%{\sf REVISION HISTORY:}
%\begin{verbatim}   	05Aug98 - Jing Guo <guo@thunder> - initial prototype/prolog/code\end{verbatim}
 
%/////////////////////////////////////////////////////////////
 
\mbox{}\hrulefill\ 

  \subsubsection{I90\_LoadF() --- Loads resource file into memory. }

    Reads resource file, strips out comments, translate TAB's into
    blanks, and loads the modified file contents into memory.
    Must be called only once for each resource file.
  
\bigskip{\sf CALLING SEQUENCE:}
\begin{verbatim}       call i90_LoadF ( filen, iret )\end{verbatim}
%{\em INPUT PARAMETERS:}
%\end{verbatim}
%{\em OUTPUT PARAMETERS:}
%\begin{verbatim} 
\begin{verbatim}
       character*(*),intent(in)  :: filen    ! file name
       integer,      intent(out) :: iret     ! Return code:
                                             !   0    no error
                                             ! -98    coult not get unit number
                                             !        (strange!)
                                             ! -98    talk to a wizzard
                                             ! -99    out of memory: increase
                                             !        NBUF_MAX in 'i90.h'
                                             ! other iostat from open statement.
\end{verbatim}
{\sf BUGS:}
\begin{verbatim}    It does not perform dynamic allocation, mostly to keep vanilla f77
    compatibility. Overall amount of static memory is small (~100K
    for default NBUF_MAX = 400*256).\end{verbatim}{\sf SEE ALSO:}
\begin{verbatim}    i90_label()   selects a label (key)\end{verbatim}{\sf FILES USED:}
\begin{verbatim}    File name supplied on input. The file is opened, read and then closed.\end{verbatim}%{\sf REVISION HISTORY:}
%\begin{verbatim}    19Jun96   da Silva   Original code.\end{verbatim}
 
%/////////////////////////////////////////////////////////////
 
\mbox{}\hrulefill\ 

  \subsubsection{I90\_Label() --- Selects a label (record). }

    Once the buffer has been loaded with {\tt i90\_loadf()}, this routine
    selects a given ``line'' (record/table) associated with ``label''. 
    Think of ``label'' as a resource name or data base ``key''.
  
\bigskip{\sf CALLING SEQUENCE:}
\begin{verbatim}             
       call i90_Label ( label, iret )
\end{verbatim}
%{\em INPUT PARAMETERS:}
%\end{verbatim}
%{\em OUTPUT PARAMETERS:}
%\begin{verbatim} 
\begin{verbatim}
       character(len=*),intent(in)  :: label     ! input label
       integer,         intent(out) :: iret      ! Return code:
                                                 !   0    no error
                                                 !  -1    buffer not loaded
                                                 !  -2    could not find label
\end{verbatim}
{\sf SEE ALSO:}
\begin{verbatim}    i90_loadf()    load file into buffer
    i90_gtoken()   get next token
    i90_gline()    get next line (for tables)
    atof()         convert word (string) to float
    atoi()         convert word (string) to integer
    \end{verbatim}%{\sf REVISION HISTORY:}
%\begin{verbatim}    19Jun96   da Silva   Original code.\end{verbatim}
 
%/////////////////////////////////////////////////////////////
 
\mbox{}\hrulefill\ 
 
  \subsubsection{I90\_GLine() --- Selects next line. }
  
       Selects next line, irrespective of of label. If the next line starts
   with :: (end of table mark), then it lets the user know. This sequential
   access of the buffer is useful to assess tables, a concept introduced
   in Inpak 77 by Jing Guo. A table is a construct like this:
  
   \begin{verbatim}
   my_table_name::
    1000     3000     263.0   
     925     3000     263.0
     850     3000     263.0
     700     3000     269.0
     500     3000     287.0
     400     3000     295.8
     300     3000     295.8    
   ::
   \end{verbatim}
  
   To access this table, the user first must use {\tt i90\_label()} to 
   locate the beginning of the table, e.g.,
  
   \begin{verbatim}
         call i90_label ( 'my_table_name::', iret )
   \end{verbatim}
  
   Subsequently, {\tt i90\_gline()} can be used to gain acess to each
   row of the table. Here is a code fragment to read the above
   table (7 rows, 3 columns):
  
   \begin{verbatim}
         real          table(7,3)
         character*20  word
         integer       iret
         call i90_label ( 'my_table_name::', iret )
         do i = 1, 7
            call i90_gline ( iret )
            do j = 1, 3
               table(i,j) = fltget ( 0. )
            end do                   
         end do
   \end{verbatim}
  
    For simplicity we have assumed that the dimensions of table were
    known. It is relatively simple to infer the table dimensions
    by manipulating ``iret''.
  
\bigskip{\sf CALLING SEQUENCE:}
\begin{verbatim}       call i90_gline ( iret )\end{verbatim}
%{\em INPUT PARAMETERS:}
%\begin{verbatim}
%       None.
%\end{verbatim}
%{\em OUTPUT PARAMETERS:}
\begin{verbatim}
       integer,intent(out) :: iret       ! Return code:
                                          !   0    no error
                                          !  -1    end of buffer reached
                                          !  +1    end of table  reached
 \end{verbatim}
{\sf SEE ALSO:}
\begin{verbatim}    i90_label()    selects a line (record/table)\end{verbatim}%{\sf REVISION HISTORY:}
%\begin{verbatim}    10feb95   Guo        Wrote rdnext(), Inpak 77 extension.
%    19Jun96   da Silva   Original code with functionality of rdnext()\end{verbatim}
 
%/////////////////////////////////////////////////////////////
 
\mbox{}\hrulefill\ 

  \subsubsection{I90\_GToken() --- Gets next token. }

    Get next token from current line. The current line is defined by a
    call to {\tt i90\_label()}. Tokens are sequences of characters (including
    blanks) which may be enclosed by single (') or double (") quotes. 
    If no quotes are present, the token from the current position to the next
    blank of TAB is returned.
    
    {\em Examples of valid token:}
  
    \begin{verbatim}
                 single_token "second token on line"
                 "this is a token"
                 'Another example of a token'
                 'this is how you get a " inside a token'
                 "this is how you get a ' inside a token"
                 This is valid too   # the line ends before the #
    \end{verbatim}
    The last line has 4 valid tokens: {\tt This, is, valid} and {\tt too}.
    
    {\em Invalid string constructs:}
  
    \begin{verbatim}
                 'cannot handle mixed quotes"
                 'escaping like this \' is not implemented'
                 'this # will not work because of the #'
    \end{verbatim}
    The \# character is reserved for comments and cannot be included
    inside quotation marks.
  
\bigskip{\sf CALLING SEQUENCE:}
\begin{verbatim}       call i90_GToken ( token, iret )\end{verbatim}
%{\em INPUT PARAMETERS:}
%\begin{verbatim}
%       None.
%\end{verbatim}
%{\em OUTPUT PARAMETERS:}
\begin{verbatim}
       character*(*),intent(out) :: token     ! Next token from current line
       integer,      intent(out) :: iret      ! Return code:
                                              !   0    no error
                                              !  -1    either nothing left
                                              !        on line or mismatched
                                              !        quotation marks.
 \end{verbatim}{\sf BUGS:}
\begin{verbatim}       Standard Unix escaping is not implemented at the moment.
       \end{verbatim}{\sf SEE ALSO:}
\begin{verbatim}    i90_label()    selects a line (record/table)
    i90_gline()    get next line (for tables)
    atof()         convert word (string) to float
    atoi()         convert word (string) to integer
    \end{verbatim}%{\sf REVISION HISTORY:}
%\begin{verbatim}    19Jun96   da Silva   Original code.\end{verbatim}
 
%/////////////////////////////////////////////////////////////
 
\mbox{}\hrulefill\ 
 
  \subsubsection{i90\_GFloat() --- Returns next float number. }

    Returns next float (real number) from the current line.
    If an error occurs a zero value is returned.
  
\bigskip{\sf CALLING SEQUENCE:}
\begin{verbatim}        real  rnumber
        rnumber = i90_gfloat ( iret )\end{verbatim}
%{\em OUTPUT PARAMETERS:}
\begin{verbatim}       integer,intent(out) :: iret    ! Return code:
                                      !   0    no error
                                      !  -1    either nothing left
                                      !        on line or mismatched
                                      !        quotation marks.
                                      !  -2    parsing error
 \end{verbatim}%{\sf REVISION HISTORY:}
%\begin{verbatim}    19Jun96   da Silva   Original code.\end{verbatim}
 
%/////////////////////////////////////////////////////////////
 
\mbox{}\hrulefill\ 
 

  \subsubsection{I90\_GInt() --- Returns next integer number. }

    Returns next integer number from the current line.
    If an error occurs a zero value is returned.
  
\bigskip{\sf CALLING SEQUENCE:}
\begin{verbatim}        integer number
        number = i90_gint ( default )\end{verbatim}
%{\em OUTPUT PARAMETERS:}
\begin{verbatim}
       integer,intent(out) :: iret    ! Return code:
                                      !   0    no error
                                      !  -1    either nothing left
                                      !        on line or mismatched
                                      !        quotation marks.
                                      !  -2    parsing error
 \end{verbatim}%{\sf REVISION HISTORY:}
%\begin{verbatim}    19Jun96   da Silva   Original code.
%    24may00   da Silva   delcared x as real*8 in case this module is compiled
%                         with real*4\end{verbatim}
 
%/////////////////////////////////////////////////////////////
 
\mbox{}\hrulefill\ 
 
  \subsubsection{i90\_AtoF() --- Translates ASCII (string) to float. }

       Converts string to real number. Same as obsolete {\tt str2rn()}.
  
\bigskip{\sf CALLING SEQUENCE:}
\begin{verbatim}       real  rnumber
       rnumber = i90_atof ( string, iret )\end{verbatim}
%{\em INPUT PARAMETERS:}
%\end{verbatim}
%{\em OUTPUT PARAMETERS:}
%\begin{verbatim}
\begin{verbatim}
       character(len=*),intent(in) :: string  ! a string
       integer,intent(out)         :: iret    ! Return code:
                                              !   0    no error
                                              !  -1    could not convert, probably
                                              !        string is not a number
 \end{verbatim}%{\sf REVISION HISTORY:}
%\begin{verbatim}    19Jun96   da Silva   Original code.\end{verbatim}
 
%/////////////////////////////////////////////////////////////
 
\mbox{}\hrulefill\ 

  \subsubsection{I90\_AtoI() --- Translates ASCII (strings) to integer. }

       Converts string to integer number.
  
\bigskip{\sf CALLING SEQUENCE:}
\begin{verbatim}       integer number
       number = i90_atoi ( string, iret )\end{verbatim}
%{\em INPUT PARAMETERS:}
%\end{verbatim}
%{\em OUTPUT PARAMETERS:}
%\begin{verbatim}
\begin{verbatim}
       character*(*),inten(in) :: string   ! a string
       integer,intent(out)     :: iret     ! Return code:
                                           !   0    no error
                                           !  -1    could not convert, probably
                                           !        string is not a number
 \end{verbatim}%{\sf REVISION HISTORY:}
%\begin{verbatim}    19Jun96   da Silva   Original code.\end{verbatim}
 
%/////////////////////////////////////////////////////////////
 
\mbox{}\hrulefill\ 
 
  \subsubsection{I90\_Len() --- Returns length of string. }

    Returns the length of a string excluding trailing blanks.
    It follows that 
    \begin{verbatim}
                i90_len(string) .le. len(string),
    \end{verbatim}
    where {\tt len} is the intrinsic string length function.  
    Example:
    \begin{verbatim}
           ls = len('abc  ')       ! results in ls = 5
           ls = i90_len ('abc  ')  ! results in ls = 3
    \end{verbatim}
  
\bigskip{\sf CALLING SEQUENCE:}
\begin{verbatim}         integer ls
         ls = i90_len ( string )\end{verbatim}
{\em INPUT PARAMETERS:}
\begin{verbatim}
          character*(*),intent(in) :: string     ! a string
\end{verbatim}
%{\em OUTPUT PARAMETERS:}
%\begin{verbatim}
%          The length of the string, excluding trailing blanks.
%\end{verbatim}
%{\sf REVISION HISTORY:}
%\begin{verbatim}    01Apr94   Guo        Original code (a.k.a. luavail())
%    19Jun96   da Silva   Minor modification + prologue.\end{verbatim}
 
%/////////////////////////////////////////////////////////////
 
\mbox{}\hrulefill\ 
 
  \subsubsection{I90\_Lua() --- Returns available logical unit number. }

    Look for an available (not opened) Fortran logical unit for i/o.
  
\bigskip{\sf CALLING SEQUENCE:}
\begin{verbatim}         integer lu
         lu = i90_lua()\end{verbatim}
%{\em INPUT PARAMETERS:}
%\begin{verbatim}
%         None.
%\end{verbatim}
%{\em OUTPUT PARAMETERS:}
\begin{verbatim}
         The desired unit number if positive, -1 if unsucessful.
\end{verbatim}%{\sf REVISION HISTORY:}
%\begin{verbatim}    01Apr94   Guo        Original code (a.k.a. luavail())
%    19Jun96   da Silva   Minor modification + prologue.\end{verbatim}
 
%/////////////////////////////////////////////////////////////
 
\mbox{}\hrulefill\ 
 
  \subsubsection{I90\_Pad() --- Pad strings. }

       Pads from the right with the comment character (\#). It also
    replaces TAB's with blanks for convenience. This is a low level
    i90 routine.
  
\bigskip{\sf CALLING SEQUENCE:}
\begin{verbatim}        call i90_pad ( string )\end{verbatim}
%{\em INPUT PARAMETERS:}
%\begin{verbatim}
%        character*256 string       ! input string
% \end{verbatim}
%{\em OUTPUT PARAMETERS:}
\begin{verbatim}
        character*256,intent(inout) :: string
\end{verbatim}
{\sf BUGS:}
\begin{verbatim}        It alters TAB's even inside strings.\end{verbatim}%{\sf REVISION HISTORY:}
%\begin{verbatim}    19Jun96   da Silva   Original code.\end{verbatim}
 
%/////////////////////////////////////////////////////////////
 
\mbox{}\hrulefill\ 
 
  \subsubsection{I90\_Trim() - Removes leading blanks from strings. }

      Removes blanks and TABS from begenning of string. 
      This is a low level i90 routine.
   
\bigskip{\sf CALLING SEQUENCE:}
\begin{verbatim}       call i90_Trim ( string )\end{verbatim}
%{\em INPUT PARAMETERS:}
\begin{verbatim}
       character*256,ntent(inout) ::  string 
\end{verbatim}
%{\em OUTPUT PARAMETERS:}
%\begin{verbatim}
%       character*256 string    ! the modified string
%\end{verbatim}%{\sf REVISION HISTORY:}
%\begin{verbatim}    19Jun96   da Silva   Original code.\end{verbatim}
 
%/////////////////////////////////////////////////////////////
 
\mbox{}\hrulefill\ 

  \subsubsection{Lablin() --- Selects a Label (Inpak 77) }

      Selects a given ``line'' (record/table) associated with ``label''. 
      Similar to {\tt i90\_label()}, but prints a message to {\tt stdout}
      if it cannot locate the label. Kept for Inpak 77 upward compatibility.
  
\bigskip{\sf CALLING SEQUENCE:}
\begin{verbatim}       call lablin ( label )\end{verbatim}
%{\em INPUT PARAMETERS:}
\begin{verbatim} 
       character(len=*),intent(in) :: label   ! string with label name
\end{verbatim}
%{\em OUTPUT PARAMETERS:}
%\begin{verbatim}      None.\end{verbatim}%{\sf REVISION HISTORY:}
%\begin{verbatim}    19Jun96   da Silva   Original code.\end{verbatim}
 
%/////////////////////////////////////////////////////////////
 
\mbox{}\hrulefill\ 
 
  \subsubsection{FltGetsp() --- Returns next float (Inpak 77, single precision) }

    Returns next float (real number, single precision) from the current 
    line, or a default value if it fails to obtain the desired number.
    Kept for Inpak 77 upward compatibility.
  
\bigskip{\sf CALLING SEQUENCE:}
\begin{verbatim}        real  rnumber, default
        rnumber = fltgetsp ( default )\end{verbatim}
%{\em INPUT PARAMETERS:}
\begin{verbatim}       real(SP), intent(IN) ::    default       ! default value.
 \end{verbatim}%{\sf REVISION HISTORY:}
%\begin{verbatim}    19Jun96   da Silva   Original code.
%    12Oct99   Guo/Larson - Built from original FltGet() function.\end{verbatim}
 
%/////////////////////////////////////////////////////////////
 
\mbox{}\hrulefill\ 
 
  \subsubsection{FltGetdp() --- Returns next float (Inpak 77) }

    Returns next float (real number) from the current line, or a 
    default value (double precision) if it fails to obtain the desired 
    number.  Kept for Inpak 77 upward compatibility.
  
\bigskip{\sf CALLING SEQUENCE:}
\begin{verbatim}        real(DP) :: default
        real :: rnumber 
        rnumber = FltGetdp(default)\end{verbatim}
%{\em INPUT PARAMETERS:}
\begin{verbatim}       real(DP), intent(IN) ::    default       ! default value.
 \end{verbatim}%{\sf REVISION HISTORY:}
%\begin{verbatim}    19Jun96   da Silva   Original code.
%    12Oct99   Guo/Larson - Built from original FltGet() function.\end{verbatim}
 
%/////////////////////////////////////////////////////////////
 
\mbox{}\hrulefill\ 

  \subsubsection{IntGet() --- Returns next integer (Inpak 77).  }

    Returns next integer number from the current line, or a default 
    value if it fails to obtain the desired number.
    Kept for Inpak 77 upward compatibility.
  
\bigskip{\sf CALLING SEQUENCE:}
\begin{verbatim}        integer number, default
        number = intget ( default )\end{verbatim}
%{\em INPUT PARAMETERS:}
\begin{verbatim}        integer, intent(IN) ::    default       ! default value.
 \end{verbatim}%{\sf REVISION HISTORY:}
%\begin{verbatim}    19Jun96   da Silva   Original code.\end{verbatim}
 
%/////////////////////////////////////////////////////////////
 
\mbox{}\hrulefill\ 
 
  \subsubsection{ChrGet() --- Returns next character (Inpak 77). }

    Returns next non-blank character from the current line, or a default 
    character if it fails for whatever reason.
    Kept for Inpak 77 upward compatibility.
  
\bigskip{\sf CALLING SEQUENCE:}
\begin{verbatim}       character*1 ch, default
       ch = chrget ( default )\end{verbatim}
%{\em INPUT PARAMETERS:}
\begin{verbatim}       character*1, intent(IN) ::  default      ! default value.
 \end{verbatim}%{\sf REVISION HISTORY:}
%\begin{verbatim}    19Jun96   da Silva   Original code.\end{verbatim}
 
%/////////////////////////////////////////////////////////////
 
\mbox{}\hrulefill\ 
 
  \subsubsection{TokGet() --- Gets next token (Inpakk 77 like). }

    Returns next token from the current line, or a default 
    word if it fails for whatever reason.
  
\bigskip{\sf CALLING SEQUENCE:}
\begin{verbatim}        call TokGet ( token, default )\end{verbatim}
%{\em INPUT PARAMETERS:}
%\end{verbatim}
%{\em OUTPUT PARAMETERS:}
%\begin{verbatim}
\begin{verbatim}
        character*(*), intent(IN) :: default     ! default token
        character*(*), intent(OUT) :: token      ! desired token
\end{verbatim}%{\sf REVISION HISTORY:}
%\begin{verbatim}    19Jun96   da Silva   Original code.\end{verbatim}
 
%/////////////////////////////////////////////////////////////
 
\mbox{}\hrulefill\ 

  \subsubsection{index\_ Extension of the Fortran 77 intrinsic "index" for }

    "string" (input) with length that can exceed 2**15-1 (=MAXLEN).  
  Finds the starting location = "index\_", of the first character in "tok"  
    within "string", where string is of "arbitrary" length.  If tok occurs more than
    once in "string", then the value of index\_ is based on the first occurrence of "tok". 
  
\bigskip{\sf CALLING SEQUENCE:}
\begin{verbatim}        index_( string,tok )\end{verbatim}
%{\em INPUT PARAMETERS:}
\begin{verbatim}       character(len=*), intent(in) :: string, tok\end{verbatim}%{\sf REVISION HISTORY:}
%\begin{verbatim}    2001Apr25   G. Gaspari   Original code.\end{verbatim}

%\markboth{Left}{Source File: m\_ioutil.F90,  Date: Thu Nov 15 22:24:26 GMT 2001
%}

 
%/////////////////////////////////////////////////////////////
 
\mbox{}\hrulefill\ 

  \subsection{Module m\_ioutil - a F90 module for several convenient I/O functions }
%
%  	m\_ioutil is a module containing several portable interfaces for
%  	some highly system dependent, but frequently used I/O functions.
  
\bigskip{\sf INTERFACE:}
\begin{verbatim} 
 	public	:: opntext,clstext ! open/close a text file
 	public	:: opnieee,clsieee ! open/close a binary sequential file
 	public	:: luavail	   ! return a free logical unit
 	public	:: luflush	   ! flush the buffer of a given unit
 	!public	:: MX_LU
 \end{verbatim}%{\sf REVISION HISTORY:}
%\begin{verbatim}   	16Jul96 - J. Guo	- (to do)
%   	02Apr97 - Jing Guo <guo@eramus> - finished the coding
%  	11Feb97 - Jing Guo <guo@thunder> - added luflush()\end{verbatim}
% 
%/////////////////////////////////////////////////////////////
 
\mbox{}\hrulefill\ 

  \subsubsection{opnieee - portablly open an IEEE format file}
%
%  	Open a file in `IEEE' format.
%  
%  	`IEEE' format is refered as a FORTRAN "unformatted" file with
%  	"sequantial" access and variable record lengths.  Under common
%  	Unix, it is only a file with records packed with a leading 4-
%  	byte word and a trailing 4-byte word indicating the size of
%  	the record in bytes.  However, under UNICOS, it is also assumed
%  	to have numerical data representations represented according to
%  	the IEEE standard corresponding KIND conversions.  Under a DEC
%  	machine, it means that compilations of the source code should
%  	have the "-bigendian" option specified.
  
\bigskip{\sf INTERFACE:}
\begin{verbatim} 
     subroutine opnieee(lu,fname,status,ier,recl)
       integer,         intent(in) :: lu     ! logical unit number
       character(len=*),intent(in) :: fname  ! filename to be opended
       character(len=*),intent(in) :: status ! the value for STATUS=
       integer,         intent(out):: ier    ! the status
       integer,optional,intent(in) :: recl   ! record length
 \end{verbatim}%{\sf REVISION HISTORY:}
%\begin{verbatim}  	02Feb95 - Jing G. - First version included in PSAS.  It is not
%  		used in the libpsas.a calls, since no binary data input/
%  		output is to be handled.
%   	09Oct96 - J. Guo  - Check for any previous assign() call under
%  		UNICOS.\end{verbatim}
 
%/////////////////////////////////////////////////////////////
 
\mbox{}\hrulefill\ 
 
  \subsubsection{clsieee - Close a logical unit opened by opnieee()}

%  	The reason for a paired clsieee() for opnieee() instead of a
%  	simple close(), is for the portability reason.  For example,
%  	under UNICOS, special system calls may be need to set up the
%  	unit right, and the status of the unit should be restored upon
%  	close.
%  
\bigskip{\sf INTERFACE:}
\begin{verbatim} 
 	subroutine clsieee(lu,ier)
 	  integer, intent(in)  :: lu	! the unit used by opnieee()
 	  integer, intent(out) :: ier	! the status
 \end{verbatim}%{\sf REVISION HISTORY:}
%\begin{verbatim}   	10Oct96 - J. Guo	- (to do)\end{verbatim}
 
%/////////////////////////////////////////////////////////////
 
\mbox{}\hrulefill\ 
 
  \subsubsection{opntext - portablly open a text file}

%  	Open a text (ASCII) file.  Under FORTRAN, it is defined as
%  	"formatted" with "sequential" access.
%  
\bigskip{\sf INTERFACE:}
\begin{verbatim} 
     subroutine opntext(lu,fname,status,ier)
       integer,         intent(in) :: lu     ! logical unit number
       character(len=*),intent(in) :: fname  ! filename to be opended
       character(len=*),intent(in) :: status ! the value for STATUS=<>
       integer,         intent(out):: ier    ! the status
 
 \end{verbatim}%{\sf REVISION HISTORY:}
%\begin{verbatim}  	02Feb95 - Jing G. - First version included in PSAS and libpsas.a
%   	09Oct96 - J. Guo  - modified to allow assign() call under UNICOS
%  			  = and now, it is a module in Fortran 90.\end{verbatim}
 
%/////////////////////////////////////////////////////////////
 
\mbox{}\hrulefill\ 
 
  \subsubsection{clstext - close a text file opend with an opntext() call}

\bigskip{\sf INTERFACE:}
\begin{verbatim} 
     subroutine clstext(lu,ier)
       integer, intent(in)  :: lu  ! a logical unit to close
       integer, intent(out) :: ier ! the status
 \end{verbatim}%{\sf REVISION HISTORY:}
%\begin{verbatim}   	09Oct96 - J. Guo	- (to do)\end{verbatim}
 
%/////////////////////////////////////////////////////////////
 
\mbox{}\hrulefill\ 

  \subsubsection{luavail - locate the next available unit}

%      luavail() Look for an available (not opened and not statically
%      assigned to any I/O attributes to) logical unit.
  
\bigskip{\sf INTERFACE:}
\begin{verbatim} 
 	function luavail()
 	  integer :: luavail	! result
 \end{verbatim}%{\sf REVISION HISTORY:}
%\begin{verbatim}   	23Apr98 - Jing Guo <guo@thunder> - new prototype/prolog/code
%  			- with additional unit constraints for SunOS.
%   	: Jing Guo, [09-Oct-96]
%   		+ Checking also Cray assign() attributes, with some
%   		  changes to the code.  See also other routines.
%   	: Jing Guo, [01-Apr-94]
%   		+ Initial code.\end{verbatim}
 
%/////////////////////////////////////////////////////////////
 
\mbox{}\hrulefill\ 

  \subsubsection{luflush - a uniform interface of system flush()}

%  	Flush() calls available on many systems are often implementation
%  	dependent.  This subroutine provides a uniform interface.  It
%  	also ignores invalid logical unit value.
  
\bigskip{\sf INTERFACE:}
\begin{verbatim} 
     subroutine luflush(unit)
       integer,optional,intent(in) :: unit
 \end{verbatim}%{\sf REVISION HISTORY:}
%\begin{verbatim}   	13Mar98 - Jing Guo <guo@thunder> - initial prototype/prolog/code\end{verbatim}

%\markboth{Left}{Source File: m\_mall.F90,  Date: Thu Nov 15 22:24:26 GMT 2001
%}

 
%/////////////////////////////////////////////////////////////
 
\mbox{}\hrulefill\ 

  \subsection{Module m\_mall - A bookkeeper of user allocated memories }

\bigskip{\sf INTERFACE:}
\begin{verbatim} 
     module m_mall
       public :: mall_ci
       public :: mall_co
       public :: mall_mci
       public :: mall_mco
       public :: mall_flush
       public :: mall_reset
 
 		! mall_ activity controls
 
       public :: mall_ison
       public :: mall_set
 
       interface mall_ci
         module procedure 
           ci_
       end interface
       interface mall_co
         module procedure 
           co_
       end interface
 
       interface mall_mci
         module procedure
           ciI0_, ciI1_, ciI2_,	ciI3_, ciI4_, ciR0_, ciR1_, ciR2_, ciR3_, 
           ciR4_, ciD0_, ciD1_, ciD2_, ciD3_, ciD4_, ciL0_, ciL1_, ciL2_, 
           ciL3_, ciL4_, ciC0_, ciC1_, ciC2_, ciC3_, ciC4_
       end interface
 
       interface mall_mco
         module procedure
           coI0_, coI1_, coI2_, coI3_, coI4_, coR0_, coR1_, coR2_, coR3_, 
           coR4_, coD0_, coD1_, coD2_, coD3_, coD4_, coL0_, coL1_, coL2_, 
           coL3_, coL4_, coC0_, coC1_, coC2_, coC3_, coC4_
       end interface
 
       interface mall_flush
         module procedure 
           flush_
       end interface
       interface mall_reset
         module procedure 
           reset_
       end interface
 
       interface mall_ison
         module procedure 
           ison_
       end interface
       interface mall_set
         module procedure 
           set_
       end interface
 \end{verbatim}%{\sf REVISION HISTORY:}
%\begin{verbatim}   	13Feb98 - Jing Guo <guo@thunder> - initial prototype/prolog/code\end{verbatim}
 
%/////////////////////////////////////////////////////////////
 
\mbox{}\hrulefill\ 
 
  \subsubsection{ison\_ -}

\bigskip{\sf INTERFACE:}
\begin{verbatim} 
     function ison_()
       logical :: ison_
 \end{verbatim}%{\sf REVISION HISTORY:}
%\begin{verbatim}   	25Oct99	- Jing Guo <guo@dao.gsfc.nasa.gov>
%  		- initial prototype/prolog/code\end{verbatim}
 
%/////////////////////////////////////////////////////////////
 
\mbox{}\hrulefill\ 
 
  \subsubsection{set\_ - set the switch on}

\bigskip{\sf INTERFACE:}
\begin{verbatim} 
     subroutine set_(on)
       logical,optional,intent(in) :: on
 \end{verbatim}%{\sf REVISION HISTORY:}
%\begin{verbatim}   	25Oct99	- Jing Guo <guo@dao.gsfc.nasa.gov>
%  		- initial prototype/prolog/code\end{verbatim}
 
%/////////////////////////////////////////////////////////////
 
\mbox{}\hrulefill\ 
 
  \subsubsection{ciI0\_ - check in as an integer scalar}

\bigskip{\sf INTERFACE:}
\begin{verbatim} 
     subroutine ciI0_(marg,thread)
       integer,intent(in) :: marg
       character(len=*),intent(in) :: thread
 \end{verbatim}%{\sf REVISION HISTORY:}
%\begin{verbatim}   	21Oct99	- Jing Guo <guo@dao.gsfc.nasa.gov>
%  		- initial prototype/prolog/code\end{verbatim}
 
%/////////////////////////////////////////////////////////////
 
\mbox{}\hrulefill\ 
 
  \subsubsection{ciI1\_ - check in as an integer rank 1 array}

\bigskip{\sf INTERFACE:}
\begin{verbatim} 
     subroutine ciI1_(marg,thread)
       integer,dimension(:),intent(in) :: marg
       character(len=*),intent(in)     :: thread
 \end{verbatim}%{\sf REVISION HISTORY:}
%\begin{verbatim}   	21Oct99	- Jing Guo <guo@dao.gsfc.nasa.gov>
%  		- initial prototype/prolog/code\end{verbatim}
 
%/////////////////////////////////////////////////////////////
 
\mbox{}\hrulefill\ 

  \subsubsection{ciI2\_ - check in as an integer rank 2 array}

\bigskip{\sf INTERFACE:}
\begin{verbatim} 
     subroutine ciI2_(marg,thread)
       integer,dimension(:,:),intent(in) :: marg
       character(len=*),intent(in)       :: thread
 \end{verbatim}%{\sf REVISION HISTORY:}
%\begin{verbatim}   	21Oct99	- Jing Guo <guo@dao.gsfc.nasa.gov>
%  		- initial prototype/prolog/code\end{verbatim}
 
%/////////////////////////////////////////////////////////////
 
\mbox{}\hrulefill\ 

  \subsubsection{ciI3\_ - check in as an integer rank 3 array}

\bigskip{\sf INTERFACE:}
\begin{verbatim} 
     subroutine ciI3_(marg,thread)
       integer,dimension(:,:,:),intent(in) :: marg
       character(len=*),intent(in)         :: thread
 \end{verbatim}%{\sf REVISION HISTORY:}
%\begin{verbatim}   	21Oct99	- Jing Guo <guo@dao.gsfc.nasa.gov>
%  		- initial prototype/prolog/code\end{verbatim}
 
%/////////////////////////////////////////////////////////////
 
\mbox{}\hrulefill\ 

  \subsubsection{ciI4\_ - check in as an integer rank 4 array}

\bigskip{\sf INTERFACE:}
\begin{verbatim} 
     subroutine ciI4_(marg,thread)
       integer,dimension(:,:,:,:),intent(in) :: marg
       character(len=*),intent(in)           :: thread
 \end{verbatim}%{\sf REVISION HISTORY:}
%\begin{verbatim}   	21Oct99	- Jing Guo <guo@dao.gsfc.nasa.gov>
%  		- initial prototype/prolog/code\end{verbatim}
 
%/////////////////////////////////////////////////////////////
 
\mbox{}\hrulefill\ 
 
  \subsubsection{ciR0\_ - check in as a real(SP) scalar}

\bigskip{\sf INTERFACE:}
\begin{verbatim} 
     subroutine ciR0_(marg,thread)
       real(SP),intent(in)         :: marg
       character(len=*),intent(in) :: thread
 \end{verbatim}%{\sf REVISION HISTORY:}
%\begin{verbatim}   	21Oct99	- Jing Guo <guo@dao.gsfc.nasa.gov>
%  		- initial prototype/prolog/code\end{verbatim}
 
%/////////////////////////////////////////////////////////////
 
\mbox{}\hrulefill\ 
 
  \subsubsection{ciR1\_ - check in as a real(SP) rank 1 array}

\bigskip{\sf INTERFACE:}
\begin{verbatim} 
     subroutine ciR1_(marg,thread)
       real(SP),dimension(:),intent(in) :: marg
       character(len=*),intent(in)      :: thread
 \end{verbatim}%{\sf REVISION HISTORY:}
%\begin{verbatim}   	21Oct99	- Jing Guo <guo@dao.gsfc.nasa.gov>
%  		- initial prototype/prolog/code\end{verbatim}
 
%/////////////////////////////////////////////////////////////
 
\mbox{}\hrulefill\ 
 
  \subsubsection{ciR2\_ - check in as a real(SP) rank 2 array}

\bigskip{\sf INTERFACE:}
\begin{verbatim} 
     subroutine ciR2_(marg,thread)
       real(SP),dimension(:,:),intent(in) :: marg
       character(len=*),intent(in)        :: thread
 \end{verbatim}%{\sf REVISION HISTORY:}
%\begin{verbatim}   	21Oct99	- Jing Guo <guo@dao.gsfc.nasa.gov>
%  		- initial prototype/prolog/code\end{verbatim}
 
%/////////////////////////////////////////////////////////////
 
\mbox{}\hrulefill\ 

  \subsubsection{ciR3\_ - check in as a real(SP) rank 3 array}

\bigskip{\sf INTERFACE:}
\begin{verbatim} 
     subroutine ciR3_(marg,thread)
       real(SP),dimension(:,:,:),intent(in) :: marg
       character(len=*),intent(in)          :: thread
 \end{verbatim}%{\sf REVISION HISTORY:}
%\begin{verbatim}   	21Oct99	- Jing Guo <guo@dao.gsfc.nasa.gov>
%  		- initial prototype/prolog/code\end{verbatim}
 
%/////////////////////////////////////////////////////////////
 
\mbox{}\hrulefill\ 
 
  \subsubsection{ciR4\_ - check in as a real(SP) rank 4 array}

\bigskip{\sf INTERFACE:}
\begin{verbatim} 
     subroutine ciR4_(marg,thread)
       real(SP),dimension(:,:,:,:),intent(in) :: marg
       character(len=*),intent(in)            :: thread
 \end{verbatim}%{\sf REVISION HISTORY:}
%\begin{verbatim}   	21Oct99	- Jing Guo <guo@dao.gsfc.nasa.gov>
%  		- initial prototype/prolog/code\end{verbatim}
 
%/////////////////////////////////////////////////////////////
 
\mbox{}\hrulefill\ 
 
  \subsubsection{ciD0\_ - check in as a real(DP) scalar}

\bigskip{\sf INTERFACE:}
\begin{verbatim} 
     subroutine ciD0_(marg,thread)
       real(DP),intent(in)         :: marg
       character(len=*),intent(in) :: thread
 \end{verbatim}%{\sf REVISION HISTORY:}
%\begin{verbatim}   	21Oct99	- Jing Guo <guo@dao.gsfc.nasa.gov>
%  		- initial prototype/prolog/code\end{verbatim}
 
%/////////////////////////////////////////////////////////////
 
\mbox{}\hrulefill\ 
 
  \subsubsection{ciD1\_ - check in as a real(DP) rank 1 array}

\bigskip{\sf INTERFACE:}
\begin{verbatim} 
     subroutine ciD1_(marg,thread)
       real(DP),dimension(:),intent(in) :: marg
       character(len=*),intent(in)      :: thread
 \end{verbatim}%{\sf REVISION HISTORY:}
%\begin{verbatim}   	21Oct99	- Jing Guo <guo@dao.gsfc.nasa.gov>
%  		- initial prototype/prolog/code\end{verbatim}
 
%/////////////////////////////////////////////////////////////
 
\mbox{}\hrulefill\ 

  \subsubsection{ciD2\_ - check in as a real(DP) rank 2 array}

\bigskip{\sf INTERFACE:}
\begin{verbatim} 
     subroutine ciD2_(marg,thread)
       real(DP),dimension(:,:),intent(in) :: marg
       character(len=*),intent(in)        :: thread
 \end{verbatim}%{\sf REVISION HISTORY:}
%\begin{verbatim}   	21Oct99	- Jing Guo <guo@dao.gsfc.nasa.gov>
%  		- initial prototype/prolog/code\end{verbatim}
 
%/////////////////////////////////////////////////////////////
 
\mbox{}\hrulefill\ 
 
  \subsubsection{ciD3\_ - check in as a real(DP) rank 3 array}

\bigskip{\sf INTERFACE:}
\begin{verbatim} 
     subroutine ciD3_(marg,thread)
       real(DP),dimension(:,:,:),intent(in) :: marg
       character(len=*),intent(in)          :: thread
 \end{verbatim}%{\sf REVISION HISTORY:}
%\begin{verbatim}   	21Oct99	- Jing Guo <guo@dao.gsfc.nasa.gov>
%  		- initial prototype/prolog/code\end{verbatim}
 
%/////////////////////////////////////////////////////////////
 
\mbox{}\hrulefill\ 

  \subsubsection{ciD4\_ - check in as a real(DP) rank 4 array}

\bigskip{\sf INTERFACE:}
\begin{verbatim} 
     subroutine ciD4_(marg,thread)
       real(DP),dimension(:,:,:,:),intent(in) :: marg
       character(len=*),intent(in)            :: thread
 \end{verbatim}%{\sf REVISION HISTORY:}
%\begin{verbatim}   	21Oct99	- Jing Guo <guo@dao.gsfc.nasa.gov>
%  		- initial prototype/prolog/code\end{verbatim}
 
%/////////////////////////////////////////////////////////////
 
\mbox{}\hrulefill\ 
 
  \subsubsection{ciL0\_ - check in as a logical scalar}

\bigskip{\sf INTERFACE:}
\begin{verbatim} 
     subroutine ciL0_(marg,thread)
       logical,intent(in)          :: marg
       character(len=*),intent(in) :: thread
 \end{verbatim}%{\sf REVISION HISTORY:}
%\begin{verbatim}   	21Oct99	- Jing Guo <guo@dao.gsfc.nasa.gov>
%  		- initial prototype/prolog/code\end{verbatim}
 
%/////////////////////////////////////////////////////////////
 
\mbox{}\hrulefill\ 
 
  \subsubsection{ciL1\_ - check in as a logical rank 1 array}

\bigskip{\sf INTERFACE:}
\begin{verbatim} 
     subroutine ciL1_(marg,thread)
       logical,dimension(:),intent(in) :: marg
       character(len=*),intent(in)     :: thread
 \end{verbatim}%{\sf REVISION HISTORY:}
%\begin{verbatim}   	21Oct99	- Jing Guo <guo@dao.gsfc.nasa.gov>
%  		- initial prototype/prolog/code\end{verbatim}
 
%/////////////////////////////////////////////////////////////
 
\mbox{}\hrulefill\ 
 
  \subsubsection{ciL2\_ - check in as a logical rank 2 array}

\bigskip{\sf INTERFACE:}
\begin{verbatim} 
     subroutine ciL2_(marg,thread)
       logical,dimension(:,:),intent(in) :: marg
       character(len=*),intent(in)       :: thread
 \end{verbatim}%{\sf REVISION HISTORY:}
%\begin{verbatim}   	21Oct99	- Jing Guo <guo@dao.gsfc.nasa.gov>
%  		- initial prototype/prolog/code\end{verbatim}
 
%/////////////////////////////////////////////////////////////
 
\mbox{}\hrulefill\ 
 
  \subsubsection{ciL3\_ - check in as a logical rank 3 array}

\bigskip{\sf INTERFACE:}
\begin{verbatim} 
     subroutine ciL3_(marg,thread)
       logical,dimension(:,:,:),intent(in) :: marg
       character(len=*),intent(in)         :: thread
 \end{verbatim}%{\sf REVISION HISTORY:}
%\begin{verbatim}   	21Oct99	- Jing Guo <guo@dao.gsfc.nasa.gov>
%  		- initial prototype/prolog/code\end{verbatim}
 
%/////////////////////////////////////////////////////////////
 
\mbox{}\hrulefill\ 
 
  \subsubsection{ciL4\_ - check in as a logical rank 4 array}

\bigskip{\sf INTERFACE:}
\begin{verbatim} 
     subroutine ciL4_(marg,thread)
       logical,dimension(:,:,:,:),intent(in) :: marg
       character(len=*),intent(in)           :: thread
 \end{verbatim}%{\sf REVISION HISTORY:}
%\begin{verbatim}   	21Oct99	- Jing Guo <guo@dao.gsfc.nasa.gov>
%  		- initial prototype/prolog/code\end{verbatim}
 
%/////////////////////////////////////////////////////////////
 
\mbox{}\hrulefill\ 
 
  \subsubsection{ciC0\_ - check in as a character scalar} 
  
\bigskip{\sf INTERFACE:}
\begin{verbatim} 
     subroutine ciC0_(marg,thread)
       character(len=*),intent(in) :: marg
       character(len=*),intent(in) :: thread
 \end{verbatim}%{\sf REVISION HISTORY:}
%\begin{verbatim}   	21Oct99	- Jing Guo <guo@dao.gsfc.nasa.gov>
%  		- initial prototype/prolog/code\end{verbatim}
 
%/////////////////////////////////////////////////////////////
 
\mbox{}\hrulefill\ 

  \subsubsection{ciC1\_ - check in as a character rank 1 array}

\bigskip{\sf INTERFACE:}
\begin{verbatim} 
     subroutine ciC1_(marg,thread)
       character(len=*),dimension(:),intent(in) :: marg
       character(len=*),intent(in)              :: thread
 \end{verbatim}%{\sf REVISION HISTORY:}
%\begin{verbatim}   	21Oct99	- Jing Guo <guo@dao.gsfc.nasa.gov>
%  		- initial prototype/prolog/code\end{verbatim}
 
%/////////////////////////////////////////////////////////////
 
\mbox{}\hrulefill\ 

  \subsubsection{ciC2\_ - check in as a character rank 2 array}

\bigskip{\sf INTERFACE:}
\begin{verbatim} 
     subroutine ciC2_(marg,thread)
       character(len=*),dimension(:,:),intent(in) :: marg
       character(len=*),intent(in)                :: thread
 \end{verbatim}%{\sf REVISION HISTORY:}
%\begin{verbatim}   	21Oct99	- Jing Guo <guo@dao.gsfc.nasa.gov>
%  		- initial prototype/prolog/code\end{verbatim}
 
%/////////////////////////////////////////////////////////////
 
\mbox{}\hrulefill\ 
 
  \subsubsection{ciC3\_ - check in as a character rank 3 array}

\bigskip{\sf INTERFACE:}
\begin{verbatim} 
     subroutine ciC3_(marg,thread)
       character(len=*),dimension(:,:,:),intent(in) :: marg
       character(len=*),intent(in)                  :: thread
 \end{verbatim}%{\sf REVISION HISTORY:}
%\begin{verbatim}   	21Oct99	- Jing Guo <guo@dao.gsfc.nasa.gov>
%  		- initial prototype/prolog/code\end{verbatim}
 
%/////////////////////////////////////////////////////////////
 
\mbox{}\hrulefill\ 

  \subsubsection{ciC4\_ - check in as a character rank 4 array}

\bigskip{\sf INTERFACE:}
\begin{verbatim} 
     subroutine ciC4_(marg,thread)
       character(len=*),dimension(:,:,:,:),intent(in) :: marg
       character(len=*),intent(in)                    :: thread
 \end{verbatim}%{\sf REVISION HISTORY:}
%\begin{verbatim}   	21Oct99	- Jing Guo <guo@dao.gsfc.nasa.gov>
%  		- initial prototype/prolog/code\end{verbatim}
 
%/////////////////////////////////////////////////////////////
 
\mbox{}\hrulefill\ 
 
  \subsubsection{ci\_ - check-in allocate activity}

\bigskip{\sf INTERFACE:}
\begin{verbatim} 
     subroutine ci_(nword,thread)
       integer,intent(in)          :: nword
       character(len=*),intent(in) :: thread
 \end{verbatim}%{\sf REVISION HISTORY:}
%\begin{verbatim}   	13Feb98 - Jing Guo <guo@thunder> - initial prototype/prolog/code\end{verbatim}
 
%/////////////////////////////////////////////////////////////
 
\mbox{}\hrulefill\ 
 
  \subsubsection{coI0\_ - check in as an integer scalar}

\bigskip{\sf INTERFACE:}
\begin{verbatim} 
     subroutine coI0_(marg,thread)
       integer,intent(in)          :: marg
       character(len=*),intent(in) :: thread
 \end{verbatim}%{\sf REVISION HISTORY:}
%\begin{verbatim}   	21Oct99	- Jing Guo <guo@dao.gsfc.nasa.gov>
%  		- initial prototype/prolog/code\end{verbatim}
 
%/////////////////////////////////////////////////////////////
 
\mbox{}\hrulefill\ 
 
  \subsubsection{coI1\_ - check in as an integer rank 1 array}

\bigskip{\sf INTERFACE:}
\begin{verbatim} 
     subroutine coI1_(marg,thread)
       integer,dimension(:),intent(in) :: marg
       character(len=*),intent(in)     :: thread
 \end{verbatim}%{\sf REVISION HISTORY:}
%\begin{verbatim}   	21Oct99	- Jing Guo <guo@dao.gsfc.nasa.gov>
%  		- initial prototype/prolog/code\end{verbatim}
 
%/////////////////////////////////////////////////////////////
 
\mbox{}\hrulefill\ 
 
  \subsubsection{coI2\_ - check in as an integer rank 2 array}

\bigskip{\sf INTERFACE:}
\begin{verbatim} 
     subroutine coI2_(marg,thread)
       integer,dimension(:,:),intent(in) :: marg
       character(len=*),intent(in)       :: thread
 \end{verbatim}%{\sf REVISION HISTORY:}
%\begin{verbatim}   	21Oct99	- Jing Guo <guo@dao.gsfc.nasa.gov>
%  		- initial prototype/prolog/code\end{verbatim}
 
%/////////////////////////////////////////////////////////////
 
\mbox{}\hrulefill\ 
 
  \subsubsection{coI3\_ - check in as an integer rank 3 array}

\bigskip{\sf INTERFACE:}
\begin{verbatim} 
     subroutine coI3_(marg,thread)
       integer,dimension(:,:,:),intent(in) :: marg
       character(len=*),intent(in)         :: thread
 \end{verbatim}%{\sf REVISION HISTORY:}
%\begin{verbatim}   	21Oct99	- Jing Guo <guo@dao.gsfc.nasa.gov>
%  		- initial prototype/prolog/code\end{verbatim}
 
%/////////////////////////////////////////////////////////////
 
\mbox{}\hrulefill\ 
 
  \subsubsection{coI4\_ - check in as an integer rank 4 array}
  
\bigskip{\sf INTERFACE:}
\begin{verbatim} 
     subroutine coI4_(marg,thread)
       integer,dimension(:,:,:,:),intent(in) :: marg
       character(len=*),intent(in)           :: thread
 \end{verbatim}%{\sf REVISION HISTORY:}
%\begin{verbatim}   	21Oct99	- Jing Guo <guo@dao.gsfc.nasa.gov>
%  		- initial prototype/prolog/code\end{verbatim}
 
%/////////////////////////////////////////////////////////////
 
\mbox{}\hrulefill\ 
 
  \subsubsection{coR0\_ - check in as a real(SP) scalar}

\bigskip{\sf INTERFACE:}
\begin{verbatim} 
     subroutine coR0_(marg,thread)
       real(SP),intent(in)         :: marg
       character(len=*),intent(in) :: thread
 \end{verbatim}%{\sf REVISION HISTORY:}
%\begin{verbatim}   	21Oct99	- Jing Guo <guo@dao.gsfc.nasa.gov>
%  		- initial prototype/prolog/code\end{verbatim}
 
%/////////////////////////////////////////////////////////////
 
\mbox{}\hrulefill\ 
 
  \subsubsection{coR1\_ - check in as a real(SP) rank 1 array}

\bigskip{\sf INTERFACE:}
\begin{verbatim} 
     subroutine coR1_(marg,thread)
       real(SP),dimension(:),intent(in) :: marg
       character(len=*),intent(in)      :: thread
 \end{verbatim}%{\sf REVISION HISTORY:}
%\begin{verbatim}   	21Oct99	- Jing Guo <guo@dao.gsfc.nasa.gov>
%  		- initial prototype/prolog/code\end{verbatim}
 
%/////////////////////////////////////////////////////////////
 
\mbox{}\hrulefill\ 

  \subsubsection{coR2\_ - check in as a real(SP) rank 2 array}

\bigskip{\sf INTERFACE:}
\begin{verbatim} 
     subroutine coR2_(marg,thread)
       real(SP),dimension(:,:),intent(in) :: marg
       character(len=*),intent(in)        :: thread
 \end{verbatim}%{\sf REVISION HISTORY:}
%\begin{verbatim}   	21Oct99	- Jing Guo <guo@dao.gsfc.nasa.gov>
%  		- initial prototype/prolog/code\end{verbatim}
 
%/////////////////////////////////////////////////////////////
 
\mbox{}\hrulefill\ 
 
  \subsubsection{coR3\_ - check in as a real(SP) rank 3 array}

\bigskip{\sf INTERFACE:}
\begin{verbatim} 
     subroutine coR3_(marg,thread)
       real(SP),dimension(:,:,:),intent(in) :: marg
       character(len=*),intent(in)          :: thread
 \end{verbatim}%{\sf REVISION HISTORY:}
%\begin{verbatim}   	21Oct99	- Jing Guo <guo@dao.gsfc.nasa.gov>
%  		- initial prototype/prolog/code\end{verbatim}
 
%/////////////////////////////////////////////////////////////
 
\mbox{}\hrulefill\ 
 
  \subsubsection{coR4\_ - check in as a real(SP) rank 4 array}

\bigskip{\sf INTERFACE:}
\begin{verbatim} 
     subroutine coR4_(marg,thread)
       real(SP),dimension(:,:,:,:),intent(in) :: marg
       character(len=*),intent(in)            :: thread
 \end{verbatim}%{\sf REVISION HISTORY:}
%\begin{verbatim}   	21Oct99	- Jing Guo <guo@dao.gsfc.nasa.gov>
%  		- initial prototype/prolog/code\end{verbatim}
 
%/////////////////////////////////////////////////////////////
 
\mbox{}\hrulefill\ 
 
  \subsubsection{coD0\_ - check in as a real(DP) scalar}

\bigskip{\sf INTERFACE:}
\begin{verbatim} 
     subroutine coD0_(marg,thread)
       real(DP),intent(in)         :: marg
       character(len=*),intent(in) :: thread
 \end{verbatim}%{\sf REVISION HISTORY:}
%\begin{verbatim}   	21Oct99	- Jing Guo <guo@dao.gsfc.nasa.gov>
%  		- initial prototype/prolog/code\end{verbatim}
 
%/////////////////////////////////////////////////////////////
 
\mbox{}\hrulefill\ 
 
  \subsubsection{coD1\_ - check in as a real(DP) rank 1 array}

\bigskip{\sf INTERFACE:}
\begin{verbatim} 
     subroutine coD1_(marg,thread)
       real(DP),dimension(:),intent(in) :: marg
       character(len=*),intent(in)      :: thread
 \end{verbatim}%{\sf REVISION HISTORY:}
%\begin{verbatim}   	21Oct99	- Jing Guo <guo@dao.gsfc.nasa.gov>
%  		- initial prototype/prolog/code\end{verbatim}
 
%/////////////////////////////////////////////////////////////
 
\mbox{}\hrulefill\ 

  \subsubsection{coD2\_ - check in as a real(DP) rank 2 array}

\bigskip{\sf INTERFACE:}
\begin{verbatim} 
     subroutine coD2_(marg,thread)
       real(DP),dimension(:,:),intent(in) :: marg
       character(len=*),intent(in)        :: thread
 \end{verbatim}%{\sf REVISION HISTORY:}
%\begin{verbatim}   	21Oct99	- Jing Guo <guo@dao.gsfc.nasa.gov>
%  		- initial prototype/prolog/code\end{verbatim}
 
%/////////////////////////////////////////////////////////////
 
\mbox{}\hrulefill\ 
 
  \subsubsection{coD3\_ - check in as a real(DP) rank 3 array}

\bigskip{\sf INTERFACE:}
\begin{verbatim} 
     subroutine coD3_(marg,thread)
       real(DP),dimension(:,:,:),intent(in) :: marg
       character(len=*),intent(in)          :: thread
 \end{verbatim}%{\sf REVISION HISTORY:}
%\begin{verbatim}   	21Oct99	- Jing Guo <guo@dao.gsfc.nasa.gov>
%  		- initial prototype/prolog/code\end{verbatim}
 
%/////////////////////////////////////////////////////////////
 
\mbox{}\hrulefill\ 
 
  \subsubsection{coD4\_ - check in as a real(DP) rank 4 array}

\bigskip{\sf INTERFACE:}
\begin{verbatim} 
     subroutine coD4_(marg,thread)
       real(DP),dimension(:,:,:,:),intent(in) :: marg
       character(len=*),intent(in)            :: thread
 \end{verbatim}%{\sf REVISION HISTORY:}
%\begin{verbatim}   	21Oct99	- Jing Guo <guo@dao.gsfc.nasa.gov>
%  		- initial prototype/prolog/code\end{verbatim}
 
%/////////////////////////////////////////////////////////////
 
\mbox{}\hrulefill\ 
 
  \subsubsection{coL0\_ - check in as a logical scalar}

\bigskip{\sf INTERFACE:}
\begin{verbatim} 
     subroutine coL0_(marg,thread)
       logical,intent(in)          :: marg
       character(len=*),intent(in) :: thread
 \end{verbatim}%{\sf REVISION HISTORY:}
%\begin{verbatim}   	21Oct99	- Jing Guo <guo@dao.gsfc.nasa.gov>
%  		- initial prototype/prolog/code\end{verbatim}
 
%/////////////////////////////////////////////////////////////
 
\mbox{}\hrulefill\ 
 
  \subsubsection{coL1\_ - check in as a logical rank 1 array}

\bigskip{\sf INTERFACE:}
\begin{verbatim} 
     subroutine coL1_(marg,thread)
       logical,dimension(:),intent(in) :: marg
       character(len=*),intent(in)     :: thread
 \end{verbatim}%{\sf REVISION HISTORY:}
%\begin{verbatim}   	21Oct99	- Jing Guo <guo@dao.gsfc.nasa.gov>
%  		- initial prototype/prolog/code\end{verbatim}
 
%/////////////////////////////////////////////////////////////
 
\mbox{}\hrulefill\ 
 
  \subsubsection{coL2\_ - check in as a logical rank 2 array}

\bigskip{\sf INTERFACE:}
\begin{verbatim} 
     subroutine coL2_(marg,thread)
       logical,dimension(:,:),intent(in) :: marg
       character(len=*),intent(in)       :: thread
 \end{verbatim}%{\sf REVISION HISTORY:}
%\begin{verbatim}   	21Oct99	- Jing Guo <guo@dao.gsfc.nasa.gov>
%  		- initial prototype/prolog/code\end{verbatim}
 
%/////////////////////////////////////////////////////////////
 
\mbox{}\hrulefill\ 
 
  \subsubsection{coL3\_ - check in as a logical rank 3 array}

\bigskip{\sf INTERFACE:}
\begin{verbatim} 
     subroutine coL3_(marg,thread)
       logical,dimension(:,:,:),intent(in) :: marg
       character(len=*),intent(in)         :: thread
 \end{verbatim}%{\sf REVISION HISTORY:}
%\begin{verbatim}   	21Oct99	- Jing Guo <guo@dao.gsfc.nasa.gov>
%  		- initial prototype/prolog/code\end{verbatim}
 
%/////////////////////////////////////////////////////////////
 
\mbox{}\hrulefill\ 

  \subsubsection{coL4\_ - check in as a logical rank 4 array}

\bigskip{\sf INTERFACE:}
\begin{verbatim} 
     subroutine coL4_(marg,thread)
       logical,dimension(:,:,:,:),intent(in) :: marg
       character(len=*),intent(in)           :: thread
 \end{verbatim}%{\sf REVISION HISTORY:}
%\begin{verbatim}   	21Oct99	- Jing Guo <guo@dao.gsfc.nasa.gov>
%  		- initial prototype/prolog/code\end{verbatim}
 
%/////////////////////////////////////////////////////////////
 
\mbox{}\hrulefill\ 
 
  \subsubsection{coC0\_ - check in as a character scalar}

\bigskip{\sf INTERFACE:}
\begin{verbatim} 
     subroutine coC0_(marg,thread)
       character(len=*),intent(in) :: marg
       character(len=*),intent(in) :: thread
 \end{verbatim}%{\sf REVISION HISTORY:}
%\begin{verbatim}   	21Oct99	- Jing Guo <guo@dao.gsfc.nasa.gov>
%  		- initial prototype/prolog/code\end{verbatim}
 
%/////////////////////////////////////////////////////////////
 
\mbox{}\hrulefill\ 
 
  \subsubsection{coC1\_ - check in as a character rank 1 array}

\bigskip{\sf INTERFACE:}
\begin{verbatim} 
     subroutine coC1_(marg,thread)
       character(len=*),dimension(:),intent(in) :: marg
       character(len=*),intent(in)              :: thread
 \end{verbatim}%{\sf REVISION HISTORY:}
%\begin{verbatim}   	21Oct99	- Jing Guo <guo@dao.gsfc.nasa.gov>
%  		- initial prototype/prolog/code\end{verbatim}
 
%/////////////////////////////////////////////////////////////
 
\mbox{}\hrulefill\ 

  \subsubsection{coC2\_ - check in as a character rank 2 array}

\bigskip{\sf INTERFACE:}
\begin{verbatim} 
     subroutine coC2_(marg,thread)
       character(len=*),dimension(:,:),intent(in) :: marg
       character(len=*),intent(in)                :: thread
 \end{verbatim}%{\sf REVISION HISTORY:}
%\begin{verbatim}   	21Oct99	- Jing Guo <guo@dao.gsfc.nasa.gov>
%  		- initial prototype/prolog/code\end{verbatim}
 
%/////////////////////////////////////////////////////////////
 
\mbox{}\hrulefill\ 

  \subsubsection{coC3\_ - check in as a character rank 3 array}

\bigskip{\sf INTERFACE:}
\begin{verbatim} 
     subroutine coC3_(marg,thread)
       character(len=*),dimension(:,:,:),intent(in) :: marg
       character(len=*),intent(in)                  :: thread
 \end{verbatim}%{\sf REVISION HISTORY:}
%\begin{verbatim}   	21Oct99	- Jing Guo <guo@dao.gsfc.nasa.gov>
%  		- initial prototype/prolog/code\end{verbatim}
 
%/////////////////////////////////////////////////////////////
 
\mbox{}\hrulefill\ 
 
  \subsubsection{coC4\_ - check in as a character rank 4 array}

\bigskip{\sf INTERFACE:}
\begin{verbatim} 
     subroutine coC4_(marg,thread)
       character(len=*),dimension(:,:,:,:),intent(in) :: marg
       character(len=*),intent(in)                    :: thread
 \end{verbatim}%{\sf REVISION HISTORY:}
%\begin{verbatim}   	21Oct99	- Jing Guo <guo@dao.gsfc.nasa.gov>
%  		- initial prototype/prolog/code\end{verbatim}
 
%/////////////////////////////////////////////////////////////
 
\mbox{}\hrulefill\ 
 
  \subsubsection{co\_ - check-out allocate activity}

\bigskip{\sf INTERFACE:}
\begin{verbatim} 
     subroutine co_(nword,thread)
       integer,intent(in)          :: nword
       character(len=*),intent(in) :: thread
 \end{verbatim}%{\sf REVISION HISTORY:}
%\begin{verbatim}   	13Feb98 - Jing Guo <guo@thunder> - initial prototype/prolog/code\end{verbatim}
 
%/////////////////////////////////////////////////////////////
 
\mbox{}\hrulefill\ 
 
  \subsubsection{cix\_ - handling macro ALLOC\_() error}

\bigskip{\sf INTERFACE:}
\begin{verbatim} 
     subroutine cix_(thread,stat,fnam,line)
       character(len=*),intent(in) :: thread
       integer,intent(in)          :: stat
       character(len=*),intent(in) :: fnam
       integer,intent(in)          :: line
 
 \end{verbatim}%{\sf REVISION HISTORY:}
%\begin{verbatim}   	13Feb98 - Jing Guo <guo@thunder> - initial prototype/prolog/code\end{verbatim}
 
%/////////////////////////////////////////////////////////////
 
\mbox{}\hrulefill\ 
 
  \subsubsection{cox\_ - handling macro DEALLOC\_() error}

\bigskip{\sf INTERFACE:}
\begin{verbatim} 
     subroutine cox_(thread,stat,fnam,line)
       character(len=*),intent(in) :: thread
       integer,intent(in)          :: stat
       character(len=*),intent(in) :: fnam
       integer,intent(in)          :: line
 \end{verbatim}%{\sf REVISION HISTORY:}
%\begin{verbatim}   	13Feb98 - Jing Guo <guo@thunder> - initial prototype/prolog/code\end{verbatim}
 
%/////////////////////////////////////////////////////////////
 
\mbox{}\hrulefill\ 
 
  \subsubsection{flush\_ - balancing the up-to-date ci/co calls}

\bigskip{\sf INTERFACE:}
\begin{verbatim} 
     subroutine flush_(lu)
       integer,intent(in) :: lu
 \end{verbatim}%{\sf REVISION HISTORY:}
%\begin{verbatim}   	17Feb98 - Jing Guo <guo@thunder> - initial prototype/prolog/code\end{verbatim}
 
%/////////////////////////////////////////////////////////////
 
\mbox{}\hrulefill\ 
 
  \subsubsection{wcount\_ - generate word count output with unit}

\bigskip{\sf INTERFACE:}
\begin{verbatim} 
     subroutine wcount_(wknt,cknt)
       integer,         intent(in)  :: wknt ! given an integer value
       character(len=6),intent(out) :: cknt ! return a string value
 \end{verbatim}%{\sf REVISION HISTORY:}
%\begin{verbatim}   	17Feb98 - Jing Guo <guo@thunder> - initial prototype/prolog/code\end{verbatim}
 
%/////////////////////////////////////////////////////////////
 
\mbox{}\hrulefill\ 
 
  \subsubsection{lookup\_ - search/insert a name in a list}

\bigskip{\sf INTERFACE:}
\begin{verbatim} 
     function lookup_(thread)
       character(len=*),intent(in) :: thread
       integer                     :: lookup_
 \end{verbatim}%{\sf REVISION HISTORY:}
%\begin{verbatim}   	17Feb98 - Jing Guo <guo@thunder> - initial prototype/prolog/code\end{verbatim}
 
%/////////////////////////////////////////////////////////////
 
\mbox{}\hrulefill\ 
 
  \subsubsection{reset\_ - initialize the module data structure}

\bigskip{\sf INTERFACE:}
\begin{verbatim} 
     subroutine reset_()
 \end{verbatim}%{\sf REVISION HISTORY:}
%\begin{verbatim}   	16Mar98 - Jing Guo <guo@thunder> - initial prototype/prolog/code\end{verbatim}

%\markboth{Left}{Source File: m\_mpif90.F90,  Date: Thu Nov 15 22:24:27 GMT 2001
%}

 
%/////////////////////////////////////////////////////////////
 
\mbox{}\hrulefill\ 
 
  \subsection{Module m\_mpif90 - a Fortran 90 style MPI module interface. }

     By wrapping \verb'include "mpif.h"' into a module, \verb"m_mpif()"
     provides an easy way to
  \begin{itemize}
    \item avoid the problem with {\sl fixed} or {\sl free} formatted
  	Fortran 90 files;
    \item provide protections with only a limited set of \verb"PUBLIC"
  	variables; and
    \item be extended to a MPI Fortran 90 binding.
  \end{itemize}
  
\bigskip{\sf INTERFACE:}
\begin{verbatim} 
     module m_mpif90
       public :: MP_type
       public :: MP_2type
 
       public :: MP_INTEGER
       public :: MP_REAL
       public :: MP_DOUBLE_PRECISION
       public :: MP_LOGICAL
       public :: MP_CHARACTER
 
       public :: MP_REAL4
       public :: MP_REAL8
 
       public :: MP_2INTEGER
       public :: MP_2REAL
       public :: MP_2DOUBLE_PRECISION
 
       public :: MP_COMM_WORLD
       public :: MP_COMM_NULL
 
       public :: MP_SUM
       public :: MP_PROD
       public :: MP_MIN
       public :: MP_MAX
 
       public :: MP_MINLOC
       public :: MP_MAXLOC
 
       public :: MP_MAX_ERROR_STRING
 
       public :: MP_init
       public :: MP_initialized
       public :: MP_finalize
       public :: MP_abort
 
       public :: MP_wtime
       public :: MP_wtick
 
       public :: MP_comm_size
       public :: MP_comm_rank
       public :: MP_comm_dup
       public :: MP_comm_free
 
       public :: MP_cart_create
       public :: MP_dims_create
       public :: MP_cart_coords
       public :: MP_cart_rank
 
       public :: MP_error_string
 
       public :: MP_perr
 
       public :: MP_STATUS_SIZE
       public :: MP_status
 
       public :: MP_log2
 \end{verbatim}%{\sf REVISION HISTORY:}
%\begin{verbatim}   	09Dec97 - Jing Guo <guo@thunder> - initial prototyping/coding.
%  		. started with everything public, without any interface
%  		  declaration.
%  		. Then limited to only variables current expected to
%  		  be used.
%  	\end{verbatim}
 
%/////////////////////////////////////////////////////////////
 
\mbox{}\hrulefill\ 

  \subsubsection{typeI\_ - return MPI datatype of INTEGER}

\bigskip{\sf INTERFACE:}
\begin{verbatim} 
     function typeI_(ival)
       integer,intent(in) :: ival
       integer            :: typeI_
 \end{verbatim}%{\sf REVISION HISTORY:}
%\begin{verbatim}   	28Sep99	- Jing Guo <guo@dao.gsfc.nasa.gov>
%  		- initial prototype/prolog/code\end{verbatim}
 
%/////////////////////////////////////////////////////////////
 
\mbox{}\hrulefill\ 
 
  \subsubsection{typeL\_ - return MPI datatype of LOGICAL}

\bigskip{\sf INTERFACE:}
\begin{verbatim} 
     function typeL_(lval)
       logical,intent(in) :: lval
       integer            :: typeL_
 \end{verbatim}%{\sf REVISION HISTORY:}
%\begin{verbatim}   	28Sep99	- Jing Guo <guo@dao.gsfc.nasa.gov>
%  		- initial prototype/prolog/code\end{verbatim}
 
%/////////////////////////////////////////////////////////////
 
\mbox{}\hrulefill\ 

  \subsubsection{typeC\_ - return MPI datatype of CHARACTER}

\bigskip{\sf INTERFACE:}
\begin{verbatim} 
     function typeC_(cval)
       character(len=*),intent(in) :: cval
       integer                     :: typeC_
 \end{verbatim}%{\sf REVISION HISTORY:}
%\begin{verbatim}   	28Sep99	- Jing Guo <guo@dao.gsfc.nasa.gov>
%  		- initial prototype/prolog/code\end{verbatim}
 
%/////////////////////////////////////////////////////////////
 
\mbox{}\hrulefill\ 
 
  \subsubsection{typeSP\_ - return MPI datatype of single precision REAL}

\bigskip{\sf INTERFACE:}
\begin{verbatim} 
     function typeSP_(rval)
       real(SP),intent(in) :: rval
       integer             :: typeSP_
 \end{verbatim}%{\sf REVISION HISTORY:}
%\begin{verbatim}   	28Sep99	- Jing Guo <guo@dao.gsfc.nasa.gov>
%  		- initial prototype/prolog/code\end{verbatim}
 
%/////////////////////////////////////////////////////////////
 
\mbox{}\hrulefill\ 

  \subsubsection{typeDP\_ - return MPI datatype of double precision REAL}

\bigskip{\sf INTERFACE:}
\begin{verbatim} 
     function typeDP_(rval)
       real(DP),intent(in) :: rval
       integer             :: typeDP_
 \end{verbatim}%{\sf REVISION HISTORY:}
%\begin{verbatim}   	28Sep99	- Jing Guo <guo@dao.gsfc.nasa.gov>
%  		- initial prototype/prolog/code\end{verbatim}
 
%/////////////////////////////////////////////////////////////
 
\mbox{}\hrulefill\ 

  \subsubsection{typeI1\_ - return MPI datatype of INTEGER}

\bigskip{\sf INTERFACE:}
\begin{verbatim} 
     function typeI1_(ival)
       integer,dimension(:),intent(in) :: ival
       integer                         :: typeI1_
 \end{verbatim}%{\sf REVISION HISTORY:}
%\begin{verbatim}   	28Sep99	- Jing Guo <guo@dao.gsfc.nasa.gov>
%  		- initial prototype/prolog/code\end{verbatim}
 
%/////////////////////////////////////////////////////////////
 
\mbox{}\hrulefill\ 
 
  \subsubsection{typeL1\_ - return MPI datatype of LOGICAL}

\bigskip{\sf INTERFACE:}
\begin{verbatim} 
     function typeL1_(lval)
       logical,dimension(:),intent(in) :: lval
       integer                         :: typeL1_
 \end{verbatim}%{\sf REVISION HISTORY:}
%\begin{verbatim}   	28Sep99	- Jing Guo <guo@dao.gsfc.nasa.gov>
%  		- initial prototype/prolog/code\end{verbatim}
 
%/////////////////////////////////////////////////////////////
 
\mbox{}\hrulefill\ 
 
  \subsubsection{typeC1\_ - return MPI datatype of CHARACTER}

\bigskip{\sf INTERFACE:}
\begin{verbatim} 
     function typeC1_(cval)
       character(len=*),dimension(:),intent(in) :: cval
       integer                                  :: typeC1_
 \end{verbatim}%{\sf REVISION HISTORY:}
%\begin{verbatim}   	28Sep99	- Jing Guo <guo@dao.gsfc.nasa.gov>
%  		- initial prototype/prolog/code\end{verbatim}
 
%/////////////////////////////////////////////////////////////
 
\mbox{}\hrulefill\ 
 
  \subsubsection{typeSP1\_ - return MPI datatype of single precision REAL}

\bigskip{\sf INTERFACE:}
\begin{verbatim} 
     function typeSP1_(rval)
       real(SP),dimension(:),intent(in) :: rval
       integer                          :: typeSP1_
 \end{verbatim}%{\sf REVISION HISTORY:}
%\begin{verbatim}   	28Sep99	- Jing Guo <guo@dao.gsfc.nasa.gov>
%  		- initial prototype/prolog/code\end{verbatim}
 
%/////////////////////////////////////////////////////////////
 
\mbox{}\hrulefill\  

  \subsubsection{typeDP1\_ - return MPI datatype of double precision REAL}

\bigskip{\sf INTERFACE:}
\begin{verbatim} 
     function typeDP1_(rval)
       real(DP),dimension(:),intent(in) :: rval
       integer                          :: typeDP1_
 \end{verbatim}%{\sf REVISION HISTORY:}
%\begin{verbatim}   	28Sep99	- Jing Guo <guo@dao.gsfc.nasa.gov>
%  		- initial prototype/prolog/code\end{verbatim}
 
%/////////////////////////////////////////////////////////////
 
\mbox{}\hrulefill\ 
 
  \subsubsection{typeI2\_ - return MPI datatype of INTEGER}

\bigskip{\sf INTERFACE:}
\begin{verbatim} 
     function typeI2_(ival)
       integer,dimension(:,:),intent(in) :: ival
       integer                           :: typeI2_
 \end{verbatim}%{\sf REVISION HISTORY:}
%\begin{verbatim}   	28Sep99	- Jing Guo <guo@dao.gsfc.nasa.gov>
%  		- initial prototype/prolog/code\end{verbatim}
 
%/////////////////////////////////////////////////////////////
 
\mbox{}\hrulefill\ 
 
  \subsubsection{typeL2\_ - return MPI datatype of LOGICAL}

\bigskip{\sf INTERFACE:}
\begin{verbatim} 
     function typeL2_(lval)
       logical,dimension(:,:),intent(in) :: lval
       integer                           :: typeL2_
 \end{verbatim}%{\sf REVISION HISTORY:}
%\begin{verbatim}   	28Sep99	- Jing Guo <guo@dao.gsfc.nasa.gov>
%  		- initial prototype/prolog/code\end{verbatim}
 
%/////////////////////////////////////////////////////////////
 
\mbox{}\hrulefill\ 

  \subsubsection{typeC2\_ - return MPI datatype of CHARACTER}

\bigskip{\sf INTERFACE:}
\begin{verbatim} 
     function typeC2_(cval)
       character(len=*),dimension(:,:),intent(in) :: cval
       integer                                    :: typeC2_
 \end{verbatim}%{\sf REVISION HISTORY:}
%\begin{verbatim}   	28Sep99	- Jing Guo <guo@dao.gsfc.nasa.gov>
%  		- initial prototype/prolog/code\end{verbatim}
 
%/////////////////////////////////////////////////////////////
 
\mbox{}\hrulefill\ 
 
  \subsubsection{typeSP2\_ - return MPI datatype of single precision REAL}

\bigskip{\sf INTERFACE:}
\begin{verbatim} 
     function typeSP2_(rval)
       real(SP),dimension(:,:),intent(in) :: rval
       integer                            :: typeSP2_
 \end{verbatim}%{\sf REVISION HISTORY:}
%\begin{verbatim}   	28Sep99	- Jing Guo <guo@dao.gsfc.nasa.gov>
%  		- initial prototype/prolog/code\end{verbatim}
 
%/////////////////////////////////////////////////////////////
 
\mbox{}\hrulefill\ 
 
  \subsubsection{typeDP2\_ - return MPI datatype of double precision REAL}

\bigskip{\sf INTERFACE:}
\begin{verbatim} 
     function typeDP2_(rval)
       real(DP),dimension(:,:),intent(in) :: rval
       integer                            :: typeDP2_
 \end{verbatim}%{\sf REVISION HISTORY:}
%\begin{verbatim}   	28Sep99	- Jing Guo <guo@dao.gsfc.nasa.gov>
%  		- initial prototype/prolog/code\end{verbatim}
 
%/////////////////////////////////////////////////////////////
 
\mbox{}\hrulefill\ 

  \subsubsection{typeI3\_ - return MPI datatype of INTEGER}

\bigskip{\sf INTERFACE:}
\begin{verbatim} 
     function typeI3_(ival)
       integer,dimension(:,:,:),intent(in) :: ival
       integer                             :: typeI3_
 \end{verbatim}%{\sf REVISION HISTORY:}
%\begin{verbatim}   	28Sep99	- Jing Guo <guo@dao.gsfc.nasa.gov>
%  		- initial prototype/prolog/code\end{verbatim}
 
%/////////////////////////////////////////////////////////////
 
\mbox{}\hrulefill\ 

  \subsubsection{typeL3\_ - return MPI datatype of LOGICAL}

\bigskip{\sf INTERFACE:}
\begin{verbatim} 
     function typeL3_(lval)
       logical,dimension(:,:,:),intent(in) :: lval
       integer                             :: typeL3_
 \end{verbatim}%{\sf REVISION HISTORY:}
%\begin{verbatim}   	28Sep99	- Jing Guo <guo@dao.gsfc.nasa.gov>
%  		- initial prototype/prolog/code\end{verbatim}
 
%/////////////////////////////////////////////////////////////
 
\mbox{}\hrulefill\ 

  \subsubsection{typeC3\_ - return MPI datatype of CHARACTER}

\bigskip{\sf INTERFACE:}
\begin{verbatim} 
     function typeC3_(cval)
       character(len=*),dimension(:,:,:),intent(in) :: cval
       integer                                      :: typeC3_
 \end{verbatim}%{\sf REVISION HISTORY:}
%\begin{verbatim}   	28Sep99	- Jing Guo <guo@dao.gsfc.nasa.gov>
%  		- initial prototype/prolog/code\end{verbatim}
 
%/////////////////////////////////////////////////////////////
 
\mbox{}\hrulefill\ 
 
  \subsubsection{typeSP3\_ - return MPI datatype of single precision REAL}

\bigskip{\sf INTERFACE:}
\begin{verbatim} 
     function typeSP3_(rval)
       real(SP),dimension(:,:,:),intent(in) :: rval
       integer                              :: typeSP3_
 \end{verbatim}%{\sf REVISION HISTORY:}
%\begin{verbatim}   	28Sep99	- Jing Guo <guo@dao.gsfc.nasa.gov>
%  		- initial prototype/prolog/code\end{verbatim}
 
%/////////////////////////////////////////////////////////////
 
\mbox{}\hrulefill\ 
 
  \subsubsection{typeDP3\_ - return MPI datatype of double precision REAL}

\bigskip{\sf INTERFACE:}
\begin{verbatim} 
     function typeDP3_(rval)
       real(DP),dimension(:,:,:),intent(in) :: rval
       integer                              :: typeDP3_
 \end{verbatim}%{\sf REVISION HISTORY:}
%\begin{verbatim}   	28Sep99	- Jing Guo <guo@dao.gsfc.nasa.gov>
%  		- initial prototype/prolog/code\end{verbatim}
 
%/////////////////////////////////////////////////////////////
 
\mbox{}\hrulefill\ 
 

  \subsubsection{type2I\_ - return MPI datatype of 2INTEGER}

\bigskip{\sf INTERFACE:}
\begin{verbatim} 
     function type2I_(ival)
       integer,intent(in) :: ival
       integer            :: type2I_
 \end{verbatim}%{\sf REVISION HISTORY:}
%\begin{verbatim}   	28Sep99	- Jing Guo <guo@dao.gsfc.nasa.gov>
%  		- initial prototype/prolog/code\end{verbatim}
 
%/////////////////////////////////////////////////////////////
 
\mbox{}\hrulefill\ 
 
  \subsubsection{type2SP\_ - return MPI datatype of single precision 2REAL}

\bigskip{\sf INTERFACE:}
\begin{verbatim} 
     function type2SP_(rval)
       real(SP),intent(in) :: rval
       integer             :: type2SP_
 \end{verbatim}%{\sf REVISION HISTORY:}
%\begin{verbatim}   	28Sep99	- Jing Guo <guo@dao.gsfc.nasa.gov>
%  		- initial prototype/prolog/code\end{verbatim}
 
%/////////////////////////////////////////////////////////////
 
\mbox{}\hrulefill\ 
 
  \subsubsection{type2DP\_ - return MPI datatype of double precision 2REAL}

\bigskip{\sf INTERFACE:}
\begin{verbatim} 
     function type2DP_(rval)
       real(DP),intent(in) :: rval
       integer             :: type2DP_
 \end{verbatim}%{\sf REVISION HISTORY:}
%\begin{verbatim}   	28Sep99	- Jing Guo <guo@dao.gsfc.nasa.gov>
%  		- initial prototype/prolog/code\end{verbatim}
 
%/////////////////////////////////////////////////////////////
 
\mbox{}\hrulefill\ 
 
  \subsubsection{perr\_ - MPI error information hanlder}

\bigskip{\sf INTERFACE:}
\begin{verbatim} 
     subroutine perr_(proc,MP_proc,ierror)
       character(len=*),intent(in) :: proc
       character(len=*),intent(in) :: MP_proc
       integer,intent(in)          :: ierror
 \end{verbatim}%{\sf REVISION HISTORY:}
%\begin{verbatim}   	21Apr98 - Jing Guo <guo@thunder> - initial prototype/prolog/code\end{verbatim}
 
%/////////////////////////////////////////////////////////////
 
\mbox{}\hrulefill\ 
 
  \subsubsection{MP\_log2 - The smallest integer its power of 2 is >= nPE}

\bigskip{\sf INTERFACE:}
\begin{verbatim} 
     function MP_log2(nPE)
       integer,intent(in) :: nPE
       integer            :: MP_log2
 \end{verbatim}%{\sf REVISION HISTORY:}
%\begin{verbatim}   	01Feb00	- Jing Guo <guo@dao.gsfc.nasa.gov>
%  		- initial prototype/prolog/code\end{verbatim}

%\markboth{Left}{Source File: m\_mpout.F90,  Date: Thu Nov 15 22:24:27 GMT 2001
%}

 
%/////////////////////////////////////////////////////////////
 
\mbox{}\hrulefill\ 
 
  \subsection{Module m\_mpout - a multiple but mergable parallel output module }

\bigskip{\sf INTERFACE:}
\begin{verbatim} 
     module m_mpout
       public :: mpout	! The file handle as a Fortran logical unit
 
       public :: mpout_open      ! open the multiple output streams
       public :: mpout_close     ! close the multiple output streams
       public :: mpout_sync      ! sync. the multiple output streams
       public :: mpout_flush     ! flush the multople output streams
       public :: mpout_ison      ! verify if mpout is proper defined
       public :: mpout_log       ! write a message to mpout
       public :: mpout_MASK      ! mask of all but PE 0
       public :: mpout_setflush  ! modify default flush behavior after
                                 ! writes using mpout_log().
 
       interface mpout_open;  module procedure open_;  end interface
       interface mpout_close; module procedure close_; end interface
       interface mpout_sync;  module procedure sync_;  end interface
       interface mpout_flush; module procedure flush_; end interface
       interface mpout_ison;  module procedure ison_;  end interface
       interface mpout_log;   module procedure	&
                   logi_,	&
                   logr_,	&
                   logd_,	&
                   log_;   end interface
       interface mpout_setflush; module procedure	&
                   setflush_; end interface
 \end{verbatim}%{\sf REVISION HISTORY:}
%\begin{verbatim}  	
%         06Feb01 - Tom Clune
%  	        . Added interface to modify default flush
%                   behavior in log routines.
%  	12Oct00	- Jing Guo
%  		. Revised the module to make its behaviors closer to
%  		  what developers might want to see.  The conflict
%  		  between the principles for lately added procedures
%  		  and for the earlier implementation is merged.
%  		  The earlier design expected more disciplined usage,
%  		  while the later additions expected applications with
%  		  less care.
%  		. Added EXAMPLES for the module.
%  	28Sep99 - Jing Guo <guo@thunder>
%  		- Added additional calls to support the "Violet" system
%  		  development.
%   	25Feb98 - Jing Guo <guo@thunder> - initial prototype/prolog/code\end{verbatim}
{\sf SEE ALSO:}
\begin{verbatim}  	
   !DESIGN ISSUES:
\end{verbatim}
   \begin{itemize}
   \item	It might be considered useful to implement this module to be
  	applicable to a given {\sl communicator}.   The argument
  	taken now is to only have one multiple output stream handle
  	per excution.  This is consistent with \verb"stdout" in the
  	traditional sense. (Jing Guo, 25Feb98)
   \item \verb"mpout_log()" is implemented in a way producing output
  	only if \verb"mpout_ison()" (being \verb".true.").  The reason
  	of not implementing a default output such as \verb"stdout", is
  	hoping to provent too many unexpected output when the system is
  	switched to a multiple PE system.  The design principle for
  	this module is that \verb"mpout" is basically {\sl not} the same
  	module as \verb"stdout". (Jing Guo, 28Sep99)
   \end{itemize}
 
%/////////////////////////////////////////////////////////////
 
\mbox{}\hrulefill\ 
 
  \subsubsection{open\_ - open a multiple files with the same name prefix}

\bigskip{\sf INTERFACE:}
\begin{verbatim} 
     subroutine open_(mask,pfix,append)
       integer,optional,intent(in) :: mask
 
                ! It is a mask of PE ranks where mpout will _not_ be on
                ! ("on" == a file is opened).
                !   condition           : status
                ! .not.present(mask)    : on PE 0 only
                ! mask  == -1           : off
                !       ==  0           : on
                ! iand(mask,rank) == 0  : on
                ! iand(maxk,rank) /= 0  : off
 
       character(len=*),optional,intent(in) :: pfix
 
                ! The leading filename component of the mpout file set.
                ! If the argument is not present, pfix='mpout' is
                ! assumed.  Filenames of the mpout file set opened by
                ! this open() are in the form of "pfix.zzz", where
                ! "pfix" is the string defined by argument pfix, and
                ! "zzz" is the PE rank in hexidecimal.
 
       logical,optional,intent(in) :: append
 
                ! If the file is to be positioned as 'append'.  The
                ! default is not to append.
 
   !EXAMPLES:
     . mpout_ison() on no PE but PE 0, where mpout is opened to file
       "mpout.000":
  	call mpout_open()
     . mpout_ison() on every 4 PE starting from 0, where mpout is opened
       to files named, "out.000", "out.004", "out.007", "out.00a", etc.
 
  	call mpout_open(mask=3,pfix='out')
       Note that 3 = "11"b.  Therefore,
  	mask="000011"b
  	PE 0="000000"b	is "clean", mpout_ison()
  	   1="000001"b	is "dirty", .not. on
  	   2="000010"b	is "dirty", .not. on
  	   3="000011"b	is "dirty", .not. on
  	   4="000100"b	is "clean", mpout_ison()
  	   5="000101"b	is "dirty", .not. on
  	   6="000110"b	is "dirty", .not. on
  	   7="000111"b	is "dirty", .not. on
  	   8="001000"b	is "clean", mpout_ison()
  	   9="001001"b	is "dirty", .not. on
  	  10="001010"b	is "dirty", .not. on
  	  11="001011"b	is "dirty", .not. on
  	  12="001100"b	is "clean", mpout_ison()
  	  13="001101"b	is "dirty", .not. on
  	  14="001110"b	is "dirty", .not. on
  	  15="001111"b	is "dirty", .not. on\end{verbatim}%{\sf REVISION HISTORY:}
%\begin{verbatim}  	12Oct00	- Jing Guo
%  		. Revised argument describtions.
%  		. Revised EXAMPLES.
%  		. Removed default sync_() action, which should be a
%  		  choice by users.
%   	25Feb98 - Jing Guo <guo@thunder> - initial prototype/prolog/code\end{verbatim}
 
%/////////////////////////////////////////////////////////////
 
\mbox{}\hrulefill\ 
 
  \subsubsection{close\_ - close the unit opened by open\_}

\bigskip{\sf INTERFACE:}
\begin{verbatim} 
     subroutine close_()
 \end{verbatim}%{\sf REVISION HISTORY:}
%\begin{verbatim}  	12Oct00	- Jing Guo
%  		. Removed default sync_() action, which should be a
%  		  choice of users.
%   	25Feb98 - Jing Guo <guo@thunder> - initial prototype/prolog/code\end{verbatim}
 
%/////////////////////////////////////////////////////////////
 
\mbox{}\hrulefill\ 

  \subsubsection{sync\_ - write a mark for posible later file merging}

\bigskip{\sf INTERFACE:}
\begin{verbatim} 
     subroutine sync_(tag)
       character(len=*),intent(in) :: tag
 \end{verbatim}%{\sf REVISION HISTORY:}
%\begin{verbatim}   	25Feb98 - Jing Guo <guo@thunder> - initial prototype/prolog/code
%\end{verbatim}
{\sf SEE ALSO:}
\begin{verbatim}
   !DESIGN ISSUES:
\end{verbatim}
   \begin{itemize}
   \item	Should the variable \verb"tag" be implemented as an optional
  	argument?  Because the current implementation does not require
  	actual synchronization between all threads of the multiple
  	output streams, forcing the user to supply a unique \verb"tag"
  	would make the final multi-stream merging verifiable.  However,
  	since the \verb"tag"s have not been forced to be unique, the
  	synchronization operations are still symbolic.
  	
   \end{itemize}
 
%/////////////////////////////////////////////////////////////
 
\mbox{}\hrulefill\ 
 
  \subsubsection{flush\_ - flush the multiple output streams}

\bigskip{\sf INTERFACE:}
\begin{verbatim} 
     subroutine flush_()
 \end{verbatim}%{\sf REVISION HISTORY:}
%\begin{verbatim}   	27Feb98 - Jing Guo <guo@thunder> - initial prototype/prolog/code\end{verbatim}
 
%/////////////////////////////////////////////////////////////
 
\mbox{}\hrulefill\ 

  \subsubsection{ison\_ - decide if the current PE has a defined mpout}

     It needs to be checked to avoid undesired output.
  
\bigskip{\sf INTERFACE:}
\begin{verbatim} 
     function ison_()
       logical :: ison_
 \end{verbatim}%{\sf REVISION HISTORY:}
%\begin{verbatim}   	14Sep99	- Jing Guo <guo@dao.gsfc.nasa.gov>
%  		- initial prototype/prolog/code\end{verbatim}
 
%/////////////////////////////////////////////////////////////
 
\mbox{}\hrulefill\ 
 
  \subsubsection{log\_ - write a message to mpout}

\bigskip{\sf INTERFACE:}
\begin{verbatim} 
     subroutine log_(where,message,showrank,flush)
       character(len=*),intent(in) :: where
       character(len=*),intent(in) :: message
       logical,optional,intent(in) :: showrank
       logical,optional,intent(in) :: flush
 \end{verbatim}%{\sf REVISION HISTORY:}
%\begin{verbatim}   	14Sep99	- Jing Guo <guo@dao.gsfc.nasa.gov>
%  		- initial prototype/prolog/code\end{verbatim}
 
%/////////////////////////////////////////////////////////////
 
\mbox{}\hrulefill\ 

  \subsubsection{logi\_ - write a message to mpout}

\bigskip{\sf INTERFACE:}
\begin{verbatim} 
     subroutine logi_(where,message,num,showrank,flush)
       character(len=*),intent(in) :: where
       character(len=*),intent(in) :: message
       integer,intent(in)          :: num
       logical,optional,intent(in) :: showrank
       logical,optional,intent(in) :: flush
 \end{verbatim}%{\sf REVISION HISTORY:}
%\begin{verbatim}   	14Sep99	- Jing Guo <guo@dao.gsfc.nasa.gov>
%  		- initial prototype/prolog/code\end{verbatim}
 
%/////////////////////////////////////////////////////////////
 
\mbox{}\hrulefill\ 

  \subsubsection{logd\_ - write a message to mpout}

\bigskip{\sf INTERFACE:}
\begin{verbatim} 
     subroutine logd_(where,message,val,showrank,flush)
       character(len=*),intent(in) :: where
       character(len=*),intent(in) :: message
       real(DP),intent(in)         :: val
       logical,optional,intent(in) :: showrank
       logical,optional,intent(in) :: flush
 \end{verbatim}%{\sf REVISION HISTORY:}
%\begin{verbatim}   	14Sep99	- Jing Guo <guo@dao.gsfc.nasa.gov>
%  		- initial prototype/prolog/code\end{verbatim}
 
%/////////////////////////////////////////////////////////////
 
\mbox{}\hrulefill\ 
 
  \subsubsection{logr\_ - write a message to mpout}

\bigskip{\sf INTERFACE:}
\begin{verbatim} 
     subroutine logr_(where,message,val,showrank,flush)
       character(len=*),intent(in) :: where
       character(len=*),intent(in) :: message
       real(SP),intent(in)         :: val
       logical,optional,intent(in) :: showrank
       logical,optional,intent(in) :: flush
 \end{verbatim}%{\sf REVISION HISTORY:}
%\begin{verbatim}   	14Sep99	- Jing Guo <guo@dao.gsfc.nasa.gov>
%  		- initial prototype/prolog/code\end{verbatim}
 
%/////////////////////////////////////////////////////////////
 
\mbox{}\hrulefill\ 
 
  \subsubsection{initialize\_ - initialize to a default setting}

\bigskip{\sf INTERFACE:}
\begin{verbatim} 
     subroutine initialize_()
 \end{verbatim}%{\sf REVISION HISTORY:}
%\begin{verbatim}   	05Oct00	- Jing Guo <guo@dao.gsfc.nasa.gov>
%  		- initial prototype/prolog/code\end{verbatim}
 
%/////////////////////////////////////////////////////////////
 
\mbox{}\hrulefill\ 

  \subsubsection{setflush\_ - initialize to a default setting}

\bigskip{\sf INTERFACE:}
\begin{verbatim} 
     subroutine setflush_(flush)
       logical, intent(in) :: flush
 \end{verbatim}%{\sf REVISION HISTORY:}
%\begin{verbatim}   	06Feb01	- Tom Clune <clune@sgi.com>
%  		- initial prototype/prolog/code\end{verbatim}

%\markboth{Left}{Source File: m\_rankMerge.F90,  Date: Thu Nov 15 22:24:28 GMT 2001
%}

 
%/////////////////////////////////////////////////////////////
 
\mbox{}\hrulefill\ 

  \subsection{Module m\_rankMerge - A merging tool through ranking }

\bigskip{\sf INTERFACE:}
\begin{verbatim} 
     module m_rankMerge
       public :: rankSet		! set inital ranks
       public :: rankMerge	! merge two ranks
       public :: IndexedRankMerge ! index-merge two array segments
 
       interface rankSet; module procedure set_; end interface
 
       interface rankMerge; module procedure	&
          imerge_,   &    ! rank-merging two integer arrays
          rmerge_,   &    ! rank-merging two real arrays
          dmerge_,   &    ! rank-merging two dble arrays
          uniq_	          ! merging to rank arrays
       end interface
 
       interface IndexedRankMerge; module procedure	&
          iindexmerge_,   &       ! merging two index arrays of integers
          rindexmerge_,   &       ! merging two index arrays of reals
          dindexmerge_            ! merging two index arrays of dbles
       end interface
 \end{verbatim}%{\sf REVISION HISTORY:}
%\begin{verbatim}   	13Mar00	- Jing Guo <guo@dao.gsfc.nasa.gov>
%  		- initial prototype/prolog/code\end{verbatim}
 
%/////////////////////////////////////////////////////////////
 
\mbox{}\hrulefill\ 
 
  \subsubsection{set\_ - set initial ranking}

\bigskip{\sf INTERFACE:}
\begin{verbatim} 
     subroutine set_(rank)
       integer,dimension(:),intent(out) :: rank
 \end{verbatim}%{\sf REVISION HISTORY:}
%\begin{verbatim}   	13Mar00	- Jing Guo <guo@dao.gsfc.nasa.gov>
%  		- initial prototype/prolog/code\end{verbatim}
 
%/////////////////////////////////////////////////////////////
 
\mbox{}\hrulefill\ 

  \subsubsection{imerge\_ - merge two sorted integer arrays by ranking}

\bigskip{\sf INTERFACE:}
\begin{verbatim} 
     subroutine imerge_(value_i,value_j,krank_i,krank_j,descend)
       integer,dimension(:),intent(in)    :: value_j	! value of j-vec
       integer,dimension(:),intent(in)    :: value_i	! value of i-vec
 
       integer,dimension(:),intent(inout) :: krank_i	! rank of i-vec
       integer,dimension(:),intent(inout) :: krank_j	! rank of j-vec
 
       logical,optional,intent(in) :: descend
 \end{verbatim}%{\sf REVISION HISTORY:}
%\begin{verbatim}   	13Mar00	- Jing Guo <guo@dao.gsfc.nasa.gov>
%  		- initial prototype/prolog/code\end{verbatim}
 
%/////////////////////////////////////////////////////////////
 
\mbox{}\hrulefill\ 
 
  \subsubsection{rmerge\_ - merge two sorted real arrays by ranking}

\bigskip{\sf INTERFACE:}
\begin{verbatim} 
     subroutine rmerge_(value_i,value_j,krank_i,krank_j,descend)
       real(SP),dimension(:),intent(in)    :: value_i	! value of i-vec
       real(SP),dimension(:),intent(in)    :: value_j	! value of j-vec
 
       integer,dimension(:),intent(inout) :: krank_i	! rank of i-vec
       integer,dimension(:),intent(inout) :: krank_j	! rank of j-vec
 
       logical,optional,intent(in) :: descend
 \end{verbatim}%{\sf REVISION HISTORY:}
%\begin{verbatim}   	13Mar00	- Jing Guo <guo@dao.gsfc.nasa.gov>
%  		- initial prototype/prolog/code\end{verbatim}
 
%/////////////////////////////////////////////////////////////
 
\mbox{}\hrulefill\ 
 
  \subsubsection{dmerge\_ - merge two sorted real arrays by ranking}

\bigskip{\sf INTERFACE:}
\begin{verbatim} 
     subroutine dmerge_(value_i,value_j,krank_i,krank_j,descend)
       real(DP),dimension(:),intent(in)   :: value_i	! value of i-vec
       real(DP),dimension(:),intent(in)   :: value_j	! value of j-vec
 
       integer,dimension(:),intent(inout) :: krank_i	! rank of i-vec
       integer,dimension(:),intent(inout) :: krank_j	! rank of j-vec
 
       logical,optional,intent(in) :: descend
 \end{verbatim}%{\sf REVISION HISTORY:}
%\begin{verbatim}   	13Mar00	- Jing Guo <guo@dao.gsfc.nasa.gov>
%  		- initial prototype/prolog/code\end{verbatim}
 
%/////////////////////////////////////////////////////////////
 
\mbox{}\hrulefill\ 
 
  \subsubsection{iindexmerge\_ - merge two sorted integer arrays by ranking}

\bigskip{\sf INTERFACE:}
\begin{verbatim} 
     subroutine iindexmerge_(indx_i,indx_j,value,krank_i,krank_j,descend)
       integer,dimension(:),intent(in)    :: indx_i	! of the i-vec
       integer,dimension(:),intent(in)    :: indx_j	! of the j-vec
       integer,dimension(:),intent(in)    :: value	! of the full
 
       integer,dimension(:),intent(inout) :: krank_i	! rank of i-vec
       integer,dimension(:),intent(inout) :: krank_j	! rank of j-vec
 
       logical,optional,intent(in) :: descend
 \end{verbatim}%{\sf REVISION HISTORY:}
%\begin{verbatim}   	13Mar00	- Jing Guo <guo@dao.gsfc.nasa.gov>
%  		- initial prototype/prolog/code\end{verbatim}
 
%/////////////////////////////////////////////////////////////
 
\mbox{}\hrulefill\ 

  \subsubsection{rindexmerge\_ - merge two sorted real arrays by ranking}

\bigskip{\sf INTERFACE:}
\begin{verbatim} 
     subroutine rindexmerge_(indx_i,indx_j,value,krank_i,krank_j,descend)
       integer,dimension(:),intent(in)    :: indx_i	! of the i-vec
       integer,dimension(:),intent(in)    :: indx_j	! of the j-vec
       real(SP),dimension(:),intent(in)   :: value	! of the full
 
       integer,dimension(:),intent(inout) :: krank_i	! rank of i-vec
       integer,dimension(:),intent(inout) :: krank_j	! rank of j-vec
 
       logical,optional,intent(in) :: descend
 \end{verbatim}%{\sf REVISION HISTORY:}
%\begin{verbatim}   	13Mar00	- Jing Guo <guo@dao.gsfc.nasa.gov>
%  		- initial prototype/prolog/code\end{verbatim}
 
%/////////////////////////////////////////////////////////////
 
\mbox{}\hrulefill\ 
 
  \subsubsection{dindexmerge\_ - merge two sorted real arrays by ranking}

\bigskip{\sf INTERFACE:}
\begin{verbatim} 
     subroutine dindexmerge_(indx_i,indx_j,value,krank_i,krank_j,descend)
       integer,dimension(:),intent(in)    :: indx_i	! of the i-vec
       integer,dimension(:),intent(in)    :: indx_j	! of the j-vec
       real(DP),dimension(:),intent(in)   :: value	! of the full
 
       integer,dimension(:),intent(inout) :: krank_i	! rank of i-vec
       integer,dimension(:),intent(inout) :: krank_j	! rank of j-vec
 
       logical,optional,intent(in) :: descend
 \end{verbatim}%{\sf REVISION HISTORY:}
%\begin{verbatim}   	13Mar00	- Jing Guo <guo@dao.gsfc.nasa.gov>
%  		- initial prototype/prolog/code\end{verbatim}
 
%/////////////////////////////////////////////////////////////
 
\mbox{}\hrulefill\ 
 
  \subsubsection{uniq\_ - merge two rank arrays with unique rank values}

\bigskip{\sf INTERFACE:}
\begin{verbatim} 
     subroutine uniq_(krank_i,krank_j)
       integer,dimension(:),intent(inout) :: krank_i	! rank of i-vec
       integer,dimension(:),intent(inout) :: krank_j	! rank of j-vec
 \end{verbatim}%{\sf REVISION HISTORY:}
%\begin{verbatim}   	13Mar00	- Jing Guo <guo@dao.gsfc.nasa.gov>
%  		- initial prototype/prolog/code\end{verbatim}

%\markboth{Left}{Source File: m\_realkinds.F90,  Date: Thu Nov 15 22:24:28 GMT 2001
%}

 
%/////////////////////////////////////////////////////////////
 
\mbox{}\hrulefill\ 
 
  \subsection{Module m\_realkinds - real KIND definitions }

\bigskip{\sf INTERFACE:}
\begin{verbatim} 
     module m_realkinds
       public :: kind_r4		! real*4
       public :: kind_r8		! real*8
       public :: kind_r		! default real
       public :: SP		! default REAL
       public :: DP		! default DOUBLE_PRECISION
 
       real*4,parameter :: R4=1.
       real*8,parameter :: R8=1.
       real,  parameter :: R =1.
 
       integer,parameter :: SP = kind(1.  )
       integer,parameter :: DP = kind(1.D0)
 
       integer,parameter :: kind_r4=kind(R4)
       integer,parameter :: kind_r8=kind(R8)
       integer,parameter :: kind_r =kind(R )
 \end{verbatim}%{\sf REVISION HISTORY:}
%\begin{verbatim}   	19Feb98 - Jing Guo <guo@thunder> - initial prototype/prolog/code\end{verbatim}
%
%\markboth{Left}{Source File: m\_showDistrib.F90,  Date: Thu Nov 15 22:24:28 GMT 2001
%}
 
%/////////////////////////////////////////////////////////////
 
\mbox{}\hrulefill\ 
 
  \subsection{Module m\_showDistrib - show distributions }

\bigskip{\sf INTERFACE:}
\begin{verbatim} 
     module m_showDistrib
       public :: showDistrib		! The class data structure
 
       interface showDistrib; module procedure	&
 	 showalli_,	&
 	 showallr_,	&
 	 showalld_
       end interface
 \end{verbatim}%{\sf REVISION HISTORY:}
%\begin{verbatim}   	02Mar01	- Jing Guo <guo@dao.gsfc.nasa.gov>
%  		- initial prototype/prolog/code\end{verbatim}
 
%/////////////////////////////////////////////////////////////
 
\mbox{}\hrulefill\ 

  \subsubsection{showalli\_ - show all INTEGER values}

\bigskip{\sf INTERFACE:}
\begin{verbatim} 
     subroutine showalli_(lu,vnam,ival,root,comm,listall)
       integer,intent(in) :: lu		! output unit
       character(len=*),intent(in) :: vnam	! variable name
       integer,intent(in) :: ival	! to be listed value
       integer,intent(in) :: root	! root PE
       integer,intent(in) :: comm	! communicator
       logical,optional,intent(in) :: listall
 \end{verbatim}%{\sf REVISION HISTORY:}
%\begin{verbatim}   	02Mar01	- Jing Guo <guo@dao.gsfc.nasa.gov>
%  		- initial prototype/prolog/code\end{verbatim}
 
%/////////////////////////////////////////////////////////////
 
\mbox{}\hrulefill\ 
 
  \subsubsection{showallr\_ - show all INTEGER values}

\bigskip{\sf INTERFACE:}
\begin{verbatim} 
     subroutine showallr_(lu,vnam,rval,root,comm,listall,scalar)
       integer,intent(in) :: lu		! output unit
       character(len=*),intent(in) :: vnam	! variable name
       real(SP),intent(in) :: rval	! to be listed value
       integer,intent(in) :: root	! root PE
       integer,intent(in) :: comm	! communicator
       logical,optional,intent(in) :: listall
       real(SP),optional,intent(in) :: scalar
 \end{verbatim}%{\sf REVISION HISTORY:}
%\begin{verbatim}   	02Mar01	- Jing Guo <guo@dao.gsfc.nasa.gov>
%  		- initial prototype/prolog/code\end{verbatim}
 
%/////////////////////////////////////////////////////////////
 
\mbox{}\hrulefill\ 
 
  \subsubsection{showalld\_ - show all INTEGER values}

\bigskip{\sf INTERFACE:}
\begin{verbatim} 
     subroutine showalld_(lu,vnam,rval,root,comm,listall,scalar)
       integer,intent(in) :: lu		! output unit
       character(len=*),intent(in) :: vnam	! variable name
       real(DP),intent(in) :: rval	! to be listed value
       integer,intent(in) :: root	! root PE
       integer,intent(in) :: comm	! communicator
       logical,optional,intent(in) :: listall
       real(DP),optional,intent(in) :: scalar
 \end{verbatim}%{\sf REVISION HISTORY:}
%\begin{verbatim}   	02Mar01	- Jing Guo <guo@dao.gsfc.nasa.gov>
%  		- initial prototype/prolog/code\end{verbatim}

%\markboth{Left}{Source File: m\_stdio.F90,  Date: Thu Nov 15 22:24:28 GMT 2001
%}

 
%/////////////////////////////////////////////////////////////
 
\mbox{}\hrulefill\ 
 
  \subsection{Module m\_stdio - a F90 module defines std. I/O parameters }

  	Define system dependent I/O parameters.
  
\bigskip{\sf INTERFACE:}
\begin{verbatim} 
     module m_stdio
       public	:: stdin	! a unit linked to UNIX stdin
       public	:: stdout	! a unit linked to UNIX stdout
       public	:: stderr	! a unit linked to UNIX stderr
 
       public	:: LEN_FILENAME
 \end{verbatim}%{\sf REVISION HISTORY:}
%\begin{verbatim}  	10oct96 - Jing G.	- Defined\end{verbatim}

%\markboth{Left}{Source File: m\_zeit.F90,  Date: Thu Nov 15 22:24:28 GMT 2001
%}

%/////////////////////////////////////////////////////////////
 
\mbox{}\hrulefill\ 
 
  \subsection{Module m\_zeit - a multi-timer of process times and wall-clock times }

\bigskip{\sf INTERFACE:}
\begin{verbatim} 
     module m_zeit
       public :: zeit_ci		! push a new name to the timer
       public :: zeit_co		! pop the current name on the timer
       public :: zeit_flush	! print per PE timing
       public :: zeit_allflush	! print all PE timing
       public :: zeit_reset	! reset the timers to its initial state
 
 	! Flags of all printable timers
 
       public ::  MWTIME	! MPI_Wtime() wall-clock time
       public ::  XWTIME	! times() wall-clock time
       public ::  PUTIME	! times() process user time
       public ::  PSTIME	! times() process system time
       public ::  CUTIME	! times() user time of all child-processes
       public ::  CSTIME	! times() system time of all child-processes
       public :: ALLTIME	! all of above
       public ::  UWRATE ! (putime+cutime)/xwtime
 
       interface zeit_ci;    module procedure ci_;    end interface
       interface zeit_co;    module procedure co_;    end interface
       interface zeit_flush; module procedure flush_; end interface
       interface zeit_allflush; module procedure allflush_; end interface
       interface zeit_reset; module procedure reset_; end interface
 \end{verbatim}%{\sf REVISION HISTORY:}
%\begin{verbatim}   	05Mar98 - Jing Guo <guo@thunder>	-
%  		. rewritten for possible MPI applications, with
%  		  additional functionalities and new performance
%  		  analysis information.
%  		. Interface names have been redefined to ensure all
%  		  use cases to be verified.
%  		. removed the type(pzeit) data structure, therefore,
%  		  limited to single _instance_ applications.
%  		. added additional data components for more detailed
%  		  timing analysis.
%  		. used times() for the XPG4 standard conforming
%  		  timing functions.
%  		. used MPI_Wtime() for the MPI standard conforming
%  		  high-resolution timing functions.
%   	20Feb97 - Jing Guo <guo@eramus>		-
%  		. rewritten in Fortran 90 as the first modular
%  		  version, with a type(pzeit) data structure.
%  	10may96 - Jing G. -	Add _TZEITS macro for the testing code
%  	09may96 - Jing G. -	Changed output format also modifed
%  				comments
%  	11Oct95 - Jing G. -	Removed earlier way of letting clock
%  				timing (clkknt and clktot) to be no less
%  				then the CPU timing, following a
%  				suggestion by James Abeles from Cray.
%  				This way, users may use the routings to
%  				timing multitasking speedup as well.
%  	12May95	- Jing G. -	Merged zeitCRAY.f and zeitIRIS.f.
%  	Before	- ?	  -	See zeitCRAY.f and zeitIRIS.f for more
%  				information.  Authors of those files are
%  				not known to me.
%\end{verbatim}
%\begin{verbatim}
%   !DESIGN ISSUES:
%  	05Mar98	- Jing Guo <guo@thunder>	-
%  		. Removing the data structure may be consider as a 
%  		  limitation to future changes to multiple _instance_
%  		  applications.  However, it is unlikely there will be
%  		  any neccessary multi-_intance_ application soon, if
%  		  ever for this module.
%  		. Without an additional layer with the derived
%  		  datatype, one may worry less the tricky performance
%  		  issues associated with ci_/co_.
%  		. Performance issue with the flush_() calls are not
%  		  considered.
%  	20Feb97	- Jing Guo <guo@eramus>		-
%  		. Currently a single threaded module.  May be easily
%  		  extended to multi-threaded module by adding the name
%  		  of an instance of the class to the argument list.  It
%  		  requires some but very limited interface extensions.
%  		  Right now, the backward compatibility is the main
%  		  issue.
%        10may96 - Jing Guo <guo@eramus>		-
%               + This `zeit' subroutine collection replaces original zeit files
%  	used in PSAS on both systems, UNICOS and IRIX, with following
%  	changes:
%  	      +	Removed the some bugs in zeitCRAY.f that overite the
%  		first user defined name entry in a special situation
%  		(but not being able to correct in zeitCRAY.f).
%  	      + Unified both zeitCRAY.f and zeitIRIS.f in to one file
%  		(this file), that handles system dependency in only
%  		one subroutine syszeit_() with a couple of lines of
%  		differences.
%  	      + Added system CPU time counts for system supporting
%  		the function.
%  	      + Added some error checking and reporting functions.
%  	      + According to zeitCRAY.f, "zeit" is "time" in Germen.
%  		The name is used through the code as another name for
%  		"time".
%  	      + This version does not work for parallelized processes.
%       + Elapsed time records since the first call are used.  Although
%  	it may loose accuracy when the values of the time records
%  	become large, it will keep the total time values conserved.
%       +	The accuracy of the elapsed times at a IEEE real*4 accuracy
%  	(ffrac = 2^23 ~= 1.19e-7) should be no worse than +- 1 second
%  	in 97 days, if only the numerical accuracy is considered.
%       +	The precision of "wall clock" time returned by syszeit_() is
%  	only required to be reliable upto seconds.
%       +	The "wall clock" time for individual `name' (clkknt) is
%  	accumulated by adding the differences between two integer
%  	values, iclk and iclksv.  Care must be taken to compute the
%  	differences of iclk and iclksv first.  That is, doing
%  		clkknt()=clkknt() + (iclk-iclksv)
%  	not
%  		clkknt()=clkknt() + iclk-iclksv
%  	The latter statement may ignore the difference between the two
%  	integer values (iclk and iclksv).
%\end{verbatim}
 
%/////////////////////////////////////////////////////////////
 
\mbox{}\hrulefill\ 
 
  \subsubsection{ci\_ - push an entry into the timer}

\bigskip{\sf INTERFACE:}
\begin{verbatim} 
     subroutine ci_(name)
       character(len=*), intent(in) :: name
 \end{verbatim}%{\sf REVISION HISTORY:}
%\begin{verbatim}   	05Mar98 - Jing Guo <guo@thunder> - initial prototype/prolog/code\end{verbatim}
 
%/////////////////////////////////////////////////////////////
 
\mbox{}\hrulefill\ 
 
  \subsubsection{co\_ - pop the current level}

\bigskip{\sf INTERFACE:}
\begin{verbatim} 
     subroutine co_(name,tms)
       character(len=*), intent(in) :: name	! account name
       real*8,optional,dimension(0:5,0:1),intent(out) :: tms ! timings
 
       The returned variable tms(0:5,0:1) contains two sets of timing
     information.  tms(0:5,0) is the NET timing data charged under the
     account name only, and tms(0:5,1) is the SCOPE timing data since
     the last ci() with the same account name and at the out most level.\end{verbatim}%{\sf REVISION HISTORY:}
%\begin{verbatim}   	11Oct99 - J.W. Larson - <jlarson@dao> explicit definition of 
%                   tms as real*8
%   	05Mar98 - Jing Guo <guo@thunder> - initial prototype/prolog/code\end{verbatim}
 
%/////////////////////////////////////////////////////////////
 
\mbox{}\hrulefill\ 
 

  \subsubsection{reset\_ - reset module m\_zeit to an initial state}

\bigskip{\sf INTERFACE:}
\begin{verbatim} 
     subroutine reset_()
 \end{verbatim}%{\sf REVISION HISTORY:}
%\begin{verbatim}   	04Mar98 - Jing Guo <guo@thunder> - initial prototype/prolog/code\end{verbatim}
 
%/////////////////////////////////////////////////////////////
 
\mbox{}\hrulefill\ 
 
  \subsubsection{lookup\_ search/insert a name}

\bigskip{\sf INTERFACE:}
\begin{verbatim} 
     function lookup_(name)
       character(len=*),intent(in) :: name
       integer :: lookup_
 \end{verbatim}%{\sf REVISION HISTORY:}
%\begin{verbatim}   	04Mar98 - Jing Guo <guo@thunder> - initial prototype/prolog/code\end{verbatim}
 
%/////////////////////////////////////////////////////////////
 
\mbox{}\hrulefill\ 
 
  \subsubsection{flush\_ - print the timing data}

\bigskip{\sf INTERFACE:}
\begin{verbatim} 
     subroutine flush_(lu,umask)
       integer,intent(in) :: lu	! logical unit for the output
       integer,optional,intent(in) :: umask
 \end{verbatim}%{\sf REVISION HISTORY:}
%\begin{verbatim}   	05Mar98 - Jing Guo <guo@thunder> - initial prototype/prolog/code\end{verbatim}
 
%/////////////////////////////////////////////////////////////
 
\mbox{}\hrulefill\ 
 
  \subsubsection{sp\_balances\_ - print a table of a given timer}

\bigskip{\sf INTERFACE:}
\begin{verbatim} 
     subroutine sp_balances_(lu,itm,zti)
       integer,intent(in) :: lu
       integer,intent(in) :: itm
       real*8,intent(in) :: zti
 \end{verbatim}%{\sf REVISION HISTORY:}
%\begin{verbatim}   	06Mar98 - Jing Guo <guo@thunder> - initial prototype/prolog/code\end{verbatim}
 
%/////////////////////////////////////////////////////////////
 
\mbox{}\hrulefill\ 
 
  \subsubsection{allflush\_ - print a summary of all PEs.}

\bigskip{\sf INTERFACE:}
\begin{verbatim} 
     subroutine allflush_(comm,root,lu,umask)
       integer,intent(in) :: comm
       integer,intent(in) :: root
       integer,intent(in) :: lu
       integer,optional,intent(in) :: umask
 \end{verbatim}%{\sf REVISION HISTORY:}
%\begin{verbatim}   	09Mar98 - Jing Guo <guo@thunder> - initial prototype/prolog/code\end{verbatim}
 
%/////////////////////////////////////////////////////////////
 
\mbox{}\hrulefill\ 
 
  \subsubsection{mp\_balances\_ - summarize the timing data of all PEs}

%   \newcommand{\tb}{\overline{t}}
  
  	\verb"mp_balances_"() summarizes the timing data of all PEs
     with quantified load balancing measures:
     \begin{eqnarray*}
  	x &=& \frac{\max(t) - \tb}{N\tb}	\times 100\%	\\
  	i &=& \frac{\max(t) - \tb}{\max(t)}	\times 100\%	\\
  	r &=& \frac{1}{N\tb} \sum^{t>\tb}{(t-\tb)}
  		\times 100\%
     \end{eqnarray*}
     where
     \begin{center}
       \begin{tabular}{rl}
         $t$: & time by any process element			\\
       $\tb$: & mean time by all process elements		\\
  	$x$: & the ma{\bf x}imum percentage load deviation	\\
  	$i$: & percentage {\bf i}dle process-time or
  					load {\bf i}mbalance	\\
  	$r$: & percentage {\bf r}elocatable loads		\\
  	$N$: & {\bf n}umber of process elements
       \end{tabular}
     \end{center}
  
\bigskip{\sf INTERFACE:}
\begin{verbatim} 
     subroutine mp_balances_(lu,item,nPEs,names,zsum,zavg,zabv,zmax)
       integer,intent(in) :: lu
       integer,intent(in) :: item
       integer,intent(in) :: nPEs
       character(len=1),dimension(:,0:),intent(in) :: names
       real*8,   dimension(0:,0:,0:   ),intent(in) :: zsum
       real*8,   dimension(0:,0:,0:   ),intent(in) :: zavg
       real*8,   dimension(0:,0:,0:   ),intent(in) :: zabv
       real*8,   dimension(0:,0:,0:,0:),intent(in) :: zmax
\end{verbatim}
%{\sf REVISION HISTORY:}
%\begin{verbatim}   	10Mar98 - Jing Guo <guo@thunder> - initial prototype/prolog/code\end{verbatim}
 
%/////////////////////////////////////////////////////////////
 
\mbox{}\hrulefill\ 
 
  \subsubsection{allreduce\_ -}

\bigskip{\sf INTERFACE:}
\begin{verbatim} 
     subroutine allreduce_(ztmp,zsum,zavg,zabv,zmax,myID,nPEs,comm)
       real*8,dimension(0:,0:,0:   ),intent(in ) :: ztmp
       real*8,dimension(0:,0:,0:   ),intent(out) :: zsum
       real*8,dimension(0:,0:,0:   ),intent(out) :: zavg
       real*8,dimension(0:,0:,0:   ),intent(out) :: zabv
       real*8,dimension(0:,0:,0:,0:),intent(out) :: zmax
       integer,intent(in) :: myID
       integer,intent(in) :: nPEs
       integer,intent(in) :: comm
 \end{verbatim}
%{\sf REVISION HISTORY:}
%\begin{verbatim}   	15Mar01	- Jing Guo <guo@dao.gsfc.nasa.gov>
%  		- initial prototype/prolog/code\end{verbatim}

